\chapter{集成电路}

集成电路(Integrated Circuit,IC)是将大量电子元件(晶体管、电阻、电容等)集成在一块半导体晶片上的电子器件。自 1958 年第一块集成电路诞生以来,IC 技术经历了飞速发展,从最初只有几个晶体管的简单电路,发展到今天包含数十亿晶体管的复杂系统芯片(SoC)。本章将系统介绍各类集成电路的工作原理、特性参数、应用电路和选型指南。

\section{集成电路基础}

\subsection{集成电路的发展历史}

集成电路的发展历程可以分为以下几个重要阶段:

\begin{enumerate}
  \item \textbf{1958--1960 年代}:小规模集成电路(SSI),集成度 10--100 个元件
  \item \textbf{1970 年代}:中规模集成电路(MSI),集成度 100--1000 个元件
  \item \textbf{1980 年代}:大规模集成电路(LSI),集成度 1000--10 万个元件
  \item \textbf{1990 年代}:超大规模集成电路(VLSI),集成度 10 万--1000 万个元件
  \item \textbf{2000 年代至今}:超大规模集成电路(ULSI)和系统芯片(SoC),集成度超过 1000 万个元件
\end{enumerate}

\subsection{集成电路的分类}

\subsubsection{按功能分类}

\begin{enumerate}
  \item \textbf{模拟集成电路}
    \begin{itemize}
      \item 运算放大器
      \item 比较器
      \item 基准电压源
      \item 模拟开关和多路复用器
      \item 数据转换器(ADC/DAC)
      \item 电源管理 IC
      \item 射频 IC
    \end{itemize}

  \item \textbf{数字集成电路}
    \begin{itemize}
      \item 逻辑门电路
      \item 触发器
      \item 计数器
      \item 存储器
      \item 微处理器和微控制器
      \item 数字信号处理器(DSP)
      \item 可编程逻辑器件(FPGA/CPLD)
    \end{itemize}

  \item \textbf{混合信号集成电路}
    \begin{itemize}
      \item 同时包含模拟和数字电路
      \item 数据转换器
      \item 接口电路
    \end{itemize}
\end{enumerate}

\subsubsection{按制造工艺分类}

\begin{enumerate}
  \item \textbf{双极型工艺}:速度快,但功耗大,集成度低
  \item \textbf{CMOS 工艺}:功耗低,集成度高,是现代 IC 的主流工艺
  \item \textbf{BiCMOS 工艺}:结合双极型和 CMOS 的优点
  \item \textbf{GaAs 工艺}:用于高频、高速应用
\end{enumerate}

\subsubsection{按封装分类}

\begin{enumerate}
  \item \textbf{通孔封装}:DIP、TO 等
  \item \textbf{表面贴装封装}:SOP、QFP、BGA 等
  \item \textbf{芯片级封装}:CSP、WLCSP 等
\end{enumerate}

\subsection{集成电路的制造工艺}

\subsubsection{基本工艺流程}

集成电路制造的主要步骤包括:

\begin{enumerate}
  \item \textbf{晶圆制备}:从硅锭切割出晶圆
  \item \textbf{氧化}:在晶圆表面生长二氧化硅层
  \item \textbf{光刻}:使用光刻胶和掩膜版定义电路图形
  \item \textbf{刻蚀}:去除不需要的材料
  \item \textbf{掺杂}:通过离子注入或扩散引入杂质
  \item \textbf{沉积}:沉积金属或多晶硅层
  \item \textbf{化学机械抛光(CMP)}:平整化表面
  \item \textbf{测试和封装}:测试芯片功能并封装
\end{enumerate}

\subsubsection{特征尺寸}

特征尺寸(Feature Size)是指制造工艺能够实现的最小线宽,通常用纳米(nm)表示。特征尺寸越小,集成度越高,性能越好,但制造难度也越大。

\subsection{集成电路的封装}

\subsubsection{封装的作用}

封装的主要作用包括:
\begin{itemize}
  \item 保护芯片免受机械损伤和环境影响
  \item 提供电气连接(引脚)
  \item 散热
  \item 便于安装和测试
\end{itemize}

\subsubsection{常见封装类型}

\begin{enumerate}
  \item \textbf{DIP(Dual In-line Package)}:双列直插封装,适用于通孔安装
  \item \textbf{SOP/SOIC(Small Outline Package)}:小外形封装,表面贴装
  \item \textbf{QFP(Quad Flat Package)}:四边扁平封装,引脚密度高
  \item \textbf{BGA(Ball Grid Array)}:球栅阵列封装,引脚在底部
  \item \textbf{QFN(Quad Flat No-leads)}:无引脚四边扁平封装,散热好
\end{enumerate}

\section{电源管理集成电路}

电源管理集成电路(Power Management IC,PMIC)是现代电子系统的重要组成部分,负责为系统提供稳定、高效的电源。电源管理 IC 包括线性稳压器、开关稳压器、电池管理、电源监控等功能。

\subsection{线性稳压器(LDO)}

线性稳压器(Low Dropout Regulator,LDO)是一种使用线性调节方式的稳压器,具有结构简单、噪声低、响应快等优点。

\subsubsection{LDO 的工作原理}

LDO 的基本结构如图~\ref{fig:ldo-basic} 所示,包括误差放大器、调整管(通常为 PNP 或 PMOS 晶体管)和反馈网络。

\begin{figure}[H]
  \centering
  \begin{circuitikz}[american,scale=1.0]
    % Error Amplifier
    \draw (0,2) node[op amp] (opamp) {};

    % Pass Transistor (PNP)
    \draw (opamp.out) to[short] ++(0.5,0) coordinate (base);
    \draw (base) to[short] ++(0,0.5) node[pnp, anchor=B] (Q1) {};

    % Connections to OpAmp
    \draw (opamp.+) to[short] ++(-0.5,0) to[short, -o] ++(-0.5,0) node[left] {$V_{ref}$};

    % Input
    \draw (Q1.E) to[short, -o] ++(0,0.5) node[above] {$V_{in}$};

    % Output and Feedback
    \draw (Q1.C) to[short, *-o] ++(2,0) coordinate (out) node[right] {$V_{out}$};
    \draw (Q1.C) to[short] ++(0,-0.5) coordinate (fb_top);
    \draw (fb_top) to[R=$R_1$] ++(0,-2) coordinate (fb_mid);
    \draw (fb_mid) to[R=$R_2$] ++(0,-2) node[ground] {};

    % Feedback connection
    \draw (fb_mid) to[short, *-] (fb_mid -| opamp.-) -- (opamp.-);
  \end{circuitikz}
  \caption{LDO 基本结构}\label{fig:ldo-basic}
\end{figure}

工作原理:
\begin{enumerate}
  \item 误差放大器比较参考电压 $V_{ref}$ 和反馈电压 $V_{fb}$
  \item 根据误差信号控制调整管的导通程度
  \item 反馈网络将输出电压分压后反馈到误差放大器
  \item 通过负反馈使输出电压稳定在设定值
\end{enumerate}

输出电压为:

\begin{equation}
V_{out} = V_{ref} \left(1 + \frac{R_1}{R_2}\right) \label{eq:ldo-output}
\end{equation}

\subsubsection{LDO 的关键参数}

\begin{enumerate}
  \item \textbf{压差(Dropout Voltage)$V_{DO}$}
    \begin{itemize}
      \item 定义:维持正常稳压所需的最小输入输出压差
      \item 典型值:0.1--0.5 V(低压差 LDO)
      \item 压差越小,效率越高
    \end{itemize}

  \item \textbf{负载调整率(Load Regulation)}
    \begin{itemize}
      \item 定义:负载电流变化时输出电压的变化
      \item 通常用百分比或 mV 表示
    \end{itemize}

  \item \textbf{线性调整率(Line Regulation)}
    \begin{itemize}
      \item 定义:输入电压变化时输出电压的变化
      \item 反映 LDO 对输入电压变化的抑制能力
    \end{itemize}

  \item \textbf{静态电流(Quiescent Current)$I_Q$}
    \begin{itemize}
      \item 定义:无负载时 LDO 自身消耗的电流
      \item 影响轻载效率,低功耗应用要求 $I_Q$ 小
    \end{itemize}

  \item \textbf{电源抑制比(PSRR)}
    \begin{itemize}
      \item 定义:LDO 对输入电源纹波的抑制能力
      \item 频率越高,PSRR 通常越低
    \end{itemize}

  \item \textbf{输出噪声}
    \begin{itemize}
      \item LDO 内部产生的噪声
      \item 低噪声 LDO 用于精密应用
    \end{itemize}

  \item \textbf{瞬态响应}
    \begin{itemize}
      \item 负载电流突变时输出电压的恢复时间
      \item 影响动态性能
    \end{itemize}
\end{enumerate}

\subsubsection{LDO 的效率}

LDO 的效率为:

\begin{equation}
\eta = \frac{P_{out}}{P_{in}} = \frac{V_{out} I_{out}}{V_{in} I_{in}} \approx \frac{V_{out}}{V_{in}} \label{eq:ldo-efficiency}
\end{equation}

由于 $V_{out} < V_{in}$,LDO 的效率总是小于 100\%,且压差越大,效率越低。这是 LDO 的主要缺点。

\subsubsection{LDO 的应用}

LDO 适用于以下场景:
\begin{itemize}
  \item 低噪声要求的应用(如模拟电路供电)
  \item 小电流应用(< 1 A)
  \item 压差较小的应用
  \item 需要快速响应的应用
  \item 成本敏感的应用
\end{itemize}

\subsubsection{LDO 的选型}

选择 LDO 时需要考虑:
\begin{enumerate}
  \item 输入输出电压范围
  \item 最大输出电流
  \item 压差要求
  \item 静态电流(低功耗应用)
  \item PSRR(对电源噪声敏感的应用)
  \item 输出噪声(精密应用)
  \item 封装和热设计
\end{enumerate}

\subsection{开关稳压器(DC-DC)}

开关稳压器(Switching Regulator)通过开关管的开关动作,将输入电压转换为所需的输出电压,具有效率高、功率密度大等优点。

\subsubsection{开关稳压器的基本工作原理}

开关稳压器的基本结构包括开关管、电感、电容、二极管(或同步开关管)和控制电路,如图~\ref{fig:buck-basic} 所示(以 Buck 转换器为例)。

\begin{figure}[H]
  \centering
  \begin{circuitikz}[american,scale=0.8]
    \draw
    (0,3) node[above] {$V_{in}$} to[switch, name=S1] (2,3)
    (2,3) to[L=$L$] (4,3)
    (4,3) to[C=$C$] (4,0) node[ground] {}
    (4,3) to[short] (5,3) node[right] {$V_{out}$}
    (2,0) node[ground] {}
    (2,3) to[D, name=D1] (2,0)
    (0,0) to[short] (5,0);
  \end{circuitikz}
  \caption{Buck 转换器基本结构}\label{fig:buck-basic}
\end{figure}

工作原理:
\begin{enumerate}
  \item 开关管导通时,输入电压通过电感和负载,电感储能
  \item 开关管关断时,电感通过二极管(或同步开关管)续流,向负载供电
  \item 通过控制开关管的占空比,调节输出电压
\end{enumerate}

\subsubsection{开关稳压器的拓扑结构}

\paragraph{Buck 转换器(降压)}

Buck 转换器将高电压转换为低电压,输出电压为:

\begin{equation}
V_{out} = D \cdot V_{in} \label{eq:buck-output}
\end{equation}

其中 $D$ 是占空比($0 < D < 1$)。

特点:
\begin{itemize}
  \item 输出电压低于输入电压
  \item 效率高(通常 80--95\%)
  \item 输出电流连续
\end{itemize}

\paragraph{Boost 转换器(升压)}

Boost 转换器将低电压转换为高电压,输出电压为:

\begin{equation}
V_{out} = \frac{V_{in}}{1 - D} \label{eq:boost-output}
\end{equation}

特点:
\begin{itemize}
  \item 输出电压高于输入电压
  \item 输入电流连续
  \item 输出电流断续
\end{itemize}

\paragraph{Buck-Boost 转换器}

Buck-Boost 转换器可以实现升压和降压,输出电压为:

\begin{equation}
V_{out} = \frac{D}{1 - D} V_{in} \label{eq:buck-boost-output}
\end{equation}

特点:
\begin{itemize}
  \item 输出电压可以高于或低于输入电压
  \item 输出电压与输入电压反相
\end{itemize}

\paragraph{SEPIC 和 Cuk 转换器}

SEPIC(Single-Ended Primary Inductor Converter)和 Cuk 转换器也可以实现升降压,且输出电压与输入电压同相。

\subsubsection{开关稳压器的控制方式}

\begin{enumerate}
  \item \textbf{脉宽调制(PWM)}
    \begin{itemize}
      \item 开关频率固定,调节占空比
      \item 适用于大功率应用
      \item 效率高,但轻载效率低
    \end{itemize}

  \item \textbf{脉冲频率调制(PFM)}
    \begin{itemize}
      \item 占空比固定或变化,调节开关频率
      \item 适用于轻载应用
      \item 轻载效率高,但输出纹波较大
    \end{itemize}

  \item \textbf{混合模式}
    \begin{itemize}
      \item 重载时使用 PWM,轻载时使用 PFM
      \item 兼顾效率和性能
    \end{itemize}
\end{enumerate}

\subsubsection{开关稳压器的关键参数}

\begin{enumerate}
  \item \textbf{效率}
    \begin{itemize}
      \item 开关稳压器的效率通常为 80--95\%
      \item 效率与负载电流、输入输出电压、开关频率等因素有关
    \end{itemize}

  \item \textbf{开关频率}
    \begin{itemize}
      \item 典型值:几百 kHz 到几 MHz
      \item 频率越高,电感和电容可以越小,但开关损耗增加
    \end{itemize}

  \item \textbf{输出纹波}
    \begin{itemize}
      \item 输出电压的交流分量
      \item 与开关频率、电感和电容值有关
    \end{itemize}

  \item \textbf{负载调整率和线性调整率}
    \begin{itemize}
      \item 与 LDO 类似,但通常比 LDO 差
    \end{itemize}

  \item \textbf{瞬态响应}
    \begin{itemize}
      \item 负载突变时输出电压的恢复时间
      \item 影响动态性能
    \end{itemize}
\end{enumerate}

\subsubsection{开关稳压器的设计考虑}

\paragraph{电感选择}

电感值影响输出纹波和瞬态响应:

\begin{equation}
L = \frac{(V_{in} - V_{out}) D}{f_{sw} \Delta I_L} \label{eq:inductor-selection}
\end{equation}

其中 $\Delta I_L$ 是电感电流纹波,通常选择为负载电流的 20--40\%。

\paragraph{电容选择}

输出电容影响输出纹波:

\begin{equation}
C_{out} \geq \frac{\Delta I_L}{8 f_{sw} \Delta V_{out}} \label{eq:capacitor-selection}
\end{equation}

其中 $\Delta V_{out}$ 是允许的输出纹波。

\paragraph{开关频率选择}

开关频率的选择需要在效率和体积之间权衡:
\begin{itemize}
  \item 频率高:电感和电容小,但开关损耗大
  \item 频率低:效率高,但电感和电容大
\end{itemize}

\subsubsection{LDO 与开关稳压器的比较}

\begin{table}[H]
  \centering
  \caption{LDO 与开关稳压器的比较}
  \begin{tabular}{|l|l|l|}
    \hline
    特性 & LDO & 开关稳压器 \\
    \hline
    效率 & 低(30--70\%) & 高(80--95\%) \\
    输出噪声 & 低 & 高 \\
    响应速度 & 快 & 较慢 \\
    成本 & 低 & 较高 \\
    复杂度 & 简单 & 复杂 \\
    适用电流 & 小(< 1 A) & 大(> 100 mA) \\
    压差 & 需要 & 不需要 \\
    \hline
  \end{tabular}
\end{table}

\subsection{电源管理单元(PMIC)}

电源管理单元(Power Management Unit,PMIC)是集成了多种电源管理功能的芯片,通常包括:

\begin{itemize}
  \item 多个 LDO 和开关稳压器
  \item 电池充电管理
  \item 电源监控和保护
  \item 电源时序控制
  \item 低功耗模式管理
\end{itemize}

PMIC 广泛应用于便携式设备、智能手机、平板电脑等,可以简化电源设计,提高系统效率。

\section{定时器集成电路}

定时器 IC 用于产生精确的时间延迟、脉冲宽度调制(PWM)信号、振荡信号等。最著名的是 555 定时器。

\subsection{555 定时器}

555 定时器是一种多功能的模拟-数字混合集成电路,可以用于产生各种定时和振荡功能。

\subsubsection{555 定时器的内部结构}

555 定时器内部包括:
\begin{itemize}
  \item 两个比较器(上比较器和下比较器)
  \item 一个 RS 触发器
  \item 一个放电晶体管
  \item 一个输出缓冲器
  \item 三个 5 k$\Omega$ 电阻组成的分压网络(555 名称的由来)
\end{itemize}

\subsubsection{555 定时器的引脚}

555 定时器(8 引脚 DIP 封装)的引脚功能:
\begin{enumerate}
  \item \textbf{GND}:地
  \item \textbf{TRIG}:触发输入
  \item \textbf{OUT}:输出
  \item \textbf{RESET}:复位输入
  \item \textbf{CTRL}:控制电压(用于调制)
  \item \textbf{THRES}:阈值输入
  \item \textbf{DISCH}:放电端
  \item \textbf{$V_{CC}$}:电源
\end{enumerate}

\subsubsection{555 定时器的工作模式}

\paragraph{单稳态模式(Monostable Mode)}

单稳态模式用于产生固定宽度的脉冲,电路如图~\ref{fig:555-monostable} 所示。

% \begin{figure}[H]
%   \centering
%   \begin{circuitikz}[american,scale=1.0]
%     \tikzset{ic555/.style={muxdemux,
%             muxdemux def={Lh=10, NL=5, Rh=10, NR=5,
%             NB=2, w=6, NT=2, square pins=1},
%         no input leads, external pins width=0.4,
%         circuitikz/muxdemuxes/fill=blue!10}
%     }
%     % 555 Timer Chip
%     \node [ic555, font=\small\ttfamily,align=center](A) at (4,2.5) {555\\Timer};

%     % Pin assignments based on standard 555 pinout
%     % Left: 2 (Trigger), 6 (Threshold), 7 (Discharge) - arbitrary mapping to mux pins
%     % Right: 3 (Output)
%     % Top: 8 (VCC), 4 (Reset)
%     % Bottom: 1 (GND), 5 (Control)

%     % Define pins for readability
%     % Using muxdemux pins: lpin (left), rpin (right), tpin (top), bpin (bottom)
%     % We map schematic logic to visual pins

%     % Left side
%     \draw (A.lpin 2) -- (A.blpin 2) node[midway, blue, font=\tiny, above]{7} node[right, font=\tiny]{DISCH};
%     \draw (A.lpin 3) -- (A.blpin 3) node[midway, blue, font=\tiny, above]{6} node[right, font=\tiny]{THRES};
%     \draw (A.lpin 4) -- (A.blpin 4) node[midway, blue, font=\tiny, above]{2} node[right, font=\tiny]{TRIG};

%     % Right side
%     \draw (A.rpin 3) -- (A.brpin 3) node[midway, blue, font=\tiny, above]{3} node[left, font=\tiny]{OUT};

%     % Top side
%     \draw (A.tpin 1) -- (A.btpin 1) node[midway, blue, font=\tiny, left]{8} node[below, font=\tiny]{VCC};
%     \draw (A.tpin 2) -- (A.btpin 2) node[midway, blue, font=\tiny, left]{4} node[below, font=\tiny]{RESET};

%     % Bottom side
%     \draw (A.bpin 1) -- (A.bbpin 1) node[midway, blue, font=\tiny, left]{5} node[above, font=\tiny]{CTRL};
%     \draw (A.bpin 2) -- (A.bbpin 2) node[midway, blue, font=\tiny, left]{1} node[above, font=\tiny]{GND};

%     % External Circuitry for Monostable

%     % VCC connection
%     \draw (A.btpin 1) -- ++(0,1) node[vcc] {$V_{CC}$};
%     \draw (A.btpin 2) -- ++(0,0.5) -- ++(-1.5,0) -- ++(0,0.5); % Reset to VCC

%     % RC Network
%     \draw (A.btpin 1) ++(0,1) -- ++(-4,0) coordinate (vcc_line);
%     \draw (vcc_line) to[R=$R$] ++(0,-2.5) coordinate (rc_node);
%     \draw (rc_node) to[C=$C$] ++(0,-2) coordinate (gnd_line) node[ground] {};

%     % Connections to chip
%     \draw (rc_node) -- (rc_node -| A.blpin 2) -- (A.blpin 2); % To Discharge
%     \draw (rc_node) -- (rc_node -| A.blpin 3) -- (A.blpin 3); % To Threshold

%     % Trigger Input
%     \draw (A.blpin 4) -- ++(-1,0) node[left] {Trigger Input};

%     % Output
%     \draw (A.brpin 3) -- ++(1,0) node[right] {Output};

%     % Ground
%     \draw (A.bbpin 2) -- ++(0,-0.5) node[ground] {};

%     % Control Voltage (Capacitor usually)
%     \draw (A.bbpin 1) to[C=10nF] ++(0,-1) node[ground] {};

%   \end{circuitikz}
%   \caption{555 单稳态电路}\label{fig:555-monostable}
% \end{figure}

\Figure[caption={555 单稳态电路}, label={fig:555-monostable}, width=0.95]{555-monostable}

输出脉冲宽度为:

\begin{equation}
T = 1.1 RC \label{eq:555-monostable}
\end{equation}

\paragraph{无稳态模式(Astable Mode)}

无稳态模式用于产生方波振荡,电路如图~\ref{fig:555-astable} 所示。

% \begin{figure}[H]
%   \centering
%   \begin{circuitikz}[american,scale=1.0]
%     \tikzset{ic555/.style={muxdemux,
%             muxdemux def={Lh=10, NL=5, Rh=10, NR=5,
%             NB=2, w=6, NT=2, square pins=1},
%         no input leads, external pins width=0.4,
%         circuitikz/muxdemuxes/fill=blue!10}
%     }
%     % 555 Timer Chip
%     \node [ic555, font=\small\ttfamily,align=center](A) at (4,2.5) {555\\Timer};

%     % Pin Mapping
%     % Left: 7 (Discharge), 6 (Threshold), 2 (Trigger)
%     \draw (A.lpin 2) -- (A.blpin 2) node[midway, blue, font=\tiny, above]{7} node[right, font=\tiny]{DISCH};
%     \draw (A.lpin 3) -- (A.blpin 3) node[midway, blue, font=\tiny, above]{6} node[right, font=\tiny]{THRES};
%     \draw (A.lpin 4) -- (A.blpin 4) node[midway, blue, font=\tiny, above]{2} node[right, font=\tiny]{TRIG};

%     % Right: 3 (Output)
%     \draw (A.rpin 3) -- (A.brpin 3) node[midway, blue, font=\tiny, above]{3} node[left, font=\tiny]{OUT};

%     % Top: 8 (VCC), 4 (Reset)
%     \draw (A.tpin 1) -- (A.btpin 1) node[midway, blue, font=\tiny, left]{8} node[below, font=\tiny]{VCC};
%     \draw (A.tpin 2) -- (A.btpin 2) node[midway, blue, font=\tiny, left]{4} node[below, font=\tiny]{RESET};

%     % Bottom: 5 (Control), 1 (GND)
%     \draw (A.bpin 1) -- (A.bbpin 1) node[midway, blue, font=\tiny, left]{5} node[above, font=\tiny]{CTRL};
%     \draw (A.bpin 2) -- (A.bbpin 2) node[midway, blue, font=\tiny, left]{1} node[above, font=\tiny]{GND};

%     % External Circuitry for Astable

%     % VCC connection
%     \draw (A.btpin 1) -- ++(0,1) node[vcc] {$V_{CC}$};
%     \draw (A.btpin 2) -- ++(0,0.5) -- ++(-1.5,0) -- ++(0,0.5); % Reset to VCC

%     % Resistor Network
%     \draw (A.btpin 1) ++(0,1) -- ++(-4,0) coordinate (vcc_line);
%     \draw (vcc_line) to[R=$R_1$] ++(0,-2) coordinate (r1_node);
%     \draw (r1_node) to[R=$R_2$] ++(0,-2) coordinate (r2_node);
%     \draw (r2_node) to[C=$C$] ++(0,-1.5) coordinate (gnd_line) node[ground] {};

%     % Connections to chip
%     \draw (r1_node) -- (r1_node -| A.blpin 2) -- (A.blpin 2); % To Discharge (Pin 7)

%     \draw (r2_node) -- (r2_node -| A.blpin 3) -- (A.blpin 3); % To Threshold (Pin 6)
%     \draw (r2_node) -- ++(0.5,0) |- (A.blpin 4); % To Trigger (Pin 2) - Trigger connected to Threshold in Astable

%     % Output
%     \draw (A.brpin 3) -- ++(1,0) node[right] {$V_{out}$};

%     % Ground
%     \draw (A.bbpin 2) -- ++(0,-0.5) node[ground] {};

%     % Control Voltage
%     \draw (A.bbpin 1) to[C=10nF] ++(0,-1) node[ground] {};

%   \end{circuitikz}
%   \caption{555 无稳态电路}\label{fig:555-astable}
% \end{figure}

\Figure[caption={555 无稳态电路}, label={fig:555-astable}, width=0.95]{555-astable}

振荡频率为:

\begin{equation}
f = \frac{1.44}{(R_1 + 2R_2)C} \label{eq:555-frequency}
\end{equation}

占空比为:

\begin{equation}
D = \frac{R_1 + R_2}{R_1 + 2R_2} \label{eq:555-duty-cycle}
\end{equation}

电容充放电与周期关系图:

\Figure[caption={电容充放电与周期关系图}, label={fig:555-astable-seq}, width=0.6]{555-astable-seq}

\paragraph{双稳态模式(Bistable Mode)}

双稳态模式用作施密特触发器,输出状态由输入信号决定。

\subsubsection{555 定时器的应用}

\begin{enumerate}
  \item 脉冲发生器
  \item PWM 信号发生器
  \item 延时电路
  \item 频率分频器
  \item 电压-频率转换器
\end{enumerate}

\subsection{其他定时器 IC}

除了 555,还有其他定时器 IC,如:
\begin{itemize}
  \item 556:双 555 定时器
  \item 558:四 555 定时器
  \item 专用定时器:如微控制器中的定时器/计数器模块
\end{itemize}

\section{比较器和基准电压源}

\subsection{比较器}

比较器用于比较两个电压的大小,输出数字信号(高电平或低电平)。

\subsubsection{比较器的特性}

比较器与运算放大器类似,但有以下区别:
\begin{itemize}
  \item 比较器设计用于开环工作,不需要负反馈
  \item 比较器通常具有更快的响应速度
  \item 比较器的输出是数字信号,可以直接驱动数字电路
  \item 比较器可能具有推挽输出或开漏输出
\end{itemize}

\subsubsection{比较器的应用}

\begin{enumerate}
  \item 电压监控和过压/欠压保护
  \item 窗口比较器(检测电压是否在范围内)
  \item 零交叉检测
  \item A/D 转换器中的比较器
  \item 施密特触发器(带滞回的比较器)
\end{enumerate}

\subsubsection{施密特触发器}

施密特触发器是具有滞回特性的比较器,可以消除噪声引起的误触发。

滞回电压(Hysteresis)为:

\begin{equation}
V_{HYS} = V_{TH+} - V_{TH-} \label{eq:schmitt-hysteresis}
\end{equation}

其中 $V_{TH+}$ 是上阈值,$V_{TH-}$ 是下阈值。

\subsection{基准电压源}

基准电压源提供精确、稳定的参考电压,用于 ADC、DAC、稳压器等电路。

\subsubsection{基准电压源的分类}

\begin{enumerate}
  \item \textbf{齐纳基准}
    \begin{itemize}
      \item 基于齐纳二极管
      \item 精度中等,温度系数较大
      \item 成本低
    \end{itemize}

  \item \textbf{带隙基准(Bandgap Reference)}
    \begin{itemize}
      \item 利用带隙电压的温度特性
      \item 精度高,温度系数小(< 50 ppm/$^\circ$C)
      \item 现代 IC 中最常用的基准
    \end{itemize}

  \item \textbf{埋层齐纳基准}
    \begin{itemize}
      \item 精度很高,温度系数很小
      \item 成本较高
    \end{itemize}
\end{enumerate}

\subsubsection{基准电压源的关键参数}

\begin{enumerate}
  \item \textbf{初始精度}:出厂时的电压误差
  \item \textbf{温度系数(TC)}:电压随温度的变化率
  \item \textbf{长期稳定性}:电压随时间的变化
  \item \textbf{负载调整率}:负载电流变化时电压的变化
  \item \textbf{线性调整率}:电源电压变化时电压的变化
  \item \textbf{噪声}:基准电压的噪声水平
\end{enumerate}

\subsubsection{常见基准电压}

\begin{itemize}
  \item 1.25 V:带隙基准的典型值
  \item 2.5 V:常用的基准电压
  \item 4.096 V:用于 12 位 ADC 的基准
  \item 5.0 V:传统基准电压
\end{itemize}

\section{模拟开关和多路复用器}

模拟开关和多路复用器用于在多个模拟信号之间进行切换。

\subsection{模拟开关}

模拟开关是电子控制的开关,可以导通或断开模拟信号路径。

\subsubsection{模拟开关的结构}

模拟开关通常使用 MOSFET 实现,包括:
\begin{itemize}
  \item 单刀单掷(SPST)
  \item 单刀双掷(SPDT)
  \item 多路开关
\end{itemize}

\subsubsection{模拟开关的关键参数}

\begin{enumerate}
  \item \textbf{导通电阻($R_{ON}$)}
    \begin{itemize}
      \item 开关导通时的电阻
      \item 典型值:几欧姆到几百欧姆
      \item 影响信号衰减和失真
    \end{itemize}

  \item \textbf{关断电阻($R_{OFF}$)}
    \begin{itemize}
      \item 开关关断时的电阻
      \item 通常 > 1 M$\Omega$
    \end{itemize}

  \item \textbf{带宽}
    \begin{itemize}
      \item 开关能够处理的信号频率范围
    \end{itemize}

  \item \textbf{串扰(Crosstalk)}
    \begin{itemize}
      \item 关断通道对导通通道的影响
    \end{itemize}

  \item \textbf{电荷注入}
    \begin{itemize}
      \item 开关切换时注入到信号路径的电荷
      \item 影响采样保持电路
    \end{itemize}
\end{enumerate}

\subsection{多路复用器}

多路复用器(Multiplexer,MUX)可以从多个输入中选择一个输出。

\subsubsection{多路复用器的应用}

\begin{enumerate}
  \item ADC 的多通道输入选择
  \item 信号路由
  \item 数据采集系统
\end{enumerate}

\section{数据转换器}

数据转换器包括模数转换器(ADC)和数模转换器(DAC),是连接模拟世界和数字世界的桥梁。

\subsection{模数转换器(ADC)}

ADC 将模拟信号转换为数字信号。

\subsubsection{ADC 的主要参数}

\begin{enumerate}
  \item \textbf{分辨率(Resolution)}
    \begin{itemize}
      \item 用位数表示,如 8 位、10 位、12 位、16 位等
      \item 分辨率越高,量化误差越小
    \end{itemize}

  \item \textbf{采样率(Sampling Rate)}
    \begin{itemize}
      \item ADC 每秒采样的次数
      \item 根据奈奎斯特定理,采样率应至少为信号最高频率的 2 倍
    \end{itemize}

  \item \textbf{量化误差}
    \begin{itemize}
      \item 由于量化产生的误差
      \item 对于 $N$ 位 ADC,量化误差为 $\pm \frac{1}{2} LSB$
    \end{itemize}

  \item \textbf{积分非线性(INL)}
    \begin{itemize}
      \item 实际转换特性与理想直线的偏差
    \end{itemize}

  \item \textbf{微分非线性(DNL)}
    \begin{itemize}
      \item 相邻码之间的实际步长与理想步长的偏差
    \end{itemize}

  \item \textbf{信噪比(SNR)}
    \begin{itemize}
      \item 信号功率与噪声功率的比值
      \item 理想 $N$ 位 ADC 的理论 SNR 为 $6.02N + 1.76$ dB
    \end{itemize}

  \item \textbf{有效位数(ENOB)}
    \begin{itemize}
      \item 考虑所有误差后的实际有效位数
      \item ENOB < 分辨率
    \end{itemize}
\end{enumerate}

\subsubsection{ADC 的架构}

\begin{enumerate}
  \item \textbf{逐次逼近型(SAR)}
    \begin{itemize}
      \item 中等速度,中等精度
      \item 功耗低
      \item 适用于一般应用
    \end{itemize}

  \item \textbf{流水线型(Pipeline)}
    \begin{itemize}
      \item 高速,中等精度
      \item 适用于高速数据采集
    \end{itemize}

  \item \textbf{Σ-Δ 型}
    \begin{itemize}
      \item 高精度,低速度
      \item 过采样技术
      \item 适用于高精度测量
    \end{itemize}

  \item \textbf{闪存型(Flash)}
    \begin{itemize}
      \item 极高速度,低精度
      \item 成本高
      \item 适用于超高速应用
    \end{itemize}

  \item \textbf{双斜率型(Dual-Slope)}
    \begin{itemize}
      \item 高精度,低速度
      \item 抗干扰能力强
      \item 适用于数字万用表等
    \end{itemize}
\end{enumerate}

\subsubsection{ADC 的接口}

\begin{enumerate}
  \item \textbf{并行接口}:速度快,但引脚多
  \item \textbf{串行接口}:引脚少,但速度较慢
    \begin{itemize}
      \item SPI
      \item I2C
      \item 串行外设接口(Serial Peripheral Interface)
    \end{itemize}
\end{enumerate}

\subsection{数模转换器(DAC)}

DAC 将数字信号转换为模拟信号。

\subsubsection{DAC 的主要参数}

\begin{enumerate}
  \item \textbf{分辨率}:与 ADC 类似
  \item \textbf{建立时间(Settling Time)}:输出达到稳定值所需的时间
  \item \textbf{毛刺(Glitch)}:数字码切换时产生的瞬态误差
  \item \textbf{INL 和 DNL}:与 ADC 类似
\end{enumerate}

\subsubsection{DAC 的架构}

\begin{enumerate}
  \item \textbf{电阻网络型}
    \begin{itemize}
      \item R-2R 梯形网络
      \item 加权电阻网络
    \end{itemize}

  \item \textbf{电流舵型(Current Steering)}
    \begin{itemize}
      \item 高速应用
    \end{itemize}

  \item \textbf{Σ-Δ 型}
    \begin{itemize}
      \item 高精度应用
    \end{itemize}
\end{enumerate}

\section{专用集成电路}

专用集成电路(Application-Specific Integrated Circuit,ASIC)是为特定应用设计的 IC,包括各种功能芯片。

\subsection{传感器接口 IC}

传感器接口 IC 用于处理各种传感器的信号。

\subsubsection{温度传感器接口}

\begin{itemize}
  \item 热敏电阻接口
  \item 热电偶放大器
  \item 数字温度传感器(如 DS18B20)
\end{itemize}

\subsubsection{压力传感器接口}

\begin{itemize}
  \item 应变计放大器
  \item 压力传感器信号调理
\end{itemize}

\subsubsection{光传感器接口}

\begin{itemize}
  \item 光电二极管放大器
  \item 光强度传感器接口
\end{itemize}

\subsection{电机驱动 IC}

电机驱动 IC 用于控制各种电机。

\subsubsection{直流电机驱动}

\begin{itemize}
  \item H 桥驱动 IC
  \item 有刷直流电机驱动
  \item 无刷直流电机(BLDC)驱动
\end{itemize}

\subsubsection{步进电机驱动}

\begin{itemize}
  \item 步进电机驱动器
  \item 微步进控制
\end{itemize}

\subsection{音频功放 IC}

音频功放 IC 用于放大音频信号。

\subsubsection{分类}

\begin{enumerate}
  \item \textbf{A 类功放}
    \begin{itemize}
      \item 线性度好,但效率低
    \end{itemize}

  \item \textbf{B 类功放}
    \begin{itemize}
      \item 效率较高,但有交越失真
    \end{itemize}

  \item \textbf{AB 类功放}
    \begin{itemize}
      \item 兼顾效率和线性度
      \item 最常用
    \end{itemize}

  \item \textbf{D 类功放}
    \begin{itemize}
      \item 效率很高(> 90\%)
      \item 开关式功放
      \item 需要输出滤波器
    \end{itemize}
\end{enumerate}

\subsubsection{关键参数}

\begin{itemize}
  \item 输出功率
  \item 总谐波失真(THD)
  \item 效率
  \item 信噪比(SNR)
\end{itemize}

\subsection{接口 IC}

接口 IC 用于不同系统之间的连接。

\subsubsection{电平转换器}

\begin{itemize}
  \item 3.3 V 与 5 V 电平转换
  \item I2C 电平转换
  \item RS-232 电平转换
\end{itemize}

\subsubsection{总线驱动器和接收器}

\begin{itemize}
  \item RS-485 驱动/接收器
  \item CAN 总线驱动/接收器
  \item USB 接口 IC
\end{itemize}

\subsection{显示驱动 IC}

显示驱动 IC 用于驱动各种显示器。

\begin{itemize}
  \item LED 驱动 IC
  \item LCD 驱动 IC
  \item OLED 驱动 IC
  \item 段式显示驱动
\end{itemize}

\section{数字逻辑集成电路}

数字逻辑 IC 实现各种逻辑功能。

\subsection{基本逻辑门}

\begin{itemize}
  \item 与门(AND)
  \item 或门(OR)
  \item 非门(NOT)
  \item 与非门(NAND)
  \item 或非门(NOR)
  \item 异或门(XOR)
\end{itemize}

\subsection{组合逻辑电路}

\begin{itemize}
  \item 编码器/解码器
  \item 多路复用器/解复用器
  \item 比较器
  \item 算术逻辑单元(ALU)
\end{itemize}

\subsection{时序逻辑电路}

\begin{itemize}
  \item 触发器(Flip-Flop)
  \item 锁存器(Latch)
  \item 计数器
  \item 移位寄存器
\end{itemize}

\subsection{可编程逻辑器件}

\begin{itemize}
  \item 可编程逻辑阵列(PLA)
  \item 可编程阵列逻辑(PAL)
  \item 复杂可编程逻辑器件(CPLD)
  \item 现场可编程门阵列(FPGA)
\end{itemize}

\section{集成电路的选型和应用}

\subsection{选型原则}

选择 IC 时需要考虑:

\begin{enumerate}
  \item \textbf{功能需求}
    \begin{itemize}
      \item 是否满足功能要求
      \item 性能指标是否足够
    \end{itemize}

  \item \textbf{电气特性}
    \begin{itemize}
      \item 电源电压范围
      \item 工作电流
      \item 输入输出电平
      \item 速度要求
    \end{itemize}

  \item \textbf{环境条件}
    \begin{itemize}
      \item 工作温度范围
      \item 湿度要求
      \item 可靠性要求
    \end{itemize}

  \item \textbf{封装}
    \begin{itemize}
      \item 封装类型
      \item 引脚数
      \item 散热要求
    \end{itemize}

  \item \textbf{成本和可获得性}
    \begin{itemize}
      \item 价格
      \item 供货情况
      \item 生命周期
    \end{itemize}
\end{enumerate}

\subsection{应用注意事项}

\subsubsection{电源设计}

\begin{itemize}
  \item 提供稳定的电源电压
  \item 使用去耦电容
  \item 注意电源时序
  \item 考虑功耗和散热
\end{itemize}

\subsubsection{PCB 布局}

\begin{itemize}
  \item 缩短信号路径
  \item 减小寄生参数
  \item 注意地平面设计
  \item 分离模拟和数字部分
\end{itemize}

\subsubsection{热设计}

\begin{itemize}
  \item 计算功耗
  \item 选择合适的封装
  \item 必要时使用散热器
  \item 考虑环境温度
\end{itemize}

\subsubsection{可靠性}

\begin{itemize}
  \item 留有余量
  \item 考虑最坏情况
  \item 进行充分测试
  \item 遵循数据手册的推荐
\end{itemize}

\section{集成电路的发展趋势}

\subsection{工艺技术}

\begin{itemize}
  \item 特征尺寸不断缩小(已进入 3 nm 时代)
  \item 三维集成技术
  \item 新材料的应用
\end{itemize}

\subsection{设计技术}

\begin{itemize}
  \item 系统级设计(SoC)
  \item IP 核复用
  \item 低功耗设计
  \item 可测试性设计
\end{itemize}

\subsection{应用领域}

\begin{itemize}
  \item 人工智能芯片
  \item 物联网(IoT)芯片
  \item 汽车电子
  \item 医疗电子
  \item 可穿戴设备
\end{itemize}
