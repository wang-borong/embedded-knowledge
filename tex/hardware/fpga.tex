\chapter{FPGA 简介}

现场可编程门阵列(Field-Programmable Gate Array,FPGA)是一种可编程逻辑器件,用户可以通过编程来配置其内部逻辑功能。FPGA 具有灵活性高、开发周期短、可重复编程等优点,广泛应用于数字信号处理、通信、图像处理、嵌入式系统等领域。本章将系统介绍 FPGA 的基本原理、架构、设计方法和应用。

\section{FPGA 基础}

\subsection{FPGA 的发展历史}

FPGA 的发展历程:

\begin{enumerate}
  \item \textbf{1980 年代}:Xilinx 和 Altera(现 Intel)推出第一代 FPGA
    \begin{itemize}
      \item 简单的可编程逻辑块
      \item 有限的资源
    \end{itemize}

  \item \textbf{1990 年代}:第二代 FPGA
    \begin{itemize}
      \item 增加嵌入式存储器
      \item 改进的布线资源
    \end{itemize}

  \item \textbf{2000 年代}:第三代 FPGA
    \begin{itemize}
      \item 嵌入式处理器(硬核和软核)
      \item DSP 块
      \item 高速串行接口
    \end{itemize}

  \item \textbf{2010 年代至今}:现代 FPGA
    \begin{itemize}
      \item 大规模集成(数百万逻辑单元)
      \item 高级工艺(7 nm、5 nm)
      \item 异构计算平台
      \item AI 加速
    \end{itemize}
\end{enumerate}

\subsection{FPGA 的基本概念}

\subsubsection{什么是 FPGA}

FPGA 是一种半定制电路,特点:
\begin{itemize}
  \item 可编程:用户可以通过编程配置逻辑功能
  \item 可重复编程:可以多次修改和更新
  \item 并行处理:天然支持并行计算
  \item 灵活性高:可以实现各种数字逻辑功能
\end{itemize}

\subsubsection{FPGA 与 ASIC 的比较}

\begin{table}[H]
  \centering
  \caption{FPGA 与 ASIC 的比较}
  \begin{tabular}{|l|l|l|}
    \hline
    特性 & FPGA & ASIC \\
    \hline
    开发成本 & 低 & 高 \\
    开发周期 & 短(几周到几个月) & 长(几个月到几年) \\
    单位成本 & 高 & 低(大批量) \\
    性能 & 中等 & 高 \\
    功耗 & 较高 & 低 \\
    灵活性 & 高 & 低 \\
    可重复编程 & 是 & 否 \\
    适用场景 & 原型、小批量、快速上市 & 大批量、高性能 \\
    \hline
  \end{tabular}
\end{table}

\subsubsection{FPGA 与 CPLD 的比较}

\begin{table}[H]
  \centering
  \caption{FPGA 与 CPLD 的比较}
  \begin{tabular}{|l|l|l|}
    \hline
    特性 & FPGA & CPLD \\
    \hline
    架构 & 查找表(LUT) & 乘积项(Product Term) \\
    容量 & 大(数万到数百万逻辑单元) & 小(数百到数万逻辑单元) \\
    速度 & 中等 & 快 \\
    功耗 & 较高 & 较低 \\
    应用 & 复杂逻辑、DSP、处理器 & 简单逻辑、状态机、接口 \\
    \hline
  \end{tabular}
\end{table}

\subsection{FPGA 的应用领域}

\begin{enumerate}
  \item \textbf{数字信号处理}
    \begin{itemize}
      \item 滤波器
      \item FFT/IFFT
      \item 图像处理
    \end{itemize}

  \item \textbf{通信系统}
    \begin{itemize}
      \item 协议处理
      \item 数据包处理
      \item 无线通信基带
    \end{itemize}

  \item \textbf{嵌入式系统}
    \begin{itemize}
      \item 系统级芯片(SoC)
      \item 协处理器
      \item 接口桥接
    \end{itemize}

  \item \textbf{高性能计算}
    \begin{itemize}
      \item 加速器
      \item AI 推理
      \item 加密解密
    \end{itemize}

  \item \textbf{原型验证}
    \begin{itemize}
      \item ASIC 原型
      \item 算法验证
    \end{itemize}
\end{enumerate}

\section{FPGA 架构}

\subsection{基本架构组成}

FPGA 的基本架构包括:

\begin{enumerate}
  \item \textbf{可配置逻辑块(CLB/Configurable Logic Block)}
    \begin{itemize}
      \item 实现组合逻辑和时序逻辑
      \item 基本逻辑单元
    \end{itemize}

  \item \textbf{输入输出块(IOB/IO Block)}
    \begin{itemize}
      \item 与外部接口
      \item 支持多种 I/O 标准
    \end{itemize}

  \item \textbf{布线资源(Routing Resources)}
    \begin{itemize}
      \item 连接逻辑块
      \item 可编程互连
    \end{itemize}

  \item \textbf{时钟资源(Clock Resources)}
    \begin{itemize}
      \item 全局时钟网络
      \item 时钟管理单元
    \end{itemize}

  \item \textbf{嵌入式资源}
    \begin{itemize}
      \item 存储器(Block RAM、Distributed RAM)
      \item DSP 块
      \item 处理器(硬核、软核)
    \end{itemize}
\end{enumerate}

\subsection{可配置逻辑块(CLB)}

\subsubsection{查找表(LUT)}

查找表(Look-Up Table,LUT)是 FPGA 的基本逻辑单元,用于实现组合逻辑。

\begin{enumerate}
  \item \textbf{工作原理}
    \begin{itemize}
      \item N 输入 LUT 可以实现任意 N 输入组合逻辑函数
      \item 通过配置存储单元(SRAM)存储真值表
      \item 输入作为地址,输出为存储的值
    \end{itemize}

  \item \textbf{LUT 大小}
    \begin{itemize}
      \item 4 输入 LUT(4-LUT):最常用
      \item 6 输入 LUT(6-LUT):现代 FPGA
      \item 更大的 LUT:某些高端 FPGA
    \end{itemize}

  \item \textbf{LUT 级联}
    \begin{itemize}
      \item 多个 LUT 可以级联实现更复杂的逻辑
      \item 通过快速进位链连接
    \end{itemize}
\end{enumerate}

\subsubsection{触发器(Flip-Flop)}

每个 CLB 通常包含多个触发器:
\begin{itemize}
  \item 用于实现时序逻辑
  \item 可以配置为 D 触发器、T 触发器、JK 触发器等
  \item 支持同步/异步复位和置位
\end{itemize}

\subsubsection{CLB 结构}

典型的 CLB 结构:
\begin{itemize}
  \item 多个 LUT(如 2--8 个)
  \item 多个触发器(如 2--16 个)
  \item 多路选择器(MUX)
  \item 快速进位链
  \item 局部布线资源
\end{itemize}

\subsection{输入输出块(IOB)}

\subsubsection{IOB 功能}

IOB 提供:
\begin{enumerate}
  \item \textbf{输入缓冲}
    \begin{itemize}
      \item 可配置的输入标准
      \item 施密特触发器输入
    \end{itemize}

  \item \textbf{输出缓冲}
    \begin{itemize}
      \item 可配置的驱动能力
      \item 开漏输出
      \item 三态输出
    \end{itemize}

  \item \textbf{输入/输出寄存器}
    \begin{itemize}
      \item 提高时序性能
      \item 减少建立/保持时间要求
    \end{itemize}

  \item \textbf{延迟控制}
    \begin{itemize}
      \item 可编程延迟
      \item 用于时序调整
    \end{itemize}
\end{enumerate}

\subsubsection{支持的 I/O 标准}

\begin{itemize}
  \item LVTTL/LVCMOS:3.3 V、2.5 V、1.8 V、1.5 V、1.2 V
  \item LVDS:低电压差分信号
  \item SSTL:存储器接口
  \item HSTL:高速存储器接口
  \item PCI/PCIe:PCI 接口
  \item 其他专用标准
\end{itemize}

\subsection{布线资源}

\subsubsection{布线层次}

FPGA 的布线资源通常分为多个层次:

\begin{enumerate}
  \item \textbf{局部布线}
    \begin{itemize}
      \item CLB 内部连接
      \item 相邻 CLB 之间的连接
    \end{itemize}

  \item \textbf{通用布线}
    \begin{itemize}
      \item 中等距离连接
      \item 可编程开关矩阵
    \end{itemize}

  \item \textbf{长线}
    \begin{itemize}
      \item 长距离连接
      \item 全局信号
    \end{itemize}

  \item \textbf{专用布线}
    \begin{itemize}
      \item 时钟网络
      \item 复位网络
      \item 高速信号
    \end{itemize}
\end{enumerate}

\subsubsection{布线延迟}

布线延迟是 FPGA 性能的重要因素:
\begin{itemize}
  \item 延迟与布线长度相关
  \item 通过时序约束控制
  \item 工具自动优化
\end{itemize}

\subsection{时钟资源}

\subsubsection{时钟网络}

FPGA 提供专门的时钟网络:
\begin{enumerate}
  \item \textbf{全局时钟}
    \begin{itemize}
      \item 低延迟、低抖动
      \item 覆盖整个芯片
      \item 数量有限(通常 10--50 个)
    \end{itemize}

  \item \textbf{区域时钟}
    \begin{itemize}
      \item 覆盖特定区域
      \item 数量较多
    \end{itemize}

  \item \textbf{I/O 时钟}
    \begin{itemize}
      \item 用于 I/O 接口
      \item 支持高速接口
    \end{itemize}
\end{enumerate}

\subsubsection{时钟管理单元}

现代 FPGA 通常包含时钟管理单元:

\begin{enumerate}
  \item \textbf{锁相环(PLL)}
    \begin{itemize}
      \item 频率合成
      \item 时钟倍频/分频
      \item 相位调整
    \end{itemize}

  \item \textbf{延迟锁定环(DLL)}
    \begin{itemize}
      \item 时钟对齐
      \item 延迟补偿
    \end{itemize}

  \item \textbf{混合模式时钟管理器(MMCM)}
    \begin{itemize}
      \item 结合 PLL 和 DLL 功能
      \item 更灵活的时钟管理
    \end{itemize}
\end{enumerate}

\subsection{存储器资源}

\subsubsection{Block RAM}

Block RAM(BRAM)是 FPGA 中的专用存储器块:

\begin{enumerate}
  \item \textbf{特点}
    \begin{itemize}
      \item 容量大(通常 18--36 kbit/块)
      \item 双端口或真双端口
      \item 可配置为 RAM、ROM、FIFO
    \end{itemize}

  \item \textbf{配置模式}
    \begin{itemize}
      \item 单端口 RAM
      \item 简单双端口 RAM
      \item 真双端口 RAM
      \item ROM
      \item FIFO
    \end{itemize}

  \item \textbf{应用}
    \begin{itemize}
      \item 数据缓冲
      \item 查找表
      \item FIFO 队列
      \item 缓存
    \end{itemize}
\end{enumerate}

\subsubsection{分布式 RAM}

分布式 RAM 使用 LUT 实现小容量存储器:
\begin{itemize}
  \item 容量小(每个 LUT 16--64 bit)
  \item 延迟小
  \item 用于小容量存储
\end{itemize}

\subsubsection{UltraRAM}

某些高端 FPGA 提供 UltraRAM:
\begin{itemize}
  \item 超大容量(数 Mbit)
  \item 用于大数据缓冲
\end{itemize}

\subsection{DSP 块}

\subsubsection{DSP 块功能}

DSP 块(DSP Slice)用于实现数字信号处理功能:

\begin{enumerate}
  \item \textbf{乘法器}
    \begin{itemize}
      \item 高速乘法运算
      \item 支持有符号/无符号
    \end{itemize}

  \item \textbf{乘加器(MAC)}
    \begin{itemize}
      \item 乘法和累加
      \item 用于滤波器、FFT 等
    \end{itemize}

  \item \textbf{预加器}
    \begin{itemize}
      \item 输入数据的预处理
    \end{itemize}

  \item \textbf{流水线}
    \begin{itemize}
      \item 支持流水线操作
      \item 提高工作频率
    \end{itemize}
\end{enumerate}

\subsubsection{DSP 块应用}

\begin{itemize}
  \item 滤波器实现
  \item FFT/IFFT
  \item 相关器
  \item 矩阵运算
\end{itemize}

\subsection{处理器资源}

\subsubsection{硬核处理器}

某些 FPGA 包含硬核处理器:
\begin{itemize}
  \item ARM Cortex-A 系列(应用处理器)
  \item ARM Cortex-R 系列(实时处理器)
  \item PowerPC(某些老款 FPGA)
\end{itemize}

\subsubsection{软核处理器}

可以在 FPGA 中实现软核处理器:
\begin{itemize}
  \item MicroBlaze(Xilinx)
  \item Nios II(Intel/Altera)
  \item RISC-V 软核
  \item 其他开源处理器
\end{itemize}

\section{硬件描述语言(HDL)}

硬件描述语言用于描述数字电路的行为和结构,是 FPGA 设计的基础。

\subsection{HDL 概述}

\subsubsection{主要 HDL}

\begin{enumerate}
  \item \textbf{Verilog}
    \begin{itemize}
      \item 语法类似 C 语言
      \item 广泛使用
      \item IEEE 1364 标准
    \end{itemize}

  \item \textbf{VHDL}
    \begin{itemize}
      \item 语法类似 Ada 语言
      \item 欧洲常用
      \item IEEE 1076 标准
    \end{itemize}

  \item \textbf{SystemVerilog}
    \begin{itemize}
      \item Verilog 的扩展
      \item 增强的验证功能
      \item IEEE 1800 标准
    \end{itemize}
\end{enumerate}

\subsubsection{HDL 的层次}

\begin{enumerate}
  \item \textbf{行为级(Behavioral)}
    \begin{itemize}
      \item 描述功能行为
      \item 抽象层次高
    \end{itemize}

  \item \textbf{寄存器传输级(RTL)}
    \begin{itemize}
      \item 描述数据流和寄存器
      \item 可综合
    \end{itemize}

  \item \textbf{门级(Gate Level)}
    \begin{itemize}
      \item 描述逻辑门连接
      \item 抽象层次低
    \end{itemize}
\end{enumerate}

\subsection{Verilog HDL}

\subsubsection{基本语法}

\paragraph{模块定义}

\begin{verbatim}
module module_name (
    input  wire signal1,
    input  wire signal2,
    output reg  signal3
);
    // 模块内容
endmodule
\end{verbatim}

\paragraph{数据类型}

\begin{itemize}
  \item \textbf{wire}:连线,用于组合逻辑
  \item \textbf{reg}:寄存器,用于时序逻辑
  \item \textbf{integer}:整数
  \item \textbf{real}:实数
\end{itemize}

\paragraph{运算符}

\begin{itemize}
  \item 算术运算符:+、-、*、/、\%
  \item 逻辑运算符:\&\&、||、!
  \item 位运算符:\&、|、\textasciicircum、\textasciitilde
  \item 关系运算符:<、>、<=、>=
  \item 移位运算符:<<、>>
\end{itemize}

\subsubsection{组合逻辑}

\paragraph{连续赋值}

\begin{verbatim}
assign out = in1 & in2;
\end{verbatim}

\paragraph{always 块(组合逻辑)}

\begin{verbatim}
always @(*) begin
    out = in1 | in2;
end
\end{verbatim}

\subsubsection{时序逻辑}

\paragraph{always 块(时序逻辑)}

\begin{verbatim}
always @(posedge clk) begin
    if (reset)
        q <= 1'b0;
    else
        q <= d;
end
\end{verbatim}

\paragraph{阻塞赋值与非阻塞赋值}

\begin{itemize}
  \item \textbf{阻塞赋值(=)}:用于组合逻辑,立即执行
  \item \textbf{非阻塞赋值(<=)}:用于时序逻辑,在时钟边沿更新
\end{itemize}

\subsubsection{常用设计模式}

\paragraph{计数器}

\begin{verbatim}
reg [7:0] counter;
always @(posedge clk) begin
    if (reset)
        counter <= 8'b0;
    else if (enable)
        counter <= counter + 1;
end
\end{verbatim}

\paragraph{状态机}

\begin{verbatim}
parameter IDLE = 2'b00;
parameter WORK = 2'b01;
parameter DONE = 2'b10;

reg [1:0] state;
always @(posedge clk) begin
    if (reset)
        state <= IDLE;
    else
        case (state)
            IDLE: if (start) state <= WORK;
            WORK: if (done)  state <= DONE;
            DONE:            state <= IDLE;
        endcase
end
\end{verbatim}

\subsection{VHDL}

\subsubsection{基本语法}

\paragraph{实体和架构}

\begin{verbatim}
entity entity_name is
    port (
        signal1 : in  std_logic;
        signal2 : in  std_logic;
        signal3 : out std_logic
    );
end entity;

architecture rtl of entity_name is
    -- 信号声明
begin
    -- 逻辑描述
end architecture;
\end{verbatim}

\paragraph{数据类型}

\begin{itemize}
  \item \textbf{std\_logic}:标准逻辑类型('0'、'1'、'Z'、'X' 等)
  \item \textbf{std\_logic\_vector}:逻辑向量
  \item \textbf{integer}:整数
  \item \textbf{unsigned/signed}:无符号/有符号数
\end{itemize}

\subsubsection{组合逻辑}

\paragraph{并发信号赋值}

\begin{verbatim}
signal3 <= signal1 and signal2;
\end{verbatim}

\paragraph{进程(Process)}

\begin{verbatim}
process(signal1, signal2)
begin
    signal3 <= signal1 or signal2;
end process;
\end{verbatim}

\subsubsection{时序逻辑}

\begin{verbatim}
process(clk)
begin
    if rising_edge(clk) then
        if reset = '1' then
            q <= '0';
        else
            q <= d;
        end if;
    end if;
end process;
\end{verbatim}

\subsection{SystemVerilog}

SystemVerilog 是 Verilog 的扩展,增加了:

\begin{enumerate}
  \item \textbf{接口(Interface)}
    \begin{itemize}
      \item 封装信号组
      \item 简化连接
    \end{itemize}

  \item \textbf{类(Class)}
    \begin{itemize}
      \item 面向对象编程
      \item 主要用于验证
    \end{itemize}

  \item \textbf{断言(Assertion)}
    \begin{itemize}
      \item 属性检查
      \item 形式化验证
    \end{itemize}

  \item \textbf{增强的数据类型}
    \begin{itemize}
      \item 枚举类型
      \item 结构体
      \item 联合体
    \end{itemize}
\end{enumerate}

\subsection{HDL 设计原则}

\begin{enumerate}
  \item \textbf{可综合代码}
    \begin{itemize}
      \item 只使用可综合的语法
      \item 避免不可综合的构造
    \end{itemize}

  \item \textbf{同步设计}
    \begin{itemize}
      \item 使用同步复位
      \item 避免异步逻辑(除非必要)
    \end{itemize}

  \item \textbf{代码风格}
    \begin{itemize}
      \item 清晰的命名
      \item 适当的注释
      \item 模块化设计
    \end{itemize}

  \item \textbf{可移植性}
    \begin{itemize}
      \item 使用标准语法
      \item 避免厂商特定特性
    \end{itemize}
\end{enumerate}

\section{FPGA 设计流程}

\subsection{设计流程概述}

FPGA 设计流程包括:

\begin{enumerate}
  \item \textbf{需求分析}
  \item \textbf{架构设计}
  \item \textbf{RTL 设计}
  \item \textbf{功能仿真}
  \item \textbf{综合(Synthesis)}
  \item \textbf{实现(Implementation)}
  \item \textbf{时序分析}
  \item \textbf{下载和测试}
\end{enumerate}

\subsection{设计输入}

\subsubsection{设计输入方式}

\begin{enumerate}
  \item \textbf{HDL 代码}
    \begin{itemize}
      \item Verilog/VHDL/SystemVerilog
      \item 最常用的方式
    \end{itemize}

  \item \textbf{原理图}
    \begin{itemize}
      \item 图形化设计
      \item 较少使用
    \end{itemize}

  \item \textbf{IP 核}
    \begin{itemize}
      \item 预设计的模块
      \item 提高设计效率
    \end{itemize}

  \item \textbf{高层次综合(HLS)}
    \begin{itemize}
      \item 从 C/C++ 生成 RTL
      \item 提高抽象层次
    \end{itemize}
\end{enumerate}

\subsection{功能仿真}

\subsubsection{仿真工具}

\begin{itemize}
  \item ModelSim/QuestaSim
  \item VCS
  \item Verilator(开源)
  \item Icarus Verilog(开源)
\end{itemize}

\subsubsection{测试平台}

测试平台(Testbench)用于验证设计:
\begin{itemize}
  \item 生成测试激励
  \item 监控输出
  \item 自动验证
\end{itemize}

\subsection{综合}

\subsubsection{综合过程}

综合将 RTL 代码转换为门级网表:
\begin{enumerate}
  \item \textbf{语法分析}
  \item \textbf{优化}
  \item \textbf{映射到目标器件}
  \item \textbf{生成网表}
\end{enumerate}

\subsubsection{综合工具}

\begin{itemize}
  \item Xilinx Vivado/ISE
  \item Intel Quartus
  \item Synopsys Synplify
  \item Mentor Precision
\end{itemize}

\subsection{实现}

\subsubsection{实现步骤}

实现包括:

\begin{enumerate}
  \item \textbf{转换(Translate)}
    \begin{itemize}
      \item 将网表转换为 FPGA 格式
    \end{itemize}

  \item \textbf{映射(Map)}
    \begin{itemize}
      \item 将逻辑映射到 CLB
      \item 分配资源
    \end{itemize}

  \item \textbf{布局(Place)}
    \begin{itemize}
      \item 确定逻辑块的位置
      \item 考虑时序和布线
    \end{itemize}

  \item \textbf{布线(Route)}
    \begin{itemize}
      \item 连接逻辑块
      \item 满足时序约束
    \end{itemize}

  \item \textbf{比特流生成(Bitstream Generation)}
    \begin{itemize}
      \item 生成配置文件
      \item 用于下载到 FPGA
    \end{itemize}
\end{enumerate}

\subsection{时序分析}

\subsubsection{时序路径}

时序路径包括:
\begin{enumerate}
  \item \textbf{建立时间路径}
    \begin{itemize}
      \item 从源触发器到目标触发器
      \item 必须满足建立时间要求
    \end{itemize}

  \item \textbf{保持时间路径}
    \begin{itemize}
      \item 必须满足保持时间要求
    \end{itemize}

  \item \textbf{时钟路径}
    \begin{itemize}
      \item 时钟分配延迟
    \end{itemize}
\end{enumerate}

\subsubsection{时序约束}

时序约束用于指导工具优化:
\begin{itemize}
  \item 时钟约束
  \item 输入/输出延迟约束
  \item 多周期路径
  \item 虚假路径
\end{itemize}

\section{时序约束和时序分析}

\subsection{时序约束基础}

\subsubsection{时钟约束}

时钟约束定义时钟特性:

\begin{verbatim}
# Xilinx Vivado
create_clock -period 10.0 -name clk [get_ports clk]

# Intel Quartus
create_clock -name clk -period 10.0 [get_ports {clk}]
\end{verbatim}

\subsubsection{输入延迟约束}

定义输入信号的延迟:

\begin{verbatim}
set_input_delay -clock clk -max 2.0 [get_ports data_in]
set_input_delay -clock clk -min 1.0 [get_ports data_in]
\end{verbatim}

\subsubsection{输出延迟约束}

定义输出信号的延迟:

\begin{verbatim}
set_output_delay -clock clk -max 2.0 [get_ports data_out]
set_output_delay -clock clk -min 1.0 [get_ports data_out]
\end{verbatim}

\subsection{时序分析}

\subsubsection{建立时间检查}

建立时间必须满足:

\begin{equation}
T_{clk} \geq T_{co} + T_{logic} + T_{routing} + T_{setup} - T_{skew} \label{eq:setup-time}
\end{equation}

其中:
\begin{itemize}
  \item $T_{clk}$:时钟周期
  \item $T_{co}$:时钟到输出延迟
  \item $T_{logic}$:组合逻辑延迟
  \item $T_{routing}$:布线延迟
  \item $T_{setup}$:建立时间
  \item $T_{skew}$:时钟偏移
\end{itemize}

\subsubsection{保持时间检查}

保持时间必须满足:

\begin{equation}
T_{hold} \leq T_{co} + T_{logic} + T_{routing} - T_{skew} \label{eq:hold-time}
\end{equation}

\subsubsection{时序裕量}

时序裕量(Slack)是实际延迟与要求延迟的差值:
\begin{itemize}
  \item 正裕量:满足时序要求
  \item 负裕量:时序违规,需要优化
\end{itemize}

\subsection{时序优化}

\subsubsection{优化方法}

\begin{enumerate}
  \item \textbf{流水线}
    \begin{itemize}
      \item 插入寄存器
      \item 减少组合逻辑延迟
    \end{itemize}

  \item \textbf{逻辑优化}
    \begin{itemize}
      \item 简化逻辑
      \item 使用专用资源(如 DSP)
    \end{itemize}

  \item \textbf{约束优化}
    \begin{itemize}
      \item 合理的时序约束
      \item 多周期路径
    \end{itemize}

  \item \textbf{布局布线优化}
    \begin{itemize}
      \item 区域约束
      \item 时序驱动布局
    \end{itemize}
\end{enumerate}

\section{FPGA 设计技巧}

\subsection{同步设计}

\subsubsection{同步复位 vs 异步复位}

\begin{enumerate}
  \item \textbf{同步复位}
    \begin{itemize}
      \item 复位信号在时钟边沿有效
      \item 推荐使用
      \item 易于时序分析
    \end{itemize}

  \item \textbf{异步复位}
    \begin{itemize}
      \item 复位信号立即有效
      \item 需要复位同步器
      \item 用于上电复位
    \end{itemize}
\end{enumerate}

\subsubsection{复位同步器}

异步复位需要同步释放:

\begin{verbatim}
reg rst_sync1, rst_sync2;
always @(posedge clk or posedge rst_async) begin
    if (rst_async) begin
        rst_sync1 <= 1'b1;
        rst_sync2 <= 1'b1;
    end else begin
        rst_sync1 <= 1'b0;
        rst_sync2 <= rst_sync1;
    end
end
\end{verbatim}

\subsection{时钟域交叉(CDC)}

\subsubsection{CDC 问题}

不同时钟域之间的信号传输需要特殊处理:
\begin{itemize}
  \item 亚稳态(Metastability)
  \item 数据丢失
  \item 数据错误
\end{itemize}

\subsubsection{CDC 技术}

\begin{enumerate}
  \item \textbf{双触发器同步器}
    \begin{itemize}
      \item 用于单比特信号
      \item 两级触发器
    \end{itemize}

  \item \textbf{握手协议}
    \begin{itemize}
      \item 用于多比特信号
      \item 请求/应答机制
    \end{itemize}

  \item \textbf{FIFO}
    \begin{itemize}
      \item 用于数据流
      \item 异步 FIFO
    \end{itemize}

  \item \textbf{格雷码}
    \begin{itemize}
      \item 用于计数器
      \item 单比特变化
    \end{itemize}
\end{enumerate}

\subsection{资源优化}

\subsubsection{逻辑优化}

\begin{enumerate}
  \item \textbf{资源共享}
    \begin{itemize}
      \item 复用逻辑
      \item 减少资源使用
    \end{itemize}

  \item \textbf{流水线}
    \begin{itemize}
      \item 平衡逻辑深度
      \item 提高工作频率
    \end{itemize}

  \item \textbf{使用专用资源}
    \begin{itemize}
      \item DSP 块用于乘法
      \item BRAM 用于存储器
    \end{itemize}
\end{enumerate}

\subsubsection{功耗优化}

\begin{enumerate}
  \item \textbf{时钟门控}
    \begin{itemize}
      \item 关闭不用的时钟域
      \item 降低动态功耗
    \end{itemize}

  \item \textbf{降低工作频率}
    \begin{itemize}
      \item 满足性能要求即可
    \end{itemize}

  \item \textbf{降低电压}
    \begin{itemize}
      \item 使用低电压 I/O
    \end{itemize}
\end{enumerate}

\section{高级主题}

\subsection{高速串行接口}

\subsubsection{SerDes}

串行器/解串器(Serializer/Deserializer,SerDes)用于高速串行通信:

\begin{itemize}
  \item PCIe
  \item SATA
  \item Ethernet(10 Gbps+)
  \item USB 3.0+
  \item 自定义高速接口
\end{itemize}

\subsubsection{设计要点}

\begin{enumerate}
  \item \textbf{时钟恢复}
    \begin{itemize}
      \item CDR(Clock Data Recovery)
    \end{itemize}

  \item \textbf{均衡}
    \begin{itemize}
      \item 预加重
      \item 去加重
      \item 自适应均衡
    \end{itemize}

  \item \textbf{眼图}
    \begin{itemize}
      \item 信号质量分析
    \end{itemize}
\end{enumerate}

\subsection{系统级芯片(SoC)}

\subsubsection{SoC 架构}

FPGA SoC 结合了处理器和可编程逻辑:
\begin{itemize}
  \item 硬核处理器(如 ARM)
  \item 可编程逻辑
  \item 高速互连(AXI 总线)
  \item 共享资源
\end{itemize}

\subsubsection{设计流程}

\begin{enumerate}
  \item 处理器系统设计
  \item 可编程逻辑设计
  \item 系统集成
  \item 软件开发
\end{enumerate}

\subsection{高层次综合(HLS)}

\subsubsection{HLS 概念}

HLS 从高级语言(C/C++)生成 RTL:
\begin{itemize}
  \item 提高抽象层次
  \item 加快开发速度
  \item 自动优化
\end{itemize}

\subsubsection{HLS 工具}

\begin{itemize}
  \item Xilinx Vivado HLS(现 Vitis HLS)
  \item Intel HLS Compiler
  \item LegUp(开源)
\end{itemize}

\subsection{部分重配置}

\subsubsection{部分重配置概念}

部分重配置允许在运行时重新配置 FPGA 的一部分:
\begin{itemize}
  \item 其他部分继续工作
  \item 动态功能切换
  \item 节省资源
\end{itemize}

\subsubsection{应用}

\begin{itemize}
  \item 多模式系统
  \item 资源复用
  \item 动态优化
\end{itemize}

\section{FPGA 应用实例}

\subsection{数字信号处理}

\subsubsection{滤波器}

\begin{itemize}
  \item FIR 滤波器
  \item IIR 滤波器
  \item 自适应滤波器
\end{itemize}

\subsubsection{FFT}

\begin{itemize}
  \item 快速傅里叶变换
  \item 用于频谱分析
  \item 使用 DSP 块加速
\end{itemize}

\subsection{图像处理}

\begin{itemize}
  \item 图像滤波
  \item 边缘检测
  \item 图像压缩
  \item 视频编解码
\end{itemize}

\subsection{通信系统}

\begin{itemize}
  \item 协议处理
  \item 数据包处理
  \item 无线通信基带
  \item 网络加速
\end{itemize}

\subsection{AI 加速}

\begin{itemize}
  \item 神经网络推理
  \item 卷积加速
  \item 矩阵运算
  \item 低延迟推理
\end{itemize}

\section{FPGA 厂商和工具}

\subsection{主要厂商}

\begin{enumerate}
  \item \textbf{Xilinx(现 AMD)}
    \begin{itemize}
      \item 市场份额最大
      \item 产品线:Artix、Kintex、Virtex、Zynq
      \item 工具:Vivado、Vitis
    \end{itemize}

  \item \textbf{Intel(原 Altera)}
    \begin{itemize}
      \item 第二大厂商
      \item 产品线:Cyclone、Arria、Stratix、Agilex
      \item 工具:Quartus Prime
    \end{itemize}

  \item \textbf{Lattice}
    \begin{itemize}
      \item 专注于低功耗、小尺寸
      \item 产品线:iCE、ECP、MachXO
      \item 工具:Lattice Diamond、Radiant
    \end{itemize}

  \item \textbf{Microsemi(现 Microchip)}
    \begin{itemize}
      \item 专注于抗辐射、高可靠性
      \item 产品线:SmartFusion、IGLOO、ProASIC
      \item 工具:Libero SoC
    \end{itemize}
\end{enumerate}

\subsection{开发工具}

\subsubsection{综合和实现工具}

\begin{itemize}
  \item Xilinx Vivado
  \item Intel Quartus Prime
  \item Lattice Radiant
  \item 第三方工具(Synplify、Precision)
\end{itemize}

\subsubsection{仿真工具}

\begin{itemize}
  \item ModelSim/QuestaSim
  \item VCS
  \item Verilator(开源)
\end{itemize}

\subsubsection{调试工具}

\begin{itemize}
  \item 逻辑分析仪(ILA、SignalTap)
  \item 示波器
  \item 协议分析仪
\end{itemize}

\section{FPGA 选型}

\subsection{选型考虑因素}

\begin{enumerate}
  \item \textbf{逻辑资源}
    \begin{itemize}
      \item LUT 数量
      \item 触发器数量
      \item 是否满足设计需求
    \end{itemize}

  \item \textbf{存储器}
    \begin{itemize}
      \item Block RAM 容量
      \item 是否需要 UltraRAM
    \end{itemize}

  \item \textbf{DSP 资源}
    \begin{itemize}
      \item DSP 块数量
      \item 是否满足 DSP 需求
    \end{itemize}

  \item \textbf{I/O}
    \begin{itemize}
      \item I/O 数量
      \item 支持的 I/O 标准
      \item 高速接口(SerDes)
    \end{itemize}

  \item \textbf{时钟资源}
    \begin{itemize}
      \item 时钟管理单元
      \item 全局时钟数量
    \end{itemize}

  \item \textbf{功耗}
    \begin{itemize}
      \item 静态功耗
      \item 动态功耗
      \item 散热要求
    \end{itemize}

  \item \textbf{成本}
    \begin{itemize}
      \item 器件成本
      \item 开发工具成本
    \end{itemize}

  \item \textbf{封装}
    \begin{itemize}
      \item 封装类型
      \item 引脚数
    \end{itemize}
\end{enumerate}

\section{总结}

FPGA 是一种强大的可编程逻辑器件,具有灵活性高、开发周期短、可重复编程等优点。掌握 FPGA 设计需要理解其架构、熟悉 HDL、掌握设计流程和时序分析。随着 FPGA 技术的不断发展,其在 AI 加速、高性能计算、通信等领域的应用越来越广泛。通过系统学习和实践,可以充分发挥 FPGA 的优势,实现高性能的数字系统设计。
