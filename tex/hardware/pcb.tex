\chapter{电路设计}

电路设计是电子系统开发的核心环节,包括从概念设计到最终产品的完整流程。本章将系统介绍原理图设计、PCB 设计、信号完整性、电源完整性、电磁兼容性等关键内容,帮助读者掌握专业的电路设计方法和技巧。

\section{原理图设计}

原理图(Schematic)是电路设计的起点,用图形符号表示电路的功能和连接关系。良好的原理图设计是成功 PCB 设计的基础。

\subsection{原理图设计基础}

\subsubsection{原理图的作用}

原理图的主要作用包括:
\begin{itemize}
  \item 表达电路功能和连接关系
  \item 作为 PCB 布局的依据
  \item 用于电路分析和仿真
  \item 作为技术文档和沟通工具
  \item 用于生成物料清单(BOM)
\end{itemize}

\subsubsection{原理图设计原则}

\begin{enumerate}
  \item \textbf{清晰性}
    \begin{itemize}
      \item 信号流向明确(通常从左到右,从上到下)
      \item 相关元件靠近放置
      \item 避免交叉线过多
      \item 使用总线(Bus)简化连接
    \end{itemize}

  \item \textbf{层次化}
    \begin{itemize}
      \item 复杂电路采用层次化设计
      \item 功能模块化
      \item 便于理解和维护
    \end{itemize}

  \item \textbf{规范性}
    \begin{itemize}
      \item 使用标准符号
      \item 统一的命名规则
      \item 完整的标注和说明
    \end{itemize}

  \item \textbf{完整性}
    \begin{itemize}
      \item 所有连接都有明确的网络名称
      \item 电源和地网络清晰标注
      \item 关键信号添加说明
    \end{itemize}
\end{enumerate}

\subsection{元件符号库}

\subsubsection{符号库的建立}

\begin{itemize}
  \item 使用标准符号(如 IEEE、IEC 标准)
  \item 保持符号的一致性
  \item 包含必要的引脚信息
  \item 添加封装信息
\end{itemize}

\subsubsection{常用元件符号}

\begin{enumerate}
  \item \textbf{无源元件}
    \begin{itemize}
      \item 电阻、电容、电感
      \item 变压器
    \end{itemize}

  \item \textbf{半导体元件}
    \begin{itemize}
      \item 二极管、晶体管
      \item 运算放大器、比较器
      \item 逻辑门
    \end{itemize}

  \item \textbf{连接器}
    \begin{itemize}
      \item 接插件
      \item 测试点
    \end{itemize}
\end{enumerate}

\subsection{网络和连接}

\subsubsection{网络命名}

\begin{itemize}
  \item 使用有意义的网络名称
  \item 电源网络:$V_{CC}$、$V_{DD}$、$V_{SS}$、GND 等
  \item 信号网络:功能描述性名称
  \item 差分信号:使用 +、- 后缀
\end{itemize}

\subsubsection{总线设计}

对于多位总线,使用总线符号可以简化原理图:

\begin{itemize}
  \item 数据总线:D[0:7]、AD[0:15] 等
  \item 地址总线:A[0:19] 等
  \item 控制总线:使用总线连接多个信号
\end{itemize}

\subsection{层次化设计}

\subsubsection{层次化结构}

复杂电路应采用层次化设计:

\begin{enumerate}
  \item \textbf{顶层}:系统级连接
  \item \textbf{子图}:功能模块
  \item \textbf{底层}:具体电路实现
\end{enumerate}

\subsubsection{层次化设计的优点}

\begin{itemize}
  \item 提高可读性
  \item 便于团队协作
  \item 支持模块复用
  \item 简化修改和维护
\end{itemize}

\subsection{设计规则检查(DRC)}

原理图设计完成后,应进行设计规则检查:

\begin{enumerate}
  \item \textbf{电气规则检查(ERC)}
    \begin{itemize}
      \item 未连接的引脚
      \item 电源冲突
      \item 输入悬空
    \end{itemize}

  \item \textbf{连接性检查}
    \begin{itemize}
      \item 所有网络是否连接
      \item 网络名称是否一致
    \end{itemize}

  \item \textbf{完整性检查}
    \begin{itemize}
      \item 元件参数是否完整
      \item 封装是否指定
    \end{itemize}
\end{enumerate}

\section{PCB 基础}

\subsection{PCB 的发展历史}

印刷电路板(Printed Circuit Board,PCB)的发展历程:

\begin{enumerate}
  \item \textbf{1940 年代}:单面板,手工制作
  \item \textbf{1950 年代}:双面板,蚀刻工艺
  \item \textbf{1960 年代}:多层板技术
  \item \textbf{1970--1980 年代}:表面贴装技术(SMT)
  \item \textbf{1990 年代至今}:高密度互连(HDI)、埋盲孔、微孔技术
\end{enumerate}

\subsection{PCB 的分类}

\subsubsection{按层数分类}

\begin{enumerate}
  \item \textbf{单面板(Single Layer)}
    \begin{itemize}
      \item 只有一面有铜箔
      \item 成本最低
      \item 适用于简单电路
    \end{itemize}

  \item \textbf{双面板(Double Layer)}
    \begin{itemize}
      \item 两面都有铜箔
      \item 通过过孔连接
      \item 最常用的类型
    \end{itemize}

  \item \textbf{多层板(Multi-Layer)}
    \begin{itemize}
      \item 4 层、6 层、8 层或更多
      \item 有内层电源和地平面
      \item 适用于复杂电路和高速设计
    \end{itemize}
\end{enumerate}

\subsubsection{按基材分类}

\begin{enumerate}
  \item \textbf{FR-4}
    \begin{itemize}
      \item 最常用的基材
      \item 玻璃纤维增强环氧树脂
      \item 介电常数 $\epsilon_r \approx 4.4$
      \item 适用于大多数应用
    \end{itemize}

  \item \textbf{高频基材}
    \begin{itemize}
      \item Rogers、Taconic 等
      \item 低介电常数和损耗
      \item 用于射频和高速应用
    \end{itemize}

  \item \textbf{柔性 PCB(FPC)}
    \begin{itemize}
      \item 聚酰亚胺基材
      \item 可弯曲
      \item 用于特殊应用
    \end{itemize}

  \item \textbf{刚柔结合板}
    \begin{itemize}
      \item 结合刚性板和柔性板
      \item 用于复杂的三维连接
    \end{itemize}
\end{enumerate}

\subsection{PCB 制造工艺}

\subsubsection{基本工艺流程}

\begin{enumerate}
  \item \textbf{基材准备}
    \begin{itemize}
      \item 选择基材和铜箔厚度
      \item 清洗和预处理
    \end{itemize}

  \item \textbf{图形转移}
    \begin{itemize}
      \item 光刻胶涂覆
      \item 曝光(使用底片或直接成像)
      \item 显影
    \end{itemize}

  \item \textbf{蚀刻}
    \begin{itemize}
      \item 去除不需要的铜
      \item 化学蚀刻或等离子蚀刻
    \end{itemize}

  \item \textbf{钻孔}
    \begin{itemize}
      \item 机械钻孔或激光钻孔
      \item 过孔、通孔、盲孔、埋孔
    \end{itemize}

  \item \textbf{孔金属化}
    \begin{itemize}
      \item 化学镀铜
      \item 电镀加厚
    \end{itemize}

  \item \textbf{阻焊层}
    \begin{itemize}
      \item 涂覆阻焊油墨
      \item 保护走线,防止短路
    \end{itemize}

  \item \textbf{丝印}
    \begin{itemize}
      \item 印刷元件标识和说明
    \end{itemize}

  \item \textbf{表面处理}
    \begin{itemize}
      \item HASL(热风整平)
      \item OSP(有机可焊性保护膜)
      \item ENIG(化学镀镍浸金)
      \item 沉银、沉锡等
    \end{itemize}

  \item \textbf{测试}
    \begin{itemize}
      \item 电气测试
      \item 外观检查
    \end{itemize}
\end{enumerate}

\subsection{PCB 材料参数}

\subsubsection{介电常数($\epsilon_r$)}

介电常数影响信号传播速度和特性阻抗:

\begin{equation}
v = \frac{c}{\sqrt{\epsilon_r}} \label{eq:signal-velocity}
\end{equation}

其中 $c$ 是光速。

\subsubsection{损耗角正切($\tan \delta$)}

损耗角正切表示介质的损耗特性,影响信号衰减:

\begin{equation}
\alpha = \frac{\omega}{2c} \sqrt{\epsilon_r} \tan \delta \label{eq:attenuation}
\end{equation}

\subsubsection{铜箔厚度}

常用铜箔厚度:
\begin{itemize}
  \item 0.5 oz(17.5 $\mu$m):内层常用
  \item 1 oz(35 $\mu$m):标准厚度
  \item 2 oz(70 $\mu$m):大电流应用
\end{itemize}

\section{PCB 布局设计}

\subsection{布局的基本原则}

\subsubsection{功能分区}

\begin{enumerate}
  \item \textbf{模拟和数字分离}
    \begin{itemize}
      \item 物理隔离
      \item 独立的电源和地
      \item 避免数字噪声影响模拟电路
    \end{itemize}

  \item \textbf{高速和低速分离}
    \begin{itemize}
      \item 高速信号远离低速信号
      \item 减少串扰
    \end{itemize}

  \item \textbf{电源分区}
    \begin{itemize}
      \item 不同电压等级分开
      \item 大功率和小功率分开
    \end{itemize}
\end{enumerate}

\subsubsection{元件放置原则}

\begin{enumerate}
  \item \textbf{关键元件优先}
    \begin{itemize}
      \item 先放置核心 IC(如 MCU、FPGA)
      \item 再放置外围元件
    \end{itemize}

  \item \textbf{信号流向}
    \begin{itemize}
      \item 按照信号流向放置元件
      \item 缩短关键信号路径
    \end{itemize}

  \item \textbf{热考虑}
    \begin{itemize}
      \item 发热元件分散放置
      \item 远离温度敏感元件
      \item 考虑散热路径
    \end{itemize}

  \item \textbf{机械考虑}
    \begin{itemize}
      \item 接插件放在边缘
      \item 考虑安装和维修
      \item 符合机械约束
    \end{itemize}
\end{enumerate}

\subsection{层叠设计}

\subsubsection{层叠设计原则}

\begin{enumerate}
  \item \textbf{对称性}
    \begin{itemize}
      \item 层叠结构对称
      \item 减少翘曲
    \end{itemize}

  \item \textbf{信号层和地平面相邻}
    \begin{itemize}
      \item 提供返回路径
      \item 控制特性阻抗
    \end{itemize}

  \item \textbf{电源平面分割}
    \begin{itemize}
      \item 不同电源分开
      \item 保持完整性
    \end{itemize}
\end{enumerate}

\subsubsection{常见层叠结构}

\paragraph{4 层板}

典型结构:
\begin{itemize}
  \item 顶层(信号)
  \item 内层 1(地平面)
  \item 内层 2(电源平面)
  \item 底层(信号)
\end{itemize}

\paragraph{6 层板}

典型结构:
\begin{itemize}
  \item 顶层(信号)
  \item 内层 1(地平面)
  \item 内层 2(信号)
  \item 内层 3(信号)
  \item 内层 4(电源平面)
  \item 底层(信号)
\end{itemize}

\paragraph{8 层板}

典型结构:
\begin{itemize}
  \item 顶层(信号)
  \item 内层 1(地平面)
  \item 内层 2(信号)
  \item 内层 3(地平面)
  \item 内层 4(电源平面)
  \item 内层 5(信号)
  \item 内层 6(地平面)
  \item 底层(信号)
\end{itemize}

\subsection{走线设计}

\subsubsection{走线宽度}

走线宽度影响:
\begin{itemize}
  \item 载流能力
  \item 特性阻抗
  \item 可制造性
\end{itemize}

载流能力估算(基于 IPC-2221):

\begin{equation}
I = k \Delta T^{0.44} A^{0.725} \label{eq:trace-current}
\end{equation}

其中:
\begin{itemize}
  \item $I$:电流(A)
  \item $k$:常数(外层 0.048,内层 0.024)
  \item $\Delta T$:温升($^\circ$C)
  \item $A$:横截面积(mil$^2$)
\end{itemize}

\subsubsection{走线间距}

走线间距影响:
\begin{itemize}
  \item 串扰
  \item 可制造性
  \item 成本
\end{itemize}

最小间距通常为:
\begin{itemize}
  \item 信号线:3--4 mil(0.075--0.1 mm)
  \item 电源线:根据电压等级确定
  \item 差分对:根据阻抗要求确定
\end{itemize}

\subsubsection{过孔设计}

\paragraph{过孔类型}

\begin{enumerate}
  \item \textbf{通孔(Through Via)}
    \begin{itemize}
      \item 贯穿所有层
      \item 成本低,但占用空间大
    \end{itemize}

  \item \textbf{盲孔(Blind Via)}
    \begin{itemize}
      \item 从表面到内层
      \item 节省空间
    \end{itemize}

  \item \textbf{埋孔(Buried Via)}
    \begin{itemize}
      \item 完全在内层
      \item 最节省空间
    \end{itemize}

  \item \textbf{微孔(Microvia)}
    \begin{itemize}
      \item 直径 < 150 $\mu$m
      \item 用于 HDI 设计
    \end{itemize}
\end{enumerate}

\paragraph{过孔参数}

\begin{itemize}
  \item 过孔直径:影响成本和可制造性
  \item 焊盘直径:通常为过孔直径 + 8--12 mil
  \item 反焊盘(Antipad):过孔与平面之间的间隙
\end{itemize}

\subsubsection{走线规则}

\begin{enumerate}
  \item \textbf{避免锐角}
    \begin{itemize}
      \item 使用 45$^\circ$ 或圆弧
      \item 减少反射和制造问题
    \end{itemize}

  \item \textbf{避免长距离平行走线}
    \begin{itemize}
      \item 减少串扰
      \item 必要时增加间距或使用地线隔离
    \end{itemize}

  \item \textbf{关键信号优先}
    \begin{itemize}
      \item 时钟、复位等关键信号
      \item 最短路径
      \item 远离干扰源
    \end{itemize}

  \item \textbf{差分对}
    \begin{itemize}
      \item 等长走线
      \item 等间距
      \item 对称布局
    \end{itemize}
\end{enumerate}

\section{信号完整性}

信号完整性(Signal Integrity,SI)是高速 PCB 设计的核心,涉及信号在传输过程中的完整性和质量。

\subsection{传输线理论}

\subsubsection{传输线的基本概念}

当信号上升时间 $t_r$ 满足以下条件时,需要考虑传输线效应:

\begin{equation}
t_r < 2 \cdot t_{pd} \cdot l \label{eq:transmission-line-condition}
\end{equation}

其中 $t_{pd}$ 是单位长度的传播延迟,$l$ 是走线长度。

\subsubsection{特性阻抗}

特性阻抗 $Z_0$ 是传输线的重要参数,对于微带线:

\begin{equation}
Z_0 = \frac{87}{\sqrt{\epsilon_r + 1.41}} \ln\left(\frac{5.98h}{0.8w + t}\right) \label{eq:microstrip-impedance}
\end{equation}

对于带状线:

\begin{equation}
Z_0 = \frac{60}{\sqrt{\epsilon_r}} \ln\left(\frac{4h}{0.67\pi(0.8w + t)}\right) \label{eq:stripline-impedance}
\end{equation}

其中:
\begin{itemize}
  \item $w$:走线宽度
  \item $t$:走线厚度
  \item $h$:介质厚度
  \item $\epsilon_r$:介电常数
\end{itemize}

\subsubsection{传播延迟}

传播延迟 $t_{pd}$ 为:

\begin{equation}
t_{pd} = \frac{\sqrt{\epsilon_r}}{c} \label{eq:propagation-delay}
\end{equation}

对于 FR-4 基材($\epsilon_r \approx 4.4$),$t_{pd} \approx 150$ ps/inch。

\subsection{反射}

\subsubsection{反射系数}

当传输线阻抗不匹配时,会发生反射。反射系数为:

\begin{equation}
\Gamma = \frac{Z_L - Z_0}{Z_L + Z_0} \label{eq:reflection-coefficient}
\end{equation}

其中 $Z_L$ 是负载阻抗,$Z_0$ 是特性阻抗。

\subsubsection{反射的影响}

反射会导致:
\begin{itemize}
  \item 信号过冲和下冲
  \item 振铃(Ringing)
  \item 时序问题
  \item 信号完整性下降
\end{itemize}

\subsubsection{阻抗匹配}

\paragraph{源端匹配}

在信号源端串联电阻,使源阻抗等于传输线阻抗:

\begin{equation}
R_s = Z_0 - R_{out} \label{eq:source-termination}
\end{equation}

\paragraph{端接匹配}

在负载端并联电阻,使负载阻抗等于传输线阻抗:

\begin{equation}
R_t = Z_0 \label{eq:termination-resistor}
\end{equation}

\paragraph{差分阻抗}

差分阻抗 $Z_{diff}$ 为:

\begin{equation}
Z_{diff} = 2 Z_0 (1 - k) \label{eq:differential-impedance}
\end{equation}

其中 $k$ 是耦合系数。

\subsection{串扰}

\subsubsection{串扰的机理}

串扰(Crosstalk)是相邻走线之间的相互干扰,包括:
\begin{itemize}
  \item 容性耦合
  \item 感性耦合
\end{itemize}

\subsubsection{串扰的减小}

\begin{enumerate}
  \item \textbf{增加间距}
    \begin{itemize}
      \item 3W 规则:间距至少为走线宽度的 3 倍
    \end{itemize}

  \item \textbf{缩短平行长度}
    \begin{itemize}
      \item 减少耦合长度
    \end{itemize}

  \item \textbf{使用地线隔离}
    \begin{itemize}
      \item 在敏感信号之间走地线
    \end{itemize}

  \item \textbf{使用差分信号}
    \begin{itemize}
      \item 差分信号抗干扰能力强
    \end{itemize}

  \item \textbf{层间隔离}
    \begin{itemize}
      \item 使用地平面隔离不同层
    \end{itemize}
\end{enumerate}

\subsection{时序}

\subsubsection{建立时间和保持时间}

数字信号必须满足建立时间(Setup Time)和保持时间(Hold Time)要求:

\begin{align}
t_{setup} &< t_{clk} - t_{delay} - t_{skew} \label{eq:setup-time} \\
t_{hold} &< t_{delay} - t_{skew} \label{eq:hold-time}
\end{align}

\subsubsection{时钟抖动}

时钟抖动(Jitter)会影响时序裕量,需要:
\begin{itemize}
  \item 使用低抖动时钟源
  \item 良好的时钟分配网络
  \item 减少干扰
\end{itemize}

\subsubsection{等长走线}

对于并行总线,需要等长走线以保证时序:

\begin{itemize}
  \item 数据线和地址线等长
  \item 时钟和数据等长或固定延迟
  \item 使用蛇形走线(Serpentine)调整长度
\end{itemize}

\section{电源完整性}

电源完整性(Power Integrity,PI)确保电源分配网络(PDN)能够为所有元件提供稳定、干净的电源。

\subsection{电源分配网络(PDN)}

\subsubsection{PDN 的组成}

PDN 包括:
\begin{itemize}
  \item 电源平面
  \item 地平面
  \item 去耦电容
  \item 过孔
  \item 走线
\end{itemize}

\subsubsection{目标阻抗}

PDN 的目标阻抗 $Z_{target}$ 为:

\begin{equation}
Z_{target} = \frac{V_{DD} \cdot \text{容差}}{I_{max}} \label{eq:target-impedance}
\end{equation}

例如,对于 3.3 V 电源,5\% 容差,1 A 最大电流:

\begin{equation}
Z_{target} = \frac{3.3 \times 0.05}{1} = 165 \text{ m}\Omega
\end{equation}

\subsection{去耦电容}

\subsubsection{去耦电容的作用}

\begin{enumerate}
  \item \textbf{提供瞬态电流}
    \begin{itemize}
      \item 响应快速的电流需求
    \end{itemize}

  \item \textbf{减小电源噪声}
    \begin{itemize}
      \item 滤除高频噪声
    \end{itemize}

  \item \textbf{降低 PDN 阻抗}
    \begin{itemize}
      \item 在特定频率范围内
    \end{itemize}
\end{enumerate}

\subsubsection{去耦电容的选择}

\begin{enumerate}
  \item \textbf{大容量电容(Bulk Capacitor)}
    \begin{itemize}
      \item 10--100 $\mu$F
      \item 提供低频去耦
      \item 通常放在电源入口
    \end{itemize}

  \item \textbf{中容量电容}
    \begin{itemize}
      \item 0.1--1 $\mu$F
      \item 提供中频去耦
      \item 通常放在 IC 附近
    \end{itemize}

  \item \textbf{小容量电容}
    \begin{itemize}
      \item 1--100 nF
      \item 提供高频去耦
      \item 紧贴 IC 电源引脚
    \end{itemize}
\end{enumerate}

\subsubsection{去耦电容的布局}

\begin{enumerate}
  \item \textbf{靠近 IC}
    \begin{itemize}
      \item 减小回路电感
      \item 提高去耦效果
    \end{itemize}

  \item \textbf{多电容并联}
    \begin{itemize}
      \item 覆盖不同频率范围
      \item 降低总阻抗
    \end{itemize}

  \item \textbf{过孔位置}
    \begin{itemize}
      \item 过孔靠近电容和 IC
      \item 减小回路面积
    \end{itemize}
\end{enumerate}

\subsection{地平面设计}

\subsubsection{地平面的作用}

\begin{itemize}
  \item 提供低阻抗返回路径
  \item 减小环路面积
  \item 屏蔽和隔离
  \item 散热
\end{itemize}

\subsubsection{地平面设计原则}

\begin{enumerate}
  \item \textbf{完整性}
    \begin{itemize}
      \item 尽量保持地平面完整
      \item 避免分割
    \end{itemize}

  \item \textbf{多点接地}
    \begin{itemize}
      \item 数字地和模拟地分开,单点连接
      \item 或使用统一地平面,但分区
    \end{itemize}

  \item \textbf{过孔}
    \begin{itemize}
      \item 地过孔靠近信号过孔
      \item 提供返回路径
    \end{itemize}
\end{enumerate}

\subsection{电源平面分割}

\subsubsection{分割原则}

\begin{enumerate}
  \item \textbf{不同电压分开}
    \begin{itemize}
      \item 不同电压等级使用不同平面或区域
    \end{itemize}

  \item \textbf{保持完整性}
    \begin{itemize}
      \item 避免过度分割
      \item 考虑电流路径
    \end{itemize}

  \item \textbf{跨分割}
    \begin{itemize}
      \item 避免信号跨电源平面分割
      \item 使用去耦电容桥接
    \end{itemize}
\end{enumerate}

\section{电磁兼容性(EMC)}

电磁兼容性(Electromagnetic Compatibility,EMC)包括:
\begin{itemize}
  \item 电磁干扰(EMI):设备产生的干扰
  \item 电磁敏感性(EMS):设备对外界干扰的抵抗能力
\end{itemize}

\subsection{EMI 的产生机理}

\subsubsection{差模和共模}

\begin{enumerate}
  \item \textbf{差模(Differential Mode)}
    \begin{itemize}
      \item 信号线和返回路径之间的电流
      \item 主要产生近场干扰
    \end{itemize}

  \item \textbf{共模(Common Mode)}
    \begin{itemize}
      \item 信号线和地之间的共模电流
      \item 主要产生远场辐射
    \end{itemize}
\end{enumerate}

\subsubsection{辐射路径}

\begin{itemize}
  \item 走线作为天线
  \item 环路辐射
  \item 共模电流
\end{itemize}

\subsection{EMC 设计方法}

\subsubsection{减小环路面积}

\begin{itemize}
  \item 信号和返回路径靠近
  \item 使用地平面
  \item 缩短走线长度
\end{itemize}

\subsubsection{滤波}

\begin{itemize}
  \item 电源滤波
  \item 信号滤波
  \item 使用磁珠、共模扼流圈
\end{itemize}

\subsubsection{屏蔽}

\begin{itemize}
  \item 使用屏蔽层
  \item 屏蔽罩
  \item 屏蔽走线
\end{itemize}

\subsubsection{布局和走线}

\begin{itemize}
  \item 敏感电路远离干扰源
  \item 时钟电路特殊处理
  \item 使用地线隔离
\end{itemize}

\section{热设计}

\subsection{热设计的重要性}

电子元件在工作时会产生热量,如果散热不当,会导致:
\begin{itemize}
  \item 元件温度过高,性能下降
  \item 可靠性降低
  \item 寿命缩短
\end{itemize}

\subsection{热阻}

\subsubsection{热阻的定义}

热阻 $R_{\theta}$ 定义为:

\begin{equation}
R_{\theta} = \frac{\Delta T}{P} \label{eq:thermal-resistance}
\end{equation}

其中 $\Delta T$ 是温差,$P$ 是功耗。

\subsubsection{热阻网络}

总热阻为各热阻之和:

\begin{equation}
R_{\theta,total} = R_{\theta,JC} + R_{\theta,CS} + R_{\theta,SA} \label{eq:total-thermal-resistance}
\end{equation}

其中:
\begin{itemize}
  \item $R_{\theta,JC}$:结到外壳热阻
  \item $R_{\theta,CS}$:外壳到散热器热阻
  \item $R_{\theta,SA}$:散热器到环境热阻
\end{itemize}

\subsection{散热方法}

\subsubsection{自然对流}

\begin{itemize}
  \item 利用空气自然流动
  \item 适用于低功耗应用
  \item 需要足够的空间和通风
\end{itemize}

\subsubsection{强制对流}

\begin{itemize}
  \item 使用风扇
  \item 提高散热效率
  \item 适用于高功耗应用
\end{itemize}

\subsubsection{热设计措施}

\begin{enumerate}
  \item \textbf{PCB 设计}
    \begin{itemize}
      \item 使用大面积铜箔
      \item 热过孔(Thermal Via)
      \item 发热元件分散放置
    \end{itemize}

  \item \textbf{元件选择}
    \begin{itemize}
      \item 选择低功耗元件
      \item 选择热阻小的封装
    \end{itemize}

  \item \textbf{散热器}
    \begin{itemize}
      \item 选择合适的散热器
      \item 良好的热接触
    \end{itemize}
\end{enumerate}

\section{高速 PCB 设计}

高速 PCB 设计需要考虑信号完整性、电源完整性、EMC 等多个方面。

\subsection{高速信号的定义}

当信号上升时间满足以下条件时,需要考虑高速设计:

\begin{equation}
t_r < \frac{1}{6 f_{max}} \label{eq:high-speed-condition}
\end{equation}

或走线长度满足:

\begin{equation}
l > \frac{t_r \cdot v}{6} \label{eq:high-speed-length}
\end{equation}

\subsection{差分信号}

\subsubsection{差分信号的优势}

\begin{itemize}
  \item 抗共模干扰
  \item 减小 EMI
  \item 提高信号质量
  \item 降低电压摆幅
\end{itemize}

\subsubsection{差分对设计}

\begin{enumerate}
  \item \textbf{等长}
    \begin{itemize}
      \item 长度差 < 5 mil(0.127 mm)
    \end{itemize}

  \item \textbf{等间距}
    \begin{itemize}
      \item 保持一致的间距
      \item 控制差分阻抗
    \end{itemize}

  \item \textbf{对称}
    \begin{itemize}
      \item 对称布局
      \item 对称过孔
    \end{itemize}

  \item \textbf{参考平面}
    \begin{itemize}
      \item 完整的参考平面
      \item 避免跨分割
    \end{itemize}
\end{enumerate}

\subsection{时钟设计}

\subsubsection{时钟分配}

\begin{enumerate}
  \item \textbf{时钟树}
    \begin{itemize}
      \item 使用时钟树分配
      \item 等长分支
    \end{itemize}

  \item \textbf{时钟缓冲器}
    \begin{itemize}
      \item 使用专用时钟缓冲器
      \item 低抖动
    \end{itemize}

  \item \textbf{时钟走线}
    \begin{itemize}
      \item 最短路径
      \item 远离干扰源
      \item 使用地线保护
    \end{itemize}
\end{enumerate}

\subsubsection{时钟抖动}

时钟抖动会影响时序,需要:
\begin{itemize}
  \item 使用低抖动时钟源
  \item 良好的电源去耦
  \item 减少干扰
\end{itemize}

\subsection{高速接口设计}

\subsubsection{常见高速接口}

\begin{enumerate}
  \item \textbf{PCIe}
    \begin{itemize}
      \item 差分阻抗:85 $\Omega$
      \item 等长要求严格
    \end{itemize}

  \item \textbf{USB}
    \begin{itemize}
      \item USB 2.0:差分阻抗 90 $\Omega$
      \item USB 3.0:更严格的要求
    \end{itemize}

  \item \textbf{SATA}
    \begin{itemize}
      \item 差分阻抗:100 $\Omega$
      \item 高速串行接口
    \end{itemize}

  \item \textbf{DDR}
    \begin{itemize}
      \item 单端阻抗:50 $\Omega$
      \item 严格的等长要求
      \item 拓扑结构重要
    \end{itemize}

  \item \textbf{HDMI}
    \begin{itemize}
      \item 差分阻抗:100 $\Omega$
      \item 高速视频接口
    \end{itemize}
\end{enumerate}

\subsubsection{接口设计要点}

\begin{enumerate}
  \item \textbf{阻抗控制}
    \begin{itemize}
      \item 严格按照规范设计阻抗
      \item 使用阻抗计算工具
      \item 与 PCB 厂商确认
    \end{itemize}

  \item \textbf{等长}
    \begin{itemize}
      \item 数据线等长
      \item 时钟和数据等长或固定延迟
    \end{itemize}

  \item \textbf{去耦}
    \begin{itemize}
      \item 充分的去耦电容
      \item 靠近接口芯片
    \end{itemize}

  \item \textbf{ESD 保护}
    \begin{itemize}
      \item 接口处添加 ESD 保护器件
    \end{itemize}
\end{enumerate}

\subsection{背板设计}

背板(Backplane)用于连接多个板卡,设计要点:

\begin{enumerate}
  \item \textbf{连接器}
    \begin{itemize}
      \item 选择高速连接器
      \item 考虑机械可靠性
    \end{itemize}

  \item \textbf{走线}
    \begin{itemize}
      \item 严格控制阻抗
      \item 等长走线
    \end{itemize}

  \item \textbf{层叠}
    \begin{itemize}
      \item 通常需要更多层
      \item 专用信号层
    \end{itemize}
\end{enumerate}

\section{射频 PCB 设计}

射频(RF)PCB 设计有特殊的要求和考虑。

\subsection{射频设计特点}

\begin{enumerate}
  \item \textbf{频率高}
    \begin{itemize}
      \item 通常 > 100 MHz
      \item 需要考虑分布参数
    \end{itemize}

  \item \textbf{阻抗匹配}
    \begin{itemize}
      \item 50 $\Omega$ 或 75 $\Omega$ 系统阻抗
      \item 严格的阻抗控制
    \end{itemize}

  \item \textbf{损耗敏感}
    \begin{itemize}
      \item 使用低损耗基材
      \item 减小走线损耗
    \end{itemize}
\end{enumerate}

\subsection{射频布局}

\begin{enumerate}
  \item \textbf{分区}
    \begin{itemize}
      \item RF 电路单独分区
      \item 物理隔离
    \end{itemize}

  \item \textbf{走线}
    \begin{itemize}
      \item 最短路径
      \item 避免直角
      \item 使用圆弧或切角
    \end{itemize}

  \item \textbf{过孔}
    \begin{itemize}
      \item 尽量减少过孔
      \item 过孔会增加寄生参数
    \end{itemize}
\end{enumerate}

\subsection{射频基材}

\begin{itemize}
  \item Rogers 系列:RO4003、RO4350 等
  \item Taconic 系列
  \item 低介电常数和损耗
  \item 温度稳定性好
\end{itemize}

\section{可制造性设计(DFM)}

可制造性设计(Design for Manufacturing,DFM)确保设计能够顺利制造。

\subsection{DFM 原则}

\begin{enumerate}
  \item \textbf{最小线宽和间距}
    \begin{itemize}
      \item 符合制造能力
      \item 留有余量
    \end{itemize}

  \item \textbf{过孔设计}
    \begin{itemize}
      \item 过孔尺寸合理
      \item 过孔密度适中
    \end{itemize}

  \item \textbf{阻焊设计}
    \begin{itemize}
      \item 阻焊开窗合适
      \item 避免阻焊桥
    \end{itemize}

  \item \textbf{丝印}
    \begin{itemize}
      \item 清晰可读
      \item 不覆盖焊盘
    \end{itemize}

  \item \textbf{测试点}
    \begin{itemize}
      \item 关键信号添加测试点
      \item 便于调试和测试
    \end{itemize}
\end{enumerate}

\subsection{拼板设计}

对于小尺寸 PCB,通常需要拼板:

\begin{itemize}
  \item 提高制造效率
  \item 降低成本
  \item 添加工艺边
  \item 使用 V-cut 或邮票孔
\end{itemize}

\section{测试和调试}

\subsection{测试方法}

\begin{enumerate}
  \item \textbf{电气测试}
    \begin{itemize}
      \item 飞针测试
      \item 针床测试
      \item 边界扫描(JTAG)
    \end{itemize}

  \item \textbf{功能测试}
    \begin{itemize}
      \item 上电测试
      \item 功能验证
    \end{itemize}

  \item \textbf{信号完整性测试}
    \begin{itemize}
      \item 使用示波器
      \item TDR(时域反射计)
      \item 网络分析仪
    \end{itemize}
\end{enumerate}

\subsection{调试技巧}

\begin{enumerate}
  \item \textbf{测试点}
    \begin{itemize}
      \item 关键信号预留测试点
      \item 便于测量
    \end{itemize}

  \item \textbf{跳线}
    \begin{itemize}
      \item 预留跳线位置
      \item 便于修改
    \end{itemize}

  \item \textbf{指示灯}
    \begin{itemize}
      \item 电源指示
      \item 状态指示
    \end{itemize}

  \item \textbf{调试接口}
    \begin{itemize}
      \item JTAG 接口
      \item UART 接口
      \item 便于调试
    \end{itemize}
\end{enumerate}

\section{设计工具和流程}

\subsection{常用设计工具}

\begin{enumerate}
  \item \textbf{原理图工具}
    \begin{itemize}
      \item Altium Designer
      \item Cadence OrCAD
      \item KiCad(开源)
    \end{itemize}

  \item \textbf{PCB 布局工具}
    \begin{itemize}
      \item Altium Designer
      \item Cadence Allegro
      \item Mentor PADS
      \item KiCad(开源)
    \end{itemize}

  \item \textbf{仿真工具}
    \begin{itemize}
      \item SPICE 仿真
      \item SI/PI 仿真工具
      \item 3D EM 仿真
    \end{itemize}
\end{enumerate}

\subsection{设计流程}

\begin{enumerate}
  \item \textbf{需求分析}
    \begin{itemize}
      \item 功能需求
      \item 性能指标
      \item 约束条件
    \end{itemize}

  \item \textbf{原理图设计}
    \begin{itemize}
      \item 电路设计
      \item 元件选型
      \item 仿真验证
    \end{itemize}

  \item \textbf{PCB 布局}
    \begin{itemize}
      \item 元件布局
      \item 走线设计
      \item 层叠设计
    \end{itemize}

  \item \textbf{仿真验证}
    \begin{itemize}
      \item SI 仿真
      \item PI 仿真
      \item EMC 仿真
    \end{itemize}

  \item \textbf{设计评审}
    \begin{itemize}
      \item DRC 检查
      \item 设计评审
      \item 修改完善
    \end{itemize}

  \item \textbf{制造文件}
    \begin{itemize}
      \item Gerber 文件
      \item 钻孔文件
      \item 装配图
      \item BOM
    \end{itemize}

  \item \textbf{制造和测试}
    \begin{itemize}
      \item PCB 制造
      \item 元件贴装
      \item 测试验证
    \end{itemize}
\end{enumerate}

\section{设计检查清单}

\subsection{原理图检查}

\begin{enumerate}
  \item 所有网络连接正确
  \item 电源和地网络完整
  \item 元件参数正确
  \item 封装指定正确
  \item 设计规则检查通过
\end{enumerate}

\subsection{PCB 检查}

\begin{enumerate}
  \item DRC 检查通过
  \item 阻抗控制正确
  \item 等长走线满足要求
  \item 去耦电容充分
  \item 地平面完整
  \item 电源分配合理
  \item 热设计考虑
  \item EMC 考虑
  \item DFM 检查通过
  \item 测试点充分
\end{enumerate}

\section{总结}

PCB 设计是一个复杂的系统工程,需要综合考虑电气性能、机械结构、热设计、EMC、可制造性等多个方面。随着电子系统向高速、高密度、多功能方向发展,PCB 设计的挑战也在不断增加。掌握系统化的设计方法和工具,遵循设计规范和最佳实践,是成功完成 PCB 设计的关键。
