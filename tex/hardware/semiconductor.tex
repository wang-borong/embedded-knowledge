\chapter{分立半导体元件}

半导体元件是现代电子系统的基础,理解其工作原理对于电路设计和分析至关重要。本章将系统介绍二极管、双极型晶体管(BJT)、结型场效应管(JFET)和金属氧化物半导体场效应管(MOSFET)等主要分立半导体元件。

\section{二极管}

二极管是最基本的半导体元件,由一个 PN 结构成。二极管具有单向导电性,在电子电路中广泛应用于整流、稳压、开关、限幅等功能。

\subsection{PN 结基础}

\subsubsection{PN 结的形成}

PN 结是半导体器件的基础结构。当 P 型半导体和 N 型半导体紧密接触时,由于载流子浓度梯度的存在,会发生载流子的扩散运动:

\begin{itemize}
  \item P 区的空穴向 N 区扩散
  \item N 区的电子向 P 区扩散
\end{itemize}

扩散的结果是在接触面附近形成一个空间电荷区(耗尽层),其中:
\begin{itemize}
  \item P 区一侧留下带负电的受主离子
  \item N 区一侧留下带正电的施主离子
\end{itemize}

这些固定电荷形成内建电场,方向从 N 区指向 P 区,阻碍载流子的进一步扩散。当扩散电流与漂移电流达到动态平衡时,PN 结处于平衡状态。

\subsubsection{内建电势}

平衡状态下,PN 结存在内建电势差 $V_{bi}$:

\begin{equation}
V_{bi} = \frac{kT}{q} \ln\left(\frac{N_A N_D}{n_i^2}\right) \label{eq:built-in-potential}
\end{equation}

其中:
\begin{itemize}
  \item $k$:玻尔兹曼常数($1.38 \times 10^{-23}$ J/K)
  \item $T$:绝对温度(K)
  \item $q$:电子电荷($1.6 \times 10^{-19}$ C)
  \item $N_A$:P 区受主浓度
  \item $N_D$:N 区施主浓度
  \item $n_i$:本征载流子浓度
\end{itemize}

在室温(300 K)下,$V_{bi}$ 的典型值约为 0.6--0.8 V(对于硅材料)。

\subsubsection{耗尽层宽度}

耗尽层的宽度 $W$ 与掺杂浓度和偏置电压有关:

\begin{equation}
W = \sqrt{\frac{2\epsilon_s}{q}\left(\frac{1}{N_A} + \frac{1}{N_D}\right)(V_{bi} - V)} \label{eq:depletion-width}
\end{equation}

其中 $\epsilon_s$ 是半导体的介电常数,$V$ 是外加电压(正偏时 $V > 0$,反偏时 $V < 0$)。

\subsection{二极管的伏安特性}

\subsubsection{理想二极管的特性}

理想二极管在正向偏置(P 区接正,N 区接负)时完全导通,相当于短路;在反向偏置时完全截止,相当于开路。

\subsubsection{实际二极管的伏安特性}

实际二极管的电流-电压关系由肖克利方程(Shockley equation)描述:

\begin{equation}
I = I_S \left(e^{V/(nV_T)} - 1\right) \label{eq:diode-current}
\end{equation}

其中:
\begin{itemize}
  \item $I_S$:反向饱和电流(Reverse Saturation Current),典型值在 $10^{-12}$--$10^{-6}$ A 范围
  \item $V$:二极管两端电压(正偏时为正)
  \item $n$:发射系数(Emission Coefficient),理想情况下为 1,实际值通常在 1--2 之间
  \item $V_T = kT/q$:热电压(Thermal Voltage),室温下约为 26 mV
\end{itemize}

\subsubsection{正向特性}

当 $V > 0$ 且 $V \gg V_T$ 时,式~\ref{eq:diode-current} 可简化为:

\begin{equation}
I \approx I_S e^{V/(nV_T)} \label{eq:diode-forward}
\end{equation}

正向导通时,二极管两端存在一个近似恒定的压降:
\begin{itemize}
  \item 硅二极管:约 0.6--0.7 V
  \item 锗二极管:约 0.2--0.3 V
  \item 肖特基二极管:约 0.3--0.4 V
\end{itemize}

\subsubsection{反向特性}

当 $V < 0$ 且 $|V| \gg V_T$ 时,式~\ref{eq:diode-current} 可简化为:

\begin{equation}
I \approx -I_S \label{eq:diode-reverse}
\end{equation}

反向电流很小,近似等于反向饱和电流 $I_S$。但当反向电压达到击穿电压 $V_{BR}$ 时,会发生击穿,反向电流急剧增加。

\subsubsection{温度特性}

二极管特性受温度影响显著:

\begin{itemize}
  \item 反向饱和电流 $I_S$ 随温度升高而指数增加,温度每升高 10$^\circ$C,$I_S$ 约增加一倍
  \item 正向压降具有负温度系数,温度每升高 1$^\circ$C,正向压降约减小 2 mV
  \item 击穿电压具有正温度系数
\end{itemize}

\subsection{齐纳二极管}

齐纳二极管(Zener Diode)是一种特殊设计的二极管,能够在反向击穿状态下稳定工作,主要用于电压基准和稳压电路。

\subsubsection{工作原理}

齐纳二极管的工作原理基于 PN 结的反向击穿机制。根据击穿电压的不同,击穿机制主要有两种:

\paragraph{齐纳击穿(Zener Breakdown)}

当击穿电压较低(通常 $V_Z < 5.6$ V)时,主要发生齐纳击穿。齐纳二极管采用重掺杂的 PN 结,耗尽层非常薄(小于 1 $\mu$m)。即使在较小的反向偏压下(约 5 V),耗尽层内的电场强度也非常高(约 500 kV/m)。

在强电场作用下,价带电子通过量子隧穿效应直接进入导带,形成大量载流子,导致反向电流急剧增加。齐纳击穿具有负温度系数,即温度升高时击穿电压降低。

\paragraph{雪崩击穿(Avalanche Breakdown)}

当击穿电压较高(通常 $V_Z > 5.6$ V)时,主要发生雪崩击穿。此时耗尽层较宽,电场强度虽然不足以直接产生隧穿,但足以使载流子获得足够能量。

高能载流子与晶格原子碰撞,产生电子-空穴对,这些新产生的载流子又被加速,继续碰撞产生更多载流子,形成雪崩倍增效应。雪崩击穿具有正温度系数,即温度升高时击穿电压升高。

\paragraph{击穿电压范围}

实际齐纳二极管中,两种击穿机制往往同时存在:
\begin{itemize}
  \item 低电压($V_Z < 5$ V):齐纳效应占主导
  \item 中等电压($V_Z \approx 5.6$ V):两种效应相当,温度系数接近零
  \item 高电压($V_Z > 5.6$ V):雪崩效应占主导
\end{itemize}

商用齐纳二极管的击穿电压范围通常为 1.2 V 到 200 V,精度可达 0.07\%,常见精度为 5\% 和 10\%。

\subsubsection{齐纳二极管的特性参数}

\paragraph{齐纳电压 $V_Z$}

齐纳电压是齐纳二极管在指定测试电流下的反向击穿电压。该电压在较宽的电流范围内保持相对稳定,这是齐纳二极管作为电压基准的基础。

\paragraph{动态电阻 $r_Z$}

动态电阻定义为齐纳电压变化量与电流变化量的比值:

\begin{equation}
r_Z = \frac{\Delta V_Z}{\Delta I_Z} \label{eq:zener-resistance}
\end{equation}

$r_Z$ 越小,齐纳二极管的稳压性能越好。典型值在几欧姆到几十欧姆之间。

\paragraph{温度系数}

温度系数表示齐纳电压随温度的变化率,单位为 mV/$^\circ$C 或 ppm/$^\circ$C。如前所述,不同击穿电压的齐纳二极管具有不同的温度系数。

\paragraph{最大功耗 $P_{max}$}

最大功耗限制了齐纳二极管的最大工作电流:

\begin{equation}
I_{Z,max} = \frac{P_{max}}{V_Z} \label{eq:zener-max-current}
\end{equation}

\subsubsection{齐纳稳压电路}

齐纳二极管最典型的应用是作为并联稳压器(Shunt Regulator),如图~\ref{fig:zener-stabilized-circuit} 所示。

\Figure[caption={齐纳稳压电路}, label={fig:zener-stabilized-circuit}, width=0.4]{Zener_diode_voltage_regulator}

在该电路中,输入电压 $V_{in}$ 通过限流电阻 $R$ 连接到齐纳二极管。当输入电压高于齐纳电压时,齐纳二极管反向击穿,将输出电压 $V_{out}$ 稳定在 $V_Z$ 附近。

\paragraph{工作原理}

\begin{enumerate}
  \item 当 $V_{in} < V_Z$ 时,齐纳二极管未击穿,处于截止状态,输出电压随输入电压变化
  \item 当 $V_{in} \geq V_Z$ 时,齐纳二极管击穿,输出电压被箝位在 $V_Z$ 附近
  \item 当输入电压或负载电流变化时,通过齐纳二极管电流的自动调节,维持输出电压基本恒定
\end{enumerate}

\paragraph{电路分析}

根据基尔霍夫电流定律,流过限流电阻 $R$ 的电流为:

\begin{equation}
I_R = I_Z + I_L = \frac{V_{in} - V_{out}}{R} \label{eq:zener-current-balance}
\end{equation}

其中 $I_Z$ 是齐纳二极管电流,$I_L$ 是负载电流。

齐纳二极管电流为:

\begin{equation}
I_Z = \frac{V_{in} - V_{out}}{R} - I_L \label{eq:zener-diode-current}
\end{equation}

\paragraph{限流电阻的选择}

限流电阻 $R$ 的选择必须满足两个条件:

\begin{enumerate}
  \item \textbf{最小电流条件}:$R$ 必须足够小,确保即使在最小输入电压和最大负载电流的情况下,流过齐纳二极管的电流 $I_Z$ 仍大于最小齐纳电流 $I_{Z,min}$(通常为 1--5 mA),以维持齐纳二极管处于击穿状态。

  最小电阻值为:
  \begin{equation}
  R_{max} = \frac{V_{in,min} - V_Z}{I_{Z,min} + I_{L,max}} \label{eq:zener-r-min}
  \end{equation}

  \item \textbf{最大功耗条件}:$R$ 必须足够大,确保即使在最大输入电压和最小负载电流的情况下,齐纳二极管的功耗不超过最大允许值。

  最大电阻值为:
  \begin{equation}
  R_{min} = \frac{V_{in,max} - V_Z}{I_{Z,max} + I_{L,min}} = \frac{V_{in,max} - V_Z}{P_{max}/V_Z + I_{L,min}} \label{eq:zener-r-max}
  \end{equation}

  因此,$R$ 的选择范围为:$R_{min} < R < R_{max}$
\end{enumerate}

\paragraph{稳压性能}

齐纳稳压电路的稳压性能主要受以下因素影响:

\begin{itemize}
  \item \textbf{输入电压变化}:输出电压变化 $\Delta V_{out} \approx r_Z \Delta I_Z$,其中 $\Delta I_Z$ 由输入电压变化引起
  \item \textbf{负载电流变化}:负载电流变化时,齐纳电流反向变化以补偿,但受动态电阻 $r_Z$ 影响
  \item \textbf{温度变化}:齐纳电压的温度系数导致输出电压随温度漂移
\end{itemize}

\paragraph{应用限制}

并联稳压器虽然简单,但存在以下缺点:

\begin{itemize}
  \item \textbf{效率低}:限流电阻和齐纳二极管始终消耗功率,特别是在轻载或无载情况下
  \item \textbf{功率限制}:受齐纳二极管最大功耗限制,只能用于小功率应用
  \item \textbf{精度有限}:稳压精度受齐纳二极管动态电阻和温度系数影响
\end{itemize}

因此,齐纳稳压电路主要用于:
\begin{itemize}
  \item 低功率电压基准
  \item 简单稳压电路
  \item 更复杂稳压电路的参考电压源
\end{itemize}

\subsubsection{其他应用}

\paragraph{温度补偿}

齐纳二极管可以与晶体管的基极-发射极结串联使用,利用齐纳击穿和雪崩击穿的不同温度系数,实现温度补偿,提高电路的温度稳定性。这在精密稳压电源的误差放大器中经常使用。

\paragraph{过压保护}

齐纳二极管可用于浪涌保护器(Surge Protector)中,限制瞬态电压尖峰,保护敏感电路免受静电放电(ESD)或电源浪涌的损害。

\paragraph{电压箝位}

在信号处理电路中,齐纳二极管可用于电压箝位,限制信号幅度。

\subsection{其他类型二极管}

\subsubsection{肖特基二极管}

肖特基二极管(Schottky Diode)由金属和半导体接触形成,而非 PN 结。其主要特点:

\begin{itemize}
  \item 正向压降低(约 0.3--0.4 V)
  \item 开关速度快,反向恢复时间极短
  \item 反向漏电流较大
  \item 主要用于高频整流和开关电路
\end{itemize}

\subsubsection{发光二极管(LED)}

发光二极管(Light Emitting Diode,LED)在正向偏置时会发光,广泛应用于显示和照明。不同材料可产生不同颜色的光。

\subsubsection{光电二极管}

光电二极管(Photodiode)在光照下会产生电流,用于光检测和光通信。

\section{双极型晶体管(BJT)}

双极型晶体管(Bipolar Junction Transistor,BJT)是一种电流控制型三端器件,由两个 PN 结组成。BJT 具有放大作用,是模拟电路和数字电路中的重要元件。

\subsection{BJT 的基本结构}

BJT 有两种基本类型:
\begin{itemize}
  \item \textbf{NPN 型}:中间是 P 型区(基区),两侧是 N 型区(发射区和集电区)
  \item \textbf{PNP 型}:中间是 N 型区(基区),两侧是 P 型区(发射区和集电区)
\end{itemize}

三个电极分别为:
\begin{itemize}
  \item \textbf{发射极(Emitter,E)}:发射载流子
  \item \textbf{基极(Base,B)}:控制载流子
  \item \textbf{集电极(Collector,C)}:收集载流子
\end{itemize}

\subsection{BJT 的工作原理}

以 NPN 型 BJT 为例说明工作原理。

\subsubsection{工作模式}

BJT 有三种工作模式,由两个 PN 结的偏置状态决定:

\begin{enumerate}
  \item \textbf{放大模式(Active Mode)}:发射结正偏,集电结反偏
    \begin{itemize}
      \item 这是 BJT 的正常放大工作状态
      \item 集电极电流 $I_C$ 受基极电流 $I_B$ 控制
      \item $I_C = \beta I_B$,其中 $\beta$ 是电流放大系数
    \end{itemize}

  \item \textbf{饱和模式(Saturation Mode)}:发射结正偏,集电结正偏
    \begin{itemize}
      \item BJT 相当于一个闭合的开关
      \item $V_{CE} \approx 0.2$ V(饱和压降)
      \item 集电极电流由外电路决定
    \end{itemize}

  \item \textbf{截止模式(Cutoff Mode)}:发射结反偏,集电结反偏
    \begin{itemize}
      \item BJT 相当于一个断开的开关
      \item 各极电流近似为零
    \end{itemize}
\end{enumerate}

\subsubsection{放大原理}

在放大模式下,BJT 的放大作用基于以下物理过程:

\begin{enumerate}
  \item 发射结正偏,大量电子从发射区注入基区
  \item 由于基区很薄且轻掺杂,大部分电子能够扩散到集电结
  \item 集电结反偏,强电场将到达集电结的电子拉入集电区,形成集电极电流
  \item 只有少量电子在基区与空穴复合,形成基极电流
\end{enumerate}

因此,集电极电流远大于基极电流,实现电流放大。

\subsubsection{电流关系}

在放大模式下,BJT 的电流关系为:

\begin{align}
I_E &= I_B + I_C \label{eq:bjt-current-sum} \\
I_C &= \beta I_B \label{eq:bjt-current-gain} \\
I_E &= (\beta + 1) I_B = \alpha I_C \label{eq:bjt-emitter-current}
\end{align}

其中:
\begin{itemize}
  \item $\beta$:共发射极电流放大系数,典型值 50--300
  \item $\alpha = \frac{\beta}{\beta + 1}$:共基极电流放大系数,接近 1(通常 0.95--0.99)
\end{itemize}

\subsection{BJT 的伏安特性}

\subsubsection{输入特性}

输入特性描述 $I_B$ 与 $V_{BE}$ 的关系,类似于二极管的伏安特性:

\begin{equation}
I_B = I_{S,B} \left(e^{V_{BE}/(nV_T)} - 1\right) \label{eq:bjt-base-current}
\end{equation}

其中 $I_{S,B}$ 是基极反向饱和电流。

\subsubsection{输出特性}

输出特性描述 $I_C$ 与 $V_{CE}$ 的关系,以 $I_B$ 为参变量。输出特性曲线分为三个区域:

\begin{enumerate}
  \item \textbf{截止区}:$I_B \approx 0$,$I_C \approx 0$
  \item \textbf{放大区}:$I_C$ 主要由 $I_B$ 决定,$V_{CE}$ 对 $I_C$ 影响较小(存在 Early 效应)
  \item \textbf{饱和区}:$V_{CE}$ 很小,$I_C$ 主要由外电路决定
\end{enumerate}

\subsection{BJT 的直流分析}

\subsubsection{静态工作点(Q 点)}

静态工作点是指无信号输入时,BJT 各极的直流电压和电流值,通常用 $I_{CQ}$、$V_{CEQ}$、$I_{BQ}$、$V_{BEQ}$ 表示。

静态工作点的选择对放大器的性能至关重要:
\begin{itemize}
  \item Q 点应位于放大区中央,以获得最大动态范围
  \item Q 点应稳定,不受温度等因素影响
  \item Q 点应满足功耗要求
\end{itemize}

\subsubsection{偏置电路}

偏置电路的作用是建立合适的静态工作点。常见的偏置方式包括:

\paragraph{固定偏置}

最简单的偏置方式,但温度稳定性差,很少使用。

\paragraph{分压式偏置(最常用)}

采用基极分压电阻和发射极电阻,具有良好的温度稳定性。

\paragraph{电流源偏置}

使用电流源提供基极电流,稳定性最好,但电路复杂。

\subsubsection{直流分析步骤}

\begin{enumerate}
  \item 画出直流通路(将所有电容视为开路,交流源视为短路)
  \item 假设 BJT 工作于放大模式($V_{BE} \approx 0.7$ V,$I_C = \beta I_B$)
  \item 应用基尔霍夫定律列写方程
  \item 求解得到静态工作点
  \item 验证 BJT 确实工作于放大模式($V_{CE} > V_{CE,sat} \approx 0.2$ V)
\end{enumerate}

\subsection{BJT 的小信号模型}

\subsubsection{混合 $\pi$ 模型}

在低频小信号情况下,BJT 可以用混合 $\pi$ 模型表示,主要参数包括:

\begin{itemize}
  \item $r_{\pi} = \frac{V_T}{I_B}$:基极输入电阻
  \item $g_m = \frac{I_C}{V_T} = \frac{\beta}{r_{\pi}}$:跨导
  \item $r_o = \frac{V_A}{I_C}$:输出电阻($V_A$ 是 Early 电压)
  \item $C_{\pi}$、$C_{\mu}$:结电容(高频时考虑)
\end{itemize}

\subsubsection{简化 $h$ 参数模型}

在低频分析中,常使用简化的 $h$ 参数模型:

\begin{itemize}
  \item $h_{ie} = r_{\pi}$:输入电阻
  \item $h_{fe} = \beta$:电流放大系数
  \item $h_{oe} = 1/r_o$:输出电导(通常可忽略)
  \item $h_{re}$:反向电压传输比(通常可忽略)
\end{itemize}

\subsubsection{交流分析步骤}

\begin{enumerate}
  \item 确定静态工作点
  \item 画出交流通路(将所有电容视为短路,直流源视为短路)
  \item 用 BJT 小信号模型替换 BJT
  \item 应用电路分析方法(节点法、网孔法等)求解
  \item 计算电压增益、电流增益、输入电阻、输出电阻等参数
\end{enumerate}

\subsection{基本放大电路}

\subsubsection{共发射极(CE)放大器}

共发射极放大器是最常用的放大器配置,具有以下特点:

\begin{itemize}
  \item \textbf{电压增益}:高(通常 10--1000)
  \item \textbf{电流增益}:高($\beta$)
  \item \textbf{输入电阻}:中等
  \item \textbf{输出电阻}:中等
  \item \textbf{相位关系}:输入与输出反相(180$^\circ$ 相位差)
\end{itemize}

共发射极放大器电路如图~\ref{fig:ce-amplifier} 所示。

\begin{figure}[htbp]
  \centering
  \begin{circuitikz}[american,scale=1.0]
    % Common Emitter Amplifier
    % Transistor
    \node[npn] (Q) at (7.75, 5.25){};

    % Bias Network
    \node[ground] at (6.5, 2){};
    \draw (6.5, 8) to[american resistor, l={$R_{B1}$}] (6.5, 6);
    \draw (6.5, 4.5) to[american resistor, l={$R_{B2}$}] (6.5, 2.5);
    \draw (6.5, 6) -- (6.5, 4.5); % Vertical connection
    \draw (6.5, 2.5) -- (6.5, 2); % To ground
    \draw (6.5, 5.25) to[short, *-] (Q.B); % To Base, with dot

    % Input
    \node[ground] at (4.5, 2){};
    \draw (4.5, 2.5) to[sinusoidal voltage source, l={$v_{in}$}] (4.5, 4.5);
    \draw (4.5, 2.5) -- (4.5, 2);
    \draw (4.5, 4.5) |- (5, 5.25) to[capacitor, l={$C_1$}] (6.5, 5.25);

    % Collector
    \draw (7.75, 8) to[american resistor, l={$R_C$}] (7.75, 6);
    \draw (Q.C) -- (7.75, 6);

    % Emitter
    \draw (7.75, 4.5) to[american resistor, l={$R_E$}] (7.75, 2.5);
    \draw (Q.E) -- (7.75, 4.5);
    \draw (7.75, 2.5) -- (7.75, 2) node[ground]{};

    % Emitter Bypass
    \draw (8.75, 4) to[capacitor, l={$C_E$}] (8.75, 2.75);
    \draw (7.75, 4.5) to[short, *-] (8.75, 4.5) -- (8.75, 4); % Dot at junction
    \draw (8.75, 2.75) -- (8.75, 2) node[ground]{};

    % VCC
    \node[vcc] at (7.75, 8.5){$V_{CC}$};
    \draw (7.75, 8) -- (7.75, 8.5);
    \draw (6.5, 8) |- (7.75, 8.25) node[circ]{}; % Dot at junction

    % Output
    \draw (7.75, 6) to[short, *-] (9, 6) to[capacitor, l={$C_2$}] (10.5, 6); % Dot at junction
    \draw (10.5, 5) to[american resistor, l={$R_L$}] (10.5, 3);
    \draw (10.5, 6) -- (10.5, 5);
    \draw (10.5, 3) -- (10.5, 2) node[ground]{};
    \draw (10.5, 6) to[short, *-o] (11, 6) node[right]{$v_{out}$}; % Dot and circle
  \end{circuitikz}
  \caption{共发射极(CE)放大器电路}
  \label{fig:ce-amplifier}
\end{figure}

电压增益为:

\begin{equation}
A_v = \frac{v_{out}}{v_{in}} = -\frac{R_C \parallel R_L}{r_e + R_E} \approx -\frac{R_C \parallel R_L}{r_e} \label{eq:ce-voltage-gain}
\end{equation}

其中 $r_e = \frac{V_T}{I_E} \approx \frac{26\text{ mV}}{I_E}$ 是发射极动态电阻。

\subsubsection{共集电极(CC)放大器(射极跟随器)}

共集电极放大器也称为射极跟随器(Emitter Follower),具有以下特点:

\begin{itemize}
  \item \textbf{电压增益}:接近 1(略小于 1)
  \item \textbf{电流增益}:高($\beta + 1$)
  \item \textbf{输入电阻}:高
  \item \textbf{输出电阻}:低
  \item \textbf{相位关系}:输入与输出同相
\end{itemize}

共集电极放大器电路如图~\ref{fig:cc-amplifier} 所示。

\begin{figure}[htbp]
  \centering
  \begin{circuitikz}[american,scale=1.0]
    % Common Collector Amplifier (Emitter Follower)
    % Transistor
    \node[npn] (Q) at (7.75, 5.25){};

    % Bias Network
    \node[ground] at (6.5, 2){};
    \draw (6.5, 8) to[american resistor, l={$R_{B1}$}] (6.5, 6);
    \draw (6.5, 4.5) to[american resistor, l={$R_{B2}$}] (6.5, 2.5);
    \draw (6.5, 6) -- (6.5, 4.5); % Connect resistors
    \draw (6.5, 2.5) -- (6.5, 2);
    \draw (6.5, 5.25) to[short, *-] (Q.B); % Dot at base

    % Input
    \node[ground] at (4.5, 2){};
    \draw (4.5, 2.5) to[sinusoidal voltage source, l={$v_{in}$}] (4.5, 4.5);
    \draw (4.5, 2.5) -- (4.5, 2);
    \draw (4.5, 4.5) |- (5, 5.25) to[capacitor, l={$C_1$}] (6.5, 5.25);

    % Collector (Direct to VCC)
    \draw (Q.C) -- (7.75, 8);
    \draw (7.75, 8) -- (7.75, 8.5);

    % Emitter
    \draw (7.75, 4.5) to[american resistor, l={$R_E$}] (7.75, 2.5);
    \draw (Q.E) -- (7.75, 4.5);
    \draw (7.75, 2.5) -- (7.75, 2) node[ground]{};

    % VCC
    \node[vcc] at (7.75, 8.5){$V_{CC}$};
    \draw (6.5, 8) |- (7.75, 8.25) node[circ]{}; % Dot at VCC

    % Output (from Emitter)
    \draw (7.75, 4.5) to[short, *-] (9, 4.5) to[capacitor, l={$C_2$}] (10.5, 4.5); % Dot at Emitter
    \draw (10.5, 4.5) to[american resistor, l={$R_L$}] (10.5, 2) node[ground]{};
    \draw (10.5, 4.5) to[short, *-o] (11, 4.5) node[right]{$v_{out}$}; % Dot and circle
  \end{circuitikz}
  \caption{共集电极(CC)放大器电路(射极跟随器)}
  \label{fig:cc-amplifier}
\end{figure}

电压增益为:

\begin{equation}
A_v = \frac{(R_E \parallel R_L)}{r_e + (R_E \parallel R_L)} \approx 1 \label{eq:cc-voltage-gain}
\end{equation}

射极跟随器主要用于阻抗变换和缓冲。

\subsubsection{共基极(CB)放大器}

共基极放大器具有以下特点:

\begin{itemize}
  \item \textbf{电压增益}:高(与 CE 相当)
  \item \textbf{电流增益}:低($\alpha \approx 1$)
  \item \textbf{输入电阻}:低
  \item \textbf{输出电阻}:高
  \item \textbf{相位关系}:输入与输出同相
\end{itemize}

共基极放大器电路如图~\ref{fig:cb-amplifier} 所示。

\begin{figure}[htbp]
  \centering
  \begin{circuitikz}[american,scale=1.0]
    % Common Base Amplifier
    % Transistor
    \node[npn] (Q) at (7.75, 5.25){};

    % Bias Network (Base)
    \node[ground] at (6.5, 2){};
    \draw (6.5, 8) to[american resistor, l={$R_{B1}$}] (6.5, 6);
    \draw (6.5, 4.5) to[american resistor, l={$R_{B2}$}] (6.5, 2.5);
    \draw (6.5, 6) -- (6.5, 4.5); % Connect resistors
    \draw (6.5, 2.5) -- (6.5, 2);
    \draw (6.5, 5.25) to[short, *-] (Q.B); % Dot at Base

    % Base Capacitor to Ground
    \draw (6.5, 5.25) -- (5.5, 5.25) to[capacitor, l={$C_B$}] (5.5, 2) node[ground]{};

    % Input (to Emitter)
    \node[ground] at (4.5, 2){};
    \draw (4.5, 2.5) to[sinusoidal voltage source, l={$v_{in}$}] (4.5, 4.5);
    \draw (4.5, 2.5) -- (4.5, 2);
    \draw (4.5, 4.5) -- (5, 4.5) to[capacitor, l={$C_1$}] (7.75, 4.5); % Connect to Emitter node

    % Collector
    \draw (7.75, 8) to[american resistor, l={$R_C$}] (7.75, 6);
    \draw (Q.C) -- (7.75, 6);

    % Emitter
    \draw (7.75, 4.5) to[american resistor, l={$R_E$}] (7.75, 2.5);
    \draw (Q.E) -- (7.75, 4.5);
    \draw (7.75, 2.5) -- (7.75, 2) node[ground]{};
    \draw (7.75, 4.5) node[circ]{}; % Dot at Emitter junction

    % VCC
    \node[vcc] at (7.75, 8.5){$V_{CC}$};
    \draw (7.75, 8) -- (7.75, 8.5);
    \draw (6.5, 8) |- (7.75, 8.25) node[circ]{}; % Dot at VCC

    % Output (from Collector)
    \draw (7.75, 6) to[short, *-] (9, 6) to[capacitor, l={$C_2$}] (10.5, 6); % Dot at Collector
    \draw (10.5, 5) to[american resistor, l={$R_L$}] (10.5, 3);
    \draw (10.5, 6) -- (10.5, 5);
    \draw (10.5, 3) -- (10.5, 2) node[ground]{};
    \draw (10.5, 6) to[short, *-o] (11, 6) node[right]{$v_{out}$}; % Dot and circle
  \end{circuitikz}
  \caption{共基极(CB)放大器电路}
  \label{fig:cb-amplifier}
\end{figure}

电压增益为:

\begin{equation}
A_v = \frac{R_C \parallel R_L}{r_e} \label{eq:cb-voltage-gain}
\end{equation}

共基极放大器主要用于高频应用,因为其频率响应好。

\section{结型场效应管(JFET)}

结型场效应管(Junction Field-Effect Transistor,JFET)是一种电压控制型单极器件,通过栅极电压控制沟道电阻,从而控制漏极电流。

\subsection{JFET 的基本结构}

JFET 有两种类型:
\begin{itemize}
  \item \textbf{N 沟道 JFET}:P 型栅极,N 型沟道
  \item \textbf{P 沟道 JFET}:N 型栅极,P 型沟道
\end{itemize}

三个电极:
\begin{itemize}
  \item \textbf{源极(Source,S)}:载流子源
  \item \textbf{栅极(Gate,G)}:控制电极
  \item \textbf{漏极(Drain,D)}:载流子漏
\end{itemize}

\subsection{JFET 的工作原理}

以 N 沟道 JFET 为例。

\subsubsection{工作模式}

\begin{enumerate}
  \item \textbf{截止区}:$V_{GS} < V_P$(夹断电压),沟道完全夹断,$I_D = 0$
  \item \textbf{线性区(欧姆区)}:$V_{GS} > V_P$ 且 $V_{DS} < V_{GS} - V_P$,沟道未夹断,JFET 相当于可变电阻
  \item \textbf{饱和区(恒流区)}:$V_{GS} > V_P$ 且 $V_{DS} > V_{GS} - V_P$,沟道在漏极端夹断,$I_D$ 主要由 $V_{GS}$ 控制
\end{enumerate}

\subsubsection{输出特性}

在饱和区,漏极电流为:

\begin{equation}
I_D = I_{DSS} \left(1 - \frac{V_{GS}}{V_P}\right)^2 \label{eq:jfet-saturation}
\end{equation}

其中:
\begin{itemize}
  \item $I_{DSS}$:$V_{GS} = 0$ 时的饱和漏极电流
  \item $V_P$:夹断电压(Pinch-off Voltage),为负值
\end{itemize}

\subsubsection{转移特性}

转移特性描述 $I_D$ 与 $V_{GS}$ 的关系,在饱和区满足式~\ref{eq:jfet-saturation}。

\subsection{JFET 的特点}

\begin{itemize}
  \item 输入阻抗极高(栅极反偏,输入电流极小)
  \item 噪声低
  \item 温度稳定性好
  \item 制造工艺相对简单
  \item 但跨导较低,增益不如 BJT
\end{itemize}

\section{金属氧化物半导体场效应管(MOSFET)}

金属氧化物半导体场效应管(Metal-Oxide-Semiconductor Field-Effect Transistor,MOSFET)是现代集成电路的基础,具有输入阻抗高、功耗低、易于集成等优点。

\subsection{MOSFET 的基本结构}

MOSFET 由金属栅极、氧化物绝缘层(SiO$_2$)和半导体衬底组成。根据沟道类型和导电方式,分为:

\begin{itemize}
  \item \textbf{N 沟道增强型(NMOS)}:$V_{GS} = 0$ 时无沟道,需要正栅压形成沟道
  \item \textbf{P 沟道增强型(PMOS)}:$V_{GS} = 0$ 时无沟道,需要负栅压形成沟道
  \item \textbf{N 沟道耗尽型}:$V_{GS} = 0$ 时已有沟道
  \item \textbf{P 沟道耗尽型}:$V_{GS} = 0$ 时已有沟道
\end{itemize}

四个电极:
\begin{itemize}
  \item \textbf{源极(Source,S)}
  \item \textbf{栅极(Gate,G)}
  \item \textbf{漏极(Drain,D)}
  \item \textbf{衬底(Body,B)}:通常与源极相连
\end{itemize}

\subsection{MOSFET 的工作原理}

以 N 沟道增强型 MOSFET 为例。

\subsubsection{阈值电压}

当栅极电压 $V_{GS}$ 达到阈值电压 $V_{TH}$ 时,栅极下方的半导体表面反型,形成导电沟道。

\subsubsection{工作模式}

\begin{enumerate}
  \item \textbf{截止区}:$V_{GS} < V_{TH}$,无沟道,$I_D = 0$
  \item \textbf{线性区(三极管区)}:$V_{GS} > V_{TH}$ 且 $V_{DS} < V_{GS} - V_{TH}$,沟道未夹断
  \item \textbf{饱和区}:$V_{GS} > V_{TH}$ 且 $V_{DS} > V_{GS} - V_{TH}$,沟道在漏极端夹断
\end{enumerate}

\subsubsection{输出特性}

在饱和区,漏极电流为:

\begin{equation}
I_D = \frac{1}{2}\mu_n C_{ox} \frac{W}{L} (V_{GS} - V_{TH})^2 (1 + \lambda V_{DS}) \label{eq:mosfet-saturation}
\end{equation}

其中:
\begin{itemize}
  \item $\mu_n$:电子迁移率
  \item $C_{ox}$:栅氧化层单位面积电容
  \item $W/L$:沟道宽长比
  \item $\lambda$:沟道长度调制系数
\end{itemize}

定义跨导参数:

\begin{equation}
k_n = \mu_n C_{ox} \frac{W}{L} \label{eq:mosfet-kn}
\end{equation}

则:

\begin{equation}
I_D = \frac{k_n}{2} (V_{GS} - V_{TH})^2 (1 + \lambda V_{DS}) \label{eq:mosfet-saturation-simple}
\end{equation}

在线性区:

\begin{equation}
I_D = k_n \left[(V_{GS} - V_{TH})V_{DS} - \frac{V_{DS}^2}{2}\right] \label{eq:mosfet-linear}
\end{equation}

\subsection{MOSFET 的特点}

\begin{itemize}
  \item \textbf{输入阻抗极高}:栅极与沟道之间由氧化物绝缘,直流输入电流几乎为零
  \item \textbf{功耗低}:静态功耗极低,适合大规模集成
  \item \textbf{易于集成}:制造工艺简单,适合大规模生产
  \item \textbf{开关速度快}:适合数字电路应用
  \item \textbf{跨导较低}:模拟应用时增益不如 BJT
\end{itemize}

\subsection{MOSFET 的应用}

\subsubsection{数字电路}

MOSFET 是 CMOS(Complementary MOS)数字电路的基础,用于构建逻辑门、存储器、微处理器等。

\subsubsection{模拟电路}

MOSFET 用于构建放大器、电流源、开关等模拟电路。

\subsubsection{功率应用}

功率 MOSFET 用于开关电源、电机驱动等大功率应用。

\subsection{CMOS 技术}

CMOS(Complementary MOS)技术将 NMOS 和 PMOS 互补使用,具有以下优点:

\begin{itemize}
  \item 静态功耗极低(只有动态功耗)
  \item 噪声容限大
  \item 工作电压范围宽
  \item 集成度高
\end{itemize}

CMOS 是现代数字集成电路的主流技术。
