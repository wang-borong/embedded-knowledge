\chapter{外设及通信协议}

现代嵌入式系统需要与各种外设和外部设备进行通信,通信协议的选择和实现直接影响系统的性能、可靠性和成本。本章将系统介绍常用的外设接口和通信协议,包括串行通信协议(UART、I2C、SPI)、工业总线(CAN、RS485)、高速接口(USB、PCIe、Ethernet)等,帮助读者理解和应用这些重要的通信技术。

\section{UART}

通用异步收发传输器(Universal Asynchronous Receiver/Transmitter,UART)是最常用的串行通信接口之一,具有简单、可靠、成本低等优点。

\subsection{UART 基础}

\subsubsection{基本概念}

UART 是一种异步串行通信协议,特点:
\begin{itemize}
  \item 异步通信:不需要时钟信号
  \item 全双工:可以同时发送和接收
  \item 点对点通信:通常连接两个设备
  \item 简单可靠:硬件实现简单,应用广泛
\end{itemize}

\subsubsection{信号线}

UART 通常需要至少两条信号线:
\begin{itemize}
  \item \textbf{TX(Transmit)}:发送数据线
  \item \textbf{RX(Receive)}:接收数据线
  \item \textbf{GND}:地线(必需)
  \item 可选:RTS/CTS(硬件流控)、DTR/DSR(数据终端就绪)
\end{itemize}

\subsection{UART 工作原理}

\subsubsection{数据帧格式}

UART 数据帧包括:
\begin{enumerate}
  \item \textbf{起始位(Start Bit)}:1 位,低电平(0)
  \item \textbf{数据位(Data Bits)}:5--9 位,通常 7 或 8 位
  \item \textbf{校验位(Parity Bit)}:可选,奇校验或偶校验
  \item \textbf{停止位(Stop Bit)}:1 或 2 位,高电平(1)
\end{enumerate}

典型帧格式:1 起始位 + 8 数据位 + 无校验 + 1 停止位 = 10 位

\subsubsection{波特率}

波特率(Baud Rate)是每秒传输的符号数,对于 UART,通常等于比特率(bps)。

常用波特率:9600、19200、38400、57600、115200、230400、460800、921600 bps

\subsubsection{数据发送}

发送过程:
\begin{enumerate}
  \item 空闲时,TX 线保持高电平
  \item 发送起始位(低电平)
  \item 按顺序发送数据位(LSB 或 MSB 优先)
  \item 发送校验位(如果启用)
  \item 发送停止位(高电平)
  \item 返回空闲状态
\end{enumerate}

\subsubsection{数据接收}

接收过程:
\begin{enumerate}
  \item 检测起始位(下降沿)
  \item 在数据位中点采样
  \item 读取数据位
  \item 检查校验位(如果启用)
  \item 检测停止位
\end{enumerate}

\subsection{UART 配置参数}

\subsubsection{基本参数}

\begin{enumerate}
  \item \textbf{波特率}
    \begin{itemize}
      \item 发送和接收必须相同
      \item 允许一定误差(通常 < 3\%)
    \end{itemize}

  \item \textbf{数据位}
    \begin{itemize}
      \item 5--9 位
      \item 通常使用 7 位(ASCII)或 8 位(二进制)
    \end{itemize}

  \item \textbf{校验位}
    \begin{itemize}
      \item 无校验(None)
      \item 奇校验(Odd)
      \item 偶校验(Even)
      \item 标记(Mark,总是 1)
      \item 空格(Space,总是 0)
    \end{itemize}

  \item \textbf{停止位}
    \begin{itemize}
      \item 1 位或 2 位
      \item 通常使用 1 位
    \end{itemize}

  \item \textbf{数据顺序}
    \begin{itemize}
      \item LSB 优先(最常见)
      \item MSB 优先
    \end{itemize}
\end{enumerate}

\subsection{流控制}

\subsubsection{软件流控制}

使用特殊字符:
\begin{itemize}
  \item XON(0x11):继续发送
  \item XOFF(0x13):暂停发送
\end{itemize}

\subsubsection{硬件流控制}

使用硬件信号线:
\begin{itemize}
  \item \textbf{RTS(Request To Send)}:请求发送
  \item \textbf{CTS(Clear To Send)}:清除发送
  \item \textbf{DTR(Data Terminal Ready)}:数据终端就绪
  \item \textbf{DSR(Data Set Ready)}:数据设备就绪
\end{itemize}

\subsection{UART 应用}

\begin{enumerate}
  \item 调试接口(Console)
  \item 与传感器通信
  \item 与模块通信(GPS、WiFi、蓝牙等)
  \item 系统间通信
\end{enumerate}

\section{定时器(Timer)}

定时器是嵌入式系统中最基本的外设之一,用于产生精确的时间间隔、PWM 信号、测量时间等。

\subsection{定时器的基本概念}

\subsubsection{定时器的功能}

\begin{enumerate}
  \item \textbf{定时}
    \begin{itemize}
      \item 产生精确的时间延迟
      \item 周期性中断
    \end{itemize}

  \item \textbf{计数}
    \begin{itemize}
      \item 对外部事件计数
      \item 测量频率
    \end{itemize}

  \item \textbf{PWM 生成}
    \begin{itemize}
      \item 脉宽调制信号
      \item 用于电机控制、LED 调光等
    \end{itemize}

  \item \textbf{输入捕获}
    \begin{itemize}
      \item 捕获外部信号的时间
      \item 测量脉冲宽度
    \end{itemize}

  \item \textbf{输出比较}
    \begin{itemize}
      \item 在特定时间产生输出
      \item 产生精确的时序
    \end{itemize}
\end{enumerate}

\subsection{定时器的工作原理}

\subsubsection{基本结构}

定时器通常包括:
\begin{itemize}
  \item 计数器(Counter):向上或向下计数
  \item 预分频器(Prescaler):对时钟分频
  \item 自动重装载寄存器(ARR/Auto-Reload Register):设置计数上限
  \item 比较寄存器(CCR/Compare Register):用于比较和捕获
  \item 控制寄存器:配置工作模式
\end{itemize}

\subsubsection{计数模式}

\begin{enumerate}
  \item \textbf{向上计数(Up Counting)}
    \begin{itemize}
      \item 从 0 计数到 ARR
      \item 溢出后重新开始
    \end{itemize}

  \item \textbf{向下计数(Down Counting)}
    \begin{itemize}
      \item 从 ARR 计数到 0
      \item 到 0 后重新开始
    \end{itemize}

  \item \textbf{中央对齐(Center-Aligned)}
    \begin{itemize}
      \item 先向上计数,再向下计数
      \item 用于对称 PWM
    \end{itemize}
\end{enumerate}

\subsubsection{时钟源}

定时器可以使用不同的时钟源:
\begin{itemize}
  \item 内部时钟(系统时钟)
  \item 外部时钟(外部引脚)
  \item 其他定时器的输出
\end{itemize}

\subsection{PWM 生成}

\subsubsection{PWM 原理}

脉宽调制(Pulse Width Modulation,PWM)通过改变脉冲宽度来模拟模拟信号。

PWM 参数:
\begin{itemize}
  \item \textbf{周期(Period)}:$T = \frac{ARR + 1}{f_{clk}}$
  \item \textbf{频率}:$f = \frac{1}{T}$
  \item \textbf{占空比(Duty Cycle)}:$D = \frac{CCR}{ARR + 1} \times 100\%$
\end{itemize}

\subsubsection{PWM 模式}

\begin{enumerate}
  \item \textbf{PWM 模式 1}
    \begin{itemize}
      \item CCR < 计数器值:输出有效电平
      \item CCR > 计数器值:输出无效电平
    \end{itemize}

  \item \textbf{PWM 模式 2}
    \begin{itemize}
      \item 与模式 1 相反
    \end{itemize}
\end{enumerate}

\subsection{输入捕获}

输入捕获用于测量外部信号的参数:
\begin{itemize}
  \item 频率测量
  \item 脉冲宽度测量
  \item 周期测量
\end{itemize}

工作原理:
\begin{enumerate}
  \item 检测外部信号边沿(上升沿、下降沿或双边沿)
  \item 捕获当前计数器值
  \item 计算时间差
\end{enumerate}

\subsection{输出比较}

输出比较用于在特定时间产生输出:
\begin{itemize}
  \item 产生精确的时序
  \item 单脉冲输出
  \item 定时输出
\end{itemize}

\section{看门狗定时器(WDT)}

看门狗定时器(Watchdog Timer,WDT)用于检测系统故障并自动复位系统,提高系统可靠性。

\subsection{看门狗的基本原理}

\subsubsection{工作原理}

\begin{enumerate}
  \item 看门狗定时器启动后开始倒计时
  \item 程序需要在定时器溢出前"喂狗"(刷新定时器)
  \item 如果程序正常运行,定时器会被定期刷新,不会溢出
  \item 如果程序死机或跑飞,无法刷新定时器,定时器溢出
  \item 定时器溢出后产生复位信号,重启系统
\end{enumerate}

\subsubsection{看门狗的作用}

\begin{itemize}
  \item 检测程序死机
  \item 检测程序跑飞
  \item 提高系统可靠性
  \item 自动恢复
\end{itemize}

\subsection{看门狗的类型}

\subsubsection{硬件看门狗}

\begin{itemize}
  \item 独立的硬件电路
  \item 即使 CPU 死机也能工作
  \item 可靠性高
  \item 需要外部复位电路
\end{itemize}

\subsubsection{软件看门狗}

\begin{itemize}
  \item 集成在 MCU 内部
  \item 使用系统时钟
  \item 成本低
  \item 如果时钟失效,看门狗也失效
\end{itemize}

\subsection{看门狗的配置}

\subsubsection{超时时间}

超时时间的选择:
\begin{itemize}
  \item 太短:正常程序可能来不及喂狗
  \item 太长:故障检测不及时
  \item 通常选择为最长任务执行时间的 2--3 倍
\end{itemize}

\subsubsection{窗口看门狗}

窗口看门狗(Window Watchdog)要求在特定时间窗口内喂狗:
\begin{itemize}
  \item 防止过早喂狗
  \item 防止过晚喂狗
  \item 提高检测能力
\end{itemize}

\subsection{看门狗的使用注意事项}

\begin{enumerate}
  \item 在关键任务中定期喂狗
  \item 避免在中断中喂狗(除非中断是唯一执行路径)
  \item 不要在死循环中忘记喂狗
  \item 考虑看门狗复位后的恢复策略
  \item 记录看门狗复位原因,便于调试
\end{enumerate}

\section{I2C/I3C}

I2C(Inter-Integrated Circuit)是一种串行通信总线,广泛用于连接低速外设。I3C(Improved Inter-Integrated Circuit)是 I2C 的改进版本。

\subsection{I2C 基础}

\subsubsection{基本特性}

\begin{itemize}
  \item 两线制:SDA(数据线)和 SCL(时钟线)
  \item 多主多从:支持多个主设备和从设备
  \item 地址寻址:每个从设备有唯一地址
  \item 低速通信:标准模式 100 kHz,快速模式 400 kHz,高速模式 3.4 MHz
  \item 开漏输出:需要上拉电阻
\end{itemize}

\subsubsection{电气特性}

\begin{itemize}
  \item 电压电平:通常 3.3 V 或 5 V
  \item 上拉电阻:通常 2.2--10 k$\Omega$
  \item 总线电容:限制总线长度和设备数量
  \item 开漏输出:线与逻辑
\end{itemize}

\subsection{I2C 协议}

\subsubsection{总线状态}

\begin{enumerate}
  \item \textbf{空闲}:SDA 和 SCL 都为高电平
  \item \textbf{起始条件(START)}:SCL 为高时,SDA 从高变低
  \item \textbf{停止条件(STOP)}:SCL 为高时,SDA 从低变高
  \item \textbf{重复起始(Repeated START)}:在停止条件前发送新的起始条件
\end{enumerate}

\subsubsection{数据传输}

\begin{enumerate}
  \item 数据在 SCL 为高时有效
  \item 数据在 SCL 为低时改变
  \item 每个字节后跟随一个应答位(ACK/NACK)
  \item MSB 优先传输
\end{enumerate}

\subsubsection{帧格式}

\begin{enumerate}
  \item \textbf{起始位}
  \item \textbf{从设备地址(7 位或 10 位)}
  \item \textbf{读写位(R/W)}
  \item \textbf{应答位(ACK)}
  \item \textbf{数据字节}
  \item \textbf{应答位}
  \item \textbf{停止位}
\end{enumerate}

\subsubsection{地址格式}

\begin{enumerate}
  \item \textbf{7 位地址}
    \begin{itemize}
      \item 地址范围:0x08--0x77
      \item 0x00--0x07 和 0x78--0x7F 为保留地址
    \end{itemize}

  \item \textbf{10 位地址}
    \begin{itemize}
      \item 地址范围:0x000--0x3FF
      \item 需要两个字节传输
    \end{itemize}
\end{enumerate}

\subsection{I2C 工作模式}

\subsubsection{主设备发送}

\begin{enumerate}
  \item 发送起始条件
  \item 发送从设备地址 + 写位
  \item 等待应答
  \item 发送数据字节
  \item 等待应答
  \item 重复步骤 4--5 或发送停止条件
\end{enumerate}

\subsubsection{主设备接收}

\begin{enumerate}
  \item 发送起始条件
  \item 发送从设备地址 + 读位
  \item 等待应答
  \item 接收数据字节
  \item 发送应答(ACK)或非应答(NACK)
  \item 重复步骤 4--5 或发送停止条件
\end{enumerate}

\subsection{I2C 应用}

\begin{enumerate}
  \item 传感器接口(温度、压力、加速度等)
  \item EEPROM 接口
  \item 实时时钟(RTC)
  \item ADC/DAC 接口
  \item LCD 控制器
\end{enumerate}

\subsection{I3C 改进}

I3C 相对于 I2C 的改进:

\begin{enumerate}
  \item \textbf{更高的速度}
    \begin{itemize}
      \item 标准数据速率:12.5 MHz
      \item 高速数据速率:25 MHz
    \end{itemize}

  \item \textbf{动态地址分配}
    \begin{itemize}
      \item 支持热插拔
      \item 自动地址分配
    \end{itemize}

  \item \textbf{带内中断}
    \begin{itemize}
      \item 不需要额外的中断线
      \item 通过总线发送中断请求
    \end{itemize}

  \item \textbf{向后兼容}
    \begin{itemize}
      \item 兼容 I2C 设备
      \item 可以在同一总线上混合使用
    \end{itemize}

  \item \textbf{改进的错误检测}
    \begin{itemize}
      \item CRC 校验
      \item 更好的错误处理
    \end{itemize}
\end{enumerate}

\section{SPI/QSPI}

串行外设接口(Serial Peripheral Interface,SPI)是一种高速、全双工的同步串行通信协议。

\subsection{SPI 基础}

\subsubsection{基本特性}

\begin{itemize}
  \item 四线制:MOSI、MISO、SCLK、CS
  \item 全双工通信
  \item 主从模式
  \item 高速通信:可达几十 MHz
  \item 简单协议:没有复杂的帧格式
\end{itemize}

\subsubsection{信号线}

\begin{enumerate}
  \item \textbf{MOSI(Master Out Slave In)}
    \begin{itemize}
      \item 主设备输出,从设备输入
      \item 主设备发送数据到从设备
    \end{itemize}

  \item \textbf{MISO(Master In Slave Out)}
    \begin{itemize}
      \item 主设备输入,从设备输出
      \item 从设备发送数据到主设备
    \end{itemize}

  \item \textbf{SCLK(Serial Clock)}
    \begin{itemize}
      \item 时钟信号
      \item 由主设备产生
    \end{itemize}

  \item \textbf{CS/SS(Chip Select/Slave Select)}
    \begin{itemize}
      \item 片选信号
      \item 低电平有效
      \item 每个从设备需要独立的 CS 线
    \end{itemize}
\end{enumerate}

\subsection{SPI 协议}

\subsubsection{时钟极性和相位}

SPI 有四种工作模式,由时钟极性(CPOL)和时钟相位(CPHA)决定:

\begin{enumerate}
  \item \textbf{模式 0(CPOL=0, CPHA=0)}
    \begin{itemize}
      \item 空闲时时钟为低
      \item 数据在时钟上升沿采样,下降沿改变
    \end{itemize}

  \item \textbf{模式 1(CPOL=0, CPHA=1)}
    \begin{itemize}
      \item 空闲时时钟为低
      \item 数据在时钟下降沿采样,上升沿改变
    \end{itemize}

  \item \textbf{模式 2(CPOL=1, CPHA=0)}
    \begin{itemize}
      \item 空闲时时钟为高
      \item 数据在时钟下降沿采样,上升沿改变
    \end{itemize}

  \item \textbf{模式 3(CPOL=1, CPHA=1)}
    \begin{itemize}
      \item 空闲时时钟为高
      \item 数据在时钟上升沿采样,下降沿改变
    \end{itemize}
\end{enumerate}

\subsubsection{数据传输}

\begin{enumerate}
  \item 主设备拉低 CS 信号,选中从设备
  \item 主设备产生时钟信号
  \item 主设备通过 MOSI 发送数据,从设备通过 MISO 发送数据
  \item 数据在时钟边沿采样
  \item 传输完成后,主设备拉高 CS 信号
\end{enumerate}

\subsubsection{多从设备连接}

\begin{enumerate}
  \item \textbf{独立 CS 方式}
    \begin{itemize}
      \item 每个从设备有独立的 CS 线
      \item 主设备通过不同的 CS 选择从设备
      \item 最常用的方式
    \end{itemize}

  \item \textbf{菊花链方式}
    \begin{itemize}
      \item 多个从设备串联
      \item 数据从一个从设备传递到下一个
      \item 节省 CS 线,但速度较慢
    \end{itemize}
\end{enumerate}

\subsection{SPI 应用}

\begin{enumerate}
  \item Flash 存储器接口
  \item ADC/DAC 接口
  \item 显示控制器接口
  \item 传感器接口
  \item 无线模块接口
\end{enumerate}

\subsection{QSPI}

四线 SPI(Quad SPI,QSPI)是 SPI 的扩展,使用 4 条数据线提高传输速度。

\subsubsection{QSPI 特性}

\begin{itemize}
  \item 4 条数据线:IO0、IO1、IO2、IO3
  \item 可以是输入或输出
  \item 传输速度是标准 SPI 的 4 倍
  \item 常用于 Flash 存储器接口
\end{itemize}

\subsubsection{QSPI 模式}

\begin{enumerate}
  \item \textbf{标准 SPI 模式}
    \begin{itemize}
      \item 兼容标准 SPI
      \item IO0 作为 MOSI,IO1 作为 MISO
    \end{itemize}

  \item \textbf{双线模式}
    \begin{itemize}
      \item IO0 和 IO1 同时用于输入输出
      \item 速度是标准 SPI 的 2 倍
    \end{itemize}

  \item \textbf{四线模式}
    \begin{itemize}
      \item 所有 4 条线同时用于数据传输
      \item 速度是标准 SPI 的 4 倍
    \end{itemize}
\end{enumerate}

\section{RS232/RS422/RS485}

RS232、RS422 和 RS485 是常用的串行通信标准,用于工业控制和长距离通信。

\subsection{RS232}

\subsubsection{基本特性}

\begin{itemize}
  \item 点对点通信
  \item 单端信号
  \item 传输距离:通常 < 15 米
  \item 传输速率:通常 < 115.2 kbps
  \item 使用负逻辑:+3--+15 V 表示 0,-3--15 V 表示 1
\end{itemize}

\subsubsection{信号定义}

常用信号:
\begin{itemize}
  \item TXD:发送数据
  \item RXD:接收数据
  \item RTS:请求发送
  \item CTS:清除发送
  \item DTR:数据终端就绪
  \item DSR:数据设备就绪
  \item DCD:载波检测
  \item RI:振铃指示
  \item GND:地
\end{itemize}

\subsubsection{连接方式}

\begin{enumerate}
  \item \textbf{直连}
    \begin{itemize}
      \item 交叉连接 TXD 和 RXD
      \item 用于 DTE 和 DCE 之间
    \end{itemize}

  \item \textbf{空 Modem}
    \begin{itemize}
      \item 用于两个 DTE 之间
      \item 需要交叉连接
    \end{itemize}
\end{enumerate}

\subsection{RS422}

\subsubsection{基本特性}

\begin{itemize}
  \item 差分信号
  \item 点对点或多点通信(一主多从)
  \item 传输距离:可达 1200 米
  \item 传输速率:可达 10 Mbps(短距离)
  \item 抗干扰能力强
\end{itemize}

\subsubsection{信号定义}

\begin{itemize}
  \item A+、A-:差分信号对 A
  \item B+、B-:差分信号对 B
  \item GND:地(可选)
\end{itemize}

\subsubsection{电气特性}

\begin{itemize}
  \item 差分电压:±2--±6 V
  \item 共模电压范围:-7--+12 V
  \item 输入阻抗:≥ 4 k$\Omega$
  \item 输出电流:±150 mA
\end{itemize}

\subsection{RS485}

\subsubsection{基本特性}

\begin{itemize}
  \item 差分信号
  \item 多点通信(多主多从)
  \item 传输距离:可达 1200 米
  \item 传输速率:可达 10 Mbps(短距离)
  \item 半双工或全双工
  \item 最常用的工业总线
\end{itemize}

\subsubsection{信号定义}

\begin{itemize}
  \item A+、A-:差分信号对(半双工)
  \item D+、D-:发送差分对(全双工)
  \item R+、R-:接收差分对(全双工)
  \item GND:地
\end{itemize}

\subsubsection{网络拓扑}

\begin{enumerate}
  \item \textbf{总线型}
    \begin{itemize}
      \item 所有设备连接到同一总线
      \item 需要终端电阻(120 $\Omega$)
      \item 最常用的拓扑
    \end{itemize}

  \item \textbf{星型}
    \begin{itemize}
      \item 通过集线器连接
      \item 较少使用
    \end{itemize}
\end{enumerate}

\subsubsection{终端电阻}

\begin{itemize}
  \item 总线两端需要终端电阻
  \item 阻值:120 $\Omega$(匹配特性阻抗)
  \item 减少信号反射
\end{itemize}

\subsubsection{偏置电阻}

\begin{itemize}
  \item 确保总线空闲时的确定状态
  \item 通常使用上拉和下拉电阻
  \item 防止总线浮空
\end{itemize}

\subsubsection{RS485 应用}

\begin{enumerate}
  \item 工业自动化
  \item 楼宇自动化
  \item 数据采集系统
  \item Modbus 协议(基于 RS485)
\end{enumerate}

\section{CAN}

控制器局域网(Controller Area Network,CAN)是一种用于实时应用的串行通信协议,广泛应用于汽车和工业控制。

\subsection{CAN 基础}

\subsubsection{基本特性}

\begin{itemize}
  \item 多主总线
  \item 基于消息的协议
  \item 非破坏性仲裁
  \item 错误检测和错误处理
  \item 高可靠性
  \item 实时性好
\end{itemize}

\subsubsection{信号定义}

\begin{itemize}
  \item CANH:CAN 高线
  \item CANL:CAN 低线
  \item 差分信号
\end{itemize}

\subsection{CAN 协议}

\subsubsection{帧类型}

\begin{enumerate}
  \item \textbf{数据帧(Data Frame)}
    \begin{itemize}
      \item 携带数据
      \item 标准帧(11 位 ID)或扩展帧(29 位 ID)
    \end{itemize}

  \item \textbf{远程帧(Remote Frame)}
    \begin{itemize}
      \item 请求发送特定 ID 的数据
      \item 不携带数据
    \end{itemize}

  \item \textbf{错误帧(Error Frame)}
    \begin{itemize}
      \item 检测到错误时发送
      \item 用于错误处理
    \end{itemize}

  \item \textbf{过载帧(Overload Frame)}
    \begin{itemize}
      \item 节点需要延迟时发送
    \end{itemize}
\end{enumerate}

\subsubsection{数据帧格式}

标准数据帧包括:
\begin{enumerate}
  \item \textbf{起始位(SOF)}:1 位
  \item \textbf{仲裁场}:11 位 ID + RTR + IDE + r0
  \item \textbf{控制场}:DLC(4 位)+ r0 + r1
  \item \textbf{数据场}:0--8 字节
  \item \textbf{CRC 场}:15 位 CRC + 1 位分隔符
  \item \textbf{应答场}:ACK + 分隔符
  \item \textbf{结束场}:7 位隐性位
\end{enumerate}

\subsubsection{非破坏性仲裁}

CAN 使用非破坏性仲裁机制:
\begin{itemize}
  \item 基于 ID 的优先级
  \item ID 越小,优先级越高
  \item 显性位(0)覆盖隐性位(1)
  \item 高优先级消息自动获得总线
  \item 低优先级消息自动退出发送
\end{itemize}

\subsection{CAN 错误处理}

\subsubsection{错误类型}

\begin{enumerate}
  \item \textbf{位错误}
  \item \textbf{填充错误}
  \item \textbf{CRC 错误}
  \item \textbf{格式错误}
  \item \textbf{应答错误}
\end{enumerate}

\subsubsection{错误状态}

每个节点有三种错误状态:
\begin{enumerate}
  \item \textbf{主动错误(Error Active)}
    \begin{itemize}
      \item 正常状态
      \item 可以正常发送和接收
    \end{itemize}

  \item \textbf{被动错误(Error Passive)}
    \begin{itemize}
      \item 错误计数较高
      \item 发送时增加延迟
    \end{itemize}

  \item \textbf{总线关闭(Bus Off)}
    \begin{itemize}
      \item 错误计数过高
      \item 不能发送和接收
      \item 需要复位才能恢复
    \end{itemize}
\end{enumerate}

\subsection{CAN 应用}

\begin{enumerate}
  \item 汽车电子(ECU 通信)
  \item 工业自动化
  \item 医疗设备
  \item 航空航天
\end{enumerate}

\section{Ethernet}

以太网(Ethernet)是最常用的局域网技术,也广泛应用于嵌入式系统。

\subsection{Ethernet 基础}

\subsubsection{基本特性}

\begin{itemize}
  \item 基于 CSMA/CD(载波监听多路访问/冲突检测)
  \item 星型拓扑(使用交换机)
  \item 高速通信:10 Mbps、100 Mbps、1 Gbps、10 Gbps
  \item 标准化:IEEE 802.3
\end{itemize}

\subsubsection{物理层}

\begin{enumerate}
  \item \textbf{10BASE-T}
    \begin{itemize}
      \item 10 Mbps
      \item 双绞线
      \item 100 米传输距离
    \end{itemize}

  \item \textbf{100BASE-TX}
    \begin{itemize}
      \item 100 Mbps(Fast Ethernet)
      \item 双绞线
      \item 100 米传输距离
    \end{itemize}

  \item \textbf{1000BASE-T}
    \begin{itemize}
      \item 1 Gbps(Gigabit Ethernet)
      \item 双绞线
      \item 100 米传输距离
    \end{itemize}
\end{enumerate}

\subsection{Ethernet 帧格式}

以太网帧包括:
\begin{enumerate}
  \item \textbf{前导码(Preamble)}:7 字节
  \item \textbf{帧起始定界符(SFD)}:1 字节
  \item \textbf{目标 MAC 地址}:6 字节
  \item \textbf{源 MAC 地址}:6 字节
  \item \textbf{类型/长度}:2 字节
  \item \textbf{数据}:46--1500 字节
  \item \textbf{帧校验序列(FCS)}:4 字节 CRC
\end{enumerate}

\subsection{嵌入式 Ethernet}

\subsubsection{以太网控制器}

嵌入式系统通常使用以太网控制器芯片:
\begin{itemize}
  \item 集成 MAC 和 PHY
  \item 或分离的 MAC 和 PHY
  \item 通过 SPI 或并行接口连接
\end{itemize}

\subsubsection{TCP/IP 协议栈}

嵌入式系统需要实现 TCP/IP 协议栈:
\begin{itemize}
  \item 可以使用轻量级协议栈(如 lwIP)
  \item 或使用硬件加速的协议栈
\end{itemize}

\section{USB}

通用串行总线(Universal Serial Bus,USB)是最常用的计算机外设接口。

\subsection{USB 基础}

\subsubsection{基本特性}

\begin{itemize}
  \item 即插即用
  \item 热插拔
  \item 电源供电(某些设备)
  \item 高速通信
  \item 标准化接口
\end{itemize}

\subsubsection{USB 版本}

\begin{enumerate}
  \item \textbf{USB 1.0/1.1}
    \begin{itemize}
      \item 低速:1.5 Mbps
      \item 全速:12 Mbps
    \end{itemize}

  \item \textbf{USB 2.0}
    \begin{itemize}
      \item 高速:480 Mbps
      \item 向后兼容 USB 1.x
    \end{itemize}

  \item \textbf{USB 3.0/3.1/3.2}
    \begin{itemize}
      \item SuperSpeed:5 Gbps(USB 3.0)
      \item SuperSpeed+:10 Gbps(USB 3.1 Gen 2)
      \item SuperSpeed+:20 Gbps(USB 3.2 Gen 2x2)
    \end{itemize}

  \item \textbf{USB 4}
    \begin{itemize}
      \item 最高 40 Gbps
      \item 基于 Thunderbolt 3
    \end{itemize}
\end{enumerate}

\subsubsection{USB 连接器}

\begin{enumerate}
  \item \textbf{Type-A}:标准 USB 接口
  \item \textbf{Type-B}:设备端接口
  \item \textbf{Type-C}:可逆接口,支持 USB 3.1+ 和 Power Delivery
  \item \textbf{Mini-USB}:小型设备
  \item \textbf{Micro-USB}:更小型设备
\end{enumerate}

\subsection{USB 架构}

\subsubsection{拓扑结构}

USB 采用树形拓扑:
\begin{itemize}
  \item 主机(Host):根节点
  \item 集线器(Hub):扩展节点
  \item 设备(Device):叶子节点
  \item 最多 127 个设备
  \item 最多 5 层(不包括根)
\end{itemize}

\subsubsection{通信模型}

USB 采用分层通信模型:
\begin{enumerate}
  \item \textbf{物理层}:电气和机械规范
  \item \textbf{数据链路层}:数据包传输
  \item \textbf{协议层}:USB 协议
  \item \textbf{应用层}:设备功能
\end{enumerate}

\subsection{USB 传输类型}

\begin{enumerate}
  \item \textbf{控制传输(Control)}
    \begin{itemize}
      \item 设备配置和状态查询
      \item 双向传输
    \end{itemize}

  \item \textbf{中断传输(Interrupt)}
    \begin{itemize}
      \item 周期性数据传输
      \item 用于键盘、鼠标等
    \end{itemize}

  \item \textbf{批量传输(Bulk)}
    \begin{itemize}
      \item 大量数据传输
      \item 用于存储设备等
    \end{itemize}

  \item \textbf{等时传输(Isochronous)}
    \begin{itemize}
      \item 实时数据传输
      \item 用于音频、视频等
    \end{itemize}
\end{enumerate}

\subsection{USB 在嵌入式系统中的应用}

\begin{enumerate}
  \item USB 设备(Device)
    \begin{itemize}
      \item 实现 USB 设备功能
      \item 需要 USB 设备控制器
    \end{itemize}

  \item USB 主机(Host)
    \begin{itemize}
      \item 连接 USB 设备
      \item 需要 USB 主机控制器
    \end{itemize}

  \item USB OTG(On-The-Go)
    \begin{itemize}
      \item 既可以作为主机,也可以作为设备
      \item 用于移动设备
    \end{itemize}
\end{enumerate}

\section{PCI/PCIe}

PCI(Peripheral Component Interconnect)和 PCIe(PCI Express)是计算机扩展总线标准。

\subsection{PCI}

\subsubsection{基本特性}

\begin{itemize}
  \item 并行总线
  \item 32 位或 64 位数据宽度
  \item 33 MHz 或 66 MHz 时钟
  \item 共享总线架构
  \item 已逐渐被 PCIe 取代
\end{itemize}

\subsubsection{配置空间}

PCI 设备有 256 字节的配置空间,用于设备识别和配置。

\Figure[caption={PCI 配置空间}, label={fig:pci-config}, width=0.95]{PCI-config-space}

配置空间包括:
\begin{itemize}
  \item 设备 ID 和厂商 ID
  \item 状态和命令寄存器
  \item 基址寄存器(BAR)
  \item 中断线
\end{itemize}

\subsection{PCIe}

\subsubsection{基本特性}

\begin{itemize}
  \item 串行点对点连接
  \item 高速通信:每通道 2.5 Gbps(Gen 1)到 32 Gbps(Gen 5)
  \item 可扩展:x1、x2、x4、x8、x16 通道
  \item 全双工通信
  \item 向后兼容 PCI 软件
\end{itemize}

\subsubsection{PCIe 架构}

\begin{enumerate}
  \item \textbf{物理层}
    \begin{itemize}
      \item 差分信号对
      \item 8b/10b 或 128b/130b 编码
      \item 时钟恢复
    \end{itemize}

  \item \textbf{数据链路层}
    \begin{itemize}
      \item 数据包序列化
      \item 错误检测和重传
      \item 流控制
    \end{itemize}

  \item \textbf{事务层}
    \begin{itemize}
      \item 事务处理
      \item 虚拟通道
      \item 服务质量(QoS)
    \end{itemize}
\end{enumerate}

\subsubsection{PCIe 链路}

\begin{itemize}
  \item 每个链路由多个通道组成
  \item 每个通道是双向的
  \item 通道数:x1、x2、x4、x8、x16
  \item 总带宽 = 单通道带宽 × 通道数
\end{itemize}

\subsubsection{PCIe 版本}

\begin{enumerate}
  \item \textbf{PCIe 1.0}
    \begin{itemize}
      \item 每通道 2.5 Gbps
      \item 编码效率 80\%
      \item 有效带宽 2 Gbps/通道
    \end{itemize}

  \item \textbf{PCIe 2.0}
    \begin{itemize}
      \item 每通道 5 Gbps
      \item 有效带宽 4 Gbps/通道
    \end{itemize}

  \item \textbf{PCIe 3.0}
    \begin{itemize}
      \item 每通道 8 Gbps
      \item 128b/130b 编码
      \item 有效带宽 7.877 Gbps/通道
    \end{itemize}

  \item \textbf{PCIe 4.0}
    \begin{itemize}
      \item 每通道 16 Gbps
      \item 有效带宽 15.754 Gbps/通道
    \end{itemize}

  \item \textbf{PCIe 5.0}
    \begin{itemize}
      \item 每通道 32 Gbps
      \item 有效带宽 31.508 Gbps/通道
    \end{itemize}
\end{enumerate}

\subsubsection{PCIe 应用}

\begin{enumerate}
  \item 显卡接口
  \item 高速网卡
  \item 存储控制器
  \item 高速数据采集卡
  \item 嵌入式系统扩展
\end{enumerate}

\section{协议选择指南}

\subsection{选择考虑因素}

\begin{enumerate}
  \item \textbf{速度要求}
    \begin{itemize}
      \item 低速:UART、I2C
      \item 中速:SPI、CAN
      \item 高速:USB、PCIe、Ethernet
    \end{itemize}

  \item \textbf{距离要求}
    \begin{itemize}
      \item 板内:SPI、I2C
      \item 短距离:UART、USB
      \item 长距离:RS485、CAN、Ethernet
    \end{itemize}

  \item \textbf{设备数量}
    \begin{itemize}
      \item 点对点:UART、SPI
      \item 多点:I2C、CAN、RS485
      \item 网络:Ethernet
    \end{itemize}

  \item \textbf{成本}
    \begin{itemize}
      \item 低成本:UART、I2C、SPI
      \item 中等成本:CAN、RS485
      \item 高成本:USB、PCIe、Ethernet
    \end{itemize}

  \item \textbf{可靠性}
    \begin{itemize}
      \item 高可靠性:CAN、RS485
      \item 一般可靠性:UART、I2C、SPI
    \end{itemize}

  \item \textbf{实时性}
    \begin{itemize}
      \item 实时:CAN
      \item 准实时:Ethernet(带 QoS)
      \item 非实时:UART、I2C、SPI
    \end{itemize}
\end{enumerate}

\subsection{协议比较}

\begin{table}[H]
  \centering
  \caption{通信协议比较}
  \begin{tabular}{|l|l|l|l|l|}
    \hline
    协议 & 速度 & 距离 & 设备数 & 主要应用 \\
    \hline
    UART & 低 & 短 & 2 & 调试、简单通信 \\
    I2C & 低 & 短 & 多 & 传感器、低速外设 \\
    SPI & 中 & 短 & 多 & Flash、高速外设 \\
    RS485 & 中 & 长 & 多 & 工业控制 \\
    CAN & 中 & 中 & 多 & 汽车、工业 \\
    USB & 高 & 短 & 127 & 计算机外设 \\
    Ethernet & 高 & 长 & 多 & 网络通信 \\
    PCIe & 很高 & 短 & 点对点 & 高速扩展 \\
    \hline
  \end{tabular}
\end{table}

\section{总结}

外设和通信协议是嵌入式系统的重要组成部分,选择合适的协议对于系统性能、可靠性和成本至关重要。本章介绍了常用的串行通信协议、工业总线和高速接口,每种协议都有其特定的应用场景和优势。在实际应用中,需要根据具体需求选择合适的协议,并注意协议实现的细节和最佳实践。
