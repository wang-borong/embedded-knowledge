\chapter{运算放大器及反馈系统}

运算放大器(Operational Amplifier,Op-Amp)是模拟电路中最重要、应用最广泛的集成电路之一。运算放大器具有高增益、高输入阻抗、低输出阻抗等特点,配合反馈网络可以实现各种信号处理功能,如放大、滤波、积分、微分、比较等。本章将系统介绍运算放大器的工作原理、反馈理论、基本电路和应用技巧。

\section{运算放大器}

\subsection{运算放大器的基本概念}

运算放大器是一种高增益的差分放大器,通常具有两个输入端(同相输入端和反相输入端)和一个输出端。理想运算放大器具有以下特性:

\begin{itemize}
  \item 开环电压增益 $A_{OL} \to \infty$
  \item 输入阻抗 $R_{in} \to \infty$
  \item 输出阻抗 $R_{out} \to 0$
  \item 带宽 $BW \to \infty$
  \item 共模抑制比 $CMRR \to \infty$
  \item 输入失调电压 $V_{OS} = 0$
  \item 输入失调电流 $I_{OS} = 0$
  \item 输入偏置电流 $I_B = 0$
\end{itemize}

\subsection{运算放大器的符号与端子}

运算放大器的电路符号如图~\ref{fig:opamp-symbol} 所示。

\begin{figure}[H]
  \centering
  \begin{circuitikz}[american,scale=1.0]
    \draw
    (0,0) node[op amp] (opamp) {}
    (opamp.+) node[left] {$V_+$}
    (opamp.-) node[left] {$V_-$}
    (opamp.out) node[right] {$V_{out}$}
    (opamp.up) node[above=0.5cm] {$V_{CC+}$}
    (opamp.down) node[below=0.5cm] {$V_{CC-}$};

    \node[left=1cm,above=1.2cm] at (opamp.+) {同相输入端};
    \node[left=1cm,below=1.2cm] at (opamp.-) {反相输入端};
    \node[right=1cm] at (opamp.out) {输出端};
  \end{circuitikz}
  \caption{运算放大器电路符号}\label{fig:opamp-symbol}
\end{figure}

主要端子包括:
\begin{itemize}
  \item \textbf{同相输入端($V_+$)}:输入信号与输出信号同相
  \item \textbf{反相输入端($V_-$)}:输入信号与输出信号反相
  \item \textbf{输出端($V_{out}$)}:输出信号
  \item \textbf{正电源($V_{CC+}$)}:正电源电压
  \item \textbf{负电源($V_{CC-}$)}:负电源电压(或地)
\end{itemize}

\subsection{理想运算放大器的特性}

理想运算放大器遵循两个基本规则:

\subsubsection{虚短(Virtual Short)}

由于开环增益无穷大,当运算放大器工作在线性区时,两个输入端之间的电压差为零:

\begin{equation}
V_+ - V_- = 0 \quad \text{或} \quad V_+ = V_- \label{eq:virtual-short}
\end{equation}

这称为"虚短",意味着两个输入端可以视为短路,但实际上没有电流流过。

\subsubsection{虚断(Virtual Open)}

由于输入阻抗无穷大,流入两个输入端的电流为零:

\begin{equation}
I_+ = I_- = 0 \label{eq:virtual-open}
\end{equation}

这称为"虚断",意味着两个输入端可以视为开路,但实际上有电压存在。

\subsection{实际运算放大器的特性}

实际运算放大器与理想模型存在差异,主要参数包括:

\subsubsection{开环增益}

实际运算放大器的开环增益 $A_{OL}$ 是有限的,典型值在 $10^4$--$10^6$(80--120 dB)之间。开环增益通常随频率下降,可以用单极点模型近似:

\begin{equation}
A_{OL}(s) = \frac{A_{OL0}}{1 + s/\omega_p} \label{eq:opamp-open-loop-gain}
\end{equation}

其中 $A_{OL0}$ 是直流开环增益,$\omega_p$ 是主极点频率。

\subsubsection{输入阻抗}

实际运算放大器的输入阻抗是有限的:
\begin{itemize}
  \item 双极型输入:输入阻抗较低,典型值 $10^6$--$10^9$ $\Omega$
  \item CMOS 输入:输入阻抗很高,典型值 $10^{12}$--$10^{15}$ $\Omega$
\end{itemize}

\subsubsection{输出阻抗}

实际运算放大器的输出阻抗不为零,典型值在 10--1000 $\Omega$ 之间。在负反馈作用下,输出阻抗会显著降低。

\subsubsection{输入失调电压}

输入失调电压 $V_{OS}$ 是指使输出电压为零时,需要在输入端施加的电压差。典型值在 0.1--10 mV 之间。失调电压会随温度漂移,温度系数通常为 1--10 $\mu$V/$^\circ$C。

\subsubsection{输入偏置电流和失调电流}

输入偏置电流 $I_B$ 是流入两个输入端的平均电流,典型值在 1 pA--1 $\mu$A 之间。

输入失调电流 $I_{OS}$ 是两个输入端偏置电流的差值:

\begin{equation}
I_{OS} = |I_{B+} - I_{B-}| \label{eq:input-offset-current}
\end{equation}

\subsubsection{共模抑制比}

共模抑制比(Common-Mode Rejection Ratio,CMRR)定义为:

\begin{equation}
CMRR = \frac{A_{DM}}{A_{CM}} = 20\log\left(\frac{A_{DM}}{A_{CM}}\right) \text{ dB} \label{eq:cmrr}
\end{equation}

其中 $A_{DM}$ 是差模增益,$A_{CM}$ 是共模增益。CMRR 典型值在 60--120 dB 之间。

\subsubsection{电源抑制比}

电源抑制比(Power Supply Rejection Ratio,PSRR)定义为:

\begin{equation}
PSRR = 20\log\left(\frac{A_{OL}}{\Delta V_{out}/\Delta V_{supply}}\right) \text{ dB} \label{eq:psrr}
\end{equation}

PSRR 表示运算放大器对电源电压变化的抑制能力,典型值在 60--100 dB 之间。

\subsubsection{带宽和压摆率}

\paragraph{单位增益带宽(Unity Gain Bandwidth,$f_{unity}$)}

单位增益带宽是开环增益降为 1(0 dB)时的频率。对于单极点模型:

\begin{equation}
f_{unity} = A_{OL0} \cdot f_p \label{eq:unity-gain-bandwidth}
\end{equation}

\paragraph{增益带宽积(Gain-Bandwidth Product,GBP)}

增益带宽积是增益与带宽的乘积,对于单极点模型,GBP 为常数:

\begin{equation}
GBP = A_{OL} \cdot f_{-3dB} = \text{常数} \label{eq:gain-bandwidth-product}
\end{equation}

\paragraph{压摆率(Slew Rate,SR)}

压摆率是输出电压的最大变化速率:

\begin{equation}
SR = \max\left|\frac{dV_{out}}{dt}\right| \label{eq:slew-rate}
\end{equation}

压摆率限制了运算放大器处理大信号的能力,典型值在 0.1--1000 V/$\mu$s 之间。

\subsubsection{输入电压范围}

输入电压范围是指运算放大器正常工作时的输入电压范围:
\begin{itemize}
  \item \textbf{轨到轨输入(Rail-to-Rail Input)}:输入电压可以接近电源电压
  \item \textbf{非轨到轨输入}:输入电压范围受限,通常比电源电压小 1--3 V
\end{itemize}

\subsubsection{输出电压范围}

输出电压范围是指运算放大器能够输出的电压范围:
\begin{itemize}
  \item \textbf{轨到轨输出(Rail-to-Rail Output)}:输出电压可以接近电源电压
  \item \textbf{非轨到轨输出}:输出电压范围受限,通常比电源电压小 1--3 V
\end{itemize}

\subsection{运算放大器的内部结构}

运算放大器通常由以下部分组成:

\begin{enumerate}
  \item \textbf{输入级}:差分放大器,提供高输入阻抗和共模抑制
  \item \textbf{中间级}:高增益放大器,提供主要的电压放大
  \item \textbf{输出级}:功率放大器,提供低输出阻抗和足够的驱动能力
  \item \textbf{偏置电路}:为各级提供稳定的偏置电流
\end{enumerate}

\section{反馈系统}

反馈是运算放大器应用的基础。通过引入反馈,可以控制放大器的增益、带宽、输入输出阻抗等特性,并提高稳定性。

\subsection{反馈的基本概念}

反馈是指将输出信号的一部分或全部返回到输入端,与输入信号进行比较的过程。

\subsubsection{反馈的分类}

根据反馈信号与输入信号的关系,反馈分为:

\begin{enumerate}
  \item \textbf{负反馈(Negative Feedback)}
    \begin{itemize}
      \item 反馈信号与输入信号反相
      \item 减小增益,提高稳定性
      \item 扩展带宽,减小失真
      \item 改变输入输出阻抗
    \end{itemize}

  \item \textbf{正反馈(Positive Feedback)}
    \begin{itemize}
      \item 反馈信号与输入信号同相
      \item 增大增益,可能引起振荡
      \item 用于振荡器和比较器
    \end{itemize}
\end{enumerate}

\subsubsection{反馈的采样方式}

根据反馈信号的采样方式,分为:

\begin{enumerate}
  \item \textbf{电压反馈(Voltage Feedback)}:反馈信号采样输出电压
  \item \textbf{电流反馈(Current Feedback)}:反馈信号采样输出电流
\end{enumerate}

\subsubsection{反馈的混合方式}

根据反馈信号与输入信号的混合方式,分为:

\begin{enumerate}
  \item \textbf{串联反馈(Series Feedback)}:反馈信号与输入信号串联(电压叠加)
  \item \textbf{并联反馈(Shunt Feedback)}:反馈信号与输入信号并联(电流叠加)
\end{enumerate}

\subsection{负反馈的基本理论}

\subsubsection{反馈系统的框图}

负反馈系统的基本框图如图~\ref{fig:feedback-block} 所示。

\begin{figure}[H]
  \centering
  \begin{circuitikz}[american,scale=1.0]
    % Styles
    \tikzstyle{block} = [draw, rectangle, minimum height=3em, minimum width=3em]
    \tikzstyle{sum} = [draw, circle, node distance=1cm]
    \tikzstyle{input} = [coordinate]
    \tikzstyle{output} = [coordinate]
    \tikzstyle{pinstyle} = [pin edge={to-,thin,black}]

    % Nodes
    \node [input, name=input] {};
    \node [sum, right of=input] (sum) {};
    \node [block, right of=sum] (system) {$A$};
    \node [output, right of=system] (output) {};
    \node [block, below of=system] (feedback) {$\beta$};

    % Connections
    \draw [->] (input) -- node {$V_{in}$} node[pos=0.9] {$+$} (sum);
    \draw [->] (sum) -- node {$V_e$} (system);
    \draw [->] (system) -- node [name=y] {$V_{out}$}(output);
    \draw [->] (y) |- (feedback);
    \draw [->] (feedback) -| node[pos=0.99] {$-$} node [near end] {$V_f$} (sum);
  \end{circuitikz}
  \caption{负反馈系统框图}\label{fig:feedback-block}
\end{figure}

其中:
\begin{itemize}
  \item $A$:前向增益(开环增益)
  \item $\beta$:反馈系数
  \item $V_{in}$:输入信号
  \item $V_{out}$:输出信号
  \item $V_f = \beta V_{out}$:反馈信号
  \item $V_e = V_{in} - V_f$:误差信号
\end{itemize}

\subsubsection{闭环增益}

闭环增益 $A_{CL}$ 为:

\begin{equation}
A_{CL} = \frac{V_{out}}{V_{in}} = \frac{A}{1 + A\beta} \label{eq:closed-loop-gain}
\end{equation}

当 $A\beta \gg 1$ 时(深度负反馈):

\begin{equation}
A_{CL} \approx \frac{1}{\beta} \label{eq:closed-loop-gain-approx}
\end{equation}

这表明在深度负反馈条件下,闭环增益主要由反馈网络决定,与开环增益 $A$ 无关。

\subsubsection{反馈深度}

反馈深度 $1 + A\beta$ 是衡量反馈强度的指标:
\begin{itemize}
  \item 反馈深度越大,负反馈作用越强
  \item 当 $A\beta \gg 1$ 时,称为深度负反馈
\end{itemize}

\subsubsection{环路增益}

环路增益 $T = A\beta$ 是反馈系统的重要参数,决定了系统的稳定性和性能。

\subsection{负反馈对放大器性能的影响}

\subsubsection{增益稳定性}

负反馈提高了增益的稳定性。增益的相对变化为:

\begin{equation}
\frac{dA_{CL}}{A_{CL}} = \frac{1}{1 + A\beta} \cdot \frac{dA}{A} \label{eq:gain-stability}
\end{equation}

可见,增益的相对变化减小了 $1 + A\beta$ 倍。

\subsubsection{带宽扩展}

负反馈扩展了放大器的带宽。对于单极点系统,闭环带宽为:

\begin{equation}
f_{CL} = f_{OL} \cdot (1 + A\beta) \label{eq:bandwidth-extension}
\end{equation}

其中 $f_{OL}$ 是开环带宽。

\subsubsection{输入阻抗的变化}

\begin{itemize}
  \item \textbf{串联反馈}:输入阻抗增加 $1 + A\beta$ 倍
  \item \textbf{并联反馈}:输入阻抗减小 $1 + A\beta$ 倍
\end{itemize}

\subsubsection{输出阻抗的变化}

\begin{itemize}
  \item \textbf{电压反馈}:输出阻抗减小 $1 + A\beta$ 倍
  \item \textbf{电流反馈}:输出阻抗增加 $1 + A\beta$ 倍
\end{itemize}

\subsubsection{非线性失真的减小}

负反馈可以减小非线性失真,失真减小 $1 + A\beta$ 倍。

\subsubsection{噪声的影响}

负反馈不能减小输入信号中的噪声,但可以减小放大器内部产生的噪声。

\subsection{反馈的稳定性}

\subsubsection{稳定性判据}

反馈系统稳定的条件是:当环路增益 $T(s) = A(s)\beta(s)$ 的相位为 $-180^\circ$ 时,其幅度必须小于 1。

用数学表达:

\begin{equation}
|T(j\omega_{180})| < 1 \label{eq:stability-criterion}
\end{equation}

其中 $\omega_{180}$ 是相位为 $-180^\circ$ 时的频率。

\subsubsection{相位裕度}

相位裕度(Phase Margin,PM)定义为:

\begin{equation}
PM = 180^\circ + \angle T(j\omega_0) \label{eq:phase-margin}
\end{equation}

其中 $\omega_0$ 是 $|T(j\omega)| = 1$ 时的频率(增益穿越频率)。

相位裕度越大,系统越稳定。通常要求 $PM > 45^\circ$,最好 $PM > 60^\circ$。

\subsubsection{增益裕度}

增益裕度(Gain Margin,GM)定义为:

\begin{equation}
GM = -20\log|T(j\omega_{180})| \text{ dB} \label{eq:gain-margin}
\end{equation}

增益裕度越大,系统越稳定。通常要求 $GM > 10$ dB。

\subsubsection{补偿技术}

当反馈系统不稳定时,需要进行补偿:

\begin{enumerate}
  \item \textbf{主极点补偿}:在反馈网络中引入主极点,降低高频增益
  \item \textbf{零点补偿}:引入零点,改善相位特性
  \item \textbf{米勒补偿}:利用米勒效应进行补偿
\end{enumerate}

\section{运算放大器的性能指标与选型}

\subsection{主要性能指标}

选择运算放大器时,需要考虑以下主要指标:

\begin{enumerate}
  \item \textbf{输入特性}
    \begin{itemize}
      \item 输入失调电压 $V_{OS}$ 及其温度系数
      \item 输入偏置电流 $I_B$ 和失调电流 $I_{OS}$
      \item 输入阻抗 $R_{in}$
      \item 输入电压范围
      \item 共模抑制比 CMRR
    \end{itemize}

  \item \textbf{输出特性}
    \begin{itemize}
      \item 输出电压范围
      \item 输出电流能力
      \item 输出阻抗
    \end{itemize}

  \item \textbf{动态特性}
    \begin{itemize}
      \item 开环增益 $A_{OL}$
      \item 单位增益带宽 $f_{unity}$
      \item 压摆率 SR
      \item 建立时间(Settling Time)
    \end{itemize}

  \item \textbf{电源特性}
    \begin{itemize}
      \item 电源电压范围
      \item 静态电流 $I_Q$
      \item 电源抑制比 PSRR
    \end{itemize}

  \item \textbf{环境特性}
    \begin{itemize}
      \item 工作温度范围
      \item 温度漂移
    \end{itemize}
\end{enumerate}

\subsection{运算放大器的分类}

根据应用需求,运算放大器可分为:

\subsubsection{通用型运算放大器}

适用于一般应用,性能平衡,价格低廉。典型型号:LM358、LM324、TL072 等。

\subsubsection{精密运算放大器}

具有低失调电压、低漂移、高精度等特点。典型型号:OP07、OP27、AD8628 等。

\subsubsection{高速运算放大器}

具有高带宽、高压摆率等特点。典型型号:AD811、LM6171 等。

\subsubsection{低功耗运算放大器}

静态电流很小,适用于电池供电应用。典型型号:LMC6482、MCP6002 等。

\subsubsection{轨到轨运算放大器}

输入和/或输出可以接近电源电压。典型型号:LMV358、MCP6002 等。

\subsubsection{高电压运算放大器}

可以工作在较高的电源电压下。典型型号:OPA445、PA85 等。

\subsubsection{电流反馈运算放大器}

采用电流反馈结构,具有高带宽、高压摆率等特点。典型型号:AD811、AD8001 等。

\subsection{选型指南}

选择运算放大器时,应遵循以下步骤:

\begin{enumerate}
  \item \textbf{确定应用需求}
    \begin{itemize}
      \item 信号类型(直流、交流、频率范围)
      \item 精度要求
      \item 速度要求
      \item 功耗限制
      \item 电源电压
      \item 成本限制
    \end{itemize}

  \item \textbf{初步筛选}
    \begin{itemize}
      \item 根据应用类型选择运算放大器类别
      \item 检查电源电压范围
      \item 检查输入输出范围
    \end{itemize}

  \item \textbf{详细评估}
    \begin{itemize}
      \item 检查关键性能指标是否满足要求
      \item 考虑温度影响
      \item 评估成本
    \end{itemize}

  \item \textbf{验证}
    \begin{itemize}
      \item 查阅数据手册
      \item 进行仿真验证
      \item 制作样机测试
    \end{itemize}
\end{enumerate}

\section{运算放大器的基本电路}

\subsection{同相放大器}

同相放大器(Non-Inverting Amplifier)的电路如图~\ref{fig:non-inverting-amp} 所示。

\begin{figure}[H]
  \centering
  \begin{circuitikz}[american,scale=1.0]
    \draw
    (0,0) node[op amp] (opamp) {}
    (opamp.+) to[short] ++(-1,0) to[short, -o] ++(-0.5,0) node[left] {$V_{in}$}
    (opamp.out) to[short, *-o] ++(1,0) node[right] {$V_{out}$}
    (opamp.out) to[R=$R_2$] ++(0,-2) coordinate (fb)
    (fb) to[R=$R_1$] ++(-2,0) coordinate (gnd_node) node[ground] {}
    (opamp.-) to[short] (opamp.- |- fb) to[short] (fb);
    \draw (fb) to[short, *-] (fb); % Ensure dot at feedback connection
  \end{circuitikz}
  \caption{同相放大器}\label{fig:non-inverting-amp}
\end{figure}

根据虚短和虚断,可以推导出:

\begin{equation}
V_{out} = \left(1 + \frac{R_2}{R_1}\right) V_{in} \label{eq:non-inverting-gain}
\end{equation}

闭环增益为:

\begin{equation}
A_{CL} = 1 + \frac{R_2}{R_1} \label{eq:non-inverting-closed-loop}
\end{equation}

特点:
\begin{itemize}
  \item 输入阻抗高(接近运算放大器的输入阻抗)
  \item 输出阻抗低
  \item 输入与输出同相
  \item 增益总是大于等于 1
\end{itemize}

\subsection{反相放大器}

反相放大器(Inverting Amplifier)的电路如图~\ref{fig:inverting-amp} 所示。

\begin{figure}[H]
  \centering
  \begin{circuitikz}[american,scale=1.0]
    \draw
    (0,0) node[op amp] (opamp) {}
    (opamp.+) node[ground] {}
    (opamp.-) to[short] ++(-0.5,0) coordinate (in_node)
    (in_node) to[R=$R_1$, -o] ++(-2,0) node[left] {$V_{in}$}
    (in_node) to[short, *-] ++(0,1) coordinate (fb_node)
    (opamp.out) to[short, *-o] ++(1,0) node[right] {$V_{out}$}
    (opamp.out) to[short] ++(0,1) coordinate (out_fb)
    (out_fb) to[R=$R_2$] (fb_node);
  \end{circuitikz}
  \caption{反相放大器}\label{fig:inverting-amp}
\end{figure}

根据虚短和虚断,可以推导出:

\begin{equation}
V_{out} = -\frac{R_2}{R_1} V_{in} \label{eq:inverting-gain}
\end{equation}

闭环增益为:

\begin{equation}
A_{CL} = -\frac{R_2}{R_1} \label{eq:inverting-closed-loop}
\end{equation}

特点:
\begin{itemize}
  \item 输入阻抗等于 $R_1$(相对较低)
  \item 输出阻抗低
  \item 输入与输出反相
  \item 增益可以小于 1
\end{itemize}

\subsection{差分放大器}

差分放大器(Differential Amplifier)的电路如图~\ref{fig:differential-amp} 所示。

\begin{figure}[H]
  \centering
  \begin{circuitikz}[american,scale=1.0]
    \draw
    (0,0) node[op amp] (opamp) {}
    (opamp.+) to[short] ++(-0.5,0) coordinate (p_node)
    (p_node) to[R=$R_3$, -o] ++(-2,0) node[left] {$V_+$}
    (p_node) to[R=$R_4$, *-] ++(0,-2) node[ground] {}
    (opamp.-) to[short] ++(-0.5,0) coordinate (n_node)
    (n_node) to[R=$R_1$, -o] ++(-2,0) node[left] {$V_-$}
    (n_node) to[short, *-] ++(0,1) coordinate (fb_node)
    (opamp.out) to[short, *-o] ++(1,0) node[right] {$V_{out}$}
    (opamp.out) to[short] ++(0,1) coordinate (out_fb)
    (out_fb) to[R=$R_2$] (fb_node);
  \end{circuitikz}
  \caption{差分放大器}\label{fig:differential-amp}
\end{figure}

当 $R_1 = R_3$ 且 $R_2 = R_4$ 时,输出电压为:

\begin{equation}
V_{out} = \frac{R_2}{R_1} (V_+ - V_-) \label{eq:differential-gain}
\end{equation}

差分放大器可以放大两个输入信号的差值,抑制共模信号。

\subsection{求和放大器}

求和放大器(Summing Amplifier)的电路如图~\ref{fig:summing-amp} 所示。

\begin{figure}[H]
  \centering
  \begin{circuitikz}[american,scale=1.0]
    \draw
    (0,0) node[op amp] (opamp) {}
    (opamp.+) node[ground] {}
    (opamp.-) to[short] ++(-0.5,0) coordinate (sum_node)
    (sum_node) to[short, *-] ++(0,1) coordinate (fb_node)
    (sum_node) to[R=$R_1$, -o] ++(-2,0) node[left] {$V_1$}
    (sum_node) to[short] ++(0,-1) coordinate (in2_node)
    (in2_node) to[R=$R_2$, -o] ++(-2,0) node[left] {$V_2$}
    (sum_node) to[short] (in2_node)
    (opamp.out) to[short, *-o] ++(1,0) node[right] {$V_{out}$}
    (opamp.out) to[short] ++(0,1) coordinate (out_fb)
    (out_fb) to[R=$R_f$] (fb_node);
  \end{circuitikz}
  \caption{求和放大器}\label{fig:summing-amp}
\end{figure}

当所有输入电阻相等($R_1 = R_2 = \cdots = R_n$)时:

\begin{equation}
V_{out} = -\frac{R_f}{R_1} (V_1 + V_2 + \cdots + V_n) \label{eq:summing-gain}
\end{equation}

\subsection{积分器}

积分器(Integrator)的电路如图~\ref{fig:integrator} 所示。

\begin{figure}[H]
  \centering
  \begin{circuitikz}[american,scale=1.0]
    \draw
    (0,0) node[op amp] (opamp) {}
    (opamp.+) node[ground] {}
    (opamp.-) to[short] ++(-0.5,0) coordinate (in_node)
    (in_node) to[C=$C$, -o] ++(-2,0) node[left] {$V_{in}$}
    (in_node) to[short, *-] ++(0,1) coordinate (fb_node)
    (opamp.out) to[short, *-o] ++(1,0) node[right] {$V_{out}$}
    (opamp.out) to[short] ++(0,1) coordinate (out_fb)
    (out_fb) to[R=$R$] (fb_node);
  \end{circuitikz}
  \caption{积分器}\label{fig:integrator}
\end{figure}

输出电压为:

\begin{equation}
V_{out}(t) = -\frac{1}{RC} \int_0^t V_{in}(\tau) d\tau + V_{out}(0) \label{eq:integrator}
\end{equation}

在频域中,传递函数为:

\begin{equation}
H(s) = -\frac{1}{RCs} \label{eq:integrator-transfer}
\end{equation}

积分器常用于:
\begin{itemize}
  \item 波形变换(如方波变三角波)
  \item 低通滤波
  \item 相位补偿
\end{itemize}

\subsection{微分器}

微分器(Differentiator)的电路如图~\ref{fig:differentiator} 所示。

\begin{figure}[H]
  \centering
  \begin{circuitikz}[american,scale=1.0]
    \draw
    (0,0) node[op amp] (opamp) {}
    (opamp.+) node[ground] {}
    (opamp.-) to[short] ++(-0.5,0) coordinate (in_node)
    (in_node) to[R=$R$, -o] ++(-2,0) node[left] {$V_{in}$}
    (in_node) to[short, *-] ++(0,1) coordinate (fb_node)
    (opamp.out) to[short, *-o] ++(1,0) node[right] {$V_{out}$}
    (opamp.out) to[short] ++(0,1) coordinate (out_fb)
    (out_fb) to[C=$C$] (fb_node);
  \end{circuitikz}
  \caption{微分器}\label{fig:differentiator}
\end{figure}

输出电压为:

\begin{equation}
V_{out}(t) = -RC \frac{dV_{in}(t)}{dt} \label{eq:differentiator}
\end{equation}

在频域中,传递函数为:

\begin{equation}
H(s) = -RCs \label{eq:differentiator-transfer}
\end{equation}

微分器对高频噪声敏感,实际应用中通常需要添加限制电阻。

\subsection{比较器}

比较器(Comparator)用于比较两个电压的大小。虽然可以使用运算放大器作为比较器,但专用的比较器具有更好的性能。

比较器的输出:
\begin{itemize}
  \item 当 $V_+ > V_-$ 时,输出高电平
  \item 当 $V_+ < V_-$ 时,输出低电平
\end{itemize}

\subsection{仪表放大器}

仪表放大器(Instrumentation Amplifier)由三个运算放大器组成,具有高输入阻抗、高共模抑制比等特点,常用于信号调理。

\section{运算放大器的应用技巧}

\subsection{失调电压的补偿}

\subsubsection{外部调零}

对于具有调零端子的运算放大器,可以通过外部电位器进行调零。

\subsubsection{输入失调补偿}

在同相输入端添加补偿电压,抵消失调电压的影响。

\subsection{输入偏置电流的补偿}

为了减小输入偏置电流的影响,在同相输入端和地之间连接一个电阻,其值等于反相输入端看到的等效电阻。

\subsection{单电源应用}

\subsubsection{偏置设置}

单电源应用时,需要将输入和输出偏置到电源电压的中点($V_{CC}/2$)。

\subsubsection{交流耦合}

使用电容进行交流耦合,隔离直流偏置。

\subsection{去耦和旁路}

\subsubsection{电源去耦}

在运算放大器的电源引脚附近添加去耦电容(通常 0.1 $\mu$F 陶瓷电容 + 10 $\mu$F 电解电容),减小电源噪声。

\subsubsection{旁路电容}

在反馈网络中适当位置添加旁路电容,改善频率响应。

\subsection{保护电路}

\subsubsection{输入保护}

使用限流电阻和箝位二极管保护运算放大器的输入端,防止过压。

\subsubsection{输出保护}

使用限流电阻保护输出端,防止短路损坏。

\subsection{PCB 布局注意事项}

\begin{enumerate}
  \item 缩短输入引线,减小寄生电容和电感
  \item 将去耦电容尽量靠近运算放大器
  \item 使用地平面,减小地线阻抗
  \item 避免数字信号和模拟信号交叉
  \item 注意热设计,避免温度梯度
\end{enumerate}

\section{运算放大器的测试方法}

\subsection{直流参数测试}

\subsubsection{输入失调电压测试}

将两个输入端短接并接地,测量输出电压 $V_{out}$,则:

\begin{equation}
V_{OS} = \frac{V_{out}}{A_{CL}} \label{eq:test-vos}
\end{equation}

\subsubsection{输入偏置电流测试}

分别测量两个输入端的电压,计算偏置电流。

\subsubsection{开环增益测试}

使用大环路增益的反馈网络,测量开环增益。

\subsection{交流参数测试}

\subsubsection{频率响应测试}

使用网络分析仪或信号发生器 + 示波器,测量不同频率下的增益和相位。

\subsubsection{压摆率测试}

输入大幅值方波,测量输出电压的上升时间,计算压摆率。

\subsubsection{建立时间测试}

输入阶跃信号,测量输出电压达到稳定值所需的时间。

\subsection{噪声测试}

使用频谱分析仪或低噪声放大器,测量运算放大器的噪声特性。

\subsection{稳定性测试}

\subsubsection{阶跃响应测试}

观察阶跃响应是否有振荡,判断稳定性。

\subsubsection{频率响应分析}

测量相位裕度和增益裕度,评估稳定性。

\subsection{环境测试}

在不同温度下测试运算放大器的性能,评估温度特性。
