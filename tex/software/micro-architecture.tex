\chapter{微处理器体系架构}

如果暂时撇开硬件不谈,可以说微处理器体系架构属于嵌入式开发的最底层的硬件知识。
我暂时还说不清楚这部分知识属于 EE,还是 CS。
但是,可以肯定我们从处理器架构手册中获取的芯片运行原理等信息与处理器的开发有交集,
只是说处理器的开发需要更加详尽且完善的知识体系,而紧贴处理器硬件的软件开发也需要这部分原理知识以便正确且高效的使用相关的处理器。

微处理器体系架构方面的知识建立了处理器硬件与软件的桥梁,是嵌入式开发所绕不开的。
所以,我们从微处理器体系架构开始,逐步的完善并建立嵌入式开发的基础知识体系。

本章主要讲述两种微处理器架构,一种是目前占主导市场的 ARMv8 架构,两外一种则是开源的 RISC-V 指令集架构。

\section{ARMv8}

ARM 公司在 2013 年发布了它的 64-bit ARMv8 架构。
ARMv8 实现了 32-bit ARMv7 的兼容。
做出来以下重要升级:\footnote{该部分借鉴 \url{https://armv8-doc.readthedocs.io/en/latest/02.html\#armv8-a},并且后续的整理也会借鉴该翻译成果,在此对付出精力的翻译人员予以感谢与尊重。
原版手册是 {ARM® Cortex®-A Series Programmer’s Guide for ARMv8-A}\cite{armpg}}

\begin{description}
    \item[Large physical address] 更大的物理内存,使处理器能够访问超过 4GB 的物理内存。
    \item[64-bit virtual addressing] 使虚拟内存突破 4GB 限制。
    这对于使用内存映射文件 I/O 或稀疏寻址的现代桌面和服务器软件很重要。
    \item[Automatic event signaling] 可以实现节能、高性能的自旋锁。
    \item[Larger register files] 31 个 64 位通用寄存器可提高性能并减少堆栈使用。
    \item[Efficient 64-bit immediate generation] 对 literal pool 的需求较少
    \item[Large PC-relative addressing range] 一个 +/‑4GB 的寻址范围,用于在共享库和与位置无关的可执行文件中进行有效的数据寻址。
    \item[Additional 16KB and 64KB translation granules] 降低了翻译后备缓冲区 (TLB) 未命中率和页面遍历深度。
    \item[New exception model] 降低了操作系统和管理程序软件的复杂性。
    \item[Efficient cache management] 用户空间缓存操作提高了动态代码生成效率。
    使用数据缓存零指令快速清除数据缓存。
    \item[Hardware-accelerated cryptography] 提供 3 到 10 倍更好的软件加密性能。
    这对于小粒度解密和加密非常有用,因为太小而无法有效地卸载到硬件加速器,例如 https。
    \item[Load-Acquire, Store-Release instructions] 专为 C++11、C11、Java 内存模型而设计。
    它们通过消除显式内存屏障指令来提高线程安全代码的性能。
    \item[NEON double-precision floating-point advanced SIMD] 使得 SIMD 矢量化能够应用于更广泛的算法集,例如科学计算、高性能计算 (HPC) 和超级计算机。
\end{description}

\subsection{Contex-A57 处理器}

目前 ARM 处理器已经更新到 X2,但是对于学习它的体系架构来说,A57 或 A53 已经足够适用,也足够经典。
我们将手册中的框架图拿过来感性的认识处理器的内部结构。

Cortex‑A57 处理器 Cortex‑A57 处理器面向移动和企业计算应用,包括计算密集型 64 位应用,例如高端计算机、平板电脑和服务器产品。
它可以与 Cortex‑A53 处理器一起使用到 ARM big.LITTLE 配置中,以实现可扩展的性能和更高效的能源使用。

Cortex‑A57 处理器具有与其他处理器的高速缓存一致性互操作性,包括用于 GPU 计算的 ARM Mali™ 系列图形处理单元 (GPU),并为高性能企业应用程序提供可选的可靠性和可扩展性功能。
它提供了比 ARMv7 架构的 Cortex‑A15 处理器更高的性能,并具有更高的能效水平。
加密扩展的包含将加密算法的性能提高了 10 倍于上一代处理器。

\anchorfig[caption={Contex-A57}, label={fig:contex-a57}, width=0.8]{A57}

Cortex‑A57 处理器完全实现了 ARMv8‑A 架构。
它支持多核操作,在单个集群中具有一到四核多处理。
通过 AMBA5 CHI 或 AMBA 4 ACE 技术,可以实现多个一致的 SMP 集群。
可通过 CoreSight 技术进程调试和跟踪。

Cortex‑A57 处理器具有以下特性:

\begin{itemize}
  \item 乱序,15+阶段流水线。
  \item 省电功能包括路径预测、标记减少和缓存查找抑制。
  \item 通过重复执行资源增加峰值指令吞吐量。
  具有本地化解码、3 宽解码带宽的功率优化指令解码。
  \item 性能优化的 L2 缓存设计使集群中的多个核心可以同时访问 L2。
\end{itemize}

\subsection{ARMv8 异常级别}

在 ARMv8 中,执行发生在四个异常级别之一。
异常级别决定特权级别,因此在 $EL_n$ 执行对应于特权 $PL_n$。
更大的 n 值的异常级别处于更高的异常级别。

异常级别提供了适用于 ARMv8 架构的所有操作状态的软件执行权限的逻辑分离。
它类似于并支持计算机科学中常见的分层保护域的概念。

\begin{description}
    \item[EL0] Normal user applications.
    \item[EL1] Operating system kernel typically described as privileged.
    \item[EL2] Hypervisor.
    \item[EL3] Low-level firmware, including the Secure Monitor.
\end{description}

异常级别之间可以转换,但是要遵循以下规则:

\begin{itemize}
  \item 移动到更高的异常级别,例如从 EL0 到 EL1,表示软件增加执行特权。
  \item 不能将异常处理到较低的异常级别。
  \item EL0 级别没有异常处理,必须在更高的异常级别处理异常。
  \item 异常导致程序流程发生变化。
    异常处理程序的执行以高于 EL0 的异常级别从与所采取的异常相关的已定义向量开始。
    例外情况包括:IRQ 和 FIQ 等中断、内存系统中止、未定义的指令、系统调用。
    这些允许非特权软件对操作系统安全监视器或管理程序陷阱。
  \item 通过执行 ERET 指令来结束异常处理并返回到上一个异常级别。
  \item 从异常返回可以保持相同的异常级别或进入较低的异常级别。
    它不能移动到更高的异常级别。
  \item 安全状态确实会随着异常级别的变化而变化,除非从 EL3 重新调整到非安全状态。
\end{itemize}

\begin{probsolu}[title={Problem and Solution \theprob}]{
    如何切换 AArch64 EL?写个切换的实际例子?(课后作业)
  }\label{pb:el_changing}

  当处理异常时,会涉及几个寄存器的操作:
  \begin{enumerate}
    \item 处理器将当前正在执行的指令地址(PC 寄存器)存储在 ELR\_ELn(Exception link register)中。
    \item 将当前处理器的状态(PSTATE)存储在 SPSR\_ELn(Saved Program Status Register)中。
    \item 处理器根据异常向量表跳转到异常处理程序。
    异常处理程序可以修改 ELR 和 SPSR。
    \item 异常处理程序执行 eret 指令推出异常状态。
    这个指令会从 SPSR\_Eln 寄存器恢复处理器的状态,并且恢复 ELE\_Eln 中储存的指令的执行。
  \end{enumerate}
  据上所述,异常处理程序\textcolor{red}{可以修改 ELR\_ELn 和 SPSR\_ELn 寄存器},所以异常处理程序能够间接的修改 EL 等参数,达到切换 EL 的目的。

  比如,想要从 EL3 异常级切换到 EL1,示例代码如下~\ref{lst:change_el}。
  那么需要配置一些系统寄存器,然后调用 eret 指令触发处理器切换异常运行级。

  \begin{enumerate}
    \item 配置 SCTLR\_EL1(System Control Register)。
      sctlr\_eln 寄存器被用来配置处理器的不同参数。
      存在 sctlr\_el1、sctlr\_el2 和 sctlr\_el3 分别对应 EL1、EL2 和 EL3 的寄存器。
      sctlr\_el1 寄存器能够配置 EL0 和 EL1 级别的内存等配置。
      通过修改 sctlr\_el1 某些位的值能达到配置处理器在 EL0 和 EL1 级别运行时的行为。
    \item 配置 HCR\_EL2(Hypervisor Configuration Register)。
      HCR\_EL2 寄存器提供了虚拟化的配置,包括定义是否将各种操作限制在 EL2 中。
      因为只有 EL2 支持 Hypervisor,所以只存在一个 HCR\_EL2 寄存器。
    \item 配置 SCR\_EL3(Secure Configuration Register)。
      SCR\_EL3 寄存器定义当前安全状态的配置:
      \begin{itemize}
        \item EL0,EL1 和 EL2 的安全状态为 Secure 或 Non-Secure
        \item EL2 的 Execution State
      \end{itemize}
    \item 配置 SPSR\_EL3(Saved Program Status Register)。
      EL3 发生异常时,SPSR\_EL3 寄存器用来保存处理器的状态。
    \item 配置 ELR\_EL3(Exception Link Register (EL3))。
      在 EL3 进行异常处理时,ELR\_EL3 寄存器将用来指定即将要返回的地址。
  \end{enumerate}
  通过配置上述系统寄存器,然后调用 eret 触发处理器的执行状态的重恢复,就能将异常级别从 EL3 切换到 EL1。
\end{probsolu}
\begin{lstlisting}[
  language={[ARM]Assembler},
  caption={切换异常级},
  label={lst:change_el}
]
  master:
  ldr    x0, =SCTLR_VALUE_MMU_DISABLED
  msr    sctlr_el1, x0

  ldr    x0, =HCR_VALUE
  msr    hcr_el2, x0

  ldr    x0, =SCR_VALUE
  msr    scr_el3, x0

  ldr    x0, =SPSR_VALUE
  msr    spsr_el3, x0

  adr    x0, el1_entry
  msr    elr_el3, x0

  eret
\end{lstlisting}

ARMv8 架构定义了两种执行状态,AArch64 和 AArch32。
每个状态分别用于描述使用 64 位宽通用寄存器或 32 位宽通用寄存器的执行。
虽然 ARMv8 AArch32 保留了 ARMv7 对特权的定义,但在 AArch64 中,特权级别由异常级别决定。
因此,在 $EL_n$ 的执行对应于特权 $PL_n$。

当处于 AArch64 状态时,处理器执行 A64 指令集。
当处于 AArch32 状态时,处理器可以执行 A32(在早期版本的架构中称为 ARM)或 T32 (Thumb) 指令集。

\begin{probsolu}[title={Problem and Solution \theprob}]{
    如何切换 AArch64 状态到 AArch32 状态?写个切换的实际例子?(课后作业)
  }\label{pb:state_changing}
  
  例如:在 EL3 下进行运行切换,EL3 为 AArch64,将 EL2 切换成 aarch32。
  在 EL3 异常级下,设置 EL2 的架构为 aarch32,设置好返回地址,通过 ERET 指令,即可将 EL2 状态切换成 EL2。
  设置中,主要涉及配置 elr\_el3 寄存器(保存下一异常级的指令地址)和 spsr\_el3 寄存器(保存下一异常级的 pstate 值)。
  对于 spsr\_el3,要设置正确,则要参考 AArch32 的 cpsr 寄存器值进行设置。

  如果需要将 A32 状态切换到 T32 状态,则使用 bx 指令,并且跳转地址的最低位要为 1;
  从 T32 状态切回 A32 状态同样使用 bx 指令,且跳转地址最低位为 0。
  
  总结:EL2 的 A64 和 A32 状态,由 EL3 决定,也就是 SCR\_EL3.RW 寄存器决定。

  EL1 的 A64 和 A32 状态,由 EL2 决定,也就是 HCR\_EL3.RW 寄存器决定。

  EL0 的 A64 和 A32 状态,由 EL1 决定,也就是 CPSR.M[4] 位决定。
\end{probsolu}

\subsection{Registers}

AArch64 64 位通用寄存器(X0-X30),也可以只使用低 32 位(W0-W30)用于 A32 状态。
从 W 寄存器读取时,忽略相应 X 寄存器高 32 位,并保持其它不变。
写入 W 寄存器时,将 X 寄存器的高 32 位设置为零。
也就是说,将 0xFFFFFFFF 写入 W0 会将 X0 设置为 0x00000000FFFFFFFF。

另外还有一些特殊寄存器:
\begin{itemize}
  \item Zero 寄存器,注意并没有所谓的 31 号寄存器(X31/W31),编号为 31 的寄存器就是零寄存器。
  当访问零寄存器时,所有写操作都被忽略,所有读操作返回 0。
  \item PC 寄存器(Program Counter)。
  \item SP 寄存器(SP/WSP)。
  注意 A64 下的 SP 并不加前缀 X。
  \item SPSR 寄存器(Program Status Register)。
  SPSR 保存着异常发生之前的 PSTATE 的值,用于在异常返回时恢复 PSTATE 的值。
  \item ELR 寄存器(Exception Link Register)。
  保存异常返回地址。
\end{itemize}
只有 EL1 和更高的异常级存在 SPSR 和 ELR。

在 ARMv8 体系结构中,要使用的栈指针的选择在一定程度上与异常级别是分开的。
默认情况下,发生异常时会选择目标异常级别的 SP\_ELn 作为栈指针。
例如,当触发到 EL1 的异常时,就会选择 SP\_EL1 作为栈指针。
每个异常级别都有自己的栈指针,SP\_EL0、SP\_EL1、SP\_EL2 和 SP\_EL3。
EL0 永远只能访问 SP\_EL0。

\begin{table}[H]
  \begin{center}
    \caption{AArch64 SP 选项}
    \label{tbl:a64_sp_opt}
    \begin{tblr}{ccc}
      \hline[1pt]
      Exception Level & Options \\
      \hline
      EL0 & EL0t \\
      EL1 & EL1t, EL1h \\
      EL2 & EL2t, EL2h \\
      EL3 & EL3t, EL3h \\
      \hline[1pt]
    \end{tblr}
  \end{center}
\end{table}

后缀 t 表示选择 SP\_EL0,h 表示选择 SP\_ELn。

大多数指令无法使用 SP 寄存器,但是有一些形式的算术指令可以操作 SP,例如,ADD 指令可以读写当前的栈指针以调整函数中的栈指针。

原来的 ARMv7 指令集的一个特性是 R15 作为程序计数器(PC),并作为一个通用寄存器使用。
PC 寄存器的使用带来了一些编程技巧,但它为编译器和复杂的流水线的设计引入了复杂性。
在 ARMv8 中删除了对 PC 的直接访问,使返回预测更容易,并简化了 ABI 规范。

PC 永远不能作为一个命名的寄存器来访问。
但是,可以在某些指令中隐式的使用 PC,如 PC 相对加载和地址生成。
PC 不能被指定为数据处理或加载指令的目的操作数。

下表总结了 SPSR 各 bit 的含义:
\begin{table}[H]
  \begin{center}
    \caption{AArch64 SPSR bit 位含义}
    \label{tbl:a64_spsr}
    \begin{tblr}{c>{\centering\arraybackslash}X}
      \hline[1pt]
      bit & 含义 \\
      \hline
      N & 负数标志位,如果结果为负数,则 N=1;
      如果结果为非负数,则 N=0。\\
      Z & 零标志位,如果结果为零,Z=1,否则 Z=0。\\
      C & 进位标志位\\
      V & 溢出标志位\\
      SS & 软件步进标志位,表示当一个异常发生时,软件步进是否开启\\
      IL & 非法执行状态位\\
      D & 程序状态调试掩码,在异常发生时的异常级别下,来自监视点、断点和软件单步调试事件中的调试异常是否被屏蔽。\\
      A & SError(系统错误)掩码位\\
      I & IRQ 掩码位\\
      F & FIQ 掩码位\\
      M$[4]$ & 异常发生时的执行状态,0 表示 AArch64\\
      M$[3:0]$ & 异常发生时的 mode 或异常级别\\
      \hline[1pt]
    \end{tblr}
  \end{center}
\end{table}

AArch64 没有直接与 ARMv7 当前程序状态寄存器 (CPSR) 等价的寄存器。
在 AArch64 中,传统 CPSR 的组件作为可以独立访问的字段提供。
这些状态被统称为处理器状态 (PSTATE)。

\begin{table}[H]
  \begin{center}
    \caption{AArch64 PSTATE field}
    \label{tbl:a64_pstate}
    \begin{tblr}{c>{\centering\arraybackslash}X}
      \hline[1pt]
      bit & 含义 \\
      \hline
      N & Negative condition flag. \\
      Z & Zero condition flag. \\
      C & Carry condition flag. \\
      V & oVerflow condition flag. \\
      D & Debug mask bit. \\
      A & SError mask bit. \\
      I & IRQ mask bit. \\
      F & FIQ mask bit. \\
      SS & Software Step bit. \\
      IL & Illegal execution state bit. \\
      EL (2) & Exception level. \\
      nRW & Execution state 
            0 = 64-bit
            1 = 32-bit \\
      SP & Stack Pointer selector.
            0 = SP\_EL0
            1 = SP\_ELn \\
      \hline[1pt]
    \end{tblr}
  \end{center}
\end{table}

在 AArch64 中,你可以通过执行 ERET 指令从一个异常中返回,那么 SPSR\_ELn 被复制到 PSTATE 中。
包括恢复 ALU 标志、执行状态、异常级别和处理器分支。
并将从 ELR\_ELn 中的地址开始继续执行。

PSTATE.\{N, Z, C, V\} 字段可以在 EL0 级别访问。
其他的字段可以在 EL1 或更高级别访问,但是这些字段在 EL0 级别未定义。

\subsubsection{系统寄存器}

\subsubsection{NEON 和浮点}

\subsection{ISA}

\subsection{Caches}

\subsection{MMU}

\subsection{Memory ordering}

\subsection{SMP}

\subsection{Others}

\section{RISC-V}

\subsection{Privileged Architecture}

\subsection{Unprivileged ISA}

\section{内存层级}
