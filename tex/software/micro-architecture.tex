\chapter{微处理器体系架构}

如果暂时撇开硬件不谈,可以说微处理器体系架构属于嵌入式开发的最底层的硬件知识。
我暂时还说不清楚这部分知识属于 EE,还是 CS。
但是,可以肯定我们从处理器架构手册中获取的芯片运行原理等信息与处理器的开发有交集,
只是说处理器的开发需要更加详尽且完善的知识体系,而紧贴处理器硬件的软件开发也需要这部分原理知识以便正确且高效的使用相关的处理器。

微处理器体系架构方面的知识建立了处理器硬件与软件的桥梁,是嵌入式开发所绕不开的话题。
所以,我们从微处理器体系架构开始,逐步的完善并建立嵌入式开发的基础知识体系。

本章主要根据相关的编程和架构手册来讲述两种微处理器架构,一种是目前占市场主导地位的 ARMv8 架构,另外一种则是开源的 RISC-V 指令集架构。
之所以选择这两种指令集架构进行学习,是因为它们的流行度和潜力。

下面让我们从 ARMv8 开始。

\section{ARMv8}
ARM 公司在 2013 年发布了它的 64-bit ARMv8 架构。
ARMv8 实现了 32-bit ARMv7 的兼容。
做出了以下重要升级:\footnote{
  这里翻译或讲述的原版手册是 {ARM® Cortex®-A Series Programmer’s Guide for ARMv8-A}\cite{armpg}
}

\begin{description}
    \item[Large physical address] 更大的物理内存,使处理器能够访问超过 4GB 的物理内存。
    \item[64-bit virtual addressing] 使虚拟内存突破 4GB 限制。
    这对于使用内存映射文件 I/O 或稀疏寻址的现代桌面和服务器软件很重要。
    \item[Automatic event signaling] 可以实现节能、高性能的自旋锁。
    \item[Larger register files] 31 个 64 位通用寄存器可提高性能并减少堆栈使用。
    \item[Efficient 64-bit immediate generation] 降低对 literal pool 的需求。
    \item[Large PC-relative addressing range] 一个 +/‑4GB 的寻址范围,用于在共享库和与位置无关的可执行文件中进行有效的数据寻址。
    \item[Additional 16KB and 64KB translation granules] 降低了翻译后备缓冲区 (TLB) 未命中率和页面遍历深度。
    \item[New exception model] 降低了操作系统和管理程序软件的复杂性。
    \item[Efficient cache management] 用户空间缓存操作提高了动态代码生成效率。
    使用数据缓存零指令快速清除数据缓存。
    \item[Hardware-accelerated cryptography] 软件加密性能提高 3 到 10 倍。
      这对于小粒度解密和加密非常有用,因为太小而无法有效地卸载到硬件加速器(too small to offload to a hardware accelerator),例如 https。
    \item[Load-Acquire, Store-Release instructions] 专为 C++11、C11、Java 内存模型而设计。
    它们通过消除显式内存屏障指令来提高线程安全代码的性能。
    \item[NEON double-precision floating-point advanced SIMD] 使得 SIMD 矢量化能够应用于更广泛的算法集,例如科学计算、高性能计算 (HPC) 和超级计算机。
\end{description}

\subsection{Contex-A57 处理器}

目前 ARM 处理器已经更新到 X2,但是对于学习它的体系架构来说,A57 或 A53 已经足够适用,也足够经典。
我们将手册中的框架图拿过来感性的认识处理器的内部结构。

Cortex‑A57 处理器 Cortex‑A57 处理器面向移动和企业计算应用,包括计算密集型 64 位应用,例如高端计算机、平板电脑和服务器产品。
它可以与 Cortex‑A53 处理器一起使用到 ARM big.LITTLE 配置中,兼顾性能和功耗。

Cortex‑A57 处理器具有与其他处理器的高速缓存一致性互操作性,包括用于 GPU 计算的 ARM Mali™ 系列图形处理单元 (GPU),并为高性能企业应用程序提供可选的可靠性和可扩展性功能。
它提供了比 ARMv7 架构的 Cortex‑A15 处理器更高的性能,并具有更高的能效水平。
加密扩展使加密算法的性能比上一代处理器提高 10 倍以上。

\Figure[caption={Contex-A57}, label={fig:contex-a57}, width=0.7]{A57}

Cortex‑A57 处理器完全实现了 ARMv8‑A 架构。
它支持多核操作,在单个集群中具有一到四核多处理器。
通过 AMBA5 CHI 或 AMBA 4 ACE 技术,可以实现多个一致的 SMP 集群。
可通过 CoreSight 技术进程调试和跟踪。

Cortex‑A57 处理器具有以下特性:

\begin{itemize}
  \item 乱序,多于 15 级流水线。
  \item 省电功能包括路径预测、标记缩减和缓存查找抑制。
  \item 通过重复执行资源增加峰值指令吞吐量。
    具有本地化解码、3 个宽解码带宽的功率优化指令解码。
  \item 性能优化的 L2 缓存设计使集群中的多个核心可以同时访问 L2。
\end{itemize}

\subsection{ARMv8 基础}

在 ARMv8 中,代码执行在四个异常级之一。
异常级别决定特权级别,因此在 $EL_n$ 执行对应于特权 $PL_n$。
更大的 n 值的异常级别处于更高的异常级别。

异常级别提供了软件执行权限的逻辑分离,适用于ARMv8架构的所有操作状态。
它类似并支持计算机科学中常见的分层保护域的概念。

\begin{description}
    \item[EL0] Normal user applications.
    \item[EL1] Operating system kernel typically described as privileged.
    \item[EL2] Hypervisor.
    \item[EL3] Low-level firmware, including the Secure Monitor.
\end{description}

\subsubsection{异常级切换}

异常级别之间可以转换,但是要遵循以下规则:

\begin{itemize}
  \item 移动到更高的异常级别,例如从 EL0 到 EL1,表示软件增加执行特权。
  \item 不能在较低的异常级别下进行异常处理。
  \item EL0 级别没有异常处理,必须在更高的异常级别处理异常。
  \item 异常导致程序流程发生变化。
    异常处理程序的执行从高于EL0的异常级别开始,起始于与所发生的异常相关的定义向量。
    异常包括:IRQ 和 FIQ 等中断、内存系统中止、未定义的指令、系统调用和安全监视器或虚拟机管理程序 trap。
  \item 通过执行 ERET 指令来结束异常处理并返回到上一个异常级别。
  \item 从异常返回可以保持相同的异常级别或进入较低的异常级别,但不能移动到更高的异常级别。
  \item 安全状态确实会随着异常级别的变化而变化,除非从 EL3 重新调整到非安全状态。
\end{itemize}

\begin{probsolu}[title={Problem and Solution \theprob}]{
    如何切换 AArch64 EL?写个切换的实际例子?(课后作业)
  }\label{pb:el_changing}

  当处理异常时,会涉及几个寄存器的操作:
  \begin{enumerate}
    \item 处理器将当前正在执行的指令地址(PC 寄存器)存储在 ELR\_ELn(Exception link register)中。
    \item 将当前处理器的状态(PSTATE)存储在 SPSR\_ELn(Saved Program Status Register)中。
    \item 处理器根据异常向量表跳转到异常处理程序。
    异常处理程序可以修改 ELR 和 SPSR。
    \item 异常处理程序执行 eret 指令推出异常状态。
    这个指令会从 SPSR\_Eln 寄存器恢复处理器的状态,并且恢复 ELE\_Eln 中储存的指令的执行。
  \end{enumerate}
  据上所述,异常处理程序\textcolor{red}{可以修改 ELR\_ELn 和 SPSR\_ELn 寄存器},所以异常处理程序能够间接的修改 EL 等参数,达到切换 EL 的目的。

  比如,想要从 EL3 异常级切换到 EL1,示例代码如下~\ref{lst:change_el}。
  那么需要配置一些系统寄存器,然后调用 eret 指令触发处理器切换异常运行级。

  \begin{enumerate}
    \item 配置 SCTLR\_EL1(System Control Register)。
      sctlr\_eln 寄存器被用来配置处理器的不同参数。
      存在 sctlr\_el1、sctlr\_el2 和 sctlr\_el3 分别对应 EL1、EL2 和 EL3 的寄存器。
      sctlr\_el1 寄存器能够配置 EL0 和 EL1 级别的内存等配置。
      通过修改 sctlr\_el1 某些位的值能达到配置处理器在 EL0 和 EL1 级别运行时的行为。
    \item 配置 HCR\_EL2(Hypervisor Configuration Register)。
      HCR\_EL2 寄存器提供了虚拟化的配置,包括定义是否将各种操作限制在 EL2 中。
      因为只有 EL2 支持 Hypervisor,所以只存在一个 HCR\_EL2 寄存器。
    \item 配置 SCR\_EL3(Secure Configuration Register)。
      SCR\_EL3 寄存器定义当前安全状态的配置:
      \begin{itemize}
        \item EL0,EL1 和 EL2 的安全状态为 Secure 或 Non-Secure
        \item EL2 的 Execution State
      \end{itemize}
    \item 配置 SPSR\_EL3(Saved Program Status Register)。
      EL3 发生异常时,SPSR\_EL3 寄存器用来保存处理器的状态。
    \item 配置 ELR\_EL3(Exception Link Register (EL3))。
      在 EL3 进行异常处理时,ELR\_EL3 寄存器将用来指定即将要返回的地址。
  \end{enumerate}
  通过配置上述系统寄存器,然后调用 eret 触发处理器的执行状态的重恢复,就能将异常级别从 EL3 切换到 EL1。
\end{probsolu}
\begin{lstlisting}[
  language={[ARM]Assembler},
  caption={切换异常级},
  label={lst:change_el}
]
  master:
  ldr    x0, =SCTLR_VALUE_MMU_DISABLED
  msr    sctlr_el1, x0

  ldr    x0, =HCR_VALUE
  msr    hcr_el2, x0

  ldr    x0, =SCR_VALUE
  msr    scr_el3, x0

  ldr    x0, =SPSR_VALUE
  msr    spsr_el3, x0

  adr    x0, el1_entry
  msr    elr_el3, x0

  eret
\end{lstlisting}

\subsubsection{运行状态切换}

ARMv8 架构定义了两种执行状态,AArch64 和 AArch32。
每个状态分别用于描述使用 64 位宽通用寄存器或 32 位宽通用寄存器的执行。
虽然 ARMv8 AArch32 保留了 ARMv7 对特权的定义,但在 AArch64 中,特权级别由异常级别决定。
因此,在 $EL_n$ 的执行对应于特权 $PL_n$。

当处于 AArch64 状态时,处理器执行 A64 指令集。
当处于 AArch32 状态时,处理器可以执行 A32(在早期版本的架构中称为 ARM)或 T32 (Thumb) 指令集。

\begin{probsolu}[title={Problem and Solution \theprob}]{
    如何切换 AArch64 状态到 AArch32 状态?写个切换的实际例子?(课后作业)
  }\label{pb:state_changing}
  
  例如:在 EL3 下进行运行切换,EL3 为 AArch64,将 EL2 切换成 aarch32。
  在 EL3 异常级下,设置 EL2 的架构为 aarch32,设置好返回地址,通过 ERET 指令,即可将 EL2 状态切换成 EL2。
  设置中,主要涉及配置 elr\_el3 寄存器(保存下一异常级的指令地址)和 spsr\_el3 寄存器(保存下一异常级的 pstate 值)。
  对于 spsr\_el3,要设置正确,则要参考 AArch32 的 cpsr 寄存器值进行设置。

  如果需要将 A32 状态切换到 T32 状态,则使用 bx 指令,并且跳转地址的最低位要为 1;
  从 T32 状态切回 A32 状态同样使用 bx 指令,且跳转地址最低位为 0。
  
  总结:EL2 的 A64 和 A32 状态,由 EL3 决定,也就是 SCR\_EL3.RW 寄存器决定。

  EL1 的 A64 和 A32 状态,由 EL2 决定,也就是 HCR\_EL3.RW 寄存器决定。

  EL0 的 A64 和 A32 状态,由 EL1 决定,也就是 CPSR.M[4] 位决定。
\end{probsolu}

\subsection{寄存器}

AArch64 64 位通用寄存器(X0-X30),也可以只使用低 32 位(W0-W30)用于 A32 状态。
从 W 寄存器读取时,忽略相应 X 寄存器高 32 位,并保持其它不变。
写入 W 寄存器时,将 X 寄存器的高 32 位设置为零。
也就是说,将 0xFFFFFFFF 写入 W0 会将 X0 设置为 0x00000000FFFFFFFF。

另外还有一些特殊寄存器:
\begin{itemize}
  \item Zero 寄存器,注意并没有所谓的 31 号寄存器(X31/W31),编号为 31 的寄存器就是零寄存器。
    当访问零寄存器时,所有写操作都被忽略,所有读操作返回 0。
  \item PC 寄存器(Program Counter)。
  \item SP 寄存器(SP/WSP)。
    注意 A64 下的 SP 并不加前缀 X。
  \item SPSR 寄存器(Program Status Register)。
    SPSR 保存着异常发生之前的 PSTATE 的值,用于在异常返回时恢复 PSTATE 的值。
  \item ELR 寄存器(Exception Link Register)。
    保存异常返回地址。
\end{itemize}

只有 EL1 和更高的异常级存在 SPSR 和 ELR。

在 ARMv8 体系结构中,选择使用栈指针寄存器在一定程度上与异常级别是分开的。
默认情况下,发生异常时会选择目标异常级别的 SP\_ELn 作为栈指针。
例如,当触发到 EL1 的异常时,就会选择 SP\_EL1 作为栈指针。
每个异常级别都有自己的栈指针,SP\_EL0、SP\_EL1、SP\_EL2 和 SP\_EL3。
EL0 永远只能访问 SP\_EL0。

\begin{stblr}
  {AArch64 SP 选项}
  {a64-sp-opt}
  {cc}
  \hline[1pt]
  Exception Level & Options \\
  \hline
  EL0 & EL0t \\
  EL1 & EL1t, EL1h \\
  EL2 & EL2t, EL2h \\
  EL3 & EL3t, EL3h \\
  \hline[1pt]
\end{stblr}

后缀 t 表示选择 SP\_EL0,h 表示选择 SP\_ELn。

大多数指令无法使用 SP 寄存器,但是有一些形式的算术指令可以操作 SP,例如,ADD 指令可以读写当前的栈指针以调整函数中的栈指针。

原来的 ARMv7 指令集的一个特性是 R15 作为程序计数器(PC),并作为一个通用寄存器使用。
PC 寄存器的使用带来了一些编程技巧,但它为编译器和复杂的流水线的设计引入了复杂性。
在 ARMv8 中删除了对 PC 的直接访问,使返回预测更容易,并简化了 ABI 规范。

PC 永远不能作为一个命名的寄存器来访问。
但是,可以在某些指令中隐式的使用 PC,如 PC 相对加载和地址生成。
PC 不能被指定为数据处理或加载指令的目的操作数。

下表总结了 SPSR 各 bit 的含义:
\begin{ltblr}[caption={AArch64 SPSR bit 位含义}, label={tbl:a64_spsr}]
  {colspec={c>{\centering\arraybackslash}X}}
    \hline[1pt]
    bit & 含义 \\
    \hline
    N & 负数标志位,如果结果为负数,则 N=1;
    如果结果为非负数,则 N=0。\\
    Z & 零标志位,如果结果为零,Z=1,否则 Z=0。\\
    C & 进位标志位\\
    V & 溢出标志位\\
    SS & 软件步进标志位,表示当一个异常发生时,软件步进是否开启\\
    IL & 非法执行状态位\\
    D & 程序状态调试掩码,在异常发生时的异常级别下,来自监视点、断点和软件单步调试事件中的调试异常是否被屏蔽。\\
    A & SError(系统错误)掩码位\\
    I & IRQ 掩码位\\
    F & FIQ 掩码位\\
    M$[4]$ & 异常发生时的执行状态,0 表示 AArch64\\
    M$[3:0]$ & 异常发生时的 mode 或异常级别\\
    \hline[1pt]
\end{ltblr}

AArch64 没有直接与 ARMv7 当前程序状态寄存器 (CPSR) 等价的寄存器。
在 AArch64 中,传统 CPSR 的组件作为可以独立访问的字段提供。
这些状态被统称为处理器状态 (PSTATE)。

\begin{ltblr}[caption={AArch64 PSTATE field}, label={tbl:a64_pstate}]
  {colspec={c>{\centering\arraybackslash}X}}
    \hline[1pt]
    bit & 含义 \\
    \hline
    N & Negative condition flag. \\
    Z & Zero condition flag. \\
    C & Carry condition flag. \\
    V & oVerflow condition flag. \\
    D & Debug mask bit. \\
    A & SError mask bit. \\
    I & IRQ mask bit. \\
    F & FIQ mask bit. \\
    SS & Software Step bit. \\
    IL & Illegal execution state bit. \\
    EL (2) & Exception level. \\
    nRW & Execution state 
          0 = 64-bit
          1 = 32-bit \\
    SP & Stack Pointer selector.
          0 = SP\_EL0
          1 = SP\_ELn \\
    \hline[1pt]
\end{ltblr}

在 AArch64 中,您可以通过执行 ERET 指令从一个异常中返回,那么 SPSR\_ELn 被复制到 PSTATE 中。
包括恢复 ALU 标志、执行状态、异常级别和处理器分支。
并将从 ELR\_ELn 中的地址开始继续执行。

PSTATE.\{N, Z, C, V\} 字段可以在 EL0 级别访问。
其他的字段可以在 EL1 或更高级别访问,但是这些字段在 EL0 级别未定义。

\subsubsection{系统寄存器}

在 AArch64 中,系统配置通过系统寄存器进行控制,并使用 MSR 和 MRS 指令进行访问。
简化了 ARMv7 架构通过协处理器 CP15 来操作系统寄存器的方式。

高异常级下可以访问本异常级和低异常级的系统寄存器。
EL0 异常级具有最低的权限,并且只有极少数的系统系统器可以在 EL0 下访问,例如:CTR\_EL0。

详细的系统寄存器列表请参见 Arm® Architecture Reference Manual for A-profile architecture \cite{armrefman}。

\paragraph*{系统控制寄存器 SCTLR}

系统控制寄存器是一个很重要的系统寄存器,用于控制内存、配置系统能力和提供处理器核的状态信息。
这个寄存器在 EL0 下拥有更多的可获取 bit,更高的异常级则更少。

% npx bit-field -i <json> --fontsize=9 > figures/<bit_field>.svg
\Figure[caption={SCTLR BIT FIELD EL1}, label={fig:sctlr-bit-field-el1}, width=1]{sctlr-bit-field-el1}
\Figure[caption={SCTLR BIT FIELD EL2/3}, label={fig:sctlr-bit-field-el23}, width=1]{sctlr-bit-field-el23}

\begin{description}
  \item[UCI] 设置该域将使能 AArch64 下的 EL0 异常级对 DC CVAU、DC CIVAC、DC CVAC 和 IC IVAU 指令的访问权限。
  \item[EE] 控制异常 Endianness。
    \begin{description}
      \item[0] 小端 
      \item[1] 大端
    \end{description}
  \item[EOE] EL0 下 Explicit data 访问字节序
    \begin{description}
      \item[0] 小端 
      \item[1] 大端
    \end{description}
  \item[WXN] 写权限下应用不可执行权限 XN(eXecute Never)
    \begin{description}
      \item[0] 可写区域可执行
      \item[1] 可写区域强制不可执行
    \end{description}
  \item[nTWE] 不陷入 WFE,此标志为 1 表示 WFE 作为普通指令执行。
  \item[nTWI] 不陷入 WFI, 此标志为 1 表示 WFI 作为普通指令执行
  \item[UCT] 此标志为 1 时,开启 AArch64 的 EL0 下对 CTR\_EL0 寄存器访问权限。
  \item[DZE] EL0 下对 DC AVA 指令的访问权限。
    \begin{description}
      \item[0] 禁止访问
      \item[1] 允许访问
    \end{description}
  \item[I] 开启指令缓存,这是在 EL0 和 EL1 下的指令缓存的启用位。
    对可缓存的正常内存的指令访问被缓存。
  \item[UMA] 用户中断屏蔽控制,EL0 运行在 AArch64 状态下有效。
  \item[SED] 控制 AArch64 状态下的 EL0 是否可以使用 SETEND 指令。
    \begin{description}
      \item[0] 可用
      \item[1] 禁用
    \end{description}
  \item[ITD] 禁止 IT 指令
    \begin{description}
      \item[0] IT 指令有效
      \item[1] IT 指令被当作 16 位指令。
        仅另外 16 位指令或 32 位指令的头 16 位可以使用,这依赖于实现。
    \end{description}
  \item[CP15BEN] CP15 barrier 使能。
    如果实现了,它是 AArch32 CP15 DMB、DSB 和 ISB barrier 操作的使能位
  \item[SA0] EL0 的栈对齐检查使能位
  \item[SA] 栈对齐检查使能位
  \item[C] 数据 cache 使能。
    EL0 和 EL1 的数据访问使能位。
    cacheable 普通内存的数据访问都被缓存。
  \item[A] 对齐检查使能位。
  \item[M] 使能 MMU。
\end{description}

\begin{probsolu}[title={Problem and Solution \theprob}]{
    手册上写的 Explicit data access 怎么理解?}

  “Explicit data access”是指在编程中明确地访问数据的操作。
  这意味着程序员直接指定要访问的数据和操作,而不依赖于编译器或运行时系统的隐式处理。

  这种方式通常用于对内存中的数据进行读取、写入或者其他操作。
  与之相对应的是隐式访问方式,其中编译器或者运行时系统负责管理数据的访问。
  在隐式访问中,程序员通常只需指定数据的名称,而无需关心数据的存储位置或者具体的访问方式。

\end{probsolu}

\begin{remark}
  The caches in the processor must be invalidated before caching of data and instructions is
enabled in any of the Exception levels.
\end{remark}

\paragraph*{大小端的设置}

每个异常级别的数据的大小端都被单独控制。
对于 EL3,EL2 和 EL1,通过 SCTLR\_ELn.EE 设置大小端。
EL1 中其他位,SCTLR\_EL1.E0E 控制 EL0 的数据大小端的设置。
在 AArch64 执行状态中,数据访问可以为 LE 或 BE,但指令的获取通常为 LE。

\subsubsection{NEON 和浮点}

除了通用寄存器之外,ARMv8 还有 32 个 128 位浮点寄存器,标记为 V0-V31。
32 个寄存器用于保存标量浮点指令的浮点操作数,以及 NEON 操作的标量操作数和向量操作数。

操作数使用 H (Half)、S (Single) 和 D (Double) 分别用来指定 V 寄存器的低 16、32 和 64 位。

\begin{itemize}
  \item
    \textbf{v0-v7}: 用于传递浮点参数和返回值。
  \item
    \textbf{v8-v15}: 临时寄存器,用于浮点和SIMD操作。
  \item
    \textbf{v16-v31}: 被调用者保存的寄存器,需要在函数调用之间保持其值。
\end{itemize}

\subsection{ISA 概述}

ARMv8 架构中引入的最重要的变化之一是增加了 64 位指令集。
该指令集补充了现有的 32 位指令集架构。
这一指令集提供了对 64 位宽整数寄存器和数据操作的访问,以及使用 64 位内存指针的能力。
新的指令集被称为 A64,并且在 AArch64 状态下执行。
ARMv8 架构还包括原始的 ARM 指令集(现称为 A32)和 Thumb(T32)指令集。
A32 和 T32 都以 AArch32 状态执行,并且向后与 ARMv7 架构兼容。

虽然 ARMv8-A 向后兼容了 32 位 ARM 架构的特性,但 A64 指令集与旧的 ISA 指令是独立且不同的,而且他们的编码方式也不同。
A64 增加了一些额外的功能,同时也删除了影响高性能或功耗的功能。
ARMv8 架构还包括对 32 位指令集(A32 和 T32)的一些增强性功能。
然而,使用这些功能的代码与旧的 ARMv7 不兼容。
需要注意的是,A64 指令集中的指令操作码长度仍然是 32 位,而不是 64 位。

\subsubsection{ARMv8 指令集简介}

新的 A64 指令集与 A32 类似,都是 32 bit 宽度,并且语法类似。
该指令集使用通用的命名,原先的 32-bit 指令集则称为 A32 和 T32(16-bit 指令,以性能换空间)。
运行在 AArch64 新的指令集(64-bit 操作)则命名为 A64。

A64 指令集有两种整型指令形式,即通用寄存器保存 32-bit 或是 64-bit 值。
当查看指令中的寄存器名称时,如果是 X 开头则使用的是 64-bit 值;
W 开头则是 32-bit 值。
当使用 32-bit 形式时,会有以下体现:
\begin{itemize}
  \item 右移和旋转操作止于 31 位,而非 63 位。
  \item 由指令设置的状态标志是从低 32 位里计算而来。
  \item 向 W 寄存器写入时,X 寄存器的 $[63:32]$ 位自动设为 0。
\end{itemize}

64-bit 指令集极大扩展了地址空间。
因此,在程序中访问大量内存变得更加简单。
在 32-bit 的 CPU 核上执行一个线程,会将内存的访问限制在 4GB 空间。
大部分地址空间保留给 OS 内核、库代码和外设等成员使用。
所以,程序会面临内存不足的问题,那么程序在执行时可能需要将内存的一些数据映射出去或再映射回来。
拥有更大的内存空间,更大的 64-bit 指针,便可以避免以上问题。
而且类似文件内存映射等技术将更加具有可用性和方便性。
在这种情况下,即使文件的内容大小超出了物理 RAM 的大小,文件内容也可以映射到线程的内存中。

其它改进包括:独占访问、增加相对 PC 的偏移地址、支持未对齐地址、批量传输、加载 / 存储和对齐检查。

在问题~\ref{pb:state_changing} 中,我们分析了如何状态切换。
下面是手册中给出的切换图。

\Figure[caption={Switching between instruction sets}, label={fig:state_changing}, width=0.8]{state_changing}

\subsection{A64 指令集}

在使用汇编编程时,一般情况下我们只需要记住指令的助记符即可。
而不需要在意更底层的指令编码(如果需要那就查阅参考手册),指令编码的事情留给汇编器来完成。

我们将指令集按照其功能分类,有如下几类指令:\footnote{
  关于指令集的快速参考可以使用 \url{
    https://courses.cs.washington.edu/courses/cse469/19wi/arm64.pdf
  }
}

\begin{itemize}
  \item 数据处理指令
    % \begin{itemize}
    %   \item 算术与逻辑运算指令
    %   \item 乘法与除法指令
    %   \item 移位指令
    %   \item 位和字节操作指令
    %   \item 状态位操作指令
    % \end{itemize}
  \item 内存访问指令
  \item 流程控制指令
  \item 系统控制及其它指令
    % \begin{itemize}
    %   \item 异常处理指令
    %   \item 系统寄存器操作指令
    %   \item Debug 指令
    %   \item Hint 指令
    %   \item NEON 指令
    %   \item 浮点指令
    %   \item 加密算法指令
    % \end{itemize}
\end{itemize}

\subsubsection{数据处理指令}

\paragraph{算术与逻辑运算指令}

\begin{stblr}
  {算术和逻辑运算指令}
  {a64-isa-al}
  {c>{\centering\arraybackslash}X}
  \hline[1pt]
  类型 & 指令 \\
  \hline
  算术 & ADD, SUB, ADC, SBC, NEG \\
  逻辑 & AND, BIC, ORR, ORN, EOR, EON \\
  比较 & CMP, CMN, TST \\
  转移 & MOV, MVN \\
  \hline[1pt]
\end{stblr}

另外,有些指令的后缀带 S,表明这些指令会更新处理器的状态位(flags)。
这些指令可以配合跳转指令使用。
注意,CMP、CMN 和 TST 也会更新状态位但是不带后缀 S。

\paragraph{乘法与除法指令}

\begin{ltblr}[caption={乘法与除法指令}, label={tbl:a64-isa-md}]
  {colspec={c>{\centering\arraybackslash}X}}
    \hline[1pt]
    指令 & 说明 \\
    \hline
    \textbf{乘法} & \\
    MADD & Multiply add \\
    MNEG & Multiply negate \\
    MSUB & Multiply subtract \\
    MUL & Multiply \\
    SMADDL & Signed multiply-add long \\
    SMNEGL & Signed multiply-negate long \\
    SMSUBL & Signed multiply-subtract long \\
    SMULH & Signed multiply returning high half \\
    SMULL & Signed multiply long \\
    UMADDL & Unsigned multiply-add long \\
    UMNEGL & Unsigned multiply-negate long \\
    UMSUBL & Unsigned multiply-subtract long \\
    UMULH & Unsigned multiply returning high half \\
    UMULL & Unsigned multiply long \\
    \hline
    \textbf{除法} & \\
    SDIV & Signed divide \\
    UDIV & Unsigned divide \\
    \hline[1pt]
\end{ltblr}

\paragraph{移位指令}

\begin{ltblr}[caption={移位指令}, label={tbl:a64-isa-so}]
  {colspec={c>{\centering\arraybackslash}X}}
    \hline[1pt]
    指令 & 说明 \\
    \hline
    \textbf{移位} & \\
    ASR & Arithmetic shift right \\
    LSL & Logical shift left \\
    LSR & Logical shift right \\
    ROR & Rotate right \\
    \hline
    \textbf{转移} & \\
    MOV & Move \\
    MVN & Bitwise NOT \\
    \hline[1pt]
\end{ltblr}

下图形象展示了移位指令的操作。

\Figure[caption={移位操作}, label={fig:shift-ops}, width=1]{shift-ops}

\paragraph{位和字节操作指令}

\begin{ltblr}[caption={位和字节操作指令}, label={tbl:a64-isa-bB}]
  {colspec={c>{\centering\arraybackslash}X}, width=1\textwidth}
  \hline[1pt]
  指令 & 说明 \\
  \hline
  BFI rd, rn, \#p, \#n & $rd_{p+n-1:p} = rn_{n-1:0}$ \\
  BFXIL rd, rn, \#p, \#n & $rd_{n−1:0} = rn_{p+n−1:p}$ \\
  CLS rd, rn & $rd = CountLeadingOnes(rn)$ \\
  CLZ rd, rn & $rd = CountLeadingZeros(rn)$ \\
  EXTR rd, rn, rm, \#p & $rd = rn_{p−1:0}:rm_{N0}$ \\
  RBIT rd, rn & $rd = ReverseBits(rn)$ \\
  REV rd, rn & $rd = BSwap(rn)$ \\
  REV16 rd, rn & $for(n=0..1|3) rd_{Hn}=BSwap(rn_{Hn})$ \\
  REV32 Xd, Xn & $Xd=BSwap(Xn_{63:32}):BSwap(Xn_{31:0})$ \\
  \{S,U\}BFIZ rd, rn, \#p, \#n & $rd = rn^?_{n−1:0} << p$ \\
  \{S,U\}BFX rd, rn, \#p, \#n & $rd = rn^?_{p+n−1:p}$ \\
  \{S,U\}XT\{B,H\} rd, Wn & $rd = Wn^?_{N0}$ \\
  SXTW Xd, Wn & $Xd = Wn^±$ \\
  \hline[1pt]
\end{ltblr}

\paragraph{状态位操作指令}

A64 支持处理器的状态位有 NZCV,分别为 Negative、Zero、Carry 和 Overflow。
C 状态位在无符号数操作溢出时被设置,而 V 状态位类似,但是是在有符号数操作溢出时被设置。

\begin{ltblr}[caption={Condition codes}, label={tbl:condcode}]
  {colspec={cc>{\centering\arraybackslash}X>{\centering\arraybackslash}Xc}, width=1\textwidth}
  \hline[1pt]
  Code & Encoding & Meaning (when set by CMP) & Meaning (when set by FCMP) & Condition flags \\
  \hline
  EQ & 0b0000 & Equal to. & Equal to. & Z =1 \\
  NE & 0b0001 & Not equal to. & Unordered, or not equal to. & Z = 0 \\
  CS & 0b0010 & Carry set (identical to HS). & Greater than, equal to, or unordered (identical to HS). & C = 1 \\
  HS & 0b0010 & Greater than, equal to (unsigned) (identical to CS). & Greater than, equal to, or unordered (identical to CS). & C = 1 \\
  CC & 0b0011 & Carry clear (identical to LO). & Less than (identical to LO). & C = 0 \\
  LO & 0b0011 & Unsigned less than (identical to CC). & Less than (identical to CC). & C = 0 \\
  MI & 0b0100 & Minus, Negative. & Less than. & N = 1 \\
  PL & 0b0101 & Positive or zero. & Greater than, equal to, or unordered. & N = 0 \\
  VS & 0b0110 & Signed overflow. & Unordered. (At least one argument was NaN). & V = 1 \\
  VC & 0b0111 & No signed overflow. & Not unordered. (No argument was NaN). & V = 0 \\
  HI & 0b1000 & Greater than (unsigned). & Greater than or unordered. & (C = 1) \&\& (Z = 0) \\
  LS & 0b1001 & Less than or equal to (unsigned). & Less than or equal to. & (C = 0) || (Z = 1) \\
  GE & 0b1010 & Greater than or equal to (signed). & Greater than or equal to. & N==V \\
  LT & 0b1011 & Less than (signed). & Less than or unordered. & N!=V \\
  GT & 0b1100 & Greater than (signed). & Greater than. & (Z==0) \&\& (N==V) \\
  LE & 0b1101 & Less than or equal to (signed). & Less than, equal to or unordered. & (Z==1) || (N!=V) \\
  AL & 0b1110 & Always executed. & Default. Always executed. & Any \\
  NV & 0b1111 & Always executed. & Always executed. & Any \\
  \hline[1pt]
\end{ltblr}

有一小部份的条件数据处理指令是无条件执行的,但使用条件标志作为指令的额外输入。
提供这组指令是为了取代 ARM 代码中条件执行的常见用法。

\subparagraph*{加 / 减}

例如,用于多精度算术和校验和的传统 ARM 指令。

\subparagraph*{带有可选增量、否定或反转的条件选择}

有条件地在一个源寄存器和第二个增量、否定、倒置或未修改的源寄存器之间进行选择。

这些是 A32 和 T32 中单个条件指令最常见的用途。
典型的用途包括有条件计数或计算有符号数量的绝对值。

\subparagraph*{条件操作}

有别于 A32 和 T32(大多数指令可以使用条件码预测),A64 只有流程控制类的跳转指令才会使用到条件码。
A64 中使用到条件码的指令,大致可以总结为:

\begin{description}
  \item[条件选择(移动)] 包括 CSEL、CSINC、CSINV 和 CSNEG。
    \begin{itemize}
      \item CSEL 根据一个条件在两个寄存器之间进行选择。
        无条件指令,然后是条件选择,可以取代简短的条件序列。
      \item CSINC 根据一个条件在两个寄存器之间进行选择。
        返回第一个源寄存器或第二个源寄存器增加一个。
      \item CSINV 根据条件在两个寄存器之间进行选择。
        返回第一个源寄存器或倒置的第二个源寄存器。
      \item CSNEG 根据条件在两个寄存器之间进行选择。
        返回第一个源寄存器或被否定的第二个源寄存器。
    \end{itemize}
  \item[条件设置] 有条件地在 0 和 1(CSET)或 0 和 -1(CSETM)之间进行选择。
    例如,用于在一般寄存器中将条件标志设置为布尔值或掩码。
  \item[条件比较](CMP 和 CMN)如果原始条件为真,则将条件标志设置为比较结果。
    如果不是真,条件标志将设置为指定的条件标志状态。
    条件比较指令对于表示嵌套或复合比较非常有用。
\end{description}

\subsubsection{内存访问指令}

与之前的所有 ARM 处理器一样,ARMv8 架构是一个加载 / 存储架构。
这意味着没有数据处理指令直接对内存中的数据进行操作。
数据必须首先被加载到寄存器中,进行修改,然后存储到内存中。
必须在程序中指定一个地址,要传输的数据大小,以及一个源寄存器或目标寄存器。
还有一些额外的加载和存储指令,提供了更多的选择,如非时间性的加载 / 存储,排他性的加载 / 存储,以及获取 / 释放指令等。

内存指令可以以非对齐方式访问普通内存(见 Memory ordering 章节~\ref{sec:memory-ordering})。
但是独占访问、加载获取或存储释放等变种访问方式不支持非对齐方式访问。
如果不需要非对齐访问,可以将上述变种访问方式配置为 exception。

\paragraph{Load}

Load 指令的一般形式如下:

\begin{lstcode}[language={[ARM]Assembler}]
  LDR Rt, <addr>
\end{lstcode}

您可以选择加载数据的大小到整数寄存器中。
例如,要加载一个比指定的寄存器值小的尺寸,可以在 LDR 指令中加入以下对应后缀:
\begin{itemize}
  \item LDRB (8-bit, zero extend)
  \item LDRSB (8-bit, sign extend)
  \item LDRH (16-bit, zero extend)
  \item LDRSH (16-bit, sign extend)
  \item LDRSW (32-bit, sign extend)
\end{itemize}

无需指定 zero-extended(高位扩展为 0)加载指令将数据加载到 X 寄存器,因为向 W 寄存器写数据会自动 zero extend 整个寄存器宽度。

\paragraph{Store}

类似的,存储指令的一般形式如下:

\begin{lstcode}[language={[ARM]Assembler}]
  STR Rn, <addr>
\end{lstcode}

要存储的大小同样可能比寄存器小。
那么您可以添加类似 LDR 指令的后缀到 STR。
在这种情况下,存储的总是寄存器中的最低有效部分。

\paragraph{浮点和 NEON 向量的 Load 和 Store}

Load 和 Store 指令同样支持访问浮点 / NEON 寄存器。
此时,仅由所加载或存储的寄存器(B、H、S、D 或 Q 寄存器中的任意一个)来决定操作大小。
具体情况总结如下表:

加载指令:

\begin{stblr}
  {加载位数}
  {load-bits}
  {cccccccc}
  \hline[1pt]
  Load & Xt & Wt & Qt & Dt & St & Ht & Bt \\
  \hline
  LDR & 64 & 32 & 128 & 64 & 32 & 16 & 9 \\
  LDP & 128 & 64 & 256 & 128 & 64 & - & - \\
  LDRB & - & 8 & - & - & - & - & - \\
  LDRH & - & 16 & - & - & - & - & - \\
  LDRSB & 8 & 8 & - & - & - & - & - \\
  LDRSH & 16 & 16 & - & - & - & - & - \\
  LDRSW & 32 & - & - & - & - & - & - \\
  LDPSW & - & - & - & - & - & - & - \\
  \hline[1pt]
\end{stblr}

存储指令:

\begin{stblr}
  {存储位数}
  {store-bits}
  {cccccccc}
  \hline[1pt]
  Store & Xt & Wt & Qt & Dt & St & Ht & Bt \\
  \hline
  STR & 64 & 32 & 126 & 64 & 32 & 16 & 8 \\
  STP & 128 & 64 & 256 & 128 & 64 & - & - \\
  STRB & - & 8 & - & - & - & - & - \\
  STRH & - & 16 & - & - & - & - & - \\
  \hline[1pt]
\end{stblr}

加载数据到浮点或 NEON 寄存器的指令没有 sign-extension 选项。
并且,地址也是由通用寄存器指定的。

例如:

\begin{lstcode}[language={[ARM]Assembler}]
  LDR D0, [X0, X1]
\end{lstcode}

\begin{Tcbox}[title={注}]
  浮点或向量 NEON 加载和存储指令使用和整型加载和存储指令一样的寻址模式。
\end{Tcbox}

\paragraph{Load 和 Store 指令中的地址指定}

A64 可用的寻址模式与 A32 和 T32 中的相似。
有一些额外的限制以及一些新的功能,但是对于熟悉 A32 或 T32 的人来说,A64 可用的寻址模式并不陌生。

在 A64 中,一个地址操作数的基寄存器必须总是一个 X 寄存器。
但是,有几条零扩展或符号扩展的指令可以使用,来满足通过 W 寄存器来提供 32 位偏移。

\subparagraph{偏移模式}

偏移寻址模式将一个立即数或一个可选择可修改的寄存器值添加到一个 64 位的基寄存器中来产生一个地址。

\begin{stblr}
  {偏移寻址}
  {offset-addressing}
  {c>{\centering\arraybackslash}X}
  \hline[1pt]
  Example instruction & Description \\
  \hline
  \lstinline[language={[ARM]Assembler}]!LDR X0, [X1]! & Load from the address in \lstinline!X1! \\
  \lstinline[language={[ARM]Assembler}]!LDR X0, [X1, #8]! & Load from address \lstinline!X1 + 8! \\
  \lstinline[language={[ARM]Assembler}]!LDR X0, [X1, X2]! & Load from address \lstinline!X1 + X2! \\
  \lstinline[language={[ARM]Assembler}]!LDR X0, [X1, X2, LSL, #3]! & Load from address \lstinline!X1 + (X2 << 3)! \\
  \lstinline[language={[ARM]Assembler}]!LDR X0, [X1, W2, SXTW]! & Load from address \lstinline!X1 + sign_extend(W2)! \\
  \lstinline[language={[ARM]Assembler}]!LDR X0, [X1, W2, SXTW, #3]! & Load from address \lstinline!X1 + (sign_extend(W2) << 3)! \\
  \hline[1pt]
\end{stblr}

通常,当指定移位或扩展选项时,移位量可以是 0(默认值)或 $log_2(access\; size\; in\; bytes)$(因此,$Rn \ll\; <shift>$ 即是 Rn 乘以访问大小)。
所以,偏移寻址支持常见的数组索引操作。

\begin{lstcode}[language=C]
// A C example showing accesses that a compiler is likely to generate.
void example_dup(int32_t a[], int32_t length) {
  int32_t first = a[0];  // LDR W3, [X0]
  for (int32_t i = 1; i < length; i++) {
    a[i] = first;  // STR W3, [X0, W2, SXTW, #2]
  }
}
\end{lstcode}

\subparagraph{索引模式}

索引模式与偏移模式类似,但它们还会更新基地址寄存器。
这里的语法与 A32 和 T32 相同,但操作集的限制性更强。
通常情况下,只能为索引模式提供立即数偏移。

索引模式有两种变体:

\begin{itemize}
  \item 在访问内存之前施加偏移量的预索引模式。
  \item 以及在访问内存之后施加偏移量的后索引模式。
\end{itemize}

\begin{stblr}
  {索引寻址}
  {index-addressing}
  {c>{\centering\arraybackslash}X}
  \hline[1pt]
  Example instruction & Description \\
  \hline
  \lstinline[language={[ARM]Assembler}]{LDR X0, [X1]} & Load from the address in X1 \\
  \lstinline[language={[ARM]Assembler}]{LDR X0, [X1, #8]!} & Pre-index: \textit{Update X1 first} (to X1 + \#8), then load from the new address \\
  \lstinline[language={[ARM]Assembler}]{LDR X0, [X1], #8} & Post-index: \textit{Load from the unmodified address in X1 first}, then update X1 (to X1 + \#8) \\
  \lstinline[language={[ARM]Assembler}]{STP X0, X1, [SP, #-16]!} & Push X0 and X1 to the stack. \\
  \lstinline[language={[ARM]Assembler}]{LDP X0, X1, [SP], \#16} & Pop X0 and X1 off the stack. \\
  \hline[1pt]
\end{stblr}

例如,这些选项准确地映射到一些常见的 C 操作上:

\begin{lstcode}[language=C]
// A C example showing accesses that a compiler is likely to generate.
void example_strcpy(char * dst, const char * src)
{
  char c;
  do {
    c = *(src++);  // LDRB W2, [X1], #1
    *(dst++) = c;  // STRB W2, [X0], #1
  } while (c != '\0');
}
\end{lstcode}

\subparagraph{相对 PC 模式(load-literal)}

A64 添加了另外一个专门用于获取 literal pool 的寻址模式。
Literal pool 是编码成一个指令流的数据块。
它们不会被执行,但是其数据可以通过周围的代码相对于 PC 的内存地址获取到。
Literal pool 经常用于编码常数,而这些常数不能被塞进一个简单的 move-immediate 指令(因为留给立即数的编码位数不足)中。

A32 和 T32 指令集中,PC 寄存器可以当作一个通用寄存器来读,所以可以通过指定 PC 作为基地址寄存器简单的访问 literal pool。
然而,在 A64 下,PC 寄存器不可以用通常的方式获取了,但是提供了一个特殊的寻址模式(只针对 load 指令)来获取 PC 相对地址。
这个特殊作用的寻址模式也极大扩展了 PC 相对加载的范围(相比于 A32 和 T32),所以可以更稀疏地定位 literal pool 。

\begin{stblr}
  {PC 相对寻址}
  {PC-relative-addressing}
  {cc}
  \hline[1pt]
  Example instruction & Description \\
  \hline
  \lstinline[language={[ARM]Assembler}]{LDR W0, <label>} & Load 4 bytes from <label> into W0 \\
  \lstinline[language={[ARM]Assembler}]{LDR X0, <label>} & Load 8 bytes from <label> into X0 \\
  \lstinline[language={[ARM]Assembler}]{LDRSW X0, <label>} & Load 4 bytes from <label> and sign-extend into X0 \\
  \lstinline[language={[ARM]Assembler}]{LDR S0, <label>} & Load 4 bytes from <label> into S0 \\
  \lstinline[language={[ARM]Assembler}]{LDR D0, <label>} & Load 8 bytes from <label> into D0 \\
  \lstinline[language={[ARM]Assembler}]{LDR Q0, <label>} & Load 16 bytes from <label> into Q0 \\
  \hline[1pt]
\end{stblr}

\begin{remark}
  对于所有变体而言,<label> 都必须 4-byte 对齐。
\end{remark}

\paragraph{获取多个地址位置}

A64 中不存在 A32 和 T32 类似的 Load Multiple (LDM) 或 Store Multiple (STM) 指令。
但是添加了 Load Pair (LDP) 和 Store Pair (STP) 指令。

与 A32 LDRD 和 STRD 指令不同的是,LDP 和 STP 可以读写任何两个整型寄存器。
LDP 和 STP 指令从相邻的内存地址读写数据。
这两个指令的内存取址模式的选项比其它内存访问指令更加受限。
它们只能使用一个基地址寄存器附加一个成比例的 7-bit 的有符号立即数,另外一个可选的预地址或后地址增加操作。
与 32-bit 的 LDRD 和 STRD 不同,LDP 和 STP 也可以进行非对齐访问。

\paragraph{非特权访问}

A64 的 LDTR 和 STTR 指令用于非特权级的数据加载和存储操作。

\lstinline!LDTR rt, [Xn{, #i9}]! 相当于 $rt = [Xn += i^{\pm}, <Unpriv>]_N$

\lstinline!STTR rt, [Xn{, #i9}]! 相当于 $[Xn += i^{\pm}, <Unpriv>]_N = rt$

\begin{itemize}
  \item 在 EL0、EL2 或 EL3 中,它们表现为一般的加载或存储指令。
  \item 在 EL1 下执行这些指令,则表现为就像在 EL0 中执行一样。
\end{itemize}
 
这些指令等效于 A32 的 LDRT 和 STRT 指令。

\paragraph{预取内存访问}

\textit{Prefetch from Memory}(PRFM)指令向代码提供了一个给内存系统暗示的功能。
暗示内存系统一个特定的地址将很快就会被当前程序用到。
该功能的效果\textbf{由具体实现定义},典型的实现是将数据或指令加载到 Cache 中。

该指令的语法为:

\lstinline!PRFM <prfop>, <addr> | label!

其中 prfop 是以下选项的拼接:

\begin{description}
  \item[Type] PLD 或 PST (prefetch for load or store)。
  \item[Target] L1、L2 或 L3(以那个 Cache 为目的)。
  \item[Policy] KEEP 或 STRM(保持在 cache 中,还是作为数据流)。
\end{description}

例如,PLDL1KEEP,加载预取 + L1 cache + 保持在 cache 中。

这些指令与 A32 的 PLD 和 PLI 指令类似。

\paragraph{Non-temporal load and store pair}

ARMv8 增加了一个关于 non-temporal 加载和存储的新概念。
相关指令是 LDNP 和 STNP,功能是读写一对寄存器的值。
并且,它们会给内存系统一个暗示,即缓存的数据不靠谱。
该暗示阻止内存系统激活诸如地址的缓存、预加载或合并等功能。
但是,这也无法获得缓存的加速。
一个典型的用例是生产数据流,但是注意高效使用这些指令需要一种微架构的特定方式。

Non-temporal 加载和存储缓解了内存序列化的需求。
在上述例子中,即使 LDNP 指令排在 LDR 指令之后,但是也可能先被观察到。
这种情况会导致从一个存储在 X0 中的不确定的地址中读到数据。
例如:

\begin{lstcode}
  LDR X0, [X3]
  LDNP X2, X1, [X0] // X0 may not be loaded when the instruction executes!
\end{lstcode}

为了解决这个问题,需要放置一个 load barrier 指令。

\begin{lstcode}
  LDR X0, [X3]
  DMB nshld
  LDNP X2, X1, [X0]
\end{lstcode}

\paragraph{内存访问的原子性}

使用单个通用寄存器进行的对齐内存访问保证是原子性的。
使用一对通用寄存器的 Load pair 和 store pair 指令进行内存地址对齐访问确保为两个独立的原子访问。
非对齐访问不具有原子性,因为它们通常需要两次单独的访问。
另外,浮点和 SIMD 内存访问不保证是原子性的。

\paragraph{内存屏障和栅栏指令}

ARMv7 和 ARMv8 都提供了不同的 barrier 操作支持。
这些操作会在~\ref{sec:memory-ordering} 更加详尽的描述。

\begin{itemize}
  \item \textit{Data Memory Barrier} (DMB)。
    该指令强制所有按照程序顺序的早期内存访问变成全局可见后才会执行后续的访问操作。
  \item \textit{Data Synchronization Barrier} (DSB)。
    在程序运行之前,所有挂起的加载和存储指令、Cache 维护指令和所有 TLB 维护指令都必须先完成。
    DSB 类似 DMB,但是附加了其它属性。
  \item \textit{Instruction Synchronization Barrier} (ISB)。
    这个指令刷新 CPU 的流水线和预取 buffer,致使 ISB 后续的指令需要从缓存或内存中预取(或重新预取)。
\end{itemize}

ARMv8 介绍了关于释放一致性模型的单侧 fence 操作。
包括 Load-Acquire (LDAR) 和 Store-Release (STLR) 并且都是基于地址的同步原语。
这两个操作可以作为完整的 fence 成对使用。
它们只支持基地址寄存器的寻址方式,偏移或其它类型的索引寻址都不支持。

\paragraph{同步原语}

ARMv7-A 和 ARMv8-A 架构都提供了独占的内存访问操作。
A64 下是由一对 Load/Store exclusive (LDXR/STXR) 指令提供的。

LDXR 指令从一个内存地址加载一个值,并且尝试静默的给该地址上个互斥锁。
Store-Exclusive 指令只能在获取并持有锁的情况下才能向该地址写入新值。
组合使用 LDXR/STXR 能够构建标准的同步原语,如 spinlock。
一对 LDXRP 和 STXRP 指令允许代码自动更新跨越两个寄存器长度的地址。
可用的选项有 byte、halfword、word 和 doubleword。
这对指令与 Load Acquire/Store Release 成对指令一样,只支持基地址寄存器寻址方式。

不同于 ARMv7 异常入口或返回也能够清除 monitor,CLREX 指令在 ARMv8 下专门用于清除 monitor。
Monitor 也可能虚假的被清除,例如 cache 驱逐操作(evictions)或其它一些与该操作无关的原因。
\textit{软件必须要避免在 LDXR 和 STXR 指令对之间含有内存访问、系统控制寄存器更新或缓存维护指令。}

另外,还有一对 Acquire/Store Release 相关的互斥指令是 LDAXR 和 STLXR。
详见关于同步的章节~\ref{sec:synchronization}。

\subsubsection{流程控制指令}

A64 指令集提供一些不同类型的跳转指令(见下表~\ref{tbl:a64-isa-branch})。
相对于当前地址偏移的简单跳转指令则使用 B 指令。
简单的无条件相对跳转指令可以基于当前地址前后跳跃 128MB。
含有条件码后缀的简单的有条件相对跳转指令的跳转范围只有 $\pm1MB$。

当函数(subroutine)调用时,则需要使用 BL 指令保存返回地址到 link 寄存器(X30),这个指令没有有条件的跳转类型。
除了会保存返回地址(BL 指令的下一条指令)到寄存器(X30)外,BL 指令功能类似 B 指令。

\begin{ltblr}[caption={流程控制指令}, label={tbl:a64-isa-branch}]
  {colspec={c>{\centering\arraybackslash}X}}
  \hline[1pt]
  & Branch 指令 \\
  \hline
  \lstinline!B (offset)! & Program relative branch forward or back 128MB. A conditional version, for example B.EQ, has a 1MB range. \\
  \lstinline!BL (offset)! & As B but store the return address in X30, and hint to branch prediction logic that this is a function call. \\
  \lstinline!BR Xn! & Absolute branch to address in Xn. \\
  \lstinline!BLR Xn! & As BR but store the return address in X30, and hint to branch prediction logic that this is a function call. \\
  \lstinline!RET{Xn}! & As BR, but hint to branch prediction logic that this is a function return. Returns to the address in X30 by default, but a different register can be specified. \\
  \hline
  & Conditional branch 指令 \\
  \hline
  \lstinline!CBZ Rt, label! & Compare and branch if zero. If Rt is zero, branch forward or back up to 1MB. \\
  \lstinline!CBNZ Rt, label! & Compare and branch if non-zero. If Rt is not zero, branch forward or back up to 1MB. \\
  \lstinline!TBNZ Rt, bit, label! & Test and branch if zero. Branch forward or back up to 32kB. \\
  \lstinline!TBNZ Rt, bit, label! & Test and branch if non-zero. Branch forward or back up to 32kB. \\
  \hline[1pt]
\end{ltblr}

除了相对 PC 的跳转指令外,A64 指令集也包含了两个绝对跳转指令。
\lstinline!BR Xn! 指令执行一个到 Xn 寄存器的绝对跳转。
同时,\lstinline!BLR Xn! 也是类似的绝对跳转功能,但是它会保存返回地址到 link 寄存器 X30 中。
\lstinline!RET! 指令行为上类似 \lstinline!BR Xn!,但是它会暗示跳转预测逻辑它执行的是一个函数返回。
尽管 \lstinline!RET! 指令可以指定其它跳转目的寄存器,但是它默认使用 X30 所保存的返回地址。

A64 指令集还包括一些特别的条件跳转指令。
这些指令能在一些情况下提高代码的密度,因为在这些情况下没必要做明确的比较。

\begin{itemize}
  \item \lstinline!CBZ Rt, label! \quad Compare and branch if zero
  \item \lstinline!CBNZ Rt, label! \quad Compare and branch if not zero
\end{itemize}

这些指令先将 32-bit 或 64-bit 源寄存器与 0 进行比较,然后执行条件跳转。
跳转的偏移范围是 $\pm1MB$。
这些指令不会读或写状态码(NZCV)。

有两个类似的检测并跳转的指令:

\begin{itemize}
  \item \lstinline!TBZ Rt, bit, label! \quad Test and branch if Rt<bit> zero
  \item \lstinline!TBNZ Rt, bit, label! \quad Test and branch if Rt<bit> is not zero
\end{itemize}

这些指令检测由立即数指定的 bit 位与源寄存器中的 bit 位是否满足条件(Z 或 NZ),然后根据检测的结果跳转。
这个跳转指令的偏移范围是 $\pm32KB$。
这两个指令与 CBZ/CBNZ 一样不会读写状态码(NZCV)。

\subsubsection{系统控制及其它指令}

\paragraph*{异常处理指令}

A64 有三个异常处理指令用于产生一个异常。
这些指令用于产生一个高异常级的代码调用,分别是 OS(EL1)、Hypervisor(EL2)和 Secure Monitor(EL3):

\begin{itemize}
  \item \lstinline!SVC \#imm16! \quad Supervisor 调用,允许应用程序调用内核代码。
  \item \lstinline!HVC \#imm16! \quad Hypervisor 调用,允许 OS 代码 调用 hypervisor (EL2)。
  \item \lstinline!SMC \#imm16! \quad Secure Monitor 调用,允许 OS 或 hypervisor 调用 Secure Monitor (EL3)。
\end{itemize}

立即数存放到 \textit{Exception Syndrome Register} 里的 handler 中。
这是一个从 ARMv7 的改动,ARMv7 的立即数需要由所调用指令的 opcode 确定。
详情请看章节~\ref{sec:exception}。

若要从异常返回,则使用 \lstinline!ERET! 指令。
该指令从过将 SPSR\_ELn 寄存器复制到 PSTATE 来恢复处理器状态,并且跳转到保存在 ELR\_ELn 中的地址继续运行。

\paragraph*{系统寄存器访问指令}

有两条可以访问系统寄存器的指令:
\begin{itemize}
  \item \lstinline!MRS Xt, <system register>! \quad 读取系统寄存器的值到通用寄存器。

    例如 \lstinline!MRS X4, ELR_EL1 // Copies ELR_EL1 to X4!
  \item \lstinline!MSR <system register>, Xt! \quad 写入通用寄存器中的配置值到系统寄存器。

    例如 \lstinline!MSR SPSR_EL1, X0 // Copies X0 to SPSR_EL1!
\end{itemize}

MSR 或 MRS 也可以单独获取 PSTATE 寄存器中的 field。
例如,选择关联于 EL0 的 Stack Pointer 或当前的异常级:

\begin{lstcode}
  // A value of 0 or 1 in this register is used to select
  // between using EL0 stack pointer or the current exception
  // level stack pointer
  MSR SPSel, #imm
\end{lstcode}

以下两个指令有一些特殊的形式可以用于清除或设置独立的异常掩码位:

\begin{lstcode}
  MSR DAIFClr, #imm4
  MSR DAIFSet, #imm4
\end{lstcode}

\paragraph*{Debug 指令}

A64 有两个调试相关的指令:

\begin{lstcode}
  // Enters monitor mode debug, where there is on-chip debug monitor
  // code
  BRK #imm16
  // Enters halt mode debug, where external debug hardware is connected
  HLT #imm16
\end{lstcode}

\paragraph*{Hint 指令}

HINT 指令可以被合理的视作 NOP,但是注意这些指令受特定实现的影响。

\begin{stblr}
  {Hint 指令}
  {a64-isa-hint}
  {c>{\centering\arraybackslash}X}
  \hline[1pt]
  指令 & 说明 \\
  \hline
  NOP & No operation - not guaranteed to take time to execute \\
  YIELD & Hint that the current thread is performing a task that can be swapped out \\
  WFE & Wait for Event \\
  WFI & Wait for interrupt \\
  SEV & Send Event \\
  SEVL & Send Event Local \\
  \hline[1pt]
\end{stblr}

\paragraph*{NEON 指令}

NEON 指令在 A64 下也有很大加强,有些加强至关重要。
章节~\ref{sec:floating-neon} 有更多的说明细节,下面简单列出 NEON 在 A64 下的改进:

\begin{itemize}
  \item 支持双精度浮点类型,向量化 C 代码下的双精度浮点类型。
  \item 新增将标量数据存储到 NEON 寄存器的指令。
  \item 新增插入和提取向量元素的指令。
  \item 新增类型转换和饱和整型算术运算指令。
  \item 新增浮点数值正规化(normalization)指令。
  \item 新增向量 reduction、summation 和获取最大最小值的 cross-lane 指令。
\end{itemize}

已经扩展了诸如比较、加法、查找绝对值和否定(negate)等指令,得以操作 64-bit 元素。

\paragraph*{浮点指令}

A64 提供了一些类似 ARMv7 VFPv4 扩展的浮点指令,这些指令提供了对标量浮点值之间的单精度和双精度数学运算。
以下列举了一些改进和新特性:

\begin{itemize}
  \item 浮点比较直接设置状态码(NZCV)。
    A64 下不用在特意转换浮点 flag 为整型 flag。
  \item 增添了关于 IEEE754 - 2008 标准的指令,例如计算一对数字的最小值和最大值。
  \item 转换整型数据到浮点型时,可以明确指定为 rounding 模式。
    在特别的 rounding 模式下进行的简单转换,已不再需要设置全局 FPCR flag。
    ARMv8 的 A32 和 T32 也支持其中一些选项。
  \item 新增 64-bit 整型到浮点型格式转换指令。
  \item A64 下,直接在整型寄存器中进行涉及整型的浮点操作。
    进行转换操作时,已不再需要手动转换整型值和浮点值以及整型寄存器。
\end{itemize}

\paragraph*{加密算法指令}

可以通过添加可选的加密算法指令扩展来极大的提升关于 AES 加密、SHA1 和 SHA256 hashing 等任务的效率。

\subsection{浮点与 NEON} \label{sec:floating-neon}

ARM 架构下,软件支持的高级 SIMD 架构的关联实现称为 NEON 技术。
AArch32(相当于 ARMv7 NEON 指令)和 AArch64 都含有 NEON 指令集。
AArch32 和 AArch64 下的 NEON 指令都可以加速大数据量的重复操作。
经典的应用是多媒体数字信号编解码。

AArch64 的 NEON 架构使用 32 个 128-bit 寄存器,是 ARMv7 的两倍。
浮点指令也使用同样的寄存器。
所有编译的代码和子程序都遵循 EABI,EABI 定义了在特殊的子程序中哪些寄存器可以被写入使用(corrupt,可以被破坏意为能被写入使用),哪些寄存器需要保留。
编译器可以自由的在代码任意点使用 NEON/VFP 寄存器保存浮点数和 NEON 数据。
所有的标准的 ARMv8 实现都需要包括浮点和 NEON。
但是,面向特殊用途的实现中可以采取以下组合:
\begin{itemize}
  \item 不实现 NEON 或浮点
  \item 完全实现包含异常的浮点和 SIMD。
  \item 实现不包含异常的浮点和 SIMD。
\end{itemize}

\subsubsection{NEON 和浮点新特性}

AArch64 NEON 是基于 AArch32 NEON,包括以下改动:
\begin{itemize}
  \item 从 16 个寄存器增加到 32 个。
  \item 长度大的寄存器不再是通过长度小的寄存器组合而成,而是将 128-bit 的寄存器的低有效位映射为长度较小的寄存器。
    单精度的浮点数使用 128-bit 寄存器的低 32 位,双精度数使用用低 64 位。
  \item 去掉 V 前缀。
  \item 向向量寄存器中写入小于等于 64 bit 的数据,寄存器的高位则被清零。
  \item AArch64 下不存在使用通用寄存器的 SIMD 或饱和运算指令\footnote{
      饱和指令(Saturating instructions)是一类特殊的指令,用于执行饱和运算。
      饱和运算是一种数学运算,它限制结果在一个特定的范围内,超出范围的值将被截断或限制在范围边界上。

      例如,在使用 8 位无符号整数表示的情况下,如果一个运算结果大于 255(即超出了 8 位无符号整数的范围),则饱和运算将结果限制为 255。
      类似地,如果结果小于 0,则饱和运算将结果限制为 0。

      饱和指令通常用于数字信号处理(DSP)和媒体处理等应用中,这些应用对数据精度和动态范围有严格的要求。
      使用饱和指令可以确保结果不会溢出,并且可以避免由于溢出而引起的意外行为或失真。
    }。
    这些操作全部使用 NEON 寄存器。
  \item 新增 lane 插入和提取指令以支持新的寄存器 pack 模式。
  \item 添加额外的指令用于生产或消费 128-bit 向量寄存器的高 64 bit。
    会产生多于一个寄存器的结果(扩展到 256-bit 向量)或者消耗两个源(收缩成一个向量)的数据处理指令已经划分成不同的指令。
  \item 新增向量 reduction 操作集,提供 across-lane 加法、最小值和最大值操作。
  \item 扩展一些已有的指令以支持 64-bit 整数。
    例如:比较、加法、取绝对值和否定等指令,并且包括 staturating 版本。
  \item 扩展饱和指令,以便在无符号累加和有符号累加之间进行转换(to include Unsigned Accumulate into Signed, and Signed into Unsigned Accumulate)。\footnote{
      表明饱和指令现在具有更多的灵活性,可以在有符号累加和无符号累加之间进行转换,并在结果溢出时执行饱和处理,以确保结果在指定的范围内。
    }
  \item AArch64 NEON 现已支持双精度类型的浮点数和完整的 IEEE754 操作,包括 rounding 模式、非规范化数字和 NaN 处理。
\end{itemize}

AArch64 加强了浮点功能,改动如下(相对于 ARMv7):

\begin{itemize}
  \item 前缀 V 替换成 F。
  \item 先已支持 IEEE754 浮点标准定义的单精度(32-bit)和双精度(64-bit)浮点向量数据类型和运算。
    遵循 FPCR 寄存器中指定的舍入模式来执行浮点数运算\footnote{
    “Honoring the FPCR Rounding Mode field”表示处理器或者软件库遵循浮点控制寄存器(Floating Point Control Register,FPCR)中的舍入模式字段(Rounding Mode field)。

      在 IEEE 754 浮点数标准中,舍入模式指定了在进行浮点数运算时如何处理结果的舍入方式。
      常见的舍入模式包括向最接近的偶数舍入、向正无穷大舍入、向负无穷大舍入、向零舍入等。
    }
    、默认的 NaN 控制、Flush-to-Zero 控制\footnote{
      Flush-to-Zero 控制允许在执行浮点数运算时将非常小的结果舍入为零。
      当启用 Flush-to-Zero 模式时,如果计算得到的结果小于一个设定的阈值(通常是一个非常小的正数,例如 IEEE 754 中的 subnormal number),则结果将被直接截断为零,而不是保留非常小的非零值。
    }
    和异常 trap 使能位(由具体实现所支持)。
  \item FP/NEON 寄存器的 Load/Store 寻址模式与整型的 Load/Store 相统一,包括加载或存储一对浮点寄存器的操作。
  \item 添加与整型 CSEL 和 CCMP 指令等效的浮点 FCSEL 和 选择并比较指令。

    类似 ARMv7,浮点 FCMP、FCMPE、FCCMP 和 FCCMP 指令根据浮点比较结果设置 PSTATE.\{N, Z, C, V\} 标志,但是不会更改 FPSR 寄存器中的状态标志。
  \item 合并所有浮点乘加(Multiply-Add)和乘减(Multiply-Subtract)指令\footnote{
      “Fused Multiply-Add (FMA)”指的是在单个指令中执行乘加运算的能力。
      这意味着指令可以同时进行乘法和加法操作,并且结果是精确的。
      这样可以提高性能,并且在一些情况下可以提高数值计算的精度。

      该变化意味着所有的浮点数乘加和乘减指令都支持 FMA 功能。
      也就是说,这些指令在执行乘加或乘减运算时,都能够同时进行乘法和加法(或减法),而不需要将乘法结果存储到临时变量中再进行加法或减法操作。

      使用 FMA 指令可以提高性能,因为它可以将乘法和加法操作合并为一个指令,并且可以在硬件级别上并行执行这两个操作。
      这样可以减少指令的数量,降低了指令调度和执行的开销,并且可以提高代码的并行性。
    }。

    VFPv4 首次介绍了 Fused multiply,这个功能意味着在执行加法运算前不会近似乘法运算的结果。
    早期的 ARM 浮点架构乘法累加操作对中间结果和最终结果都会进行近似,导致潜在的精度丢失。

  \item 新增转换操作,例如:64-bit 整型和单精度及双精度浮点型数据转换。
    转换浮点到整型数据(FCVTxU、FCVTxS)的指令有如下的有向舍入编码模式:

    \begin{itemize}
      \item[-] 到 0。
      \item[-] 到 $+\inf$。
      \item[-] 到 $-\inf$。
      \item[-] 到一个接近的偶数。
      \item[-] 到更远离零的那个整数\footnote{在“Nearest with ties to away”这种舍入模式下,如果一个值恰好处于两个整数的中间,它会舍入到远离零的整数。
          换句话说,如果一个值恰好处于两个整数的中间,它会舍入到更远离零的那个整数。
        }。
    \end{itemize}
  \item 新增包含相同有向舍入模式并且可根据当前环境进行舍入的浮点向邻近整型舍入的指令(FRINTx)。
  \item 新增不精确的舍入到奇数的双精度到单精度向下转换指令,适合通过正确的近似(FCVTXN)进行现场向下转换到半精度类型。
  \item 添加了 FMINNM 和 FMAXNM 指令,这两个指令用来实现 IEEE754 - 2008 中的操作 minNum() 和 maxNum()。
  如果其中一个操作数是静态 NaN,则返回数值。
  \item 新增浮点向量规范化加速指令(FRECPX 和 FMULX)。
\end{itemize}

\subsubsection{NEON 和 Floating-Point 架构}

NEON 寄存器保存有相同数据类型的元素组成的向量。
一个向量被分成许多 lane,每个 lane 含有一个称为元素(element)的数据值。

NEON 向量的 lane 数量取决于向量的大小和向量中的数据元素。

通常,每个 NEON 指令会产生 n 个并行操作,n 是输入向量分成的 lane 数量。
从一个 lane 到另外一个 lane 不能涉及进位或溢出。
向量的元素顺序是从最低有效位开始,说明元素 0 使用寄存器的最低有效位。

NEON 和 浮点指令操作适用于以下类型的元素:

\begin{itemize}
  \item 32-bit 单精度和 64-bit 双精度浮点类型。
    \begin{Tcbox}[title={Note}]
      16-bit 浮点类型也是支持的,但是只能作为一种被转换的类型,而不能被直接处理。
    \end{Tcbox}
  \item 8-bit、16-bit、32-bit 或 64-bit 的无符号和有符号整型。
  \item 8-bit 和 16-bit 多项式。

    多项式类型用于代码,例如使用 2 的幂有限域\footnote{
      有限域(Finite Field)算术是一种在有限域上进行的数学运算。
      有限域也称为 Galois 域,是一个包含有限数量元素的数学结构。
      在有限域中,加法和乘法运算满足特定的性质,类似于实数域或复数域中的运算。

      有限域中的元素可以是整数模素数的余数,也可以是多项式系数模一个不可约多项式的余数。
      在密码学、编码理论、数字通信等领域,有限域的理论和运算被广泛应用。
    }
    或者在 \{0, 1\} 上的简单多项式的错误纠正。
    通常的 ARM 整型代码一般使用查表的方式进行有限域运算。
    而 AArch64 NEON 则提供了使用巨大查找表的指令。

  \item 多项式运算很难从其他运算中合成出来,因此拥有一个基本的乘法运算非常有用,可以从中合成其他更大的运算。
\end{itemize}

NEON 单元将寄存器文件视作:

分别视作 32 个 128-bit 四字寄存器 V0-V31 为:

\Figure[caption={V 寄存器拆分}, label={fig:v-reg-div}, width=0.95]{divisions-of-the-v-regs}

分别视作 32 个 64-bit 双字寄存器 D0-D31 为:

\Figure[caption={D 寄存器拆分}, label={fig:d-reg-div}, width=0.95]{divisions-of-the-d-regs}

可以在任意时间访问这些寄存器。
由于所使用的指令决定了合适的视角(view),所以软件不需要明确地调整当前该使用哪种寄存器。

\paragraph{Floating-Point}

AArch64 的浮点单元将 NEON 寄存器文件视作:

\begin{itemize}
  \item 32 个 64-bit D 寄存器 D0-D31。
    D 寄存器称为双精度寄存器,可以保存双精度浮点数。
  \item 32 个 32-bit S 寄存器 S0-S31。
    S 寄存器称为单精度寄存器,可以保存单精度浮点数。
  \item 32 个 16-bit H 寄存器 H0-H31。
    H 寄存器称为半精度寄存器,可以保存半精度浮点数。
  \item 以上视角的寄存器合并。
\end{itemize}

\Figure[caption={浮点寄存器的拆分}, label={fig:fp-reg-div}, width=0.95]{fp-reg-div}

\paragraph{标量数据和 NEON}

标量数据是一个单一值,而非包含在向量中的多个值。
一些 NEON 指令使用标量操作数。
寄存器中的标量的获取通过向量的索引。

以通用的数组形式访问向量中的单个元素的格式如下:

\lstinline!<Instruction> Vd.Ts[index1], Vn.Ts[index2]!

其中,

\begin{itemize}
  \item[] Vd 是目的寄存器。
  \item[] Vn 是第一个源寄存器。
  \item[] Ts 用于指定元素的大小。
  \item[] index 是元素的索引。
\end{itemize}

例如:

\lstinline[language={[ARM]Assembler}]!INS V0.S[1], V1.S[0]!

\Figure[caption={插入一个元素到向量}, label={fig:insert-an-element-into-a-vector}, width=0.95]{insert-an-element-into-a-vector}

\lstinline!MOV V0.B[3], W0! 指令操作是将 W0 寄存器中的最低有效 byte 拷贝到 V0 寄存器的第四个 byte 位置中。

\Figure[caption={移动一个标量数到一个 lane}, label={fig:moving-a-scalar-to-a-lane}, width=0.95]{moving-a-scalar-to-a-lane}

NEON 标量可以是 8-bit、16-bit、32-bit 或是 64-bit 数值。
除了乘法指令外,其它指令都可以从寄存器文件中获取任意元素做为标量。

乘法指令只允许 16-bit 或 32-bit 标量,并且只能获取到寄存器文件中的前 128 个标量:

\begin{itemize}
  \item 16-bit 标量限制在寄存器 \lstinline!Vn.H[x]!,其中 $0 \leq n \leq 15$。
  \item 32-bit 标量限制为寄存器 \lstinline!Vn.S[x]!。
\end{itemize}

\paragraph{Floating-Point 参数}

浮点数值通过浮点寄存器传输给函数(或者返回来)。
可以同时使用整型(通用)和浮点寄存器。
也就是说,浮点参数以 H、S 或 D 寄存器传输,其它参数则通过 X 或 W 寄存器。
AArch64 调用标准强制规定只要使用到浮点运算的地方就必须使用硬件浮点运算,所以没有软件相关的浮点运算库。

详细的指令参考《ARMv8-A Architecture Reference Manual》,下面列举了一些浮点数据处理操作:

\begin{ltblr}
  {colspec={c>{\centering\arraybackslash}X}, width=1\textwidth}
  \hline[1pt]
  \lstinline!FABS Sd, Sn! & Calculates the absolute value. \\
  \lstinline!FNEG Sd, Sn! & Negates the value. \\
  \lstinline!FSQRT Sd, Sn! & Calculates the square root. \\
  \lstinline!FADD Sd, Sn, Sm! & Adds values. \\
  \lstinline!FSUB Sd, Sn, Sm! & Subtracts values. \\
  \lstinline!FDIV Sd, Sn, Sm! & Divides one value by another. \\
  \lstinline!FMUL Sd, Sn, Sm! & Multiplies two values. \\
  \lstinline!FNMUL Sd, Sn, Sm! & Multiplies and negates. \\
  \lstinline!FMADD Sd, Sn, Sm, Sa! & Multiplies and adds (fused). \\
  \lstinline!FMSUB Sd, Sn, Sm, Sa! & Multiplies, negates and subtracts (fused). \\
  \lstinline!FNMADD Sd, Sn, Sm, Sa! & Multiplies, negates and adds (fused). \\
  \lstinline!FNMSUB Sd, Sn, Sm, Sa! & Multiplies, negates and subtracts (fused). \\
  \lstinline!FPINTy Sd, Sn! & Rounds to an integral in floating-point format (where y is one of a number of rounding mode options) \\
  \lstinline!FCMP Sn, Sm! & Performs a floating-point compare. \\
  \lstinline!FCCMP Sn, Sm, \#uimm4, cond! & Performs a floating-point conditional compare. \\
  \lstinline!FCSEL Sd, Sn, Sm, cond! & Floating-point conditional select if (cond) Sd = Sn else Sd = Sm. \\
  \lstinline!FCVTSty Rn, Sm! & Converts a floating-point value to an integer value (ty specifies type of rounding). \\
  \lstinline!SCVTF Sm, Ro! & Converts an integer value to a floating-point value. \\
  \hline[1pt]
\end{ltblr}

\subsubsection{AArch64 NEON 指令格式}

AArch64 改动了 NEON 和浮点指令的语法以协调核心整型和标量浮点指令集语法。
这些指令的助记符和 ARMv7 NEON 非常接近。

\begin{itemize}
  \item 移除 ARMv7 NEON 指令中存在的前缀 V。

    重命名了与核心指令集冲突的助记符,并且移除 V 前缀。

    这意味相同名字的指令会做相同的事情,并且可以是核心指令、NEON 指令或者浮点指令,而只是语法上有区别。
    例如:

    \lstinline!ADD W0, W1, W2{, shift #amount}!
    和
    \lstinline!ADD X0, X1, X2{, shift #amount}!

    都是 A64 基础指令。

    \lstinline!ADD D0, D1, D2!

    是一个标量浮点指令。

    \lstinline!ADD V0.4H, V1.4H, V2.4H!

    则是一个 NEON 向量指令。

  \item 指令的前缀上添加了 S、U、F 或 P 前缀,用以表示 Signed、Unsigned、Floating-point 或 Polynomial 数据类型。
    指令根据该助记符来选择相应数据类型的操作。
    例如:

    \lstinline!PMULL V0.8B, V1.8B, V2.8B!

  \item 寄存器修饰符描述向量寄存器的组织形式(元素大小和 lane 数量)。
    例如:

    \lstinline!ADD Vd.T, Vn.T, Vm.T!

    其中,Vd、Vn 和 Vm 都是寄存器的名字,而 T 则是将被使用到的寄存器细分。
    就此例而言,T 是做为排列说明符 8B、16B、4H、8H、2S、4S 或 2D 中的一个。
    任意上述的排列说明符都可以使用,取决于数据类型的宽度(64、32、16 或 8-bit)以及寄存器的位宽(64 bit 或 128 bit)。

    若要对 2 个 64-bit 的 lane 相加,则是用

    \lstinline!ADD V0.2D, V1.2D, V2.2D!

  \item ARMv7 中,一些 NEON 数据处理指令存在 Normal、Long、Wide、Narrow 和 Staturating 变种。
    Long、Wide 和 Narrow 变种以后缀标识:

    \begin{itemize}
      \item[-] \textit{Normal 指令}可以对任何向量类型进行操作,并生成与操作数向量相同大小且通常相同类型的结果向量。
      \item[-] \textit{Long 指令}或 \textit{Lengthening 指令}对双字向量操作数进行操作并产生四字向量结果。
        结果元素的宽度是操作数的两倍。
        长指令使用附加到指令的 L 来指定。
        例如:

        \lstinline!SADDL V0.4S, V1.4H, V2.4H!

        下图展示了该操作,输入操作数在运算之前被提升。

        \Figure[caption={NEON Long 指令}, label={fig:NEON-long-insts}, width=0.6]{NEON-long-insts}

    \end{itemize}

  \item \textit{Wide} 或 \textit{Widening 指令}对双字和四字向量操作数进行操作,产生四字向量。
    产生的结果元素(第一个操作数)的长度是第二个操作数元素长度的两倍。
    Wide 指令含有一个 W 后缀。
    例如,

    \lstinline!SADDW V0.4S, V1.4H, V2.4S!

    下图展示该操作,输入的双字操作数在运算前被提升。

    \Figure[caption={NEON Wide 指令}, label={fig:NEON-wide-insts}, width=0.6]{NEON-wide-insts}

  \item \textit{Narrow} 或 \textit{Narrowing 指令}对四字向量进行操作,并且生成双字向量。
    生产的元素通常为操作数元素长度的一半。
    Narrow 指令用 N 后缀指定。
    例如,

    \lstinline!SUBHN V0.4H, V1.4S, V2.4S!

    下图展示该操作,输入操作数在运算前被降级。

    \Figure[caption={NEON Narrow 指令}, label={fig:NEON-narrow-insts}, width=0.6]{NEON-narrow-insts}

  \item 一些指令存在 Signed 和 unsigned staturating 变种(以 SQ 或 UQ 前缀标识),比如 SQADD 和 UQADD。
    如果结果超出了数据类型的最大值或最小值,那么 saturating 指令则返回其最大值或最小值。
    Saturation 的限制取决于指令所用的数据类型。

    \begin{stblr}
      {Saturation ranges}
      {NEON-saturation-ranges}
      {cc}
      \hline[1pt]
      Data type & Saturation range of x \\
      \hline
      Signed byte (S8) & $-27 \leq x < 27$ \\
      Signed halfword (S16) & $-215 \leq x < 215$ \\
      Signed word (S32) & $-231 \leq x < 231$ \\
      Signed doubleword (S64) & $-263 \leq x < 263$ \\
      Unsigned byte (U8) & $0 \leq x < 28$ \\
      Unsigned halfword (U16) & $0 \leq x < 216$ \\
      Unsigned word (U32) & $0 \leq x < 232$ \\
      Unsigned doubleword (U64) & $0 \leq x < 264$ \\
      \hline[1pt]
    \end{stblr}

  \item ARMv7 中示意为 pairwise 操作的 P 前缀现已在 ARMv8 中改成后缀,比如 ADDP。
    Pairwise 指令操作相邻的双字或四字操作数对。
    例如:

    \lstinline!ADDP V0.4S, V1.4S, V2.4S!

    \Figure[caption={Pairwise 操作}, label={fig:pairwise-op}, width=0.6]{pairwise-op}

  \item 添加 V 后缀到 across-all-lanes(整个寄存器)操作,比如 ADDV。
    例如:

    \lstinline!ADDV S0, V1.4S!

    \Figure[caption={Across all lanes 操作}, label={fig:across-all-lanes-op}, width=0.6]{across-all-lanes-op}

  \item 为新的加宽、缩小或加长的第二部分指令添加了 2 后缀,称为第二和上半部分说明符。
    如果存在该后缀,那么相应的操作会保持较窄元素的寄存器的高 64 位上执行操作。

    \begin{itemize}
      \item[-] 带有 2 后缀的加宽指令从包含较窄值的向量的高编号通道获取输入数据,并将扩展结果写入 128 位目标。
        例如:

        \lstinline!SADDW2 V0.2D, V1.2D, V2.4S!

        \Figure[caption={SADDW2}, label={fig:saddw2-inst}, width=0.6]{saddw2-inst}

      \item[-] 带有 2 后缀的 Narrowing 指令从 128-bit 的源操作数获取它们的输入数据,并且将缩小后的结果插入到 128-bit 目的寄存器的高序 lane 中,低序 lane 保持不变。
        例如:

        \lstinline!XTN2 V0.4S, V1.2D!

        \Figure[caption={XTN2}, label={fig:xtn2-inst}, width=0.6]{xtn2-inst}

      \item[-] 带有 2 后缀的 Lengthening 指令从 128-bit 源向量寄存器的高序 lane 中获取它们的输入数据,并将加长的结果保存到 128-bit 目的寄存器中。
        例如:

        \lstinline!SADDL2 V0.2D, V1.4S, V2.4S!

        \Figure[caption={SADDL2}, label={fig:saddl2-inst}, width=0.6]{saddl2-inst}
    \end{itemize}

  \item 比较指令已通过使用状态码来表明当前状态,以及当前状态是有符号的还是无符号的。
    例如,CMGT 和 CMHI、CMGE 和 CMHS。

\end{itemize}

\subsubsection{NEON 编码替代方案}

NEON 编码有很多种形式。
本节只是简单的列举(详细内容可查看《ARM NEON Programmers Guide》)。
其中包括内联函数的使用、C 代码的自动矢量化、库的使用,当然还有直接用汇编语言编写。

内联函数是编译器用适当的 NEON 指令替换的 C 或 C++ 伪函数调用。
这允许您使用 NEON 实现中可用的数据类型和操作,同时允许编译器处理指令调度和寄存器分配。
这些 intrinsic 函数定义在 ARM C 语言扩展文档里。

ARM 编译器 6 下,Auto-vectorization 由 \lstinline!-fvectorize! 选项配置,然而高级优化中该选项是自动打开的(\lstinline!-O2! 和更高)。
只要 \lstinline!-O0! 选项开启,无论指不指定 \lstinline!-fvectorize! Auto-vectorization 都是关闭的。
因此您需要在 \lstinline!-O1! 优化下通过以下命令开启 auto-vectorization:

\lstinline!armclang --target=armv8a-arm-none-eabi -fvectorize -O1 -c file.c!

有多个支持 NEON 代码的软件库。
这些库随时间动态更新,因此其状态也无法把控,所以当前的支持情况并没有列举到该文档中。

虽然技术上而言,手动优化 NEON 汇编代码是可行的,但是由于流水线和内存访问时序的内部复杂依赖,手动优化是非常困难的。
ARM 强烈推荐直接使用 intrinsic 调用函数,而不是手写汇编。

\begin{itemize}
  \item 使用 instrinsic 函数比使用汇编助记符更简单。
  \item Instrinsic 函数提供了很好的跨平台可移植性。
  \item 使用 instrinsic 不需要关心内部流水线和内存访问时序。
  \item 大多数情况下能够或得很好的性能。
\end{itemize}

如果您不是一个经验丰富的汇编开发者,那么使用 intrinsic 通常能获得更好的性能。
Intrinsic 提供了和直接用汇编编写代码一样丰富的控制,但是把分配寄存器的任务分配给了编译器,因此您可以专注于算法。
这样会比汇编语言具有更好的代码维护性。


\subsection{ABI}

ARM 架构的 ABI(Application Binary Interface)指定了所有可执行本机代码模块都必须遵循的最基础规则,
只有严格遵循这些规则,这些执行程序才能正确的共同工作。
特定的编程语言(如 C++)会补充一些额外的规则。
操作系统或执行环境(例如 Linux)也会添加一些规则以满足它们特定需求。
不过这些额外的规则超出了 ARM 架构的 ABI。

AArch64 架构有若干 ABI 组成元素:

\begin{description}
  \item[Executable and Linkable Format (ELF)]
    AArch64 架构的 ELF 指定 object 和 执行文件的格式。
  \item[Procedure Call Standard (PCS)]
    AArch64 函数调用标准发行指定可以分开写出、编译并链接多少个能够一起工作的子例程。
    它指定调用例程和被调用者之间的约定,或者一个例程和它的执行环境之间的约定。
    例如,当调用一个例程或者栈空间分布时的要求。
  \item[DWARF]
    DWARF 是一个广泛使用的调试数据格式标准。
    AArch64 的 DWARF 基于 DWARF 3.0 之上添加了一些额外规则。
    详情查看 DWARF for the ARM 64-bit Architecture (AArch64)。
  \item[C and C++ libraries]
    ARM Compiler ARM C and C++ Libraries 和 Floating-Point Support User Guide 描述了 ARM C 和 C++ 库。
  \item[C++ ABI]
    C++ Application Binary Interface Standard for the ARM 64-bit Architecture 描述通用的 C++ ABI。
\end{description}

\subsubsection{通用寄存器中的参数}

通用寄存器分成了 4 组来满足函数调用:

\begin{description}
  \item[Argument registers] 包括 X0-X7。
    这些寄存器用于给函数传参并返回结果。
    在函数内部和调用其它函数之间,它们可做为暂存寄存器或调用保存寄存器变量来保存函数内部的临时数值。
    由于 AArch64 上将传参寄存器的个数增加到了 8 个,所以相比 AArch32 而言,减少了对栈的使用率,从而提升了函数调用的性能。
  \item[Caller-saved temporary registers] 包括 X9-X15。
    如果调用函数想要在调用其它函数之后保留这些寄存器的值,那么它必须要将调用其它函数后会受到影响的寄存器保存到自己的栈空间中。
    被调用的函数在修改这些寄存器并返回调用函数前,不需要考虑保存和恢复它们。
    也就是说调用函数必须保存这些寄存器,以确保寄存器的数值正确。
  \item[Callee-saved registers] 包括 X19-X29。
    这些寄存器是被调用函数需要保存的。
    只要子例程确保在修改这些寄存器之前保存这些寄存器的值,并在返回调用函数之前恢复这些寄存器的值即可。
  \item[Registers with a special purpose] 包括 X8, X16-X18, X29, X30。

    \begin{itemize}
      \item 
        X8 是间接结果寄存器,用于传输间接结果地址。
        例如,当一个函数返回一个大型结构体时。
      \item 
        % X16 and X17 are IP0 and IP1, intra-procedure-call temporary registers.
        % These can be used by call veneers and similar code, or as temporary
        % registers for intermediate values between subroutine calls. They are
        % corruptible by a function. Veneers are small pieces of code which are
        % automatically inserted by the linker, for example when the branch target is
        % out of range of the branch instruction.
        X16 和 X17 做为程序内调用临时寄存器 IP0 和 IP1。
        用于调用 veneers \footnote{
          Veneers 是一些被链接器自动插入的微小代码块,比如,当跳转目的地址超出跳转指令的适用范围时。
        }和类似代码,或者做为子例程调用之间的临时变量保存寄存器使用。
        这些寄存器容易被函数破坏。
      \item 
        X18 是平台寄存器并且保留给平台的 ABI 使用。
        这是一个没有赋予特殊意义的平台附加临时寄存器。
      \item 
        X29 用作 FP(frame pointer register)。
      \item 
        X30 用作 LR(link register)。
    \end{itemize}

\end{description}

下图展示了 64-bit X 寄存器。
其它寄存器参数信息可以参考相关章节。

\Figure[caption={ABI 中的通用寄存器}, label={fig:general-purpose-regs}, width=0.95]{general-purpose-regs}

\subsubsection{间接结果地址}

重申一下,X8(XR)寄存器用于传输间接结果地址。
以下是相关例子:

\begin{lstlisting}[
  language=C,
  caption={间接地址},
  label={lst:indirect-result-location}
]
//test.c//
struct struct_A
{
  int i0;
  int i1;
  double d0;
  double d1;
} AA;

struct struct_A foo(int i0, int i1, double d0, double d1)
{
  struct struct_A A1;
  
  A1.i0 = i0;
  A1.i1 = i1;
  A1.d0 = d0;
  A1.d1 = d1;
  
  return A1;
}

void bar()
{
  AA = foo(0, 1, 1.0, 2.0);
}
\end{lstlisting}

下面通过以下命令获取到汇编代码:

\begin{lstlisting}
  armclang -target aarch64-arm-none-eabi -c test.c
  fromelf-c test.o
  # 当然,以交叉编译链 aarch64-linux-gnu-gcc 编译也是可行的,为了引述文档说明,默认使用上面的编译命令。
  aarch64-linux-gnu-gcc -O0 -S test.c
\end{lstlisting}

\begin{Tcbox}[title={Note}]
  为了演示该机理,上面的代码要以没有优化的选项编译代码,否则编译器可能会把相关细节优化掉。
\end{Tcbox}

\begin{lstlisting}[
  language={[ARM]Assembler},
]
foo//
  SUB SP, SP, #0x30
  STR W0, [SP, #0x2C]
  STR W1, [SP, #0x28]
  STR D0, [SP, #0x20]
  STR D1, [SP, #0x18]
  LDR W0, [SP, #0x2C]
  STR W0, [SP, #0]
  LDR W0, [SP, #0x28]
  STR W0, [SP, #4]
  LDR W0, [SP, #0x20]
  STR W0, [SP, #8]
  LDR W0, [SP, #0x18]
  STR W0, [SP, #10]
  LDR X9, [SP, #0x0]
  STR X9, [X8, #0]
  LDR X9, [SP, #8]
  STR X9, [X8, #8]
  LDR X9, [SP, #0x10]
  STR X9, [X8, #0x10]
  ADD SP, SP, #0x30
  RET
bar//
  STP X29, X30, [SP, #0x10]!
  MOV X29, SP
  SUB SP, SP, #0x20
  ADD X8, SP, #8
  MOV W0, WZR
  ORR W1, WZR, #1
  FMOV D0, #1.00000000
  FMOV D1, #2.00000000
  BL foo:
  ADRP X8, {PC}, 0x78
  ADD X8, X8, #0
  LDR X9, [SP, #8]
  STR X9, [X8, #0]
  LDR X9, [SP, #0x10]
  STR X9, [X8, #8]
  LDR X9, [SP, #0x18]
  STR X9, [X8, #0x10]
  MOV SP, X29
  LDP X20, X30, [SP], #0x10
  RET
\end{lstlisting}

在这个例子中,结构体包含 16 个 byte。
根据 AArch64 的 AAPCS,结构体对象会通过 XR 返回其内存地址。

所产生的代码展示出:

\begin{itemize}
  \item W0, W1, D0 和 D1 用于传递整型和双精度型参数。
  \item bar() 函数使用栈空间保存 foo() 函数 返回的结构体值,并将 sp 存入 X8 中。
  \item 在 foo() 函数拿到地址进行后续操作之前,bar() 函数传递 X8 以及通过 W0、W1、D0 和 D1 传递其他的参数到 foo() 函数。
  \item foo() 函数可能会破坏 X8,所以 bar() 使用 SP 获取返回结构体。
\end{itemize}

使用 X8(XR)寄存器的优势是它并未降低用以传递函数参数的寄存器可用性。
AAPC64 的栈帧如下图所示。
帧指针 X29 指向保存在堆栈上的上一帧指针,LR(X30)保存在上一帧指针之后。
链上的最后一个帧指针应该设置为 0。
栈指针必须 16 byte 对齐。
堆栈框架的确切布局可能存在一些变化,特别是可变参数或无帧的函数。
具体细节可以参考 AAPCS64 文档。

\Figure[caption={栈帧}, label={fig:stack-frame}, width=0.4]{stack-frame}

\begin{Tcbox}[title={Note}]
  AAPCS 只指定了 FP 和 LR 寄存器块布局以及如何把这些块链到一起。
  上图的其它内容(包括两个函数帧之间的边界的精确位置)是未定义的,这些内容由编译器自由定义。
\end{Tcbox}

图~\ref{fig:stack-frame} 说明了一个栈帧使用了两个被调函数保存的寄存器(X19 和 X20)和一个临时变量,布局如下(左边的数字代表相对于 FP 的偏移):

\begin{lstlisting}
  40: <padding>
  32: temp
  24: X20
  16: X19
   8: LR'
   0: FP'
\end{lstlisting}

为了维持栈指针的 16 byte 对齐,填补(padding)是必要的。

\begin{lstlisting}[
  language={[ARM]Assembler},
]
  function:
  STP X29, X30, [SP, #-48]! // Push down stack pointer and store FP and LR
  MOV X29, SP               // Set the frame pointer to the bottom of the new
                            // frame
  STP X19, X20, [X29, #16]  // Save X19 and X20

  /*
   * Main body of code
   */

  LDP X19, X20, [X29, #16] // Restore X19 and X29
  LDP X29, X30, [SP], #48  // Restore FP' and LR' before setting the stack
                           // pointer to its original position
  RET                      // Return to caller
\end{lstlisting}

\subsubsection{NEON 和 浮点寄存器中的参数}

AArch64 架构还有 32 个 NEON 和 浮点操作寄存器 V0 - V31。
以寄存器名称的不同来表明访问数据的大小。

\begin{Tcbox}[title={Note}]
  与 AArch32 不同,AArch64 中的 128-bit 和 64-bit 视角的 NEON 和 浮点寄存器不会在较窄的视角下重叠多个寄存器,所以 q1、d1 和 s1 都是寄存器 bank 中的同一条目。
\end{Tcbox}

\Figure[caption={ABI 中的 SIMD 和浮点寄存器}, label={fig:simd-fp-reg-abi}, width=0.95]{simd-fp-reg-abi}

\begin{itemize}
  \item V0 - V7 用于函数调用的参数传递以及从函数调用中返回值。
    在函数中,这些寄存器也会用于保存临时值(但是通常只用于函数调用)。
  \item V8 - V15 在函数调用期间必须由被调函数保护。
    只有低 64 bit 需要保护。
  \item V16 - V31 不需要保护(或者可以由调用函数保护)。
\end{itemize}

\subsection{异常处理} \label{sec:exception}

严格来讲,中断将打断软件的运行流程。
然而,在 ARM 术语中,这实际是异常。
异常是一些状态和系统事件,这些事件需要特权软件(异常处理函数)采取一些措施来保证系统的正常运行。
每类异常都由相关的异常处理函数与其关联。
一旦异常被处理完成,特权软件将恢复 CPU 核到异常前的执行状态。

异常分为以下几类:

\BlockDesc{Interrupts}

系统有两种中断类型为 IRQ 和 FIQ。

FIQ 优先级高于 IRQ。
这两种异常类型通常与 CPU 核的输入引脚相连。
假定没有关闭中断的情况下,外部硬件发起一个中断线路请求,并且在当前的指令执行完成后(虽然一些指令可以加载多个数值,但是可以被中断),引发对应的异常类型。
FIQ 和 IRQ 是 CPU 核的物理信号,当触发这些信号时,CPU 核便会响应关联的异常(中断使能的情况下)。
绝大多数的系统中,各种中断源都是用中断控制器连接的。
中断控制器对中断进行仲裁和优先级排序,每次提供一个串行化的信号给连接到 CPU 核的 FIQ 或 IRQ 信号口。
由于中断的发生与 CPU 在任意时刻执行的软件没有直接的关系,所以它们被分类为异步异常。
详情可见于 \textit{The Generic Interrupt Controller} 章节。

\BlockDesc{Aborts}

Abort 可能由取指令错误或者数据访问错误产生。
Abort 异常可能从外部的内存系统的数据访问错误产生(表面指定的地址与系统的真实内存不对应)。
另外,Abort 也可以通过 CPU 核的 MMU 产生。
操作系统使用 MMU abort 来给应用程序动态分配内存。

在取指令的过程中,指令在 pipeline 中可被标注为 abort 异常。
指令 abort 仅发生在 CPU 核尝试执行指令的时候。
异常发生在指令执行之前。
如果 pipeline 在会产生 abort 的指令进入 pipeline 的执行阶段前被刷新了,那么系统不再产生该 abort 异常。
数据 abort 由 load 或 store 指令产生,并且发生在系统尝试读写数据之后。

如果 Abort 异常由指令流的执行或尝试执行所产生,那么可以描述为同步异常,并且返回地址提供了产生异常的详细信息。

异步的 abort 异常不是由执行指令所产生,同时返回地址也不一定会提供产生 abort 的详细原因。
ARMv8-A 架构下,指令和数据 abort 是同步的。
异步异常是 IRQ/FIQ 和系统错误(System errors, SError)。

\BlockDesc{Reset}

复位(reset)被视为所实现的最高异常级别的特殊向量。
该向量即为,当发起该异常时 ARM 处理器将跳转的指令位置。
\lstinline!RVBAR_ELn! 保存该 reset 向量地址,其中 n 是所实现的最高异常级的序号。

所有核都有一个 reset 入口并且当它们被复位后立即采取复位异常。
该异常的优先级是系统内最高的,并且不能被屏蔽。
该异常用于上电后执行核上的初始化代码。

\BlockDesc{Exception generating instructions}

某些指令的执行也会产生异常。
一般,执行这些指令用于从运行在更高异常级的软件中获取其提供的服务(例如,系统调用服务)。

\begin{itemize}
  \item SVC(Supervisor Call)指令开启了用户模式程序向操作系统请求服务的功能。
  \item HVC(Hypervisor Call)指令开启了客户操作系统向虚拟机监控程序请求服务的功能。
  \item SMC(Secure monitor)指令开启了普通程序向安全程序请求服务的功能。
\end{itemize}

如果所产生的异常是由于 EL0 下取指而生成的,那么它将被视为 EL1 下的异常。
除非在非安全状态下设置了 \lstinline!HCR_EL2.TGE! 位,那么它将被带入 EL2 下。

如果所产生的异常是由其它异常级下取指所产生的,那么异常级保持不变。

本文档的前面讲述了 ARMv8-A 架构有 4 个异常级。
处理器的运行只能通过进入或返回一个异常的方式切换运行级。
当处理器从更高的异常级切换到更低的异常级时,运行状态可以保持不变,或者处理器可以从 AArch64 切换到 AArch32。
相反,处理器从更低的异常级切换到更高的异常级时,运行状态可以保持不变,或者处理器可以从 AArch32 切换到 AArch64。

\Figure[caption={异常执行流程}, label={fig:exception-flow}, width=0.6]{exception-flow}

上图示意出在应用程序运行过程中发生异常的程序流程。
处理器跳转到包含所有异常类的入口的向量表。
向量表包含通常表明有异常原因的调度代码,并且调度代码选择调用合适的函数处理所处异常。
调度代码执行完成后并返回到高级的处理函数,该函数执行 ERET 指令返回应用程序。

\subsubsection{异常处理寄存器}

如果发生异常,PSTATE 信息将被保存到 \lstinline!SPSR_ELn!(Saved Program Status Register)寄存器中,系统有 \lstinline!SPSR_EL3!、\lstinline!SPSR_EL2! 和 \lstinline!SPSR_EL1!。

\Figure[caption={AArch64 下的 SPSR}, label={fig:aarch64-spsr}, width=0.95]{aarch64-spsr}
\Figure[caption={AArch32 下的 SPSR}, label={fig:aarch32-spsr}, width=0.95]{aarch32-spsr}

% The SPRSR.M field (bit 4) is used to record the execution state (0 indicates AArch64 and 1
% indicates AArch32).
\lstinline!SPSR.M! 字段(bit 4)用于记录运行状态(0 表明 AArch64,1 为 AArch32)。

\begin{stblr}
  {PSTATE 字段}
  {pstate-field}
  {c>{\centering\arraybackslash}X}
  \hline[1pt]
  字段 & 说明 \\
  \hline
  NZCV & Condition flags \\
  Q & Cumulative saturation bit \\
  DAIF & Exception mask bits \\
  SPSel & SP selection (EL0 or ELn), not applicable to EL0 \\
  E & Data endianness (AArch32 only) \\
  IL & Illegal flag \\
  SS & Software stepping bit \\
  \hline[1pt]
\end{stblr}

The exception bit mask bits (DAIF) allow the exception events to be masked. The exception is
not taken when the bit is set.

\begin{description}
  \item[D] Debug exceptions mask.
  \item[A] SError interrupt Process state mask, for example, asynchronous External Abort.
  \item[I] IRQ interrupt Process state mask.
  \item[F] FIQ interrupt Process state mask.
\end{description}

SPSel 字段选择当前的该使用哪个异常级栈指针 \lstinline!SP_EL0!。
切换栈指针可以在除了 EL0 之外的任何异常级进行。
后续章节会讨论如何切换。

设置 IL 字段会导致执行下一条指令以触发异常。
该字段使用非法执行的返回,例如,尝试从 AArch64 返回 EL2 而实际系统运行状态是 AArch32。

SS(Software Stepping)字段在调试相关章节讲解。
调试器使用它执行一条指令然后对后续的指令发起一个调试异常。

当采取一个异常时,一些分开的字段(CurrentEL、DAIF 和 NZCV 等等)以一种紧凑的方式拷贝到 \lstinline!SPSR_ELn!(返回时相反)。

当以事件引发一个异常,处理器会自动采取一些特定的动作。
这些动作有更新 \lstinline!SPSR_ELn!(n 是异常发生时所处异常级)、
保存正确从异常返回时所需的 PSTATE 信息、
更新 PSTATE 以反应处理器的新状态(可能意味着异常级的发起或保持不变)以及
将从异常所需的返回地址保存到 \lstinline!ELR_ELn!。

\Figure[caption={异常处理}, label={fig:exception-handling}, width=0.5]{exception-handling}

记住,\lstinline!_ELn! 后缀表示这些寄存器在不同异常级的多个备份。
例如,\lstinline!SPSR_EL1! 与 \lstinline!SPSR_EL2! 是不同的物理寄存器。
另外,在同步或 SError 异常的情况下,也会把 \lstinline!ESR_ELn! 更新为表面异常产生原因的值。

软件通过执行 ERET 指令来通知处理器何时从异常返回。
执行 ERET 将从 \lstinline!SPSR_ELn! 恢复异常前的 PSTATE 值并且从 \lstinline!ELR_ELn! 恢复之前的程序运行地址。

上面已经讲述了 SPSR 如何为异常返回记录必要的状态信息。
下面将继续讲述 link 寄存器如何存储程序地址信息。
架构架构为函数调用和异常返回提供了单独的链接寄存器。

我们已经在 A64 指令集相关章节了解到 X30 寄存器用于从子例程中返回(通过 RET 指令)。
一旦我们执行了跳转链接(branch with link)指令(BL 或 BLR),那么返回值将被更新到 X30 寄存器中。
而 \lstinline!ELR_ELn! 寄存器用于存储从异常返回的地址。
当进入一个异常时,该寄存器值将被硬件自动更新,并且当执行 ERET 指令返回时,该寄存器的值会写入到 PC 中。

\begin{Tcbox}[title={Note}]
  当从异常返回时,如果 SPSR 的值与系统寄存器的设置冲突,那么您将看到一个错误。
\end{Tcbox}

\lstinline!ELR_ELn! 保存有指定异常类型的返回地址。
对于一些异常而言,这个返回值是发生异常指令地址的下一条指令地址。
例如,当一个 SVC 指令执行时,我们希望返回时能执行下一条指令。
在其他情况下,我们可能希望重新执行发生异常的指令。

对于异步异常来说,\lstinline!ELR_ELn! 指向由于获取中断而尚未执行或完全执行的第一条指令的地址。
系统允许处理函数代码修改 \lstinline!ELR_En!,例如,如果有必要返回到中止同步异常之后的指令地址。
ARMv8-A 模型已经明显比 ARMv7-A 简单了,然而处于兼容性的原因,从特定类型的异常中返回时仍然需要从链接寄存器中减去 4 或 8。

除了 SPSR 和 ELR 寄存器,每个异常级都包含其专用的栈指针寄存器。
这些寄存器被命名为 \lstinline!SP_EL0!、\lstinline!SP_EL1!、\lstinline!SP_EL2! 和 \lstinline!SP_EL3!。
这些寄存器用于指向专用栈,这些专用栈则可用于保存一些会被异常处理函数破坏的寄存器,那么便可以在异常返回时恢复这些被保存的寄存器。

异常处理代码可能会切换 \lstinline!SP_ELn! 为 \lstinline!SP_EL0!。
例如,\lstinline!SP_EL1! 可能指向一块小型的内核可以确保永久有效的栈内存上。
\lstinline!SP_EL0! 则可能指向一块大型的内核无法确保安全的(栈溢出)任务栈上。
通过控制 SPSel 字段进行切换,方法如下:

\begin{lstlisting}[
  language={[ARM]Assembler},
]
MSR SPSel, #0  // switch to SP_EL0
MSR SPSel, #1  // switch to SP_ELn
\end{lstlisting}

\subsubsection{同步和异步异常}

AArch64 中,异常分为同步和异步异常。
同步异常是由于尝试执行指令所引起的,并且返回地址包含导致异常的指令的详细信息。
异步异常不是指令的执行所引起的,返回地址也不一定会包含产生异常的信息。

异步异常是由 IRQ、FIQ 或 SError 产生的。
有不少可能导致系统错误异常的产生原因,其中最常见的是异步数据中止(例如 cache line 中的脏数据写回外部内存所触发的中止)。

同步异常的产生源头有:

\begin{itemize}
  \item MMU 产生的指令中止。
    比如,指令的内存地址设置为不可执行。
  \item MMU 产生的数据中止。
    例如,访问权限不足或对齐检查。
  \item SP 和 PC 对齐检查。
  \item 同步的外部中止。
    比如,读取也表时发生的中止。
  \item 未分配的指令
  \item 调试异常。
\end{itemize}

\BlockDesc{同步中止}

有很多产生同步异常的原因:

\begin{itemize}
  \item MMU 产生
  \item SP 和 PC 对齐检查
  \item 未分配的指令
  \item 服务调用指令(SVC、SMC 和 HVC)
\end{itemize}

操作系统可以用这些异常实现一些正常功能。
例如,在 Linux 中,一个 task 请求分配新的内存页的功能就是通过 MMU 中止机制。

ARMv7-A 架构中,预取中止、数据中止和未定义异常是分开的条目。
但是在 AArch64 中,这些异常会统一产生一个同步中止。
异常处理函数读取 syndrome 和 FAR 寄存器获取必要的信息来区分这些异常。

\BlockDesc{处理同步异常}

异常处理函数通过读取一些寄存器来获取异常产生的原因等信息。
其中,\lstinline!ESR_ELn!(Exception Syndrome Register)提供异常的产生原因。
\lstinline!FAR_ELn!(Fault Address Register)保存有关于同步指令、数据中止和对齐错误相关的出错的虚拟地址。

\lstinline!ELR_ELn!(Exception Link Register)保存产生数据访问中止时的指令地址(Data Abort)。
这通常在内存错误之后更新,但也会在其它情况下设置,比如跳转到一个非对齐的地址。

如果将异常从使用 AArch32 的异常级别转移到使用 AArch64 的异常级别,那么异常将以目标异常级的方式写 FAR 寄存器,也就是说 \lstinline!FAR_ELn! 的高 32 bit 会全部设成 0。

对于实现了 EL2(Hypervisor)或 EL3(Secure Kernel)的系统来说,同步异常通常由当前或者更高的异常级接管。
异步异常可以被路由到更高异常级处理。
\lstinline!SCR_EL3! 寄存器指定哪些异常被路由到 EL3。
同样的,\lstinline!HCR_EL2! 指定哪些异常被路由到 EL2。
IRQ、FIQ 和 SError 由一些单独的 bit 来独立控制。

\BlockDesc{系统调用}

一些指令和系统函数只能在特定的异常级下执行。
如果运行在一个较低的异常级下的代码需要采取特权操作,例如,当应用程序向内核请求功能时。
一种方式是使用 SVC 指令,该指令允许应用程序产生一个异常。
系统调用的参数则通过通用寄存器传入,或者编码在系统调用中。

\BlockDesc{EL2/EL3 的系统调用}

与应用程序向 EL1(内核)请求系统调用类似,请求 EL2 和 EL3 的服务同样有相应指令,分别为 HVC 和 SMC。
当处理器运行在 EL0 下,它无法直接向 EL2 提交服务请求,只能通过 SVC 先向内核提交请求,然后通过 EL1 下的内核来向 EL2 发起请求。

运行在 EL1 下的 OS 内核可以通过 HVC 指令调用 hypervisor 服务,也可以通过 SMC 调用安全监控服务。
如果处理器实现了 EL3,那么 EL2 能够捕获来自 EL1 的 SMC 指令。
如果没有 EL3,那么 SMC 不会被分配并且只在当前异常级触发。

同样,EL2 下的代码可以使用 SMC 调用 EL3 代码。
如果 SVC 指令在 EL2 或 EL3 下执行,那么它仍然会同一异常级产生一个同步异常,并且异常级下的处理函数来决定如何响应。

\BlockDesc{未分配的指令}

未分配的指令产生一个同步中止,当处理器执行以下指令时产生该类异常:

\begin{itemize}
  \item 没有分配操作码的指令
  \item 需要更高特权级执行的指令
  \item 禁用的指令
  \item PSTATE.IL 字段设置后的任意指令
\end{itemize}

% -----
\BlockDesc{The Exception Syndrome Register (ESR)}

\lstinline!ESR_ELn! 寄存器保存异常的信息,可由异常处理函数从中获取异常产生的原因。
它只会在发生同步异常和 SError 异常时更新。
它不会在产生 IRQ 或 FIQ 时更新,IRQ 和 FIQ 的状态信息保存在 GIC 的寄存器中(详情查阅 \textit{The Generic Interrupt Controller})。
ESR 寄存器的编码如下:

\begin{description}
  \item[31:26] 异常类,异常处理函数可用以区分异常类别(例如未分配的指令、从 MCR/MRC 到 CP15 所产生的异常、FP 操作异常、SVC、HVC、SMC、数据中止和对齐异常)。
  \item[25] 异常指令的长度(0 代表 16-bit 指令;
    1代表 32-bit 指令),并且设置给特定异常类。
  \item[24:0] 形成 Instruction Specific Syndrome (ISS) 字段,包含该类异常的信息。
    例如:当系统调用指令(SVC、HVC 或 SMC)执行时,该字段包含指令的立即数,比如当执行 \lstinline!SVC 0x123456! 时该字段包含 0x123456。
\end{description}

\subsubsection{异常引起的执行状态和异常级的变化}

异常产生时,处理器可能改变运行状态也可能保持不变。
例如,当处理器运行在 AArch32 状态时,外部中断源可能引发中断,但是 OS 内核的中断处理函数运行在 AArch64 状态。

SPSR 寄存器会保存处理器的运行状态,以便异常返回时切换回去。
这个切换的动作是由处理器自动完成的,但是每个异常级别的运行状态的控制是通过以下方式完成的:

\begin{itemize}
  \item 硬件配置输入决定最高异常级别的初始运行状态(不一定为 EL3)。
    但是,并没有固定,因为有 \lstinline!RMR_ELn! 寄存器可以在处理器运行时改变最高异常级的运行状态。
    EL3 是安全监控代码,它是一个有特定运行状态的安全的小型代码块。
  \item EL2 和 EL1 的运行状态由 \lstinline!SCR_EL3.RW! 和 \lstinline!HCR_EL2.RW! 位。
    \lstinline!SCR_EL3.RW! 在 EL3 下配置控制 EL2 的运行状态。
    \lstinline!HCR_EL2.RW! 在 EL2 或 EL3 下配置控制 EL1/0 的运行状态。
  \item EL0 不会处理异常(只运行应用程序代码)。
\end{itemize}

如下图所示,考虑当应用程序正在 EL0 上运行时,发生了 IRQ 中断的情况。
而内核的 IRQ 处理函数运行于 EL1。
中断发生时,处理器会决定如何设置运行状态。
它通过查看更高异常级(EL2)的控制寄存器的 RW 位。
这个例子中它会查看 \lstinline!HCR_EL2.RW!。

\Figure[caption={到 EL1 的异常}, label={fig:exception-to-el1}, width=0.4]{exception-to-el1}

那么接下来还要考虑异常级别的切换。
同样的,异常发生时异常级别可以切到更高或保持不变。
老生常谈,EL0 不处理异常。

同步异常通常由当前异常级别或更高的异常级别处理。
然而,异步异常可以被路由到更高异常级。
对于安全代码来说,\lstinline!SCR_EL3! 可用于指定哪些异常可以路由到 EL3。
对于 hypervisor 代码而言,\lstinline!HCR_EL2! 可用于指定哪些异常可以路由到 EL2。

在所有的路由情况下,IRQ、FIQ 和 SError 都由独立的字段来控制。
处理器只会把异常带入到它被路由的异常级去处理。
异常不可能到比它出发的异常级低的级别去处理。
中断只能在其处理的异常级才能被屏蔽。

将异常从 AArch32 带入到 AArch64 时,需要做一些特殊考虑。
AArch64 处理代码可能需要访问 AArch32 的寄存器,因此架构将寄存器映射到 AArch32 上来对其进行访问。

AArch32 寄存器的 R0 到 R12 对应 X0 到 X12。
AArch32 不同模式的所有储备 SP 和 LR 寄存器对应于 X13 到 X23,储备 的 FIQ 寄存器 R8 到 R12 对应 X24 到 X29。
这些寄存器的 \verb![63:32]! 位在 AArch32 模式下无法获取,其值为 0 或者为上次在 AArch64 下写入的值。
架构不保证这些高位值,所以一般都以 W 寄存器去访问。

\subsubsection{AArch64 的异常向量表}

当异常发生时,处理器必须执行可以处理该异常的处理函数代码。
异常处理代码保存的内存地址称作异常向量。
ARM 架构下,异常向量存在一个表中,称为异常向量表。
除了 EL0 外,每个异常级都有一个异常向量表。
向量表中保存异常发生时要被执行的代码,而不是一些地址。
不同异常的向量被放在表中的固定地址偏移上。
分别通过向量基地址寄存器 \lstinline!VBAR_EL3!、\lstinline!VBAR_EL2! 和 \lstinline!VBAR_EL1! 设置每个表的虚拟地址。

向量表中的每个条目长度为 32 指令(单个指令为 4-byte)长度。
这相对 ARMv7 的向量表条目只有 4 byte 来说,有了重大的改变。
对于 ARMv7 来说,一个 4-byte 的条目只能存放一个跳转类型的指令,跳转到一个真正的异常处理函数地址上。
而 AArch64 的异常向量的空间足够大,可以直接在异常向量中直接处理异常。

下表展示了一个异常向量表。
基地址由 \lstinline!VBAR_ELn! 设置,而每个条目都在基于该地址的偏移上。
每个表含有 16 个条目,每个条目为 128 byte(32 条指令)。
向量表由 4 组 4 条目组成,使用哪个条目则由以下因素决定:

\begin{itemize}
  \item 异常类型(SError、FIQ、IRQ 和 同步异常)
  \item 如果异常在同一异常级产生,那么使用栈指针(SP0 或 SPx)
  \item 如果异常在较低的异常级产生,那么决定较低一级的执行状态(AArch64 或 AArch32)
\end{itemize}

\begin{ltblr}[caption={异常向量表}, label={tbl:exp-table}]
  {colspec={c>{\centering\arraybackslash}X>{\centering\arraybackslash}X}}
  \hline[1pt]
  地址 & 异常类型 & 描述\\
  \hline
  VBAR\_ELn+0x000  & Synchronous & \SetCell[r=4]{c} Current EL with SP0\\
  +0x080 & IRQ/vIRQ & \\
  +0x100 & FIQ/vFIQ & \\
  +0x180 & SError/vSError & \\
  \hline
  +0x200 & Synchronous & \SetCell[r=4]{c} Current EL with SPx\\
  +0x280 & IRQ/vIRQ & \\
  +0x300 & FIQ/vFIQ & \\
  +0x380 & SError/vSError & \\
  \hline
  +0x400 & Synchronous & \SetCell[r=4]{c} Lower EL using AArch64\\
  +0x480 & IRQ/vIRQ & \\
  +0x500 & FIQ/vFIQ & \\
  +0x580 & SError/vSError & \\
  \hline
  +0x600 & Synchronous & \SetCell[r=4]{c} Lower EL using AArch32\\
  +0x680 & IRQ/vIRQ & \\
  +0x700 & FIQ/vFIQ & \\
  +0x780 & SError/vSError & \\
  \hline[1pt]
\end{ltblr}

举个例子来理解上述的异常条目的选择。

如果内核代码正在 EL1 下运行,而此时正好发生了一个中断 IRQ,该中断并未关联到更高的异常级(hypervisor 或 secure 环境),那么该中断也是由 EL1 下的内核代码来处理(内核此时的栈是 \lstinline!SP_EL1!,并且设置在 SPSel 字段,那么异常处理代码也使用 \lstinline!SP_EL1!)。
所以,异常发生时的跳转地址即为 \lstinline!VBAR_EL1 + 0x280!。

由于 ARMv8 中不存在 \lstinline!LDR PC, [PC, #offset]! 指令,所以程序员必须使用更多的指令才能从寄存器表中读取到目的地址。
异常向量的间距的设计旨在避免未使用的向量对典型大小的指令高速缓存线造成高速缓存污染。
Reset 的地址是由\textbf{设计决定的}完全独立的地址,通常是硬件固化配置。
该地址可由 \lstinline!RVBAR_EL1/2/3! 等寄存器获取。

无论当前的异常级还是较低的异常级中每个异常都有独立的异常向量的设计给 OS 或 hypervisor 带来灵活性,即可以灵活的决定较低异常级的运行状态(AArch64 还是 AArch32)。
\lstinline!SP_ELn! 用于较低异常级产生的异常。
但是,软件可以在异常处理代码中切换到 \lstinline!SP_EL0!。
使用此机制有助于从处理程序的线程中访问变量值。

\subsubsection{中断处理}

ARM 通常使用中断(interrupt)来表示中断信号。
在 ARM A-profile 和 R-profile 处理器上,表示外部的 IRQ 或 FIQ 中断信号。
架构并未指定这些信号的使用方法。
FIQ 一般保留为安全中断源。
在早期的架构中,FIQ 和 IRQ 用于表示高中断优先级和标准中断优先级,但是 ARMv8-A 已经不是这种概念了。

当处理器在 AArch64 运行状态下产生一个异常时,所有的 PSTATE 中断屏蔽字段都会被自动设置。
这意味后面的异常都被关闭掉了。
如果软件想要支持嵌套异常,例如运行一个更高优先级的中断打断正在处理的低级的中断,那么软件必须明确的重新使能这些中断。

使用的指令如下:

% \noindent
{
  \lstinline!MSR DAIFClr, #imm!
}

立即数是一个 4-bit 长度的字段,表示:

\begin{itemize}
  \item PSTATE.A (SError)
  \item PSTATE.D (Debug)
  \item PSTATE.I (IRQ)
  \item PSTATE.F (FIQ)
\end{itemize}

\Figure[caption={C 中断处理示意}, label={fig:irq-hdl-in-c}, width=0.5]{irq-hdl-in-c}

一个汇编语言编写的 IRQ 处理函数类似为:

\begin{lstlisting}[
  language={[ARM]Assembler},
  caption={汇编 IRQ handler},
  label={lst:asm-irq-hdl}
]

IRQ_Handler
                          // Stack all corruptible registers
STP X0, X1, [SP, #-16]!   // SP = SP -16
//...
STP X2, X3, [SP, #-16]!   // SP = SP - 16
                          // unlike in ARMv7, there is no STM instruction and
                          // so we may need several STP instructions
BL read_irq_source        // a function to work out why we took an interrupt
                          // and clear the request
BL C_irq_handle           // the C interrupt handler
                          // restore from stack the corruptible registers
LDP X2, X3, [SP], #16     // S = SP + 16
LDP X0, X1, [SP], #16     // S = SP + 16
//...
ERET
\end{lstlisting}

然而,从性能考虑,下面的代码会更好一些:

\begin{lstlisting}[
  language={[ARM]Assembler},
  caption={汇编 IRQ handler (better)},
  label={lst:asm-irq-hdl-b}
]

SUB SP, SP, #<frame_size>   // SP = SP - <frame_size>
STP X0, X1, [SP}            // Store X0 and X1 at the base of the frame

STP X2, X3, [SP]            // Store X2 and X3 at the base of the frame + 16
//...                       // bytes more register storing
//...
                            // Interrupt handling
BL read_irq_source          // a function to work out why we took an interrupt
                            // and clear the request
BL C_irq_handler            // the C interrupt handler
                            // restore from stack the corruptible registers
LDP X0, X1, [SP]            // Load X0 and X1 at the base of the frame
LDP X2, X3, [SP]            // Load X2 and X3 at the base of the frame + 16
//...                       // bytes more register loading
ADD SP, SP, #<frame_size>   // Restore SP at its original value
//...
ERET 
\end{lstlisting}

\Figure[caption={处理嵌套中断}, label={fig:hdl-nest-irqs}, width=0.8]{hdl-nest-irqs}

嵌套的处理函数需要更多代码。
它必须将 \lstinline!SPSR_EL1! 和 \lstinline!ELR_EL1! 保存到栈上。
处理代码中也必须在确定(以及清除)中断来源后重新使能所有 IRQ。
然而(与 ARMv7-A 不同),用于子例程调用的链接寄存器和异常使用的链接寄存器是不同的,所以必须要避免对 LR 和 模式做任何特殊的操作。

\begin{probsolu}[title={Problem and Solution \theprob}]{
  如何处理嵌套中断?写出处理嵌套中断的代码。
  (可以从 FreeRTOS、RT-Thread 等代码中找到参考)
  }\label{pb:how-to-hdl-nest-irqs}

  经查阅,RT-Thread 在进入中断处理函数之前保存了浮点寄存器 Q0-Q31、通用寄存器 X0-X30、FPCR、FPSR、SP\_EL0、SPSR\_EL1 和 ELR\_EL1 到栈上。
  最后将栈保存到 X0 寄存器中。

  具体可查阅 RT-Thread 代码 \lstinline!aarch64/common/context_gcc.S!。
\end{probsolu}

\subsubsection{GIC}

ARM 提供了一个标准的中断控制器可用于 ARMv8-A 系统中。
该中断控制器的编程接口定义在 GIC 架构中。
GIC 架构定义了多个版本,本文档主要介绍 GICv2。
ARMv8-A 处理器与 GIC 连接,例如,GIC-400 或 GIC-500 等。
GIC 支持在多核之间路由软件产生的、私有和共享的外设中断。

\subsection{Caches}

ARM 架构开发初期,处理器的速度和访问内存的速度大体相近。
当今的处理器核心的运行速度要比内存访问速度快上许多。
然而,外部总线和存储设备的频率并没有达到与处理器相同的程度,所以可以在芯片内部实现一些与处理器内核同等速度的 RAM。
不过这些 RAM 与 DRAM 比起来非常昂贵,相比起来,DRAM 一般拥有数千倍以上的容量。
ARM 架构下,访问外部 ram 通常需要数十个处理器周期,甚至数千个。

Cache 是一个小型的快速的内存块,它存在于主内存和处理器之间,保存有主内存中的数据缓存。
访问 Cache 的速度比访问外存快很多。
处理核需要读取或者写入一个特定地址时,先去查看 Cache 里面有没有保存该地址的数据。
如果 Cache 找到该地址,那么它就使用 Cache 中的数据,并不再访问主内存。
由于减少了对外部内存的访问次数所以大大增加了系统的潜在性能。
另外,由于避免了驱动外部信号所以也减少了系统的功耗。

\Figure[caption={Cache 的基本组成}, label={fig:basic-cache-arrangement}]{basic-cache-arrangement}

ARMv8 架构处理器一般有 2 - 3 级 cache,通常每个 CPU 核都含有较小的一级 I-Cache 和 D-Cache。
基于 Cortex-A53 和 Cortex-A57 的处理器一般有 2 级及以上的 Cache,即较小的一级 I-Cache 和 D-Cache,以及更大的 Cluster 共享的二级 Cache。
另外,还可以增加 Cluster 之间共享的更大的 3 级 Cache。

系统初始状态访问数据不会得到加速,此时 Cache 也需要填充从内存中读取的数据。
然而,当 Cache 中缓存了一定量的数据,系统再次需要这些数据时,Cache 的加速功能便显现出来。
处理器在取指令或数据读写时先去检查 Cache,但是有一部分内存地址必须要标记为不可缓存的,比如外设寄存器地址空间。
由于 Cache 中只能缓存一小部分主存数据,所以必须要想出一种高效的方式来快速检测数据是否在 Cache 中。

有时,Cache 中的数据和外部主存的数据会出现不一致。
可能的情况有,处理器更新了 Cache 中的数据,但是还没有写回主存;
或者,某个硬件机构在处理器获取主存的数据后更改了主存中的数据,例如 DMA。
这种数据一致性的问题将在后续章节继续讲述。
这种问题在多核和有 DMA 的情况下特别需要注重。

\subsubsection{Cache 术语}

Von Neumann 架构定义单个 Cache 用于指令和数据的缓存(统一 Cache)。
Harvard 架构更新了该种形式,提出 I-Cache 和 D-Cache 分开缓存指令和数据的形式。
ARMv8 架构则在 L1 Cache 中使用 Harvard 形式,更高级使用统一的 Cache。

Cache 需要保存地址、数据和一些状态信息。

下图简略描述了 Cache 的术语及基础框架。

\Figure[caption={Cache 术语}, label={fig:cache-terminology}]{cache-terminology}

\begin{itemize}
  \item \textit{tag} 是保存在 Cache 中的部分内存地址,这部分地址用于连系内存地址和 Cache line。
    64 位内存地址的高字段告诉 Cache 数据信息在内存中的来源,该高字段即称为 \textit{tag}。
    总缓存大小衡量的是它可以保存的数据量,但用于保存 \textit{tag} 的 RAM 不包括在计算中。
    然而,tag 仍然是占用物理空间的。
  \item 如果每个 tag 下只存放一个 word,那么 tag 所占的比重就太大了,导致 Cache 的利用率很低。
    所以就定义了每个 tag 下可以存放 \textit{Cache line} 大小的数据。
    Cache line 是 Cache 可以提供的最小数据量,并且 Cache line 中保存的数据是内存中连续的数据 word。
    当 Cache line 中存有指令或数据时,称为有效(valid),反之则为无效(invalid)。

    Cache line 数据相关联有一个或多个状态位。
    通常,有一个有效位表明 Cache line 是否存在有效数据,意味着 tag 中含有真是数据。
    D-Cache 中也可能含有污染(dirty)位,用以表明该 Cache line 中的数据(或部分数据)与主存中数据是否不同(比其更新)。
  \item \textit{way} 是 Cache 的细分,每个 way 都具有相同的大小,相同的索引方式。
    \textit{set} 由所有 way 中的所有具有相同索引方式的 line 组成。
  \item 地址中的低字段(做为 offset),则不需要保存到 tag 中。
    Cache 给出的是一行(cache line)数据,所以地址中有 5 到 6 个最低有效位恒为 0(也就是说 cache line 有 $2^5$ - $2^6$ 个 byte)。
\end{itemize}

\BlockDesc{组(Set)相连 Cache 和 way}

ARM 架构下的 Cache 基本都是组相连的,这种类型的 Cache 极大的减少了直接映射型 Cache 的抖动(thrashing)问题\footnote{
Where main memory is accessed in a pattern that leads to multiple main memory locations competing for the same cache lines, resulting in excessive cache misses.
This is most problematic for caches that have low associativity.
 },从而提升程序的执行速度并且增加了更多可预测的执行。
不过,该种类型的 Cache 也会增加硬件的复杂度和略微增加功耗,这是由于每个 cycle 下会比较多个 tag。

在这种类型的 Cache 组织方式下,Cache 被分成许多大小相同的块,这些块就是 \textit{way}。
那么内存地址便可以映射到 way,而不是 line 上。
可以使用地址中的索引字段继续选择特定的 line,但是此时它指向的是每个 way 中的单独 line。
通常,L1 Cache 中有 2 - 4 路,Contex-A57 则有 3-way 的 L1 I-Cache。
而 L2 Cache 通常有 16 路。

外部的 L3 Cache 的大小更大,所以有更多路,因此有更高的关联性。
具有相同索引值的 line 属于同一 set,于是检查是否命中需要查看 set 中的所有 tag。

下图是一个 2-way 相连的 Cache。
主存中的 0x00、0x40 和 0x80 可能出现在其中一路中,但是不会同时存在于两路中。

\Figure[caption={2-way 组相连 Cache}, label={fig:2-way-set-associative-cache}]{2-way-set-associative-cache}

关联性的增加(即增加 way 数)可以降低 thrashing 的概率。
最理想的情况是全相连 Cache,即主存和 Cache 一一映射,每个地址都能找到对应。
如果不是想缓存很少的数据来说(例如 TLB),构建这样的 Cache 是不现实的。
实际上,8-way 以上的对与性能来说提升很小,16-way 的关联性对于更大的 L2 才更有用。

\BlockDesc{tag 和 物理地址}

一个 tag 对应一条 line,用于记录关联于 line 的外部内存物理地址。
Cache line 的大小是由\textbf{实现决定的},但是所有的 CPU 核都具有相同的 Cache line 大小,因为这些核之间是相互连系的。

要访问的物理地址决定数据缓存在 Cache 中的位置。
最低有效字段用于选择 Cache 中的相关条目。
中间字段用于选择 set 中的特定 line。
最高有效字段标识地址的剩余部分,并与保存的 tag 进行比较。
ARMv8 下数据 Cache 通常是 \textit{Physically Indexed, Physically Tagged}(PIPT),也可以是 non-aliasing \textit{Virtually Indexed, Physically Tagged}(VIPT)。

Cache line 包含以下成员:

\begin{itemize}
  \item 关联物理地址中的 tag 值。
  \item 表明 line 是否存在的有效位,亦即 tag 是否有效。
    如果缓存在多个核心之间是一致的,有效位也可以是 MESI 状态的状态位。
  \item 表明 Cache line 与外部内存不一致的 dirty 位。
\end{itemize}

一个简单的 4 路组相连的 32KB L1 D-Cache 如下所示,该 Cache 的 line 大小为 16-word(64 byte)。

\Figure[caption={32KB 4-way D-Cache}, label={fig:32KB-4-way-D-Cache}]{32KB-4-way-D-Cache}

\BlockDesc{包容或排它 Cache}

包容型 Cache 即数据可以同时存在与 L1 和 L2 Cache 中。
排它型 Cache 即数据只能同时存在于单级 Cache,同一地址不可能同时在 L1 和 L2 中出现。

\subsubsection{Cache 控制器}

Cache 控制器负责管理 Cache 缓存,它对于应用程序来说几乎是透明的。
它自动的将内存中的代码或数据写入到 Cache 中。
它从处理器核处获取内存读写请求并且执行必要的 Cache 缓存或外部内存操作。

它收到处理核的请求命令后,首先检查该请求地址是否在 Cache 中,这个步骤称为 \textit{cache look-up}。
它将请求中的部分地址位与 Cache line 对应的 tag 进行比对来完成 \textit{cache look-up} 操作。
如果上述对比成功,那么说明 Cache 中存在匹配 line,称为命中(hit),并且控制器将该行标注为有效(valid),从而使用 Cache 缓存作为对写请求的数据来源。

当读写请求无法从 Cache 中找到匹配的 tag,或者 tag 是无效的,那么即产生 Cache 未命中(miss)。
此时,请求必须传给下一级内存层级 L2 Cache,或外部内存。
这将产生一个 cache line 填充(cahe linefill),即将内存中的数据拷贝到 cache 中。
请求的数据或指令流也会同时发送给处理核。
该过程对于软件开发人员是透明的,是不直接可见的。
处理核不用等到 cache line 填充完数据才能使用。
Cache 控制器通常先从 cache line 中获取关键字(\textit{critical word})。
例如,当执行一个加载指令而发生 cahce miss 并产生一个 cache linefill 时,CPU 核先从 cache line 中获取部分请求的数据。
这部分关键数据先提供给 CPU 核的流水线(pipeline),然后 Cache 和外部总线接口在后台读取 cache line 中剩余的数据。

\subsubsection{缓存策略}

Cache 策略简而言之就是,数据 Cache 何时需要申请 line;
存储指令执行时并且数据 Cache 命中后应该发生什么事情。

Cache 的分配机制有以下几种:

\begin{description}
  \item[Write allocation (WA)] 在写未命中时分配 cache line。
    这意味 CPU 执行存储指令时会发生读操作。
    在写操作执行前,cache 会先获取数据填充对应的 cache line。
    即使只是写入一个 byte,Cache 中也会保存一个完整行,因为这是它的最小单元。
  
  \item[Read allocation (RA)] 在读为命中时分配 cache line。
\end{description}

Cache 更新机制有:

\begin{description}
  \item[Write-back (WB)] 写时更新 Cache,并且只会将 cache line 标记为污染(dirty)。
    外部内存只在该行被驱逐出去或明确的清理操作发生时更新。
  
  \item[Write-through (WT)] 写时更新 Cache 和外部内存。
    该机制不会标注 cache line 为污染。
\end{description}

两种更新机制的数据读操作命中时的 Cache 行为相同。

普通内存的可缓存属性分为 \textit{inner} 和 \textit{outer} 属性。
至于这两种属性的范围是\textbf{由实现所决定的}。
一般来说,inner 属性用于集成 Cache(CPU 核内集成的 Cache),outer 属性可在处理器内存总线上供外部 Cache 使用。

\Figure{cacheable-properties-of-memory}

处理器可以预测式的访问普通内存,也就是说处理器可以自动暗中加载数据到 cache 中。
这个行为并不需要开发人员的干涉,也无需指定请求地址。
当然,开发人员也可以暗示处理器哪些数据将被使用。
ARMv8 提供的预加载暗示(hint)指令。
但是 cache 是否支持预测和预加载是由\textbf{实现决定}。
下面的指令是架构提供的:

\begin{description}
  \item[AArch64] \lstinline!PRFM PLDL1KEEP, [Xm, #imm]! 意思是将 \lstinline!Xm + offset! 的数据加载到 L1 缓存中作为临时预取,这意味着数据可能会被使用多次。
  \item[AArch32] \lstinline!PLD Rm! 将数据从 Rm 中的地址预加载到缓存。
\end{description}

A64 预取内存指令更一般的形式:

\begin{lstcode}
  PRFM <prfop>, addr

  Where:

  <prfop>     <type><target><policy> | #uimm5
  <type>      PLD for prefetch for load
              PST for prefetch for store
  <target>    L1 for L1 cache, L2 for L2 cache, L3 for L3 cache
  <policy>    KEEP for retain or temporal prefetch means allocate in cache
              normally
              STRM for streaming or non-temporal prefetch means the memory
              is used only once
  uimm5       Represents the hint encodings as a 5-bit immediate.
              These are optional.
\end{lstcode}

\subsubsection{POC 和 POU}

对于基于 set 和基于 way 的 cache 清理和 invalidate 来说,操作是施加到指定 cache 级上的。
对于使用虚拟地址的操作,该架构定义两个观察点。

\begin{description}
  \item[\textit{Point of Coherency} (PoC)] 对于一个特定的地址,PoC 是所有观察者的点,例如,处理器、DSP 或 DMA 引擎等可以访问内存的单元,这些单元都确保可以在一个内存地址下看到同样的数据。
    通常这个观察点为外部内存。
  \item[\textit{Point of Unification} (POU)] PoU 是指令和数据 Cache 和 TLB 在内存地址可以看到的相同数据。
    例如,L2 Cache 即为 Harvard 架构的一级 Cache 和 TLB 可见的相同点。
    如果没有 L2 等外部 cache,那么内存则为 PoU。
\end{description}

了解 PoU 可以确保在取指修改后的代码时的正确性。
可以通过以下两个阶段的来处理:

\begin{itemize}
  \item 通过地址来清理相关的数据 cache 条目。
  \item 通过地址来无效指令 cache 条目。
\end{itemize}

即使对于共享内存地址来说,ARM 架构也不需要硬件来确保指令 cache 和内存的一致性。

\subsubsection{Cache 维护}

有时在软件中清除和无效 cache 是必要的。
在外部内存已经发生改变这种操作可能就是必要的,并且这时需要从 cache 中把陈旧的数据移除掉。
在 MMU 相关的活动后也有必要进行上述的 cache 操作,比如:改变内存访问权限、cache 策略或虚拟地址到物理地址的映射,或者诸如 JIT 和动态库加载产生动态代码时必须要进行指令和数据 cache 同步。

\begin{itemize}
  \item 无效 cache 或 cache line 意味着清除它的数据,即清除 cache line 中的有效位。
    \textbf{初始化后必须要无效 cache,因为此时数据是未定义的。}
    这也可以被看作是一种使缓存外的内存域中的更改对缓存的使用者可见的方式。
  \item 清理缓存或缓存行意味着将标记为污染的数据写入到下一级缓存或主内存中,并清除缓存行中的污染位。
    这使得缓存行的内容与下一级缓存或内存系统保持一致。
    这仅适用于使用写回策略的数据缓存。
    这也是使缓存中的更改对外部内存域的用户可见的一种方式,但仅适用于数据缓存。
  \item 清零。
    这会在缓存内将一块内存清零,而无需先从外部域读取其内容,仅适用于数据缓存。
\end{itemize}

您可以选择哪些条目需要应用于这些操作:

\begin{description}
  \item[All] 指整个缓存,不适用于数据缓存或统一缓存。
  \item[\textit{Modified Virtual Address} (MVA)] 是虚拟地址 (VA) 的另一种名称,是指包含特定虚拟地址的缓存行。
  \item[Set or Way] 指通过其在缓存结构中的位置选定的特定缓存行。
\end{description}

AArch64 缓存维护操作使用具有以下一般形式的指令执行:

\lstinline!<cache> <operation>{, <Xt>}!

有多种操作可用。

\begin{ltblr}[caption={数据、指令和统一缓存的操作}, label={tbl:diu-cache-ops}]
  {colspec={cc>{\centering\arraybackslash}Xc}}
  \hline[1pt]
  Cache & 操作 & 描述 & AArch32 等效\\
  \hline
     & CISW & Clean and invalidate by Set/Way & DCCISW \\ 
     & CIVAC & Clean and Invalidate by Virtual Address to Point of Coherency & DCCIMVAC \\
     & CSW & Clean by Set/Way & DCCSW \\
  DC & CVAC & Clean by Virtual Address to Point of Coherency & DCCMVAC \\
     & CVAU & Clean by Virtual Address to Point of Unification & DCCMVAU \\
     & ISW & Invalidate by Set/Way & DCISW \\
     & IVAC & Invalidate by Virtual Address, to Point of Coherency & DCIMVAC \\
  \hline
  DC &  ZVA & Cache zero by Virtual Address & - \\
  \hline
     & IALLUIS & Invalidate all, to Point of Unification, Inner Sharable & ICIALLUIS \\
  IC & IALLU & Invalidate all, to Point of Unification, Inner Shareable & ICIALLU \\
     & IVAU & Invalidate by Virtual Address to Point of Unification & ICIMVAU \\
  \hline[1pt]
\end{ltblr}

那些接受地址参数的指令使用一个 64 位寄存器,该寄存器保存要维护的虚拟地址。
该地址没有对齐限制。
接受组 / 路 / 级别参数的指令,使用一个 64 位寄存器,其低 32 位遵循 ARMv7 架构中描述的格式。
AArch64 数据缓存失效指令,\lstinline!DC IVAC!,需要写入权限,否则会生成权限故障。

所有指令缓存维护指令可以以任何顺序执行,相对于其他指令缓存维护指令、数据缓存维护指令和加载存储指令,除非在指令之间执行了 DSB。

除了 \lstinline!DC ZVA! 之外的数据缓存操作,如果它们指定了一个地址,只有在它们指定相同地址时,才保证按程序顺序执行。
指定地址的那些操作,相对于不指定地址的所有维护操作,按程序顺序执行。

考虑以下代码。

\begin{lstlisting}[
  language={[ARM]Assembler},
  caption={对于 PoU 的 cache 无效和清理操作},
  label={lst:cache-ic-pou}
]
IC IVAU, X0  // Instruction Cache Invalidate by address to Point of Unification
DC CVAC, X0  // Data Cache Clean by address to Point of Coherency
IC IVAU, X1  // Might be out of order relative to the previous operations if
             // x0 and x1 differ
\end{lstlisting}

前两条指令按顺序执行,因为它们引用相同的地址。
然而,最后一条指令可能相对于前面的操作重新排序,因为它引用了不同的地址。

\begin{lstlisting}[
  language={[ARM]Assembler},
  caption={对于 PoU 的 cache 无效操作},
  label={lst:cache-i-pou}
]
IC IVAU, X0  // I cache Invalidate by address to Point of Unification
IC IALLU     // I cache Invalidate All to Point of Unification
             // Operations execute in order
\end{lstlisting}

这仅适用于发出指令。
只有在执行 DSB 指令后才能确保完成。

在 ARMv8-A 中,使用 \lstinline!DC ZVA! 指令预加载数据缓存的零值是一项新功能。
处理器的运行速度通常显著快于外部内存系统,有时从内存加载缓存行可能需要很长时间。

缓存行清零的行为类似于预取,它是一种提示处理器未来可能会使用某些地址的方式。
然而,清零操作可能会更快,因为不需要等待外部内存访问完成。
与从内存读取实际数据到缓存不同,您会得到填充了零的缓存行。
它可以提示处理器,代码完全覆盖了缓存行的内容,因此不需要初始读取。

考虑这样一种情况:您需要一个大的临时存储缓冲区,或者正在初始化一个新的结构。
您可以让代码直接开始使用内存,或者在使用之前预取它。
两者都会消耗大量的周期和内存带宽来读取初始内容到缓存中。
通过使用缓存清零选项,您可能可以节省这种浪费的带宽,并且让代码执行得更快。

缓存维护指令执行的时机可以根据指令是通过虚拟地址 (VA) 还是通过组 / 路 (Set/Way) 操作而定。

您可以选择作用域,可以是 PoC 或 PoU,对于可以广播的操作,您可以选择共享性。

以下示例代码展示了一种通用机制,用于将整个数据缓存或统一缓存清理(对于 PoC)。

\begin{lstlisting}[
  language={[ARM]Assembler},
  caption={对于 PoC 的 cache 清除操作},
  label={lst:clean-cache-poc}
]
       MRS X0, CLIDR_EL1
       AND W3, W0, #0x07000000  // Get 2 x Level of Coherence
       LSR W3, W3, #23
       CBZ W3, Finished
       MOV W10, #0              // W10 = 2 x cache level
       MOV W8, #1               // W8 = constant 0b1
Loop1: ADD W2, W10, W10, LSR #1 // Calculate 3 x cache level
       LSR W1, W0, W2           // extract 3-bit cache type for this level
       AND W1, W1, #0x7
       CMP W1, #2
       B.LT Skip                // No data or unified cache at this level
       MSR CSSELR_EL1, X10      // Select this cache level
       ISB                      // Synchronize change of CSSELR
       MRS X1, CCSIDR_EL1       // Read CCSIDR
       AND W2, W1, #7           // W2 = log2(linelen)-4
       ADD W2, W2, #4           // W2 = log2(linelen)
       UBFX W4, W1, #3, #10     // W4 = max way number, right aligned
       CLZ W5, W4               /* W5 = 32-log2(ways), bit position of way in DC operand */
       LSL W9, W4, W5           /* W9 = max way number, aligned to position in DC operand */
       LSL W16, W8, W5          // W16 = amount to decrement way number per iteration
Loop2: UBFX W7, W1, #13, #15    // W7 = max set number, right aligned
       LSL W7, W7, W2           /* W7 = max set number, aligned to position in DC operand */
       LSL W17, W8, W2          // W17 = amount to decrement set number per iteration
       Loop3: ORR W11, W10, W9  // W11 = combine way number and cache number...
       ORR W11, W11, W7         // ... and set number for DC operand
       DC CSW, X11              // Do data cache clean by set and way
       SUBS W7, W7, W17         // Decrement set number
       B.GE Loop3
       SUBS X9, X9, X16         // Decrement way number
       B.GE Loop2
Skip: ADD W10, W10, #2          // Increment 2 x cache level
       CMP W3, W10
       DSB                      /* Ensure completion of previous cache maintenance operation */
       B.GT Loop1
\end{lstlisting}

一些需要注意的要点:

\begin{itemize}
  \item 在正常情况下,清理或使整个缓存失效是固件应该执行的任务之一,作为核心上电或关机代码序列的一部分。
  这可能需要相当长的时间,因为 L2 缓存中的行数可能相当多,需要逐个遍历它们。

    因此,这种清理绝对是为特殊场合准备的!
  \item 缓存维护操作,如 \lstinline!DC CSW!,在上述节中有描述。
  \item 在代码序列开始时必须禁用缓存,以防止在序列中段分配新行。
  如果缓存是排它性的,行可能会在级别之间迁移。
  \item 在 SMP 系统中,另一个核心可能会在序列中段从缓存中获取脏缓存行,阻止它们到达 PoC。
  Cortex-A53 和 Cortex-A7 处理器都可以做到这一点。
  \item 如果存在 EL3,则必须从安全世界使缓存失效,因为某些条目可能是“安全脏”数据,不能从普通世界使其失效。
  如果不处理,“安全脏”数据在被驱逐时可能会破坏内存系统,因为在安全世界或普通世界中进行正常缓存使用。
\end{itemize}

如果软件需要在指令执行和内存之间保持一致性,它必须使用 ISB 和 DSB 内存屏障以及缓存维护指令来管理这种一致性。
下面的代码序列可用于此目的。

\begin{lstlisting}[
  language={[ARM]Assembler},
  caption={清理一行自改代码},
  label={lst:clean-one-line}
]
/* Coherency example for data and instruction accesses within the same Inner
Shareable domain. Enter this code with <Wt> containing a new 32-bit instruction,
to be held in Cacheable space at a location pointed to by Xn. */
STR Wt, [Xn]
DC CVAU, Xn // Clean data cache by VA to point of unification (PoU)
DSB ISH     // Ensure visibility of the data cleaned from cache
IC IVAU, Xn // Invalidate instruction cache by VA to PoU
DSB ISH     // Ensure completion of the invalidations
ISB         // Synchronize the fetched instruction stream
\end{lstlisting}

这个代码序列仅适用于适合单个 I 或 D 缓存行的指令序列。

该代码根据给定的起始基地址(存储在寄存器 x0 中)和长度(存储在寄存器 x1 中),对指定区域的数据和指令缓存进行虚拟地址清理和失效操作。

\begin{lstlisting}[
  language={[ARM]Assembler},
  caption={通过虚拟地址清理 cache},
  label={lst:clean-cache-by-virt}
]

//
// X0 = base address
// X1 = length (we assume the length is not 0)
//

// Calculate end of the region
ADD x1, x1, x0  // Base Address + Length
//
// Clean the data cache by MVA
//
MRS X2, CTR_EL0 // Read Cache Type Register

// Get the minimun data cache line
//

UBFX X4, X2, #16, #4 // Extract DminLine (log2 of the cache line)
MOV X3, #4           // Dminline iss the number of words (4 bytes)
LSL X3, X3, X4       // X3 should contain the cache line
SUB X4, X3, #1       // get the mask for the cache line

BIC X4, X0, X4       // Aligned the base address of the region
clean data cache:
DC CVAU, X4          // Clean data cache line by VA to PoU
ADD X4, X4, X3       // Next cache line
CMP X4, X1           // Is X4 (current cache line) smaller than the end
                     // of the region
B.LT clean_data_cache // while (address < end_address)

DSB ISH              // Ensure visibility of the data cleaned from cache

//
//Clean the instruction cache by VA
//
// Get the minimum instruction cache line (X2 contains ctr_el0)
AND X2, X2, #0xF  // Extract IminLine (log2 of the cache line)
MOV X3, #4        // IminLine is the number of words (4 bytes)
LSL X3, X3, X2    // X3 should contain the cache line
SUB x4, x3, #1    // Get the mask for the cache line

BIC X4, X0, X4    // Aligned the base address of the region
clean_instruction_cache:
IC IVAU, X4       // Clean instruction cache line by VA to PoU
ADD X4, X4, X3    // Next cache line
CMP X4, X1        // Is X4 (current cache line) smaller than the end
                  // of the region
B.LT clean_instruction_cache // while (address < end_address)

DSB ISB           // Ensure completion of the invalidations
ISH               // Synchronize the fetched instruction stream
\end{lstlisting}

\subsubsection{Cache 探测}

缓存维护操作可以通过缓存集、路或虚拟地址进行。
与平台无关的代码可能需要了解缓存的大小、缓存行的大小、集合和路的数量,以及系统中有多少级缓存。
这种需求最有可能出现在后重置的缓存失效和清零操作中。
在架构缓存上的所有其他操作很可能是基于 PoC 或 PoU 进行的。

有一些系统控制寄存器包含了这些信息:

\begin{itemize}
  \item 软件可以通过读取缓存级别 ID 寄存器(\lstinline!CLIDR_EL1!)来确定存在的缓存级别数量。
  \item Cache line 大小可以在缓存类型寄存器中获取(\lstinline!CTR_EL0!)。
  \item 如果需要由在执行级别 EL0 运行的用户代码访问,可以通过设置系统控制寄存器(\lstinline!SCTLR/SCTLR_EL1!)的 UCT 位来实现。
\end{itemize}

要确定缓存中的集合数量(set)和路数(way),需要对两个单独的寄存器进行异常级别的访问。

\begin{enumerate}
  \item 代码必须首先写入缓存大小选择寄存器(\lstinline!CSSELR_EL1!),以选择要获取信息的缓存。
  \item 然后,代码读取缓存大小 ID 寄存器(\lstinline!CCSIDR/CCSIDR_EL1!)。
  \item 数据缓存零 ID 寄存器(\lstinline!DCZID_EL0!)包含用于零操作的块大小。
  \item \lstinline!SCTLR/SCTLR_EL1! 中的 \lstinline![DZE]! 位和 Hypervisor Configuration Register (\lstinline!HCR/HCR_EL2!) 中的\lstinline![TDZ]!位控制哪些执行级别和哪些世界可以访问 \lstinline!DCZID_EL0!。
    \lstinline!CLIDR_EL1!、\lstinline!CSSELR_EL1! 和 \lstinline!CCSIDR_EL1! 只能通过特权代码访问,即在 AArch32 中为 PL1 或更高级别,在 AArch64 中为 EL1 或更高级别。
  \item 如果在异常级别上禁止执行通过虚拟地址进行数据缓存零操作(\lstinline!DC ZVA!)指令,例如通过 \lstinline!SCTLR_EL1.DZE! 位控制 EL0 级别,或通过 \lstinline!HCR_EL2.TDZ! 位控制 EL1 和 EL0 级别的非安全执行,则读取此寄存器会返回一个指示不支持该指令的值。

  \item \lstinline!CLIDR! 寄存器只知道集成到处理器本身中的缓存有多少级。
  它无法提供关于外部内存系统中的任何缓存的信息。

    例如,如果仅集成了 L1 和 L2,则 \lstinline!CLIDR/CLIDR_EL1! 标识了两个级别的缓存,处理器不知道任何外部 L3 缓存。

    在执行缓存维护操作或维护与集成缓存的一致性的代码时,可能需要考虑非集成缓存。
\end{enumerate}

\Figure[width=0.3]{cache-discovery}

此外,在 big.LITTLE 系统中,描述的缓存层次结构可能会因核心而异,例如,Cortex-A53 和 Cortex-A57 处理器具有不同的 \lstinline!CTR.L1IP! 字段。

\subsection{MMU}

内存管理单元(MMU)的一个重要功能是使系统能够运行多个任务,作为在它们自己的私有虚拟内存空间中运行的独立程序。
它们不需要知道系统的物理内存映射,即硬件实际使用的地址,也不需要知道可能同时执行的其他程序。

\Figure[caption={内存管理单元}, label={fig:mmu}, width=0.8]{mmu}

您可以为每个程序使用相同的虚拟内存地址空间。
即使物理内存是分散的,您也可以使用连续的虚拟内存映射。
这个虚拟地址空间与系统中实际的物理内存映射是分开的。
您可以编写、编译和链接应用程序以在虚拟内存空间中运行。

下图展示了一个示例系统,即内存的虚拟视图和物理视图。
在单个系统中的不同处理器和设备可能具有不同的虚拟和物理地址映射。
操作系统将 MMU 编程以在这两种内存视图之间进行转换。

\Figure[caption={虚拟和物理内存}, label={fig:virt-phys-mem}, width=0.8]{virt-phys-mem}

为了实现这一点,在虚拟内存系统中,硬件必须提供地址转换,即将处理器发出的虚拟地址转换为主存储器中的物理地址。

虚拟地址是由您、编译器和链接器在将代码放置到内存中时使用的地址。
物理地址是实际硬件系统使用的地址。

MMU 使用虚拟地址的最高有效位来索引翻译表中的条目,并确定正在访问哪个块。
MMU 将代码和数据的虚拟地址转换为实际系统中的物理地址。
这种转换是在硬件中自动进行的,并且对应用程序是透明的。
除了地址转换外,MMU 还控制每个内存区域的内存访问权限、内存排序和缓存策略。

\Figure[caption={使用转换表的地址转换}, label={fig:address-translation-with-tb}, width=0.9]{address-translation-with-tb}

MMU 使得任务或应用程序可以编写为不需要了解系统的物理内存映射或可能同时运行的其他程序。
这使得您可以为每个程序使用相同的虚拟内存地址空间。

它还使您可以在物理内存是分散的情况下使用连续的虚拟内存映射。
这个虚拟地址空间与系统中实际的物理内存映射是分开的。
应用程序被编写、编译和链接以在虚拟内存空间中运行。

\subsubsection{TLB}

\textit{Translation Lookaside Buffer} (TLB) 是 MMU 中最近访问的页面翻译的缓存。
对于处理器执行的每次内存访问,MMU 会检查该翻译是否缓存于 TLB 中。
如果请求的地址翻译在 TLB 中命中,则地址翻译会立即可用。

每个 TLB 条目通常不仅包含物理地址和虚拟地址,还包括诸如内存类型、缓存策略、访问权限、地址空间 ID(ASID)和虚拟机 ID(VMID)等属性。
如果 TLB 不包含处理器发出的虚拟地址的有效翻译,这被称为 TLB 未命中,则会执行外部翻译表遍历或查找。
MMU 内的专用硬件使其能够读取内存中的翻译表。
如果翻译表遍历没有导致页面错误,则新加载的翻译可以缓存到 TLB 中以备将来使用。
TLB 的具体结构因 ARM 处理器的实现而异。

如果操作系统修改了可能已缓存到 TLB 中的翻译条目,那么操作系统有责任使这些过时的 TLB 条目失效。

在执行 A64 代码时,有一个 TLBI 指令,这是一个 TLB 失效指令。

\lstinline!TLBI <type><level>{IS} {, <Xt>}!

下面列举了 type 字段常用的选项,详情可见表~\ref{tbl:tlb-inv-inst}。

\begin{description}
  \item[ALL] All TLB entries.
  \item[VMALL] All TLB entries. This is stage 1 for current guest OS.
  \item[VMALLS12] All TLB entries. This is stage 1 and 2 for current guest OS.
  \item[ASID] Entries that match ASID in Xt.
  \item[VA] Entry for Virtual Address and ASID specified in Xt.
  \item[VA] Entries for Virtual Address specified in Xt, with any ASID.
\end{description}

每个异常级别,即 EL3、EL2 或 EL1,都有其自己的虚拟地址空间,操作适用于对应空间。
IS 字段指定这仅适用于内部共享条目。

\begin{Tcbox}[title={Note}]
  后续章节有更加详细的 ASID 和共享概念的介绍。
\end{Tcbox}

level 字段指定了异常级的虚拟地址空间(可以是 3、2 或 1)。

\begin{ltblr}[caption={TLB 失效指令}, label={tbl:tlb-inv-inst}]
  {colspec={c>{\centering\arraybackslash}X}}
    \hline[1pt]
    变种 & 描述 \\
    \hline
    ALLEn & TLB invalidate All, ELn. \\
    ALLEnIS & TLB invalidate All, ELn, Inner Shareable. \\
    ASIDE1 & TLB invalidate by ASID, EL1. \\
    ASIDE1IS & TLB invalidate by ASID, EL1, Inner Shareable. \\
    IPAS2E1 & TLB invalidate by IPA, Stage 2, EL1. \\
    IPAS2E1IS & TLB invalidate by IPA, Stage 2, EL1, Inner Shareable. \\
    IPAS2LE1IS & TLB invalidate by IPA, Stage 2, Last level, EL1, Inner Shareable. \\
    VAAE1 & TLB invalidate by VA, All ASID, EL1. \\
    VAAE1IS & TLB invalidate by VA, All ASID, EL1, Inner Shareable. \\
    VAALE1IS & TLB invalidate for the Last level, by VA, All ASID, EL1, Inner Shareable. \\
    VAEn & TLB invalidate by VA, ELn. \\
    VAEnIS & TLB invalidate by VA, ELn, Inner Shareable. \\
    VALEn & TLB invalidate by VA, Last level, ELn. \\
    VALEnIS & TLB invalidate by VA, Last level, ELn, Inner Shareable. \\
    VMALLE1 & TLB invalidate by VMID, All at stage 1, EL1. \\
    VMALLE1IS & TLB invalidate by VMID, EL1, Inner Shareable. \\
    VMALLS12E1 & TLB invalidate by VMID, All at Stage 1 and 2, EL1. \\
    VMALLS12E1IS & TLB invalidate by VMID, All at Stage 1 and 2, EL1 Inner Shareable. \\
    \hline[1pt]
\end{ltblr}

以下代码示例显示了对由内部共享内存支持的翻译表进行写入的顺序:

\begin{lstcode}[language={[ARM]Assembler}]
  // << Writes to Translation Tables >>
  DSB ISHST  // ensure write has completed
  TLBI ALLE1 // invalidate all TLB entries
  DSB ISH    // ensure completion of TLB invalidation
  ISB        // synchronize context and ensure that no instructions are
             // fetched using the old translation
\end{lstcode}

若要改变单个条目,使用该指令:

\lstinline!TLBI VAE1, X0!

由 X0 提供单个条目的相关联的地址。

TLB 可以容纳固定数量的条目。
通过最小化由翻译表遍历引起的外部内存访问次数,并获得高 TLB 命中率,可以实现最佳性能。
ARMv8-A 架构提供了一种称为连续块条目的功能,以有效利用 TLB 空间。
翻译表块条目每个都包含一个连续位。
当设置时,此位向 TLB 发出信号,表明它可以缓存一个单一条目,覆盖多个块的翻译。
查找可以在由连续块覆盖的地址范围中的任何地方进行索引。
因此,TLB 可以缓存一个条目,用于一组定义的地址范围,使得在 TLB 中存储更大范围的虚拟地址成为可能。

要使用连续位功能,则使用到的连续块必须是相邻的,即它们必须对应于一段连续的虚拟地址范围。
它们必须以对齐边界开始,具有一致的属性,并且在同一级别的翻译中指向一个连续的输出地址范围。
所需的对齐是对于所有地址,若是 4KB 粒度,\lstinline!VA[20:16]! 必须相同,或者对于 64KB 粒度,则 \lstinline!VA[28:21]! 必须相同。
以下是所需的连续块数量:

\begin{itemize}
  \item 4KB 粒度下,16 × 4KB 相邻块组成 64KB 条目。
  \item 16KB 粒度下,32 × 32MB 相邻块组成 1GB 条目(二级描述符),128 × 16KB 组成 2MB 条目(三级描述符)。
  \item 64KB 粒度下,32 × 64KB 组成 2MB 条目。
\end{itemize}

如果不满足这些条件,就会发生编程错误,可能导致 TLB 中止或查找损坏。
此类错误的可能示例包括:

\begin{itemize}
  \item 一个或多个表条目未设置连续位。
  \item 其中一个条目的输出指向对齐范围之外。
\end{itemize}

在 ARMv8 架构中,不正确的使用不允许权限检查在 EL0 和 EL1 之外的有效地址空间中逃逸,也不会错误地提供对 EL3 空间的访问权限。

\begin{Tcbox}[title={扩展}]
\textbf{TLB thrashing}

  TLB 抖动可能会发生在工作集页数太多,而翻译后备缓冲(TLB)作为内存管理单元(MMU)的缓存,用于将虚拟地址翻译为物理地址的大小太小的情况下。
  即使指令缓存或数据缓存没有抖动,TLB 抖动也可能发生,因为它们的缓存大小不同。
  指令和数据被缓存在小块(缓存行)中,而不是整个页面,但地址查找是在页面级别进行的。
  因此,即使代码和数据工作集适合缓存,如果工作集在许多页面中分散,虚拟地址工作集可能不适合 TLB,导致 TLB 抖动。

  更多关于 thrashing 的概念可以参考\href{https://en.wikipedia.org/wiki/Thrashing_(computer_science)}{维基百科}。
\end{Tcbox}

\subsubsection{内核和用户的虚拟地址区分}

操作系统通常会同时运行多个应用程序或任务。
每个应用程序或任务都有自己独特的翻译表集合,内核在切换上下文以在不同任务之间切换时,会从一个翻译表集合切换到另一个。
然而,大部分内存系统仅由内核使用,并具有固定的虚拟地址到物理地址映射,其中翻译表条目很少改变。
ARMv8 架构提供了一些功能,以有效地处理这个需求。

表的基地址在翻译表基寄存器(\lstinline!TTBR0_EL1!)和(\lstinline!TTBR1_EL1!)中指定。
当虚拟地址(VA)的高位全为 0 时,选择 TTBR0 指向的翻译表。
当虚拟地址的高位全为 1 时,选择 TTBR1 指向的翻译表。
您可以启用虚拟地址标记,以排除检查中的最高八位。

来自处理器的指令获取或数据访问的虚拟地址是 64 位的。
然而,必须在单个 48 位物理地址内存映射中映射上述定义的两个区域。

EL2 和 EL3 有 TTBR0,但不存在 TTBR1。
这说明:

\begin{itemize}
  \item 如果 EL2 使用 AArch64,那么它只能使用 0x0 - 0x0000FFFF\_FFFFFFFF 范围的虚拟地址空间。
  \item EL3 同理。
\end{itemize}

下图展示了如何将内核空间映射到内存的最高有效区域,并将与每个应用程序关联的虚拟地址空间映射到内存的最低有效区域。
然而,这两者都映射到一个更小得多的物理地址空间。

\Figure[caption={内核和应用程序的地址映射}, label={fig:kernel-app-mem-map}, width=0.9]{kernel-app-mem-map}

翻译控制寄存器 \lstinline!TCR_EL1! 定义了需要检查的最高有效位的确切数量。\lstinline!TCR_EL1! 包含大小字段 \lstinline!T0SZ[5:0]! 和 \lstinline!T1SZ[5:0]!。
字段中的整数表示必须全部为 0 或全部为 1 的最高有效位数。
这些字段有指定的最小值和最大值,这些值随粒度大小和起始表级别而变化。
因此,您必须始终使用这两个空间,并且在所有系统中至少需要两个翻译表。
一个没有操作系统的简单裸机系统仍然需要一个包含仅故障条目的小上表。

\Figure[caption={翻译控制配置}, label={fig:trans-tbl-ctrl-config}, width=1]{trans-tbl-ctrl-config}

\lstinline!TCR_EL1! 控制 EL1 和 EL0 的其他内存管理功能。
下图仅显示了那些控制地址范围和粒度大小的字段。

\Figure[caption={翻译表控制寄存器}, label={fig:trans-tbl-ctrl-reg}, width=1]{trans-tbl-ctrl-reg}

中间物理地址大小(IPS)字段控制最大输出地址大小。
如果翻译指定的输出地址超出此范围,则访问将出错,000=32 位物理地址,101=48 位。
两位翻译粒度(TG)字段 TG1 和 TG0 分别给出内核空间或用户空间的粒度大小,00=4KB,01=16KB,11=64KB。

您可以配置用于第一次查找的翻译表级别。
完整的翻译过程可能需要三到四级表。
您不必实现所有级别。
第一次查找的级别实际上由粒度大小和 \lstinline!TCR_ELn.TxSZ! 字段决定。
您可以通过 \lstinline!TTBR0_EL1! 和 \lstinline!TTBR1_EL1! 分别配置粒度大小。

\subsubsection{翻译虚拟地址到物理地址}

当处理器发出用于指令获取或数据访问的 64 位虚拟地址时,MMU 硬件将虚拟地址转换为相应的物理地址。
对于虚拟地址,最高的 16 位 \lstinline![63:47]! 必须全为 0 或 1,否则地址会触发故障。

然后,最不重要的位用于在选定的部分内给出偏移量,因此 MMU 将块表条目的物理地址位与原始地址的最不重要位结合起来,生成最终地址。

该架构还支持标记地址。
在这种情况下,地址的最高八位会被忽略(视为不属于地址的一部分)。
这意味着这些位可以用于其他用途,例如记录关于指针的信息。

\Figure[caption={512MB 块大小的虚实转换}, label={fig:virt-to-phys-512MB}, width=1]{virt-to-phys-512MB}

在仅涉及一级查找的简单地址翻译中,假设我们使用 64KB 粒度和 42 位虚拟地址。
MMU 按如下方式翻译虚拟地址:

\begin{enumerate}
  \item 如果 \lstinline!VA[63:42] = 1!,则 TTBR1 用作第一级页表的基地址。
    当 \lstinline!VA[63:42] = 0! 时,TTBR0 用作第一级页表的基地址。
  \item 页表包含 8192 个 64 位页表条目,并使用 \lstinline!VA[41:29]! 进行索引。
    MMU 从表中读取相关的二级页表条目。
  \item MMU 检查页表条目的有效性以及所请求内存的可访问性。
    假设它有效,并且允许内存访问。
  \item 上图中,页表条目指向一个 512MB 的页面(它是一个块描述符)。
  \item 位 \lstinline![47:29]! 取自该页表条目,并形成物理地址的位 \lstinline![47:29]!。
  \item 因为我们有一个 512MB 的页面,所以虚拟地址的位 \lstinline![28:0]! 用于形成物理地址的位 \lstinline!PA[28:0]!。
  \item 返回完整的物理地址 \lstinline!PA[47:0]!,以及来自页表条目的附加信息。
\end{enumerate}

在实际操作中,如此简单的翻译过程严重限制了您可以如何细分地址空间。
与仅使用这个一级翻译表不同,一级表条目还可以指向二级页表。

这样,操作系统可以将一个大的虚拟内存区域进一步划分为更小的页面。
对于二级表,一级描述符包含二级页表的物理基地址。
与处理器请求的虚拟地址对应的物理地址,位于二级描述符中。

下图展示了从阶段 1、级别 2 开始的 64 位粒度翻译示例,用于普通的 64KB 页面。

\Figure[caption={64KB 页大小的虚实转换}, label={fig:virt-to-phys-64KB}, width=1]{virt-to-phys-64KB}

每个二级表都与一个或多个一级条目关联。
您可以有多个一级描述符指向同一个二级表,这意味着您可以将多个虚拟位置别名为同一个物理地址。

上图描述了两级查找的情况。
同样,这假设使用 64KB 的粒度和 42 位虚拟地址空间。

\begin{enumerate}
\item
  如果 VA{[}63:42{]} = 1,则 TTBR1 用于第一级页表的基地址。
  当 VA{[}63:42{]}
  = 0 时,TTBR0 用于第一级页表的基地址。
\item
  页表包含 8192 个 64 位页表条目,通过 VA{[}41:29{]}进行索引。
  MMU 从表中读取相关的二级页表条目。
\item
  MMU 检查二级页表条目的有效性以及请求的内存访问是否被允许。
  假设它是有效的,则允许内存访问。
\item
  在上图中,二级页表条目指向三级页表的地址(它是一个表描述符)。
\item
  从二级页表条目中取出位{[}47:16{]},形成三级页表的基地址。
\item
  VA 的位{[}28:16{]}用于索引三级页表条目。
  MMU 从表中读取相关的三级页表条目。
\item
  MMU 检查三级页表条目的有效性以及请求的内存访问是否被允许。
  假设它是有效的,则允许内存访问。
\item
  在上图中,三级页表条目指向一个 64KB 的页(它是一个页描述符)。
\item
  从三级页表条目中取出位{[}47:16{]}并用来形成 PA{[}47:16{]}。
\item
  由于我们有一个 64KB 的页,VA{[}15:0{]}被用来形成 PA{[}15:0{]}。
\item
  返回完整的 PA{[}47:0{]}以及页表条目中的附加信息。
\end{enumerate}

\BlockDesc{安全和非安全地址}

在理论上,安全和非安全物理地址空间是相互独立且并行存在的。
一个系统可以被设计为拥有两个完全独立的内存系统。
然而,大多数实际系统将安全和非安全视为访问控制的属性。
正常(非安全)世界只能访问非安全物理地址空间。
安全世界可以访问两个物理地址空间。
这一切都通过翻译表进行控制。

\Figure[caption={物理地址空间}, label={fig:phys-addr-space}, width=0.9]{phys-addr-space}

这也会带来缓存一致性的问题。
例如,技术上讲,安全的 0x8000 和非安全的 0x8000 是不同的物理地址,它们可以同时存在于缓存中。

在一个安全和非安全内存位于不同位置的系统中,不会有问题。
但更有可能的是,它们位于相同位置。
理想情况下,内存系统应阻止安全访问非安全内存和非安全访问安全内存。
在实际操作中,大多数系统只阻止非安全访问安全内存。
这意味着你可能会在缓存中两次出现相同的物理内存,一次是安全的,一次是非安全的。
这始终是编程错误。
为了避免这种情况,安全世界必须始终使用非安全访问来访问非安全内存。

\BlockDesc{配置并使能 MMU}

对系统寄存器的写入控制着 MMU(内存管理单元)的操作,它们是会改变上下文的事件,并且它们之间没有顺序要求。
在发生上下文同步事件之前,这些事件的结果不能保证会被看到(参见下一章节的屏障部分)。

\begin{lstcode}
  MSR TTBR0_EL1, X0  // Set TTBR0
  MSR TTBR1_EL1, X1  // Set TTBR1
  MSR TCR_EL1, X2    // Set TCR
  ISB                // The ISB forces these changes to be seen before /
                     // the MMU is enabled.
  MRS X0, SCTLR_EL1  // Read System Control Register configuration data
  ORR X0, X0, #1     // Set [M] bit and enable the MMU.
  MSR SCTLR_EL1, X0  // Write System Control Register configuration data
  ISB                // The ISB forces these changes to be seen by the /
                     // next instruction
\end{lstcode}

这与平面映射(flat mapping)的要求无关,平面映射是为了确保我们知道在写入 SCTLR\_EL1.M 后直接执行的是哪条指令。
如果我们看到了写入的结果,那么在使用新的转换机制时执行的指令是 VA+4 处的指令。
如果我们没有看到写入的结果,那么仍然是在 VA+4 处的指令,但此时的 VA 等于 PA。
ISB 在这里无济于事,因为除非我们进行平面映射,否则无法保证它是下一个执行的指令。

当第一级 MMU 被禁用时,对于非安全的 EL0 和 EL1 访问,如果 HCR\_EL2.DC 位被设置为启用数据缓存,默认的内存类型为正常非共享的、内部写回读写分配、外部写回读写分配。

\BlockDesc{内存管理单元禁用时的操作}

当第一级 MMU 被禁用时,对于非安全的 EL0 和 EL1 访问,如果 HCR\_EL2.DC 位被设置为启用数据缓存,默认的内存类型为正常非共享的、内部写回读写分配、外部写回读写分配。

\subsubsection{ARMv8-A 中的翻译表}

ARMv8-A 架构支持三种不同的翻译表格式:

\begin{itemize}
\item
  ARMv8-A AArch64 长描述符格式。
\item
  ARMv7-A 长描述符格式,例如用于 ARMv7-A 架构的大物理地址扩展(LPAE),在例如 ARM Cortex-A15 处理器中可以找到。
\item
  ARMv7-A 短描述符格式。
\end{itemize}

在 AArch32 状态下,可以使用现有的 ARMv7-A 长描述符和短描述符格式来运行现有的客户(guest)操作系统和应用代码,而无需修改。
ARMv7-A 短描述符只能在 EL0 和 EL1 的第一级翻译中使用。
因此,不能被管理程序或安全监控代码使用。

在 AArch64 执行状态下,始终使用 ARMv8-A 长描述符格式。
这与带有大物理地址扩展的 ARMv7-A 长描述符格式非常相似。
它使用相同的 64 位长描述符格式,但做了一些更改。
它引入了一个新的第 0 级表索引,使用与第 1 级表相同的描述符格式。
支持高达 48 位的输入和输出地址。
输入虚拟地址现在来自 64 位寄存器。
然而,由于架构不支持完整的 64 位寻址,地址的第 63 到 48 位必须全部相同,即全部为 0 或全部为 1,或者顶部的八位可以用于虚拟地址标记。

AArch64 支持三种不同的翻译粒度。
这些粒度定义了翻译表最低级别的块大小,并控制使用的翻译表大小。
较大的粒度大小减少了所需的页表级别数量,这在使用管理程序提供虚拟化的系统中可能成为一个需要重要考虑的因素。

支持的粒度大小有 4KB、16KB 和 64KB,具体支持哪种粒度由实现定义。
创建页表的代码可以读取系统寄存器 ID\_AA64MMFR0\_EL1,以确定支持哪些大小。
Cortex-A53 处理器支持这三种大小,但某些处理器的早期版本(如 Cortex-A57)不支持 16KB 粒度大小。
可以在翻译控制寄存器(TCR\_EL1)中为每个翻译表配置大小。

\paragraph{AArch64 描述符格式}

描述符格式可以在所有级别的表中使用,从第 0 级到第 3 级。
第 0 级描述符只能输出第 1 级表的地址。
第 3 级描述符不能指向另一个表,只能输出块地址。
因此,第 3 级表的格式略有不同。

如下图所示,表描述符类型由条目的位 1:0 标识,可以指向:

\begin{itemize}
\item
  下一级表的地址,此情况下内存可以进一步细分为更小的块。
\item
  可变大小的内存块的地址。
\item
  表条目,可以标记为故障(Fault)或无效(Invalid)。
\end{itemize}

\Figure[caption={描述符类型}, label={fig:tbl-desc-type}, width=1]{tbl-desc-type}

\begin{Tcbox}[title={Note}]
  为了显示清晰,上图没有标注字段的宽度信息。
\end{Tcbox}

\paragraph{翻译表粒度大小的影响}

三种不同大小的粒度会影响所需的翻译表数量和大小。

\begin{Tcbox}[title={Note}]
在所有情况下,如果虚拟地址(VA)输入范围限制在 42 位,可以省略第 0 级表。
根据可能的虚拟地址范围,甚至可以减少更多级别。
例如,对于 4KB 粒度,如果 TTBCR 设置为使低地址仅跨越 1GB,那么第 0 级和第 1 级是不需要的,翻译从第 2 级开始,向下到第 3 级以获得 4KB 页。
\end{Tcbox}

\BlockDesc{4KB}
当使用 4KB 粒度大小时,硬件可以使用四级查找过程。
48 位地址的每一级翻译包含 9 个地址位,即每级有 512 个条目,最后的 12 个位用于选择 4KB 块内的字节,这些位直接来自原始地址。

具体过程如下:

\begin{itemize}
  \item
  \textbf{第 0 级(L0)}:
    虚拟地址的位 47:39 用来索引 512 条目的 L0 表。
    每个条目覆盖 512GB 范围,并指向一个 L1 表。
  \item
  \textbf{第 1 级(L1)}:
    在 512 条目的 L1 表中,位 38:30 作为索引选择一个条目,每个条目指向一个 1GB 块或一个 L2 表。
  \item
  \textbf{第 2 级(L2)}:
    位 29:21 索引到一个 512 条目的 L2 表中,每个条目指向一个 2MB 块或下一级表(L3 表)。
  \item
  \textbf{第 3 级(L3)}:
    最后,位 20:12 索引到一个 512 条目的 L3 表中,每个条目指向一个 4KB 块。
\end{itemize}

\Figure[caption={4KB 粒度}, label={fig:4KB-granule-tbl}, width=1]{4KB-granule-tbl}

\BlockDesc{16KB}

当使用 16KB 粒度大小时,硬件可以使用四级查找过程。
48 位地址的每一级翻译包含 11 个地址位,即每级有 2048 个条目,最后的 14 个位用于选择 4KB 块内的字节,这些位直接来自原始地址。

具体过程如下:

\begin{itemize}
  \item
    \textbf{第 0 级(L0)}:
    虚拟地址的第 47 位用于从两个条目的 L0 表中选择一个描述符。
    每个条目覆盖 128TB 范围,并指向一个 L1 表。
  \item
    \textbf{第 1 级(L1)}:
    在 2048 条目的 L1 表中,位 46:36 作为索引选择一个条目,每个条目指向一个 L2 表。
  \item
    \textbf{第 2 级(L2)}:
    位 35:25 索引到一个 2048 条目的 L2 表中,每个条目指向一个 32MB 块或下一级表(L3 表)。
  \item
    \textbf{第 3 级(L3)}:
    最后,位 24:14 索引到一个 2048 条目的 L3 表中,每个条目指向一个 16KB 块。
\end{itemize}

\Figure[caption={16KB 粒度}, label={fig:16KB-granule-tbl}, width=1]{16KB-granule-tbl}

\BlockDesc{64KB}

当使用 64KB 粒度大小时,硬件可以使用三级查找过程。
第 1 级表只包含 64 个条目。

具体过程如下:

\begin{itemize}
  \item
    \textbf{第 1 级(L1)}:

    虚拟地址的位 47:42 用于从 64 个条目的 L1 表中选择一个描述符。
    每个条目覆盖 4TB 范围,并指向一个 L2 表。
  \item
    \textbf{第 2 级(L2)}:
    在 8192 个条目的 L2 表中,位 41:29 作为索引选择一个条目,每个条目指向一个 512MB 块或下一级表(L3 表)。
  \item
    \textbf{第 3 级(L3)}:
    在 8192 个条目的 L3 表中,最后,位 28:16 用作索引,每个条目指向一个 64KB 块。
\end{itemize}

\Figure[caption={64KB 粒度}, label={fig:64KB-granule-tbl}, width=0.8]{64KB-granule-tbl}

\paragraph{Cache 配置}

MMU(内存管理单元)使用翻译表和翻译寄存器来控制哪些内存位置是可缓存的。
MMU 控制着缓存策略、内存属性、访问权限,并提供虚拟地址到物理地址的转换。

\Figure[caption={内存总线和 Cache}, label={fig:mem-bus-and-cache}, width=0.5]{mem-bus-and-cache}

软件配置通过系统寄存器进行,其中一些列在 ARMv8 寄存器章节中。
在某些设计中,外部存储系统可能包含进一步的特定于实现的外部存储器缓存。

\paragraph{Cache 策略}

MMU 翻译表还为内存系统中的每个块定义了缓存策略。
被定义为“Normal”的内存区域可能被标记为可缓存或不可缓存。
翻译表条目的位{[}4:2{]}指向内存属性间接寄存器(MAIR)中的八种内存属性编码之一。
然后,内存属性编码指定了在访问该内存时要使用的缓存策略。
这些是对处理器的提示,具体实现中是否支持所有缓存策略以及哪些缓存数据被视为一致,这是由实现定义的。
内存区域可以根据其共享性属性来定义。

\subsubsection{翻译表配置}

除了在 TLB 中存储单个翻译之外,你还可以配置 MMU 将翻译表存储在可缓存的内存中。
这通常比总是从外部存储器中读取表要快得多。
TCR\_EL1 具有额外的字段来控制这一点。
这些额外的字段指定了 TTBR0 和 TTBR1 的翻译表的缓存性和共享性。
相关字段称为 SH0/1 共享性、IRGN0/1 内部可缓存性和 ORGN0/1 外部可缓存性。
下表显示了缓存性的允许设置。

\begin{stblr}
  {缓存属性配置}
  {cacheability-settings}
  {>{\centering\arraybackslash}X>{\centering\arraybackslash}X}
  \hline[1pt]
  TTBR0/TTBR1 的 IRGN/ORGN 位 & 缓存属性 \\
  \hline
  00 & Normal memory, Inner Non-cacheable \\
  01 & Normal memory, Inner Write-Back Write-Allocate Cacheable \\
  02 & Normal memory, Inner Write-Through Cacheable \\
  03 & Normal memory, Inner Write-Back no Write-Allocate Cacheable \\
  \hline[1pt]
\end{stblr}

与内存共享性相关的对应表与翻译表遍历相关联。
对于设备或强顺序内存区域,该值将被忽略。

\begin{stblr}
  {内存共享性}
  {mem-sh}
  {cc}
  \hline[1pt]
  SH0 bits{[}13:12{]} & 共享性 \\
  \hline
  00 & Non-shareable \\
  01 & UNPREDICTABLE \\
  10 & Outer shareable \\
  11 & Inner shareable \\
  \hline[1pt]
\end{stblr}

在 TCR\_EL1 中指定的属性必须与存储翻译表的虚拟内存区域中指定的属性相同。
缓存翻译表是正常的默认行为。

\BlockDesc{虚拟地址标记(tagging)}

Translation Control Register(TCR\_ELn)具有一个额外的字段称为 Top Byte Ignore(TBI),提供了标记寻址支持。
通用寄存器的宽度为 64 位,但地址的最高 16 位必须全部为 0xFFFF 或 0x0000。
任何尝试使用不同的位值都会触发错误。

当启用标记寻址支持时,虚拟地址的最高八位(即{[}63:56{]})将被处理器忽略。
它内部将位{[}55{]}设置为将地址扩展为 64 位格式的符号位。
虚拟地址的最高八位可以用于传递数据。
这些位在寻址和翻译错误时被忽略。
TCR\_EL1 具有 EL0 和 EL1 的单独启用位。
ARM 不指定或强制标记寻址的特定用例。

一个可能的用例是支持面向对象编程语言。
除了具有指向对象的指针外,可能还需要保持一个引用计数,用于跟踪引用对象的引用数或指针数或句柄数,以便垃圾自动收集代码可以释放不再使用的引用对象。
这个引用计数可以作为标记地址的一部分存储,而不是在一个单独的表中存储,从而加速创建或销毁对象的过程。

\subsubsection{EL2 和 EL3 的地址转换}

ARMv8-A 架构的虚拟化扩展引入了第二阶段的转换。
当系统中存在管理程序时,可能存在一个或多个客操作系统。
这些继续使用之前描述的 TTBRn\_EL1,并且 MMU 操作看起来没有变化。

管理程序必须执行一些额外的转换步骤,以在不同的客操作系统之间共享物理内存系统的两级过程。
在第一阶段,虚拟地址(VA)被转换为中间物理地址(IPA)。
这通常在操作系统控制之下。
由管理程序控制的第二阶段然后对 IPA 进行转换,将其转换为最终的物理地址(PA)。

管理程序和安全监控还具有自己的一组第一级转换表,用于其自身的代码和数据,直接将 VA 映射到 PA。

\Figure[caption={两级转换过程}, label={fig:two-stage-trans-process}, width=1]{two-stage-trans-process}

第二阶段的转换,即将中间物理地址转换为物理地址,使用管理程序控制的额外一组表。
必须通过写入管理程序配置寄存器 HCR\_EL2 来显式启用这些表。
此过程仅适用于非安全的 EL1/0 访问。

这些第二阶段转换表的基址在虚拟化转换表基址寄存器 VTTBR0\_EL2 中指定。
它指定了一个单一连续的地址空间位于内存底部。
支持的地址空间大小在虚拟化转换控制寄存器 VTCR\_EL2 的 TSZ{[}5:0{]}字段中指定。

该寄存器的 TG 字段指定了粒度大小,而 SL0 字段控制第一级表查找。
任何超出定义的地址范围的访问都会引发转换错误。

\Figure[caption={最大 IPA 空间}, label={fig:max-ipa-space}, width=0.4]{max-ipa-space}

管理程序 EL2 和安全监控 EL3 具有它们自己的第一级表,直接将虚拟地址映射到物理地址空间。
这些表的基址分别在 TTBR0\_EL2 和 TTBR0\_EL3 中指定,从而在内存底部启用一个可变大小的单一连续地址空间。
TG 字段指定了粒度大小,SL0 字段控制了第一级表查找。
任何超出定义的地址范围的访问都会引发转换错误。

\Figure[caption={最大虚拟地址空间}, label={fig:max-virt-addr-space}, width=0.4]{max-virt-addr-space}

安全监控 EL3 具有自己专用的翻译表。
表的基址在 TTBR0\_EL3 中指定,并通过 TCR\_EL3 进行配置。
翻译表能够访问安全和非安全物理地址。
TTBR0\_EL3 仅在安全监控 EL3 模式下使用,而不是由可信内核本身使用。
当过渡到安全世界完成后,可信内核使用 EL1 转换,即由 TTBR0\_EL1 和 TTBR1\_EL1 指向的翻译表。
由于在 AArch64 中这些寄存器没有进行分行操作,安全监控代码必须为安全世界配置新的表,并保存和恢复 TTBR0\_EL1 和 TTBR1\_EL1 的副本。

在安全状态下,与非安全状态下的正常操作相比,EL1 转换模式的行为有所不同。
第二阶段的转换被禁用,EL1 转换模式现在能够指向安全或非安全物理地址。
安全世界中没有虚拟化,因此 IPA 始终与最终 PA 相同。

TLB 中的条目标记为安全或非安全,因此在安全和正常世界之间转换时永远不需要 TLB 维护。

\subsubsection{访问权限}

访问权限通过翻译表条目进行控制。
访问权限控制区域是否可读、可写,或二者兼而有之,并且可以分别设置为 EL0 以供非特权访问,以及为 EL1、EL2 和 EL3 以供特权访问,如下表所示。

\begin{stblr}
  {访问权限}
  {access-permissions}
  {ccc}
  \hline[1pt]
  AP & Unprivileged (EL0) & Privileged (EL1/2/3) \\
  \hline
  00 & No access & Read and write \\
  01 & Read and write & Read and write \\
  10 & No access & Read-only \\
  11 &  Read-only &  Read-only \\
  \hline[1pt]
\end{stblr}

操作系统内核运行在执行级别 EL1。
它定义了翻译表映射,这些映射被内核自身和在 EL0 运行的应用程序使用。
区分非特权和特权访问权限是必要的,因为内核为其自身的代码和应用程序指定了不同的权限。
运行在执行级别 EL2 的管理程序和安全监控 EL3 仅具有用于自身使用的翻译方案,因此在权限上没有特权和非特权的分割。

另一种访问权限是可执行属性。
块可以标记为可执行或不可执行(Execute Never (XN))。
您可以分别设置非特权执行不允许(UXN)和特权执行不允许(PXN)属性,并使用这些属性来防止,例如,应用程序代码以内核特权运行,或尝试在非特权状态下执行内核代码。
设置这些属性可以防止处理器对内存位置进行推测性指令获取,并确保推测性指令获取不会意外访问可能受到此类访问影响的位置,例如先进先出(FIFO)页面替换队列。
因此,设备区域必须始终标记为不可执行。

\Figure[caption={设备区域}, label={fig:device-regions}, width=0.3]{device-regions}

您可以使用 SCTLR 寄存器中的以下位配置处理器将可写区域视为不可执行:

\begin{description}
  \item[SCTLR\_EL1.WXN] 在 EL0 可写的区域在 EL0 和 EL1 被视为 XN。
  在 EL1 可写的区域在 EL1 被视为 XN。
  \item[SCTLR\_EL2 和 3.WXN] 在 ELn 可写的区域在 ELn 被视为 XN。
  \item[SCTLR.UWXN] 在 EL0 可写的区域在 EL1 被视为 XN。
  仅适用于 AArch32。
\end{description}

SCTLR\_ELn 的位可以缓存在 TLB 条目中。
因此,更改 SCTLR 中的位可能不会影响已经存在于 TLB 中的条目。
在修改这些位时,需要进行 TLB 失效和 ISB 序列。

\subsubsection{操作系统对转换表描述符的使用}

描述符中的另一个内存属性位是访问标志(AF),指示块条目何时被首次使用。

\begin{itemize}
  \item AF = 0:此块条目尚未被使用。
  \item AF = 1:此块条目已被使用。
\end{itemize}

操作系统使用访问标志位来跟踪哪些页面正在使用。
软件管理此标志。
当页面首次创建时,其条目的 AF 设置为 0。
当代码首次访问页面时,如果其 AF 为 0,则触发 MMU 故障。
页面故障处理程序记录此页面现在正在使用,并手动设置表条目中的 AF 位。
例如,Linux 内核在 ARM64 上使用{[}AF{]}位用于 PTE\_AF(Linux 内核对 AArch64 的名称),用于检查页面是否曾经被访问过。
这影响了一些内核内存管理选择。
例如,当必须将页面交换出内存时,不太可能交换出正在积极使用的页面。

描述符的位{[}58:55{]}被标记为保留给软件使用,可用于在翻译表中记录特定于操作系统的信息。
例如,Linux 内核使用其中一个位将条目标记为干净或脏。
脏状态记录页面是否已写入。
如果将页面稍后交换出内存,干净的页面可以简单地丢弃,但脏的页面必须首先保存其内容。

\Figure[caption={转换表描述符}, label={fig:trans-tbl-desc}, width=1]{trans-tbl-desc}

请参阅下一章“内存排序”以获取有关指定内存类型及其缓存性和共享性属性的其他内存属性的信息。

\subsubsection{安全性和 MMU}

ARMv8-A 架构定义了两种安全状态,安全和非安全。
它还定义了两个物理地址空间:安全和非安全,以使正常世界只能访问非安全物理地址空间。
安全世界可以访问安全和非安全物理地址空间。

在非安全状态下,翻译表中的 NS 位和 NSTable 位被忽略。
只能访问非安全内存。
在安全状态下,NS 位和 NSTable 位控制虚拟地址是否转换为安全或非安全物理地址。
您可以使用 SCR\_EL3.CIF 来防止安全世界从任何转换为非安全物理地址的虚拟地址执行。
此外,当在安全世界中时,您可以使用 SCR.CIF 位来控制是否可以对非安全物理内存进行安全指令获取。

\subsubsection{上下文切换}

实现 ARMv8-A 架构的处理器通常用于运行具有许多并发运行的应用程序或任务的复杂操作系统的系统中。
每个进程都有自己独特的翻译表,驻留在物理内存中。
当应用程序启动时,操作系统会为其分配一组翻译表条目,将应用程序使用的代码和数据映射到物理内存中。
这些表随后可以被内核修改,例如,用于映射额外的空间,并且在应用程序不再运行时会被删除。

因此,在内存系统中可能存在多个任务。
内核调度程序定期将执行从一个任务转移到另一个任务。
这称为上下文切换,需要内核保存与进程关联的所有执行状态,并将要运行的进程的状态恢复到原始状态。
内核还将转换表条目切换到要运行的下一个进程的条目。
当前未运行的任务的内存完全受到正在运行的任务的保护。

在不同的操作系统中,需要保存和恢复的内容各不相同,但是典型的进程上下文切换包括保存或恢复以下一些或全部元素:

\begin{itemize}
  \item 通用寄存器 X0-X30。
  \item 高级 SIMD 和浮点寄存器 V0-V31。
  \item 一些状态寄存器。
  \item TTBR0\_EL1 和 TTBR1\_EL1。
  \item 线程进程 ID(TPIDxxx)寄存器。
  \item 地址空间 ID(ASID)。
\end{itemize}

对于 EL0 和 EL1,有两个翻译表。
TTBR0\_EL1 提供底部虚拟地址空间的翻译,通常是应用程序空间,而 TTBR1\_EL1 涵盖顶部虚拟地址空间,通常是内核空间。
这种分割意味着操作系统映射不必在每个任务的翻译表中复制。

翻译表条目包含一个非全局(nG)位。
如果对于特定页面设置了 nG 位,则它与特定任务或应用程序相关联。
如果该位标记为 0,则该条目是全局的,并适用于所有任务。

对于非全局条目,在更新 TLB 并将条目标记为非全局时,除了常规的转换信息外,还会在 TLB 条目中存储一个值。
该值称为地址空间 ID(ASID),由操作系统分配给每个单独的任务。
后续的 TLB 查找仅在当前 ASID 与存储在条目中的 ASID 匹配时才匹配该条目。
这允许对于特定页面标记为非全局的多个有效 TLB 条目存在,但具有不同的 ASID 值。
换句话说,我们在上下文切换时不一定需要刷新 TLB。

在 AArch64 中,此 ASID 值可以由 TCR\_EL1.AS 位控制为 8 位或 16 位值。
当前 ASID 值由 TTBR0\_EL1 或 TTBR1\_EL1 指定。
TCR\_EL1 控制哪个 TTBR 保存 ASID,但通常是 TTBR0\_EL1,因为这对应于应用程序空间。

\begin{Tcbox}[title={Note}]
  将 ASID 的当前值存储在翻译表寄存器中意味着您可以在单条指令中原子地修改翻译表和 ASID。
  与 ARMv7-A 架构相比,这简化了在更改表和 ASID 时的过程。
\end{Tcbox}

此外,ARMv8-A 架构为操作系统软件提供了线程 ID 寄存器。
这些寄存器在硬件上没有意义,通常由线程库用作每个线程数据的基址。
这通常称为线程局部存储(TLS)。
例如,pthread 库使用此功能并包含以下寄存器:

\begin{itemize}
\item
  用户读写线程 ID 寄存器(TPIDR\_EL0)。
\item
  用户只读线程 ID 寄存器(TPIDRRO\_EL0)。
\item
  线程 ID 寄存器,仅特权访问(TPIDR\_EL1)。
\end{itemize}

\subsubsection{具有用户权限的内核访问}

有些指令允许在 EL1 执行的代码(例如,操作系统)以 EL0 或应用程序权限执行内存访问。
例如,这可以用于解引用与系统调用一起提供的指针,并使操作系统能够检查只访问应用程序可访问的数据。
这可以通过使用 LDTR 或 STTR 指令实现。
当在 EL1 执行时,这些指令执行加载或存储,就好像在 EL0 执行一样。
在所有其他异常级别上,LDTR 和 STTR 的行为与常规的 LDR 或 STR 指令相同。
有通常大小和有符号和无符号变体作为正常的加载和存储指令,但是偏移量较小,并且索引选项受限。


\subsection{内存排序} \label{sec:memory-ordering}

如果您的代码直接与硬件交互,或者与在其他核心上执行的代码交互,或者直接加载或写入要执行的指令,或者修改页表,您需要注意内存排序问题。

如果您是应用程序开发人员,则硬件交互可能是通过设备驱动程序进行的,与其他核心的交互是通过 Pthreads 或其他多线程 API 进行的,与分页内存系统的交互是通过操作系统进行的。
在以上这些情况下,相关代码会为您处理内存排序问题。
但是,如果您正在编写操作系统内核或设备驱动程序,或者实现超级监视器、即时编译器或多线程库,那么您必须对 ARM 架构的内存排序规则有很好的理解。
\textbf{您必须确保在您的代码需要显式排序内存访问时,通过正确使用障碍来实现这一点}。

ARMv8 架构采用了弱排序的内存模型。
一般来说,这意味着内存访问的顺序不需要与加载和存储操作的程序顺序相同。
处理器能够对内存读取操作进行重新排序。
写操作也可以被重新排序(例如,写组合)。
因此,硬件优化,例如使用缓存和写缓冲区,以提高处理器的性能,这意味着处理器与外部存储之间所需的带宽可以降低,并且与此类外部存储访问相关的长延迟被隐藏。

对于普通内存的读写可以被硬件重新排序,只受数据依赖关系和显式内存屏障指令的影响。
某些情况需要更强的排序规则。
您可以通过描述该内存的翻译表条目的内存类型属性向核心提供相关信息。

在性能非常高的系统中,可能会支持诸如 speculative memory reads、指令的多次发射、乱序执行等技术,这些技术与其他技术一起,提供了硬件重新排序内存访问的进一步可能性:

\begin{itemize}
\item
  指令的多次发射:处理器可能会在每个周期发射和执行多条指令,因此程序顺序中紧随其后的指令可以同时执行。
\item
  乱序执行:许多处理器支持非依赖性指令的乱序执行。
  每当一条指令因为等待前一条指令的结果而停滞时,处理器可以执行不具有依赖性的后续指令。
\item
  推测执行:当处理器遇到条件指令(例如分支)时,它可以在确定是否必须执行该特定指令之前进行推测性地执行指令。
  因此,如果条件解析为推测是正确的,结果将更快地得到。
\item
  推测加载:如果对可缓存位置的加载指令进行了推测性执行,这可能导致缓存行填充和潜在的现有缓存行逐出。
\item
  加载和存储优化:由于对外部存储器的读写可能具有较长的延迟,处理器可以通过将多个存储合并成一个较大的事务来减少传输次数。
\item
  外部存储系统:在许多复杂的片上系统(SoC)设备中,有许多代理可以启动传输,并且读取或写入的从设备有多个路由。
  一些设备,如 DRAM 控制器,可能能够接受来自不同主设备的同时请求。
  事务可以被缓冲或者由互连重新排序。
  这意味着来自不同主设备的访问可能需要不同数量的周期来完成,并且可能相互超越。
\item
  缓存一致的多核处理:在多核处理器中,硬件缓存一致性可以在核之间迁移缓存行。
  因此,不同的核可能以不同的顺序看到对缓存的内存位置的更新。
\item
  优化编译器:优化编译器可以重新排列指令以隐藏延迟或充分利用硬件功能。
  它通常会将内存访问向前移动,以便更早地进行,并为其提供更多的时间以在所需值之前完成。
\end{itemize}

在单核系统中,这种重新排序的效果通常对程序员来说是透明的,因为单个处理器可以检查危险并确保数据依赖关系得到尊重。
然而,在具有通过共享内存进行通信或以其他方式共享数据的多核心系统中,内存排序考虑变得更为重要。
本章讨论了与多处理(MP)操作和多个执行线程同步相关的几个主题。
它还讨论了由架构定义的内存类型和规则以及如何进行控制。

\subsubsection{内存类型}

ARMv8 架构定义了两种互斥的内存类型。
内存的所有区域都配置为这两种类型之一,即 Normal 和 Device。
第三种内存类型,强顺序(Strongly Ordered),是 ARMv7 架构的一部分。
这种类型与 Device 内存之间的差异很少,因此现在在 ARMv8 中被省略了。

除了内存类型外,属性还可以控制缓存性、共享性、访问和执行权限。
可共享和缓存属性仅适用于 Normal 内存。
设备区域始终被视为非缓存和外部共享。
对于可缓存的位置,您可以使用属性来指示处理器的缓存分配策略。

内存类型不是直接编码在翻译表项中的。
相反,每个块条目指定了一个 3 位索引,该索引指向内存类型表。
此表存储在 Memory Attribute Indirection Register MAIR\_ELn 中。
该表有八个条目,每个条目都有八位,如图 13 - 1 所示。

尽管翻译表块条目本身不直接包含内存类型编码,但处理器内部的 TLB 条目通常会为特定条目存储此信息。
因此,对 MAIR\_ELn 的更改可能直到执行 ISB 指令屏障和 TLB 失效操作之后才能观察到。

\Figure[caption={类型编码}, label={fig:type-encoding}, width=0.8]{type-encoding}

\BlockDesc{Normal}

您可以将 Normal 内存用于所有代码和大多数内存中的数据区域。
Normal 内存的示例包括物理内存中的 RAM、Flash 或 ROM 区域。
这种类型的内存提供了最高的处理器性能,因为它是弱有序的,并且对处理器施加的限制较少。
处理器可以重新排序、重复和合并对 Normal 内存的访问。

此外,标记为 Normal 的地址位置可以由处理器进行推测性访问,因此数据或指令可以从内存中读取,而无需在程序中明确引用,或者在明确引用之前执行。
这种推测性访问可能是由于分支预测、推测性缓存行填充、乱序数据加载或其他硬件优化而发生的。

为了获得最佳性能,始终将应用程序代码和数据标记为 Normal,并在需要强制执行内存顺序的情况下,可以通过使用显式的屏障操作来实现。
Normal 内存实现了一种弱有序的内存模式。
Normal 访问不需要与其他 Normal 访问或 Device 访问顺序完全一致。

但是,处理器必须始终处理由地址依赖性引起的冲突。
例如,考虑以下简单的代码序列:

\begin{lstlisting}
STR X0, [X2]
LDR X1, [X2]
\end{lstlisting}

处理器始终确保将放置在 X1 中的值是存储在 X2 中的地址写入的值。

当然,这也适用于更复杂的依赖关系。
考虑以下代码:

\begin{lstlisting}
ADD X4, X3, #3
ADD X5, X3, #2
STR X0, [X3]
STRB W1, [X4]
LDRH W2, [X5]
\end{lstlisting}

在这种情况下,访问的地址重叠。
处理器必须确保内存更新为像 STR 和 STRB 按顺序发生一样,以便 LDRH 返回最新的值。
处理器仍然可以将 STR 和 STRB 合并为一个包含最新、正确数据的单个访问,这是有效的。

\BlockDesc{Device}

你可以将设备内存用于所有可能具有副作用的内存区域。
例如,对 FIFO 位置或计时器的读取是不可重复的,因为每次读取返回的值都不同。
对控制寄存器的写入可能会触发中断。
通常仅用于系统中的外设。
设备内存类型对核心施加了更多限制。

不能对标记为设备的内存区域执行推测性数据访问。
这有一个单独的、不常见的例外情况。
如果使用 NEON 操作从设备内存读取字节,则处理器可能会读取未显式引用的字节,如果它们位于一个对齐的 16 字节块中,该块包含一个或多个显式引用的字节。

尝试从标记为设备的区域执行代码通常是不可预测的。
实现可能会将指令提取处理为具有正常非可缓存属性的内存位置,或者可能会发生权限故障。

有四种不同类型的设备内存,适用不同的规则。

\begin{itemize}
\item
  Device-nGnRnE 最严格的(相当于 ARMv7 架构中的 Strongly Ordered 内存)。
\item
  Device-nGnRE
\item
  Device-nGRE
\item
  Device-GRE 最不严格的
\end{itemize}

字母后缀表示以下三个属性:

\begin{description}
  \item [Gathering or non Gathering (G 或 nG)]

  决定是否可以将多个访问合并成单个总线事务。
  对于标记为非 Gathering(nG)的地址,执行到该位置的内存总线上的访问次数和大小必须与代码中的显式访问次数和大小完全匹配。
  如果地址标记为 Gathering(G),则处理器可以将两个字节写入合并为单个半字写入。
  对于标记为 Gathering 的区域,多个内存访问可以合并到相同的内存位置。
  例如,如果程序两次读取相同的位置,则核心只需要执行一次读取,并且可以为两个指令返回相同的结果。
  对于从标记为非 Gathering 的区域的读取,数据值必须来自终端设备,不能从写缓冲区或其他位置窃取。

  \item [Re-ordering (R 或 nR)]

  决定对同一设备的访问是否可以相互重排。
  如果地址标记为非重排序(nR),则同一块内的访问始终按程序顺序出现在总线上。
  此块的大小由实现定义。
  如果此块的大小较大,则可能跨越多个表项。
  在这种情况下,对于也标记为
  nR 的任何其他访问,都遵守排序规则。

  \item [Early Write Acknowledgement (E 或 nE)]

  确定在处理器和被访问的从设备之间是否允许有一个中间写入缓冲区发送写入完成的确认。
  如果地址标记为非早期写入确认(nE),则写入响应必须来自外围设备。
  如果地址标记为早期写入确认(E),则允许在实际接收到写入之前,互连逻辑中的缓冲区提前信号写入接受。
  这本质上是对外部内存系统的一种消息。

\end{description}

\subsubsection{屏障}

ARM 架构包含屏障指令,用于在特定点强制访问排序和访问完成。
在某些架构中,类似的指令称为栅栏。

如果你正在编写重要的排序代码,请参阅《ARM 架构参考手册 -
ARMv8-A 架构配置文件》附录 J7 中的屏障测试和《ARM 架构参考手册
ARMv7-A/R 版本》附录 G 中的屏障测试,其中包含许多示例。

ARM 架构参考手册定义了一些关键词,特别是“观察”和“必须观察”的术语。
在典型的系统中,这定义了主控件(例如,核心或 GPU)的总线接口和互连必须如何处理总线事务。
只有主控件能够观察到传输。
所有总线事务都由主控件发起。
主控件执行事务的顺序不一定与这些事务在从设备完成的顺序相同,因为事务可能会被互连重新排序,除非显式强制执行某些排序。

描述可观测性的一种简单方法是说,“当我可以读到你写的内容时,我已经观察到了你的写入,并且当我不能再更改你读取的值时,我已经观察到了你的读取”,其中我和你都指的是系统中的核心或其他主控件。

架构提供了三种类型的屏障指令:

\begin{description}
  \item [指令同步屏障(ISB)] 用于保证后续指令再次被获取,以便使用当前的 MMU 配置进行特权和访问检查。
  它用于确保先前执行的上下文切换操作(例如写入到系统控制寄存器)在 ISB 完成时已经完成。
  在硬件方面,这可能意味着刷新指令流水线,例如。
  典型的用法包括内存管理、缓存控制和上下文切换代码,或者代码在内存中移动的情况。
  \item [数据内存屏障(DMB)] 防止在屏障指令之间对数据访问指令进行重新排序。
  在 DMB 之前由该处理器执行的所有数据访问(即加载或存储),但不包括指令获取,对于指定的可共享域内的所有其他主控件都是可见的,在 DMB 之后的任何数据访问之前。
    例如:
    \begin{lstcode}
      LDR x0, [x1] // Must be seen by the memory system before the STR below.
      DMB ISHLD
      ADD x2, #1   // May be executed before or after the memory system sees LDR.
      STR x3, [x4] // Must be seen by the memory system after the LDR above.
    \end{lstcode}

    它还确保在执行任何后续数据访问之前,任何明确的前置数据或统一缓存维护操作都已完成。

    \begin{lstcode}
      DC CSW, x5   // Data clean by Set/way
      LDR x0, [x1] // Effect of data cache clean might not be seen by this
                   // instruction
      DMB ISH
      LDR x2, [x3] // Effect of data cache clean will be seen by this
      instruction
    \end{lstcode}
  \item [数据同步屏障(DSB)] 强制执行与数据内存屏障相同的排序,并额外阻止执行任何进一步的指令,而不仅仅是加载或存储,或两者,直到同步完成。
  这可用于防止执行 SEV 指令,例如,该指令会向其他核心发出信号,表示发生了事件。
  它等待由该处理器发出的所有缓存、TLB 和分支预测器维护操作针对指定的共享域都已经完成。
    例如:
    \begin{lstcode}
      DC ISW, x5     // operation must have completed before DSB can complete
      STR x0, [x1]   // Access must have completed before DSB can complete DSB ISH
      ADD x2, x2, #3 // Cannot be executed until DSB completes
    \end{lstcode}
\end{description}

正如你从上述示例中看到的那样,DMB 和 DSB 指令带有一个参数,指定屏障操作的访问类型(之前或之后),以及适用的共享域。

可用选项列在表中。

\begin{ltblr}[caption={屏障参数}, label={tbl:barrier-param}]{colspec={c>{\centering\arraybackslash}Xc}}
  \hline[1pt]
  <option>   & Ordered Accesses (before – after) & Shareability Domain \\
  \hline
  OSHLD      & Load – Load, Load – Store         &                     \\
  OSHST      & Store – Store                     & Outer shareable     \\
  OSH        & Any – Any                         &                     \\
  NSHLD      & Load – Load, Load – Store         &                     \\
  NSHST      & Store – Store                     & Non-shareable       \\
  NSH        & Any – Any                         &                     \\
  ISHLD      & Load – Load, Load – Store         & Inner shareable     \\
  ISHST      & Store – Store                     &                     \\
  ISH        & Any – Any                         &                     \\
  LD         & Load – Load, Load – Store         &                     \\
  ST         & Store – Store                     & Full system         \\
  SY         & Any – Any                                               \\
  \hline[1pt]
\end{ltblr}

有序访问字段指定屏障操作的访问类别。
有三种选择。

\begin{itemize}
\item 装载
\item 装载 / 存储

  这意味着屏障要求在屏障之前完成所有装载,但不要求存储完成。
  程序顺序中出现在屏障之后的装载和存储都必须等待屏障完成。
\item 存储 - 存储

    这意味着屏障只影响存储访问,装载可以在屏障周围自由重排序。
\item 任意 - 任意

  这意味着在屏障之前必须完成装载和存储。
  程序顺序中出现在屏障之后的所有装载和存储都必须等待屏障完成。
\end{itemize}

屏障用于防止不安全的优化发生,并强制执行特定的内存排序。
因此,不必要的屏障指令使用可能会降低软件性能。
在特定情况下,请仔细考虑是否需要屏障,以及使用哪种正确的屏障。

排序规则的更微妙影响是核心的指令界面、数据界面和 MMU 表遍历器被视为单独的观察者。
这意味着您可能需要,例如,使用 DSB 指令以确保在一个接口上的访问被保证在另一个接口上是可观察到的。

如果执行数据缓存清除和使无效指令,例如 DCCVAU,X0,您必须在此之后插入 DSB 指令,以确保后续的页表遍历、翻译表条目修改、指令获取或内存中的指令更新都可以看到新值。

例如,考虑对翻译表的更新:

\begin{lstlisting}
STR X0,[X1]    // 更新一个翻译表条目
DSB ISHST       // 确保写入已完成
TLBI VAE1IS,X2 // 使更改的 TLB 条目无效
DSB ISH         // 确保 TLB 失效已完成
ISB             // 在该处理器上同步上下文
\end{lstlisting}

必须使用 DSB 来确保维护操作完成,并且必须使用 ISB 来确保后续指令看到这些操作的影响。

处理器可能随时对标记为 Normal 的地址进行推测性访问。
因此,在考虑是否需要屏障时,不仅要考虑由装载或存储指令生成的显式访问。

\paragraph{单路屏障}

AArch64
引入了带有隐含屏障语义的新加载和存储指令。
这些指令要求在隐含屏障之前或之后的所有加载和存储操作按照程序顺序观察到。

\begin{description}
\item
  [加载 - 获取 (Load-Acquire, LDAR)]
    程序顺序中在 LDAR 之后的所有加载和存储操作,并且与目标地址的共享域匹配的,必须在 LDAR 之后观察到。
\item
  [存储 - 释放 (Store-Release, STLR)]
    程序顺序中在 STLR 之前的所有加载和存储操作,并且与目标地址的共享域匹配的,必须在 STLR 之前观察到。
\end{description}

还有这些指令的独占版本:LDAXR 和 STLXR。

与数据屏障指令不同,数据屏障指令需要一个限定符来控制哪些共享域可以看到屏障的效果,而 LDAR 和 STLR 指令使用的是被访问地址的属性。

一个 LDAR 指令保证在 LDAR 之后的任何内存访问指令,只有在加载 - 获取之后才可见。
存储 - 释放保证所有较早的内存访问在存储 - 释放变为可见之前都是可见的,并且存储在系统中所有可以存储缓存数据的部分同时可见。

\Figure[caption={单路屏障}, label={fig:one-way-barriers}, width=0.4]{one-way-barriers}

上图展示了访问如何可以在一个方向上穿过单向屏障,但在另一个方向上不能穿过。

\paragraph{详视 ISB}

ARMv8 架构定义了上下文为系统寄存器的状态,并且上下文改变操作包括缓存、TLB 和分支预测器维护操作,或者对系统控制寄存器(例如 SCTLR\_EL1、TCR\_EL1 和 TTBRn\_EL1)的更改。
上下文改变操作的效果只有在上下文同步事件发生后才能被保证看到。

有三种上下文同步事件:

\begin{itemize}
\item
  发生异常。
\item
  从异常返回。
\item
  指令同步屏障(ISB)。
\end{itemize}

ISB 会刷新流水线,从缓存或内存重新获取指令,并确保在 ISB 之前完成的任何上下文改变操作的效果对 ISB 之后的任何指令可见。
它还确保在 ISB 指令之后的任何上下文改变操作仅在 ISB 执行后生效,并且不会被 ISB 之前的指令看到。
这并不意味着在每个修改处理器寄存器的指令之后都需要一个 ISB。
例如,对 PSTATE 字段、ELRs、SPs 和 SPSRs 的读写相对于其他指令按程序顺序执行。

以下示例展示了如何在 AArch64 中启用浮点单元和 NEON,这可以通过写入 CPACR\_EL1 寄存器的第{[}20{]}位来实现。
ISB 是一个上下文同步事件,保证在任何后续或 NEON 指令执行之前启用操作已经完成。

\begin{lstlisting}
MRS X1, CPACR_EL1
ORR X1, X1, #(0x3 << 20)
MSR CPACR_EL1, X1
ISB
\end{lstlisting}

\paragraph{C 中使用屏障}

C11 和 C++11 语言提供了一个良好的平台无关的内存模型,如果可能的话,使用它们比使用内在函数更优。

所有版本的 C 和 C++都有序列点(sequence points),但 C11 和 C++11 还提供了内存模型。
序列点只防止编译器重新排序 C++源代码。
没有任何机制可以阻止处理器在生成的目标代码中重新排序指令,或读写缓冲区重新排序数据传输发送到缓存的顺序。
换句话说,它们仅适用于单线程代码。
对于多线程代码,要么使用 C11/C++11 的内存模型特性,要么使用操作系统提供的其他同步机制,例如互斥锁(mutexes)。
通常情况下,编译器不能跨越序列点重新排列语句,并且限制了编译器可以进行的优化。
代码中序列点的示例包括函数调用和对易失性变量的访问。

C 语言规范对序列点的定义如下:

“在执行序列中的某些特定点,称为序列点,之前所有评估的副作用都应完成,并且后续评估的副作用尚未发生。”

\BlockDesc{Linux 内核中的屏障}

Linux 内核包含许多平台无关的屏障函数。
更多详细信息请参见内存屏障的 Linux 内核文档 memory-barriers.txt 文件:
\url{https://git.kernel.org/cgit/linux/kernel/git/torvalds/linux.git/tree/Documentation/}。

\paragraph{non-temporal 加载和存储对}

ARMv8 引入了一个新概念,即非时间相关(non-temporal)加载和存储。
LDNP(Load Non-temporal Pair)和 STNP(Store Non-temporal Pair)指令执行一对寄存器值的读取或写入操作。
同时,它们向内存系统提供一个缓存提示,表明这些数据缓存无用。
这个提示并不禁止内存系统活动,如地址缓存、预加载或聚合,而只是表示缓存可能不会提升性能。
一个典型的使用案例是数据流处理,但需要注意的是,有效使用这些指令需要特定于微架构的方法。

非时间相关加载和存储放宽了内存排序要求。
在上述情况下,LDNP 指令可能会在前面的 LDR 指令之前被观察到,这可能导致从 X0 中读取不可预测的地址。

例如:

\begin{lstlisting}
LDR X0, [X3]
LDNP X2, X1, [X0]
\end{lstlisting}

为了纠正上述情况,需要一个显式的加载屏障:

\begin{lstlisting}
LDR X0, [X3]
DMB NSHLD
LDNP X2, X1, [X0]
\end{lstlisting}

\paragraph{内存属性}

系统的内存映射被划分成多个区域。
每个区域可能需要不同的内存属性,例如访问权限,包括不同权限级别的读写权限、内存类型和缓存策略。
功能性的代码和数据通常在内存映射中被组合在一起,并且每个区域的属性分别受到控制。
这一功能由内存管理单元(Memory Management Unit, MMU)执行。
翻译表条目使得 MMU 硬件能够将虚拟地址翻译为物理地址。
此外,这些条目还指定了与每个页面相关的一系列属性。

下图显示了在一级块条目中如何指定内存属性。
翻译表中的块条目定义了每个内存区域的属性。
二级条目有不同的布局。

\Figure[caption={1 级块内存属性}, label={fig:s1-blk-mem-attr}, width=1]{s1-blk-mem-attr}

其中:

\begin{itemize}
\item
  \textbf{UXN}(Unprivileged eXecute Never)和\textbf{PXN}(Privileged
  eXecute Never)是执行权限。
\item
  \textbf{AF}(Access Flag)是访问标志。
\item
  \textbf{SH}(Shareable attribute)是可共享属性。
\item
  \textbf{AP}(Access Permission)是访问权限。
\item
  \textbf{NS}(Non-Secure bit)是安全位,但仅在 EL3 和 Secure EL1 时使用。
\item
  \textbf{Indx}是 Memory Attribute Indirection
  Register(MAIR\_ELn)的索引。
\end{itemize}

描述符格式支持分层属性,因此在一个级别设置的属性可以被下层继承。
这意味着在 L0、L1 或 L2 表中的表项可以覆盖它所指向的表中指定的一个或多个属性。
这可以用于访问权限、安全性和执行权限。
例如,L1 表中的一个具有 NSTable=1 的条目意味着忽略了它所指向的 L2 和 L3 表中的 NS 位,并且所有条目都被视为具有 NS=1。
这个特性只限制了同一级别的后续查找。

\BlockDesc{可缓存和可共享的内存属性}

标记为 Normal 的内存区域可以被指定为缓存或非缓存。
有关可缓存内存的更多信息,请参阅第下一章节“多核处理器”。
内存缓存可以通过内外属性分别控制,以适应多级缓存。
内外属性之间的划分是实现定义的,但通常内属性用于处理器集成的缓存,而外属性则通过外部内存总线从处理器导出,因此可能被处理器或集群外部的缓存硬件使用。

共享属性用于定义一个位置是否与多个核心共享。
将一个区域标记为 Non-shareable 意味着它仅由该核心使用,而将其标记为内共享(inner shareable)或外共享(outer shareable),或两者皆是,意味着该位置与其他观察者共享,例如 GPU 或 DMA 设备可以被认为是另一个观察者。
同样,内外属性之间的划分是实现定义的。
这些属性的架构定义使我们能够定义一组观察者,对于这些观察者,共享属性使数据或统一缓存对数据访问透明。
这意味着系统提供硬件一致性管理,使得在内共享域中的两个核心必须看到标记为内共享的位置的一致副本。
如果系统中的处理器或其他主设备不支持一致性,则它必须将共享区域视为非缓存的。

\Figure[caption={内部和外部共享区域}, label={fig:inner-outer-sh-domains}, width=0.8]{inner-outer-sh-domains}

缓存一致性硬件会带来一定的开销。
数据内存访问可能会比原本需要的更长,并且消耗更多的功率。
通过将一致性保持在更少的主设备之间并确保它们在硅片上物理上靠在一起,可以最小化这种开销。
因此,架构将系统分成域,使得可以将开销限制在仅需要一致性的位置。

以下是可用的共享性域选项:

\begin{description}
\item
  [非共享] 表示仅由单个处理器或其他代理访问的内存,因此内存访问永远不需要与其他处理器同步。
  这种域通常不用于 SMP 系统。
\item
  [内部共享] 表示可以由多个处理器共享的共享性域,但不一定是系统中所有代理。
  系统可能具有多个内部共享性域。
  影响一个内部共享性域的操作不会影响系统中的其他内部共享性域。
  这样一个域的示例可能是一个四核 Cortex-A57 集群。
\item
  [外部共享] 外部共享(OSH)域被多个代理共享,可以由一个或多个内部共享域组成。
  影响外部共享域的操作也会隐含地影响其中所有的内部共享域。
  但是,它在其他方面不会像内部共享操作那样行为。
\item
  [完整系统] 在完整系统上的操作会影响系统中的所有观察者。
\end{description}


\subsection{多核处理器}

ARMv8-A 架构为包含多个处理单元的系统提供了显著的支持。
像 Cortex-A57 的 MPCore 和 Cortex-A53 的 MPCore 这样的 ARM 多核处理器可以包含一到四个核心。
使用 Cortex-A57 或 Cortex-A53 处理器的系统几乎总是以这种方式实现。
一个多核处理器可能包含几个能够独立执行指令的核心,这些核心可以被视为一个单元或集群。
ARM 的多核技术使得集群内的任何一个组成核心在不使用时可以关闭以节省功耗,例如当设备负载较轻或处于待机模式时。
当需要更高性能时,所有处理器都在使用以满足需求,同时共享工作负载以保持尽可能低的功耗。

多处理可以定义为在包含两个或多个核心的单个设备内同时运行两个或多个指令序列。
这种技术现在被广泛应用于用于通用应用处理器的系统以及更传统定义的嵌入式系统中。

多核系统的总体能耗可以显著低于基于单个处理器核心的系统。
多个核心可以使得执行更快完成,因此系统的某些功能单元可以完全关闭更长时间。
或者,一个具有多个核心的系统可能能够以比单处理器实现相同吞吐量所需的频率更低的频率运行。
较低功耗的硅工艺或较低的电源电压可以降低功耗和减少的能量使用。
大多数当前系统不允许独立更改核心的频率。
然而,每个核心可以动态时钟门控(clock gated),从而提供额外的功率和能量节省。

拥有多个核心还为系统配置提供了更多选项。
例如,你可能拥有一个系统,其中一个核心用于处理硬实时需求,另一个核心用于需要高性能且不中断的应用。
这些可以整合到一个多处理器系统中。

多核设备也可能比单核设备更具响应能力。
当中断分布在多个核心之间时,有多个核心可以响应中断,每个核心需要处理的中断更少。
多个核心还使一个重要的后台进程能够与一个重要但不相关的前台进程同时进行。

\subsubsection{多处理系统}

我们可以将系统区分为:

\begin{itemize}
  \item
  单处理器包含一个核心。
  \item
  多核处理器,例如 Cortex-A53,具有多个能够独立执行指令的核心,并且可以被系统设计者或操作系统外部视为单个单元或集群,操作系统可以从应用层抽象出底层资源。
  \item
  多个集群(如图~\ref{fig:bl-system} 所示),每个集群包含多个核心。
\end{itemize}

以下对多处理系统(multi-processing system)的描述定义了本书中使用的术语。
在其他操作系统上,这些术语可能有不同的含义。

\paragraph{确定那个核在运行代码}

全局初始化通常由在单个核心上运行的代码执行,随后在所有核心上进行本地初始化。
多处理器关联寄存器(MPIDR\_EL1)使软件能够确定它正在执行的核心,无论是在一个集群内还是在具有多个集群的系统中,它都可以确定在哪个核心和哪个集群中执行。

在一些处理器配置中,U 位指示这是一个单核心还是多核集群。
关联字段提供核心相对于其他核心位置的分层描述。
通常,关联 0 是集群内的核心 ID,关联 1 是集群 ID。

\begin{Tcbox}[title={注意}]
在 EL1 运行的软件可能在由虚拟机管理程序管理的虚拟机中运行。
为了配置虚拟机,EL2 或 EL3 可以在运行时将 MPIDR\_EL1 设置为不同的值,使特定虚拟机看到每个虚拟核心的一个一致且唯一的值。
虚拟核心与物理核心之间的关系由虚拟机管理程序控制,并可能随着时间变化。

MPIDR\_EL3 包含每个物理核心的不可更改的 ID。
没有两个核心共享相同的 MPIDR\_EL3 值。
\end{Tcbox}

\paragraph{对称多处理系统}

对称多处理(SMP)是一种软件架构,能够动态确定各个核心的角色。
集群中的每个核心对内存和共享硬件都有相同的视图。
任何应用程序、进程或任务都可以在任何核心上运行,操作系统调度程序可以动态地在核心之间迁移任务,以实现系统负载的优化。
多线程应用程序可以同时在多个核心上运行,操作系统能够隐藏应用程序的许多复杂性。

在本指南中,操作系统下运行的每个应用程序实例称为进程。
应用程序通过调用系统库来执行许多操作,系统库提供某些功能的库代码,也作为对内核操作的系统调用的包装器。
各个进程有相关的资源,包括栈、堆和常量数据区,以及诸如调度优先级设置等属性。
内核对进程的视图称为任务。
进程是共享某些公共资源的任务集合。
其他操作系统可能有不同的定义。

在描述 SMP 操作时,我们使用术语“内核”来表示操作系统中包含异常处理程序、设备驱动程序以及其他资源和进程管理代码的那部分。
我们还假设存在一个通常通过计时器中断调用的任务调度程序。
调度程序负责在多个任务之间切片核心的可用周期,动态确定各个任务的优先级,并决定下一个要运行的任务。

线程是在同一进程空间内执行的独立任务,使应用程序的不同部分能够在不同核心上并行执行。
它们还允许应用程序的一部分在等待资源时继续执行。

一般而言,一个进程内的所有线程共享多个全局资源(包括相同的内存映射以及对任何打开文件和资源句柄的访问)。
线程也有自己的局部资源,包括它们自己的栈和寄存器使用,这些在上下文切换时由内核保存和恢复。
然而,这些资源是局部的并不意味着任何线程的局部资源能够得到其他线程不正确访问的保护。
线程是独立调度的,即使在同一进程中,它们也可以有不同的优先级。

一个支持 SMP 的操作系统向应用程序提供了可用核心资源的抽象视图。
在 SMP 系统中,多个应用程序可以同时运行而无需重新编译或更改源代码。
一个传统的多任务操作系统使系统能够在单核或多核处理器中同时执行多个任务或活动。
在多核系统中,我们可以实现真正的并发,在不同核心上同时并行地运行多个任务。
管理这些任务在可用核心之间分配的角色由操作系统执行。

通常,操作系统任务调度程序可以在系统的可用核心之间分配任务。
这一功能被称为负载均衡,旨在获得更好的性能或节能,甚至两者兼有。
例如,在某些类型的工作负载中,如果工作负载的任务被调度到较少的核心上,可以实现节能。
这将使更多的资源能够闲置更长时间,从而节省能源。

在其他情况下,如果任务分布在更多的核心上,工作负载的性能可能会提高。
这些任务可以比在较少核心上运行时更快地向前推进,而不会相互干扰。

另一个案例是,可能值得在更多的核心上以较低频率运行任务,而不是在较少的核心上以较高频率运行。
这样做可以在节能和性能之间提供更好的权衡。

SMP 系统中的调度程序可以动态地重新优先级任务。
这种动态任务优先级使其他任务可以在当前任务休眠时运行。
例如,在 Linux 中,性能受处理器活动限制的任务可以降低其优先级,以便性能受 I/O 活动限制的任务可以提高其优先级。
I/O 绑定的进程中断计算绑定的进程,以便它可以启动其 I/O 操作,然后返回休眠,而处理器可以在 I/O 操作完成时执行计算绑定的代码。

中断处理也可以在核心之间进行负载均衡。
这可以帮助提高性能或节省能源。
在核心之间平衡中断或为特定类型的中断保留核心可以减少中断延迟。
这也可能导致减少缓存使用,从而帮助提高性能。

使用更少的核心来处理中断可能会使更多的资源闲置更长时间,从而在降低性能的情况下节省能源。
Linux 内核不支持自动中断负载均衡。
然而,内核提供了将中断绑定到特定核心的机制。
有一些开源项目,例如 irqbalance(https://github.com/Irqbalance/irqbalance),使用这些机制在可用核心之间分配中断。
irqbalance 能够识别系统属性,例如共享缓存层次结构(哪些核心有公共缓存)和电源域布局(哪些核心可以独立关闭)。
然后,它可以确定最佳的中断到核心的绑定。

SMP 系统按定义,在集群中的核心之间共享内存。
为了向应用软件保持所需的抽象级别,硬件必须负责为你提供一致和一致的内存视图。

对共享内存区域的更改必须对所有核心可见,而无需任何显式的软件一致性管理,尽管需要同步指令(如屏障)来确保以正确的顺序看到更新。
同样,对内存映射的任何更新(例如,由于需求分页、新内存的分配或将设备映射到当前虚拟地址空间)都必须一致地呈现给所有核心。

\paragraph{时钟(Timer)}

在支持 SMP(对称多处理)的操作系统内核中,任务调度器负责在多个任务之间分配核心的可用周期。
它动态确定各个任务的优先级,并决定每个核心下一个要运行的任务。
通常需要一个计时器,以便能够定期中断每个核心上的活动任务,从而让调度程序有机会选择其他任务继续运行。

当所有核心争夺相同的关键资源时,可能会出现问题。
每个核心都运行调度程序来决定它应该执行的任务,而这是在固定间隔内发生的。
内核调度程序代码需要使用一些共享数据,例如任务列表,这些数据可以通过排除(由互斥锁提供)来保护免受并发访问。
互斥锁允许在任何时候只有一个核心能有效地运行调度程序。

\subparagraph*{系统计时器架构}

系统计时器架构描述了一个通用的系统计数器,为每个核心提供多达四个计时器通道。
这个系统计数器应具有固定的时钟频率。
系统计时器包括安全和非安全物理计时器,以及两个用于虚拟化的计时器。
每个通道都有一个比较器,将其与一个系统范围内的 64 位计数值进行比较,该计数值从零开始计数。
可以配置计时器,使得当计数大于或等于已编程的比较器值时生成中断。

虽然系统计时器必须具有固定的频率(通常以 MHz 为单位),但允许不同的更新粒度。
这意味着计时器可以在每个时钟周期递增 1 的同时,每 10 或 100 个周期递增一个较大的数值(例如 10 或 100),以相应降低的速率更新。
这提供了相同的有效频率,但减少了更新粒度,这在实现较低功耗状态时非常有用。

\subparagraph*{关键寄存器}

\begin{itemize}
  \item
    \textbf{CNTFRQ\_EL0}:报告系统计时器的频率。
    需要注意的是,CNTFRQ\_EL0 只是每个核心的一个寄存器,但从固件的角度来看,所有其他软件应该已经看到这个寄存器在所有核心上初始化为相同的正确值。
    计数频率是全局的,并且对于所有核心是固定的。
    CNTFRQ\_EL0 为启动 ROM 或固件提供了一种方便的方式,以告诉其他软件全局计数频率,但它不控制任何硬件行为的方面。
  \item
    \textbf{CNTPCT\_EL0}:报告当前的计数值。
  \item
    \textbf{CNTKCTL\_EL1}:控制是否 EL0 可以访问系统计时器。
\end{itemize}

\subparagraph*{配置计时器的步骤}

\begin{enumerate}
  \item
    \textbf{写入比较值到 CNTP\_CVAL\_EL0}:这是一个 64 位寄存器,用于设置比较值。
  \item
    \textbf{在 CNTP\_CTL\_EL0 中启用计数器和中断生成}:这一步启动计数器并启用中断。
  \item
    \textbf{轮询 CTP\_CTL\_EL0 以报告 EL0 计时器中断的原始状态}:检查计时器中断的状态。
\end{enumerate}

\subparagraph*{使用计时器作为倒计时计时器}

在这种情况下,所需的计数被写入到 32 位的 CNTP\_TVAL\_EL0 寄存器。
硬件将为你计算正确的 CNTP\_CVAL\_EL0 值。

这种架构和机制使得在多核系统中实现精确的定时和同步变得更加容易,从而提高系统的性能和响应能力。

\paragraph{同步}\label{sec:synchronization}

在 SMP(对称多处理)系统中,数据访问通常需要限制在任意特定时间内只能由一个修改者进行。
这对于外围设备以及由多个线程或进程访问的全局变量和数据结构都是如此。
保护这些共享资源的常用方法是互斥(mutual exclusion)。
在多核系统中,可以使用自旋锁(spinlock)来实现,这实际上是一个共享的标志,通过原子不可分割的机制来测试和设置其值。

ARM 架构提供了三条与独占访问相关的指令及其变体,这些指令可用于字节、半字、字或双字大小的数据。
这些指令依赖于核心或内存系统的能力,将特定地址标记为由该核心进行独占访问监控,使用独占访问监视器。
使用这些指令在多核系统中很常见,但在单核系统中也用于在同一核心上运行的线程之间实现同步操作。

A64 指令集有以下指令来实现这样的同步功能:

\begin{itemize}
\item
  \textbf{Load Exclusive (LDXR)}:
  {\lstinline!LDXR W|Xt, [Xn]!}
\item
  \textbf{Store Exclusive (STXR)}:
  {\lstinline!STXR Ws, W|Xt, [Xn]!}
  其中 Ws 指示存储是否成功,0 表示成功。
\item
  \textbf{Clear Exclusive Access Monitor (CLREX)}:
  用于清除本地独占监视器的状态。
\end{itemize}

\subparagraph*{这些指令的工作机制}

\begin{itemize}
\item
  \textbf{LDXR}
  执行内存加载,并将物理地址标记为由该核心监视的独占访问地址。
\item
  \textbf{STXR}
  执行条件存储,只有当目标位置被该核心标记为独占访问时才会成功。
  如果存储不成功,该指令在通用寄存器 Ws 中返回非零值;
  如果存储成功,返回 0。
  汇编语法中,Ws 始终指定为 W 寄存器,即非 X 寄存器。
  此外,STXR 会清除独占标记。
\end{itemize}

\subparagraph*{独占访问指令的要求}

独占加载和存储操作仅保证在映射了以下属性的普通内存上工作:

\begin{itemize}
\item
  内部或外部可共享。
\item
  内部写回。
\item
  外部写回。
\item
  读写分配提示。
\item
  非瞬态。
\end{itemize}

\subparagraph*{使用自旋锁或互斥锁}

自旋锁或互斥锁可以用于控制对外围设备的访问。
锁的位置应在普通 RAM 中。
独占加载或存储指令不会直接访问外围设备。

每个核心只能标记一个地址。
独占监视器并不会阻止其他核心或线程读取或写入被监视的位置,而只是监视自从 LDXR 之后该位置是否被写入。

\subparagraph*{程序员的责任}

虽然架构和硬件支持独占访问的实现,但依赖于程序员强制执行正确的软件行为。
互斥锁仅仅是一个标志,而独占访问机制使得这个标志可以以原子的方式进行访问。
任何访问标志的线程或程序都可以知道它是否被正确设置。
然而,互斥锁控制的实际资源仍然可以被不正确行为的软件直接访问。
同样,用于存储互斥锁的内存没有特殊属性。
当独占访问序列完成后,它只是内存中的另一段数据。

\subparagraph*{内存顺序考虑}

在编写使用互斥锁进行资源保护的代码时,理解弱序内存模型是至关重要的。
例如,如果没有正确使用屏障和其他内存排序考虑,推测执行可能意味着在获得互斥锁之前数据已经被加载,或者在关键资源更新之前释放了互斥锁。
有关内存排序考虑的更多信息,请参见内存排序章节。

总之,正确实现和使用互斥锁以及独占访问指令对于确保多核系统中共享资源的安全和正确访问至关重要。
理解并应用内存排序的概念是确保这些机制正确运行的关键。

\paragraph{不对称多处理系统}

AMP(非对称多处理)系统允许将单个集群中的每个核心分配特定角色,这意味着每个核心都执行各自的任务,这被称为功能分布的软件架构。
通常,这意味着在各个核心上运行单独的操作系统(OS)。
这样的系统可以表现为一个单核系统,但配有用于某些关键系统服务的专用加速器。
AMP 系统不涉及将任务或中断分配给特定核心。

\subparagraph*{AMP 系统的特点和用途}

\begin{itemize}

\item
  \textbf{独立的任务视图}:每个任务可以有不同的内存视图,不能将高负载的工作转移给低负载的核心。
\item
  \textbf{硬件缓存一致性}:这类系统通常不需要硬件缓存一致性,尽管可能需要通过共享资源(可能需要专用硬件)在核心之间进行通信。
\item
  \textbf{实现原因}:实施 AMP 系统的原因可能包括安全性、实时性需求或因为各个核心专用于执行特定任务。
\item
  \textbf{SMP 和 AMP 的结合}:一些系统结合了 SMP 和 AMP 特性,多个核心运行 SMP
  OS,同时系统有不作为 SMP 系统一部分的其他元素。
  SMP 子系统可以看作 AMP 系统中的一个元素,SMP 核心之间实现了缓存一致性,但这种一致性不一定扩展到 AMP 元素。
\item
  \textbf{多操作系统}:在 AMP 系统中,各个核心可以运行不同的操作系统,这种系统称为多操作系统(Multi-OS)系统。
\end{itemize}

\subparagraph*{通信和同步}

对于需要同步的不同核心,可以通过消息传递通信协议实现,如多核通信协会 API(MCAPI)。
这种通信可以通过共享内存传递数据包,并通过软件触发中断实现所谓的“门铃”机制。

\subparagraph*{具体实现方法}

\begin{itemize}

\item
  \textbf{消息传递协议(MCAPI)}:用于核心间的通信。
\item
  \textbf{共享内存}:用于传递数据包。
\item
  \textbf{软件触发中断}:用于实现同步机制。
\end{itemize}

AMP 系统在多核处理器上实现的原因多种多样,包括:

\begin{itemize}

\item
  \textbf{安全性}:独立核心运行不同 OS 可以提高系统安全性。
\item
  \textbf{实时性}:保证实时任务的截止期限。
\item
  \textbf{专用任务}:各个核心专门用于执行特定任务,提高效率。
\end{itemize}

\subparagraph*{具体例子}

\begin{itemize}

\item
  \textbf{安全和实时性}:一个核心运行实时操作系统(RTOS)处理实时任务,另一个核心运行 Linux 处理非实时任务。
\item
  \textbf{专用任务分配}:一个核心专用于音频处理,另一个核心专用于视频处理。
\end{itemize}

通过这种方式,AMP 系统可以在满足特定需求的同时,充分利用多核处理器的能力,实现更高效和安全的系统设计。

\paragraph{异构多处理系统}

异构多处理(Heterogeneous multi-processing,HMP)这一术语在许多不同的情境中都有应用。
它经常与 AMP 混淆,用来描述由不同类型的处理器组成的系统,例如多核 ARM 应用处理器和应用特定处理器(例如基带控制器芯片或音频编解码器芯片)。

ARM 将 HMP 定义为由应用处理器集群组成的系统,它们在指令集架构上完全相同,但在微体系结构上非常不同。
所有处理器都具有完全的缓存一致性,并且属于同一一致性域。

最好的例子是使用 ARM 实现的 HMP 技术,即 big.LITTLE。
在 big.LITTLE 系统中,能源高效的 LITTLE 核心与高性能的 big 核心相结合,形成一个系统,可以以最节能的方式完成高强度和低强度任务。

\Figure[caption={一个典型的大小核系统}, label={fig:bl-system}, width=0.5]{bl-system}

big.LITTLE 的核心原则是应用软件可以在任何类型的处理器上不经修改地运行。
有关 big.LITTLE 技术及其软件执行模型的详细概述,请参阅第~\ref{sec:big-little-tec} 章。

这种技术的出现源于对能源高效性和性能的需求。
通过将高性能和低功耗核心结合在一起,big.LITTLE 技术实现了在不同工作负载下的最佳平衡。

\paragraph{独占监视器区域}

一个典型的多核系统可能包括多个独占监视器。
每个核心都有自己的本地监视器,还有一个或多个全局监视器。
与独占加载或存储指令相关的转换表条目的可共享和可缓存属性决定了使用哪个独占监视器。

在硬件中,核心包括一个名为本地监视器的设备。
该监视器观察核心。
当核心执行独占加载访问时,它会在本地监视器中记录这一事实。
当它执行独占存储时,它会检查之前是否执行了独占加载,并且如果不是这种情况,则独占存储失败。
体系结构使得各个实现能够确定监视器执行的检查级别。
核心一次只能标记一个物理地址。

本地独占监视器在每次异常返回时被清除,也就是执行 ERET 指令时。
在 Linux 内核中,多个任务在 EL1 内核上下文中运行,并且可以在没有异常返回的情况下进行上下文切换。
只有当我们返回到与其相关联的内核任务的用户空间线程时,才执行异常返回。
这与 ARMv7 架构不同,在 ARMv7 架构中,内核任务调度器必须在每次任务切换时显式清除独占访问监视器。
重置本地独占监视器是否还会重置全局独占监视器是由实现定义的。

当用于独占访问的位置标记为不可共享时,即只能由同一核心上运行的线程访问时,会使用本地监视器。
本地监视器还可以处理访问被标记为内部可共享的情况,例如,在任何核心上运行的 SMP 线程之间共享的资源受到互斥锁的保护。
对于运行在不同、不一致的核心上的线程,互斥锁位置被标记为普通、非缓存,并且在系统中需要一个全局访问监视器。

系统可能不包括全局监视器,或者全局监视器可能仅适用于某些地址区域。
如果对不存在适当监视器的位置执行独占访问,则由实现定义会发生什么。
以下是一些允许的选项:

\begin{itemize}
\item
  指令生成外部异常。
\item
  指令生成 MMU 故障。
\item
  指令被视为 NOP。
\item
  将独占指令视为标准的 LDR/STR
  指令,存储独占指令的结果寄存器中的值变为未知。
\end{itemize}

独占保留颗粒度(ERG)是独占监视器的粒度。
它的大小是由实现定义的,但通常为一个缓存行。
它提供了监视器在区分地址之间的最小间距。
将两个互斥锁放置在单个 ERG 中可能会导致误报,即执行 STXR 指令到任一互斥锁会清除两者的独占标签。
这不会阻止架构上正确的软件正常运行,但可能效率较低。
可以从特定核心的缓存类型寄存器 CTR\_EL0 中读取独占监视器的 ERG 大小。

\subsubsection{缓存一致性}

第~\ref{sec:caches} 章只考虑了单个处理器内缓存的影响。
Cortex-A53 和 Cortex-A57 处理器支持在集群中的不同核心之间进行一致性管理。
这需要将地址区域标记为正确的可共享属性。
这些处理器允许构建包含多核集群的系统,其中可以维护集群之间共享数据的一致性。
这种系统级的一致性需要一个缓存一致的互连,例如 ARM 的 CCI-400,它实现了 AMBA 4 ACE 总线规范。
参见下图。

\Figure[caption={Cache 一致性组}, label={fig:cache-coherency-groups}, width=0.5]{cache-coherency-groups}

系统中的一致性支持取决于硬件设计决策,存在许多可能的配置。
例如,一致性只能在单个集群内支持。
一个双集群 big.LITTLE 系统是可能的,其中内部域包括两个集群的核心,或者一个多集群系统,其中内部域包括一个集群,外部域包括其他集群。
有关 big.LITTLE 系统的更多信息,请参见第~\ref{sec:big-little-tec} 章的 big.LITTLE 技术。

除了硬件之外,用于在缓存之间维护数据一致性的广播缓存维护活动的能力也是必要的,该活动由在一个核心上运行的代码执行。
在重置时会对硬件配置信号进行采样,以控制内部、外部或两者缓存维护操作是否广播,以及系统屏障指令是否广播。
AMBA 4 ACE 协议允许将屏障信号传递给其他主控器,以保持维护和一致性操作的顺序。
互连逻辑可能需要由引导代码进行初始化。

软件必须定义哪些地址区域由哪组主控器使用,也就是说,哪些其他主控器共享这个地址,方法是创建适当的转换表条目。
对于正常的可缓存区域,这意味着将共享属性设置为非共享、内部共享或外部共享之一。
对于非可缓存区域,共享属性将被忽略。

在多核系统中,不可能知道特定核心是否具有覆盖其缓存中特定地址的缓存行(特别是在具有缓存功能的互连中,如 CCN-50x)。

维护操作可能需要广播到互连。
这意味着位于一个核心上的软件可以向一个地址发出缓存清理或使其失效的操作,而该地址目前可能存储在持有该地址的另一个核心的数据缓存中。
如下图所示,当维护操作被广播时,该操作将由特定共享性域中的所有核心执行。

\Figure[caption={广播 cache 操作到其它核}, label={fig:broadcast-cache-op-to-other-cores}, width=0.9]{broadcast-cache-op-to-other-cores}

SMP 操作系统通常依赖于能够广播缓存和 TLB 维护操作。
考虑一种情况,外部 DMA 引擎能够修改外部存储器的内容。

运行在特定核心上的 SMP 操作系统不知道哪个核心有哪些数据。
它只需要在集群中的任何位置使地址范围失效。
如果操作不被广播,操作系统必须在每个核心上本地发出清理或使其失效的操作。
DSB 屏障指令使得核心等待其发出的广播操作完成。
但该屏障不会强制接收到的广播操作完成。
有关屏障指令的更多信息,请参阅第~\ref{sec:memory-ordering} 章内存排序。

下表列出了第~\ref{sec:caches} 章中描述的缓存维护操作以及它们是否被广播。

% \begin{minipage}{\textwidth}
%   \centering
  \begin{ltblr}[caption={带有广播的指令}, label={tbl:inst-with-broadcast}]
    {colspec={c>{\centering\arraybackslash}Xc}}
    \hline[1pt]
    指令 & 描述 & 广播?\\
    \hline
    IC IALLUIS   & I-cache invalidate all to Point of Unification, Inner Shareable & Yes (inner only) \\
    IC IALLU     & I-cache invalidate all to Point of Unification                  & No\footnotemark[1]              \\
    IC IVAU, Xt  & I-cache invalidate by address to Point of Unification           & Maybe\footnotemark[2]           \\
    DC ZVA, Xt   & D-cache zero by address                                         & No               \\
    DC IVAC, Xt  & D-cache invalidate by address to Point of Coherency             & Yes              \\
    DC ISW, Xt   & D-cache invalidate by Set/Way                                   & No               \\
    DC CVAC, Xt  & D-cache clean by address to Point of Coherency                  & Maybe\footnotemark[2]           \\
    DC CSW, Xt   & D-cache clean by Set/Way                                        & No               \\
    DC CVAU, Xt  & D-cache clean by address to Point of Unification                & Maybe\footnotemark[2]           \\
    DC CIVAC, Xt & D-cache clean and invalidate by address to Point of Coherency   & Yes              \\
    DC CISW, Xt  & D-cache clean and invalidate by Set/Way                         & No               \\
    \hline[1pt]
  \end{ltblr}
  \footnotetext[1]{在非安全 EL1 中,如果 HCR/HCR\_EL2 的 FB 位被设置,这将覆盖正常行为。
  该位会导致以下指令在非安全模式下从 EL1 执行时在内部共享域内广播:

  EL1: TLBI VMALLE1, TLBI VAE1, TLBI ASIDE1, TLBI VAAE1, TLBI VALE1, TLBI VAALE1, IC IALLU。
  }  \footnotetext[2]{内存区域的可共享性决定了广播行为。
  }  % \footnotetext{请看上一脚注}
  % \footnotetext{请看上一脚注}
% \end{minipage}

对于 IC 指令,即指令缓存维护操作,IS 表示该功能适用于内部共享域内的所有指令缓存。

\subsubsection{集群内的多核缓存一致性}

一致性(Coherency)意味着确保系统内所有处理器或总线主控设备对共享内存具有相同的视图。
这意味着一个核心缓存中数据的变化对其他核心是可见的,防止核心看到过时或旧的数据副本。
这可以通过不缓存来处理,即对共享内存位置禁用缓存,但这通常会带来高性能成本。

\paragraph*{软件管理的一致性}

软件管理的一致性是一种更常见的数据共享处理方式。
数据被缓存,但软件(通常是设备驱动程序)必须清除脏数据或从缓存中无效旧数据。
这需要时间,增加了软件的复杂性,并且在高共享率情况下会降低性能。

\paragraph*{硬件管理的一致性}

硬件在集群内的一级数据缓存之间维护一致性。
当核心启动并启用其数据缓存(D-cache)和内存管理单元(MMU)时,如果地址被标记为一致性地址,核心会自动参与一致性方案。
然而,这种缓存一致性逻辑\textbf{不}维护数据缓存和指令缓存之间的一致性。

在 ARMv8-A 架构及其相关实现中,可能会存在硬件管理的一致性方案。
这些方案确保在硬件一致性系统中标记为共享的数据在该共享域内的所有核心和总线主控设备中具有相同的值。
这增加了互连和集群的硬件复杂性,但极大简化了软件,并使得一些仅靠软件一致性无法实现的应用成为可能。

缓存一致性方案可以通过多种标准方式运行。
ARMv8 处理器使用 MOESI 协议。
ARMv8 处理器还可以连接到 AMBA 5 CHI 互连系统,其缓存一致性协议类似于但不完全等同于 MOESI。

通过硬件管理的一致性,系统性能和软件复杂度可以达到更好的平衡,确保共享内存的正确性和有效性。

根据使用的协议,SCU(Snoop Control
Unit)将缓存中的每一行标记为以下五种属性之一:M(Modified,已修改)、O(Owned,已拥有)、E(Exclusive,独占)、S(Shared,共享)或 I(Invalid,无效)。
这些属性的具体描述如下:

\begin{description}
\item [已修改 (Modified, M)] \hfill
\begin{itemize}
\item
  \textbf{描述}:
  缓存行中包含的最新版本数据仅存在于这个缓存中,其他缓存中不存在该内存位置的副本。
  缓存行的内容与主内存已经不一致。
\item
  \textbf{关键点}: 数据已被修改且只存在于当前缓存中,不再与主内存同步。
\end{itemize}

\item [{已拥有 (Owned, O)}] \hfill
\begin{itemize}
\item
  \textbf{描述}:
  缓存行是脏数据(即已被修改)并且可能存在于多个缓存中。
  拥有状态的缓存行持有数据的最新、正确副本。
  只有一个核心可以持有数据的拥有状态,其他核心可以持有数据的共享状态。
\item
  \textbf{关键点}:
  数据是最新的、正确的,并且可能在多个缓存中存在,但只有一个缓存持有该数据的修改权限。
\end{itemize}

\item [{独占 (Exclusive, E)}] \hfill
\begin{itemize}

\item
  \textbf{描述}:
  缓存行存在于这个缓存中且与主内存一致。
  其他缓存中不存在该内存位置的副本。
\item
  \textbf{关键点}:
  数据是最新的且未被修改,只有当前缓存拥有该数据的副本。
\end{itemize}

\item [{共享 (Shared, S)}] \hfill
\begin{itemize}
\item
  \textbf{描述}:
  缓存行存在于这个缓存中,数据不一定与内存一致,因为拥有状态允许脏数据复制为共享数据。
  然而,这行数据将是最新版本。
  其他缓存中也可能存在其副本。
\item
  \textbf{关键点}:
  数据是最新的,可能与主内存不同步,并且可能在多个缓存中存在。
\end{itemize}

\item[{无效 (Invalid, I)}] \hfill
\begin{itemize}
\item
  \textbf{描述}: 缓存行无效。
\item
  \textbf{关键点}: 数据不再有效,不能被使用。
\end{itemize}

\end{description}

\begin{Tcbox}[title={扩展}]

\paragraph*{缓存一致性协议}

在多处理器系统中,为了确保所有处理器都能看到相同的内存视图,使用缓存一致性协议。
常见的协议有:

\begin{itemize}

\item
  \textbf{MESI}(Modified, Exclusive, Shared, Invalid):这是最基础的一致性协议,用于标记缓存行的状态。
\item
  \textbf{MOESI}(Modified, Owned, Exclusive, Shared, Invalid):这是 MESI 协议的扩展,增加了“Owned”状态,用于优化共享数据的访问。
\item
  \textbf{MSI}(Modified, Shared, Invalid):这是一个更简单的协议,但效率较低。
\end{itemize}

这些协议通过不同状态的组合和转换,确保在多处理器系统中,数据的一致性和正确性。
这对于现代多核处理器系统尤为重要,因为它们需要高效地管理和共享数据,同时避免数据不一致和竞态条件。

\paragraph*{ARM 缓存一致性}

在 ARM 处理器中,缓存一致性是通过硬件和软件结合的方式实现的。
硬件管理的一致性通过复杂的互连系统和缓存控制逻辑来确保数据的一致性,而软件管理的一致性则需要开发者在应用层面进行精细控制。

ARMv8 处理器通常使用 MOESI 协议,并支持 AMBA 5 CHI 互连系统,这些系统通过高级协议和信号传递机制,确保缓存一致性和系统性能。

缓存一致性对于多处理器系统的性能和正确性至关重要。
通过不同的缓存状态和一致性协议,可以有效地管理多核处理器系统中的数据共享和同步,确保系统的稳定和高效运行。

\end{Tcbox}

标准协议的实现规则如下:

\begin{itemize}
\item
  \textbf{写操作只能在缓存行处于“已修改”或“独占”状态时执行}。
  如果缓存行处于“共享”状态,必须先将所有其他缓存的副本无效化。
  写操作会将缓存行移动到“已修改”状态。
\item
  \textbf{缓存可以随时丢弃共享的缓存行},将其变为“无效”状态。
  如果缓存行处于“已修改”状态,则必须先将其写回主存。
\item
  \textbf{如果缓存行处于“已修改”状态},系统中其他缓存的读取操作将接收到该缓存的更新数据。
  通常,这是通过先将数据写回主存,然后将缓存行状态改为“共享”状态,再进行读取操作来实现的。
\item
  \textbf{如果缓存行处于“独占”状态},当其他缓存读取该缓存行时,必须将其状态改为“共享”状态。
\item
  \textbf{“共享”状态可能不精确}。
  如果一个缓存丢弃了共享的缓存行,另一个缓存可能不会意识到现在可以将该缓存行移动到“独占”状态。
\end{itemize}

\paragraph*{缓存一致性协议中的状态转换}

状态说明:

\begin{itemize}
\item
  \textbf{已修改(Modified, M)}:缓存行被修改,且数据与主存不同步。
  写操作只能在此状态或“独占”状态下执行。
\item
  \textbf{独占(Exclusive, E)}:缓存行数据与主存同步,且该缓存行的唯一副本在当前缓存中。
\item
  \textbf{共享(Shared, S)}:缓存行可能在多个缓存中存在副本,数据可能不同步。
\item
  \textbf{无效(Invalid, I)}:缓存行无效,不包含有效数据。
\end{itemize}

状态转换规则:

\begin{description}
\item [{写操作}] \hfill
  \begin{itemize}
  \item
    若缓存行为“已修改”或“独占”,可直接写入。
  \item
    若缓存行为“共享”,则先使其他缓存副本无效化,再将其状态改为“已修改”。
  \end{itemize}
\item
  [{读操作}] \hfill
  \begin{itemize}
  \item
    若缓存行为“已修改”,其他缓存需从当前缓存读取最新数据,通常是先写回主存再读。
  \item
    若缓存行为“独占”,其他缓存读取时需将其状态改为“共享”。
  \end{itemize}
\item [{丢弃操作}] \hfill
  \begin{itemize}
  \item
    “共享”缓存行可随时丢弃,变为“无效”状态。
  \item
    “已修改”缓存行丢弃前需先写回主存。
  \end{itemize}
\end{description}

处理器簇(cluster)包含一个嗅探控制单元(Snoop Control Unit,
SCU),该单元包含存储在各个 L1 数据缓存中的标签的重复副本。
因此,缓存一致性逻辑负责以下任务:

\begin{itemize}
\item
  \textbf{维护 L1 数据缓存之间的一致性}:SCU 确保所有处理器核心的 L1 数据缓存中的数据保持一致。
  如果一个核心对数据进行了修改,其他核心能够及时获知并更新其缓存。
\item
  \textbf{仲裁对 L2 接口的访问}:SCU 管理对 L2 缓存接口的访问,既包括指令也包括数据。
  这种仲裁确保了多个核心可以高效地访问共享的 L2 缓存资源,而不会发生冲突或数据不一致。
\item
  \textbf{拥有重复的标签 RAMs}:SCU 使用重复的标签 RAM 来跟踪每个核心的数据缓存中分配了哪些数据。
  这些标签用于标识缓存行的状态和位置,以便在需要时快速定位和更新数据。
\end{itemize}

\Figure[caption={缓存一致性逻辑}, label={fig:cache-coherency-logic}, width=0.8]{cache-coherency-logic}

每个核心(如上图所示)都有自己的数据缓存和指令缓存。
缓存一致性逻辑包含了数据缓存(D-cache)的本地标签副本。
然而,指令缓存不参与一致性管理。
数据缓存和一致性逻辑之间存在双向通信。

ARM 多核处理器还实现了一些优化,可以在参与的一致性 L1 缓存之间直接复制干净数据和移动脏数据,而无需访问和等待外部内存。
这些活动在多核系统中由嗅探控制单元(SCU)处理。

多核技术的关键方面包括:

\paragraph*{嗅探控制单元}

嗅探控制单元(SCU)在每个核心的 L1 数据缓存之间维持一致性,并负责管理以下互连操作:

\begin{itemize}
\item
  仲裁。
\item
  通信。
\item
  缓存到缓存和系统内存的传输。
\end{itemize}

处理器还将这些功能暴露给其他系统加速器和非缓存 DMA 驱动的外设,以提高性能并减少整个系统的功耗。
这种系统一致性还减少了在每个操作系统驱动程序中维护软件一致性时涉及的软件复杂性。

每个核心可以被单独配置为参与或不参与数据缓存一致性管理方案。
处理器内部的 SCU 设备自动在集群内的核心之间维护级别 1 数据缓存的一致性。
有关更多信息,请参阅缓存一致性章节和集群内的多核缓存一致性章节。

由于可执行代码变化较少,因此这种功能不会扩展到 L1 指令缓存。
一致性管理使用基于 MOESI 协议的方式实现,通过优化以减少外部内存访问的次数。
为了使内存访问处于一致性管理状态,以下所有条件都必须为真:

\begin{itemize}
\item
  SCU 已启用,通过位于私有内存区域的控制寄存器。
  SCU 具有可配置的访问控制,限制了哪些处理器可以配置它。
\item
  MMU 已启用。
\item
  正在访问的页面标记为 Normal
  Shareable,并具有写回、写分配的缓存策略。
  然而,设备和强序内存不可缓存,并且写透缓存在核心的视角下的行为类似于无缓存内存。
\end{itemize}

\paragraph*{加速器一致性端口}

SCU 只能在单个集群内维护一致性。
如果系统中有其他处理器或其他总线主机,并且这些主机与 MP 块共享内存,则需要显式的软件同步。

SCU 上的这个 AMBA 4
AXI 兼容从设备接口提供了一个连接点,用于与 ARMv8 处理器直接接口的主设备:

\begin{itemize}
\item
  该接口支持所有标准的读写事务,无需额外的一致性要求。
  然而,对一致性内存区域的任何读事务都会与 SCU 进行交互,以检查信息是否已经存储在 L1 缓存中。
\item
  在将写操作转发到内存系统之前,SCU 会强制执行写一致性,并可能分配到 L2 缓存中,从而消除直接写入片外内存的功耗和性能影响。
\end{itemize}

\paragraph*{集群间的缓存一致性}

集群内的多核缓存一致性展示了硬件如何在同一集群中多个处理器缓存之间保持共享数据的一致性。
系统还可以包含硬件,通过处理可共享数据事务和广播屏障和维护操作来在集群之间维护一致性。
集群可以动态地添加或移除一致性管理,例如,当整个集群(包括 L2 缓存)被关闭时。
操作系统可以通过内置的性能监控单元(PMUs)监视一致性互连上的活动。

\paragraph*{域(domain)}

在 ARMv8-A 架构中,术语“域”用于指代一组主要的总线主控器。
域确定了哪些主控器会被监听,以进行一致性事务的处理。
监听是指检查主控器的缓存,看请求的位置是否存储在那里。
有四种定义好的域类型:

\begin{itemize}
\item
  非共享域。
\item
  内部共享域。
\item
  外部共享域。
\item
  系统域。
\end{itemize}

典型的系统使用是,运行在相同操作系统下的主控器属于同一个内部共享域。
共享可缓存数据但不是紧密耦合的主控器属于同一个外部共享域。
同一内部域中的主控器也必须属于同一个外部域。
内存访问的域选择通过页表中的条目来控制。

\Figure[caption={总线主控制器一致性域}, label={fig:bus-master-coherency-domains}, width=1]{bus-master-coherency-domains}

\subsubsection{总线协议和高速缓存一致性的互连}

将硬件一致性扩展到多个集群系统需要一个一致性的总线协议。
AMBA 4
ACE 规范包括 AXI 一致性扩展(ACE)。
完整的 ACE 接口使得集群之间可以实现硬件一致性,并使得 SMP 操作系统可以在多个核心上运行。

如果有多个集群,任何对一个集群内存的共享访问都可以侦听其他集群的缓存,看数据是否存在,或者是否必须从外部内存中加载。
AMBA 4 ACE-Lite 接口是完整接口的子集,设计用于单向 IO 一致性系统主控器,例如 DMA 引擎、网络接口和 GPU。

这些设备可能没有自己的缓存,但可以从 ACE 处理器读取共享数据。
非核心主控器的缓存通常不与核心缓存保持一致。
例如,在许多系统中,核心无法在从属端口上的 GPU 缓存内进行侦听。
但反过来未必如此。

ACE-Lite 允许其他主控器在其他集群的缓存中进行侦听。
这意味着对于共享位置,如果必要,读取将从一致性缓存中满足,并且共享写操作将与来自一致性缓存行的强制清理和失效合并。
ACE 规范使得 TLB 和 I-Cache 维护操作可以广播到所有能够接收它们的设备。
数据屏障被发送到从属接口以确保它们在程序上是完整的。

CoreLink CCI-400 Cache Coherent Interface 是 AMBA 4 ACE 的首批实现之一,支持最多两个 ACE 集群,使得最多八个核心可以看到相同的内存视图,并运行 SMP 操作系统,例如,一个 Cortex-A57 处理器和 Cortex-A53 处理器的 big.LITTLE 组合,如下图所示。

\Figure[caption={多簇系统}, label={fig:multi-cluster-system}, width=0.5]{multi-cluster-system}

它还具有三个可供 DMA 控制器或 GPU 使用的 ACE-lite 一致性接口。

下图显示了从 Cortex-A53 集群到 Cortex-A57 集群进行的一致性数据读取过程。

\begin{enumerate}
\item
  Cortex-A53 集群发出一致性读请求。
\item
  CCI-400 将请求传递给 Cortex-A53 处理器,以便侦听 Cortex-A57 集群的缓存。
\item
  当收到请求时,Cortex-A57 集群检查其数据缓存的可用性,并以所需的信息作出响应。
\item
  如果请求的数据在缓存中,CCI-400 将数据从 Cortex-A57 集群移动到 Cortex-A53 集群,从而在 Cortex-A53 集群中进行缓存行填充。
\end{enumerate}

\Figure[caption={CCI 嗅探请求}, label={fig:CCI-snoop-req}, width=0.4]{CCI-snoop-req}

CCI-400 和 ACE 协议使得 Cortex-A57 和 Cortex-A53 集群之间能够实现完全的一致性,从而实现数据共享,无需进行外部内存事务。

ARMCoreLink 互联和内存控制器系统 IP 解决了在 Cortex-A 系列处理器、高性能媒体处理器和动态内存之间高效移动和存储数据的关键挑战,以优化片上系统(SoC)的系统性能和功耗。
CoreLink 系统 IP 使 SoC 设计者能够最大程度地利用系统内存带宽,并减少静态和动态延迟。

\paragraph{计算机子系统和移动应用}

下图示展示了一个移动应用处理器的示例,其中包括 Cortex-A57 和 Cortex-A53 系列处理器、CoreLink MMU-500 系统内存管理单元,以及一系列的 CoreLink 400 系统 IP。

\Figure[caption={带有 CoreLink IP 的移动应用例子}, label={fig:exp-mobi-app-proc-with-corelink-ip}, width=0.95]{exp-mobi-app-proc-with-corelink-ip}

在这个系统中,ARM
Cortex-A57 和 Cortex-A53 处理器提供了一个 big.LITTLE 集群组合,并通过 AMBA 4
ACE 连接到 CCI-400,以提供完整的硬件一致性。
ARM Mali®-T628
GPU 和 IO 一致性主机通过 AMBA 4 ACE-Lite 接口连接到 CCI-400。

ARM 提供了不同的互连选项来维护跨集群的一致性:

\begin{description}
  \item [CoreLink CCI-400 高速缓存一致性互连] \hfill \\
    支持两个多核集群,并使用 AMBA
    4 和 AMBA 一致性扩展或 ACE。
    ACE 使用 MOESI 状态机进行跨集群的一致性维护。
  \item [CoreLink CCN-504 高速缓存一致性网络] \hfill \\
    支持最多四个多核集群,并包括集成的 L3 缓存和两通道 72 位 DDR。
  \item [CoreLink CCN-508 高速缓存一致性网络] \hfill \\
    支持最多八个多核集群,32 个核心,并包括集成的 L3 缓存和四通道 72 位 DDR。
  \item [CoreLink MMU-500 系统 MMU] \hfill \\
    为系统组件提供地址转换。
  \item [CoreLink TZC-400 TrustZone 地址空间控制器] \hfill \\
    对内存或外设事务进行安全检查,并允许将内存区域标记为安全区域。
  \item [CoreLink DMC-400 动态内存控制器] \hfill \\
    提供动态内存调度和与外部 DDR2/3 或 LPDDR2 内存的接口。
  \item [CoreLink NIC-400 网络互连] \hfill \\
    是一个高度可配置的组件,能够创建完整的高性能、优化的、符合 AMBA 标准的网络基础设施。

    CoreLink NIC-400 网络互连的可能配置范围从单个桥接组件(例如 AHB 到 AXI 协议转换桥)到一个复杂的互连,其中包括多达 128 个 AMBA 协议的主控和 64 个从控。
\end{description}

\subsection{电源管理}

许多 ARM 系统是移动设备,并由电池供电。
在这些系统中,优化电力使用和总能耗是一个关键的设计约束。
程序员通常会花费大量时间来尝试节省这类系统的电池寿命。

即使在不使用电池的系统中,节能也是一个需要关注的问题。
例如,您可能希望通过减少电费、出于环保原因或为了尽量减少设备产生的热量来最小化能源使用。

ARM 内核内置了许多旨在减少电力使用的硬件设计方法。
能源使用可以分为两个组成部分:

\begin{description}
    \item[静态功耗] \hfill \\
    静态功耗,也常被称为泄漏功耗,当内核逻辑或 RAM 模块通电时就会发生。
    一般来说,泄漏电流与总硅面积成正比,这意味着芯片越大,泄漏就越高。
    随着制造工艺几何尺寸的减小,来自泄漏的功耗比例显著增加。
    \item[{动态功耗}] \hfill \\
    动态功耗是由于晶体管切换引起的,取决于内核时钟速度和每个周期内改变状态的晶体管数量。
    显然,时钟速度越高、内核越复杂,功耗就越高。

    具备电源管理功能的操作系统会动态改变内核的电源状态,平衡当前工作负载与可用计算能力,同时尽量使用最少的电量。
    这些技术中有些会动态切换内核的开关状态,或者将其置于静止状态,使其不再进行计算,从而消耗极少的电量。
    主要的这些技术示例包括:
    \begin{itemize}
      \item
        第~\ref{sec:pm-idle-man} 节的空闲管理。
      \item
        第~\ref{sec:pm-dyn-vol-freq-scale} 节的动态电压和频率调节。
    \end{itemize}
\end{description}

\subsubsection{空闲管理}\label{sec:pm-idle-man}

当一个内核处于空闲状态时,操作系统电源管理(OSPM)会将其转换为低功耗状态。
通常会有多个状态可供选择,每个状态有不同的进入和退出延迟,以及不同的功耗水平。
所使用的状态通常取决于内核再次需要的速度。
任何时候可用的电源状态也可能取决于 SoC 中其他组件的活动情况,而不仅仅是内核本身。
每个状态都由进入该状态时时钟门控或电源门控的组件集合来定义。

从低功耗状态切换到运行状态所需的时间被称为唤醒延迟。
在更深的低功耗状态中,唤醒延迟更长。
虽然空闲电源管理是由内核上的线程行为驱动的,OSPM 可以将平台置于影响许多其他组件的状态,而不仅仅是内核本身。
如果一个集群中的最后一个内核变为空闲,OSPM 可以选择影响整个集群的电源状态。
同样,如果 SoC 中的最后一个内核变为空闲,OSPM 可以选择影响整个 SoC 的电源状态。
选择也会受到系统中其他组件使用情况的驱动。
一个典型的例子是在所有内核和其他总线主控设备空闲时,将内存置于自刷新模式。

OSPM 必须提供必要的电源管理软件基础设施,以确定正确的状态选择。
在空闲管理中,当一个内核或集群被置于低功耗状态时,它可以在任何时候通过内核唤醒事件被重新激活。
也就是说,一个可以从低功耗状态唤醒内核的事件,例如中断。
不需要 OSPM 发出明确的命令来使内核或集群重新运行。
OSPM 认为受影响的内核或多个内核在任何时候都是可用的,即使它们当前处于低功耗状态。

\BlockDesc{电源和时钟}

减少能耗的一种方法是断电,这会同时消除动态和静态电流(有时称为电源门控),或者停止内核的时钟,这仅能消除动态功耗,称为时钟门控。

ARM 内核通常支持几种不同级别的电源管理,如下:

\begin{itemize}
\item
  待机模式(Standby)。
\item
  保持模式(Retention)。
\item
  关机模式(Power down)。
\item
  休眠模式(Dormant mode)。
\item
  热插拔(Hotplug)。
\end{itemize}

对于某些操作,在断电前后需要保存和恢复状态。
执行这些保存和恢复操作所需的时间以及这项额外工作消耗的电力在软件选择适当的电源管理活动时是一个重要因素。

包含内核的 SoC 设备可以有其他低功耗状态,如 STOP 和 Deep sleep。
这些状态指的是电源管理软件可以控制硬件锁相环(PLL)和电压调节器的能力。

\BlockDesc{待机模式}

在待机模式(Standby)下,内核保持通电,但其大部分时钟被停止或时钟门控。
这意味着内核的大多数部分处于静态状态,唯一的电力消耗来自于漏电流和少量用于监测唤醒条件的逻辑电路的时钟。

该模式通过 WFI(等待中断)或 WFE(等待事件)指令进入。
ARM 建议在 WFI 或 WFE 指令之前使用数据同步屏障(DSB)指令,以确保待处理的内存事务在改变状态之前完成。
如果调试通道处于活动状态,它将保持活跃。
内核会停止执行,直到检测到唤醒事件。
唤醒条件取决于进入模式的指令。
对于 WFI,一个中断或外部调试请求会唤醒内核。
对于 WFE,则有若干指定事件可以触发唤醒,包括集群中另一个内核执行 SEV 指令。

在多核系统中,来自嗅探控制单元(SCU)的请求也可以唤醒时钟以进行缓存一致性操作。
这意味着处于待机状态的内核的缓存与其他内核的缓存保持一致(但处于待机状态的内核不一定执行下一条指令)。
核心重置总是会迫使内核退出待机状态。

各种形式的动态时钟门控也可以在硬件中实现。
例如,当检测到空闲状态时,SCU、GIC、定时器、指令流水线或
NEON 块可以自动进行时钟门控以节省电力。

待机模式可以快速进入和退出(通常在两个时钟周期内)。
因此,它对内核的延迟和响应性几乎没有影响。

对于 OSPM 来说,待机状态和保持状态几乎无法区分。
这种区别在外部调试器和硬件实现中显而易见,但在操作系统的空闲管理子系统中则不明显。

\BlockDesc{保持模式}

内核状态(包括调试设置)保存在低功耗结构中,使内核至少部分关闭。
从低功耗保持状态切换到运行状态不需要重置内核。
在从低功耗保持状态切换到运行状态时,会恢复已保存的内核状态。
从操作系统的角度来看,保持状态和待机状态之间除了进入方式、延迟和使用相关的约束外,没有区别。
然而,从外部调试器的角度来看,这些状态有所不同,因为外部调试请求调试事件保持挂起状态,并且不能访问内核电源域中的调试寄存器。

\BlockDesc{关机模式}

在这种状态下,内核被断电。
设备上的软件必须保存所有内核状态,以便在断电期间保留这些状态。
从断电状态切换到运行状态必须包括:

\begin{itemize}
\item
  在恢复电源水平后重置内核。
\item
  恢复已保存的内核状态。
\end{itemize}

断电状态的定义特征是它们会破坏上下文。
这会影响给定状态下关闭的所有组件,包括内核,在更深的状态下还包括系统的其他组件,如 GIC 或特定平台的 IP。
根据调试和跟踪电源域的组织方式,在某些断电状态下,调试和跟踪上下文中的一个或两个可能会丢失。
必须提供机制以使操作系统能够为每个给定状态执行相关的上下文保存和恢复。
执行的恢复从重置向量开始,之后每个操作系统必须恢复其上下文。

\BlockDesc{休眠模式}

休眠模式(Dormant mode)是一种断电状态的实现。
在休眠模式下,内核逻辑被断电,但缓存 RAM 保持通电。
通常,这些 RAM 处于低功耗保持状态,保留其内容但不具备其他功能。
这提供了比完全关机更快的重启,因为缓存中的活动数据和代码得以保留。
同样,在多核系统中,单个内核可以进入休眠模式。

在允许集群内各个内核进入休眠模式的多核系统中,当内核断电时,无法保持一致性。
因此,这些内核必须首先将自己从一致性域中隔离。
在此之前,它们需要清除所有脏数据,并且通常通过另一个内核向外部逻辑发出信号重新通电来唤醒处于休眠模式的内核。

被唤醒的内核必须在重新加入一致性域之前恢复原始内核状态。
由于内核处于休眠模式期间内存状态可能发生变化,内核可能需要无效化缓存。
因此,休眠模式在单核环境中比在集群环境中更有用。
这是因为离开和重新加入一致性域的额外开销。
在集群中,休眠模式通常只有在其他内核已经关闭时,才由最后一个内核使用。

\BlockDesc{热插拔}

CPU 热插拔是一种可以动态切换内核开关的技术。
OSPM 可以使用热插拔根据当前的计算需求改变可用的计算能力。
热插拔有时也用于可靠性方面的原因。
热插拔与使用断电状态进行空闲管理之间有几个区别:

\begin{enumerate}
\item
  当一个内核被热插拔时,监督软件会停止该内核在中断和线程处理中的所有使用。
  调用操作系统不再将该内核视为可用。
\item
  OSPM 必须发出明确的命令才能使一个内核重新上线,即热插拔一个内核。
  在此命令之后,适当的监督软件才会开始在该内核上进行调度或启用中断。
\end{enumerate}

操作系统通常在一个主内核上执行大部分内核引导过程,并在稍后阶段使次要内核上线。
次要内核的引导行为与热插拔一个内核到系统中非常相似。
这两种情况下的操作几乎是相同的。

\subsubsection{动态电压和频率调节}\label{sec:pm-dyn-vol-freq-scale}

许多系统在负载可变的条件下运行,因此具有根据预期负载调整内核性能的能力是很有用的。
通过降低内核的时钟速度可以减少动态功耗。

动态电压和频率调节(DVFS)是一种利用以下关系的节能技术:

\begin{itemize}
\item
  功耗与工作频率之间的线性关系。
\item
  功耗与工作电压之间的平方关系。
\end{itemize}

这种关系可以表示为:

$$ P = C \times V^2 \times f $$

其中:
\begin{itemize}
\item
(P)是动态功率。
\item
(C)是逻辑电路的开关电容。
\item
(V)是工作电压。
\item
(f)是工作频率。
\end{itemize}

通过调整内核时钟的频率来实现节能。
在较低频率下,内核也可以在较低电压下工作。
降低供电电压的优点是既减少动态功耗也减少静态功耗。

对于给定电路,工作电压与该电路可安全运行的频率范围之间的关系是实现特定的。
给定的工作频率及其对应的工作电压被表示为一个元组,称为操作性能点(OPP)。
对于一个系统来说,可实现的 OPP 范围统称为系统的 DVFS 曲线。

操作系统使用 DVFS 来节省能源,并在必要时保持在热限内。
操作系统提供 DVFS 策略来管理消耗的功率和所需的性能。
旨在高性能的策略选择更高的频率并消耗更多的能源。
旨在节能的策略选择较低的频率,从而导致较低的性能。

\subsubsection{汇编电源指令}

ARM 汇编语言包括可以用于将内核置于低功耗状态的指令。
架构将这些指令定义为提示(hints),这意味着内核在执行它们时不需要采取任何特定的动作。
然而,在 Cortex-A 处理器系列中,这些指令被实现为几乎关闭核心的所有部分的时钟。
这意味着核心的功耗大大降低,只有静态泄漏电流被吸引,而没有动态功耗。

WFI 指令的作用是暂停执行,直到内核被以下条件之一唤醒:

\begin{itemize}
\item
  IRQ 中断,即使 PSTATE I 位被设置。
\item
  FIQ 中断,即使 PSTATE F 位被设置。
\item
  异步异常。
\end{itemize}

如果内核在相关的 PSTATE 中断标志被禁用时被中断唤醒,内核会执行 WFI 指令后的下一条指令。

WFI 指令在使用电池供电的系统中广泛使用。
例如,移动电话可以在等待您按下按钮时将内核置于待机模式,而这个过程可以每秒多次重复。

WFE 类似于 WFI。
它会暂停执行,直到发生事件。
这可以是列出的事件条件之一,也可以是由集群中的另一个内核发出的事件。
其他内核可以通过执行 SEV 指令来发出事件。
SEV 向所有内核发出事件信号。
通用定时器也可以编程以触发周期性事件,从而从 WFE 中唤醒一个内核。

\subsubsection{电源状态协调接口}\label{sec:pm-state-coordination-if}

Power State Coordination Interface (PSCI) 提供了一种与操作系统无关的方法,用于实现可以启动或关闭内核的电源管理用例。
这包括:

\begin{itemize}
\item
  内核空闲管理。
\item
  内核的动态添加和移除(热插拔),以及次要内核引导。
\item
  big.LITTLE 迁移。
\item
  系统关闭和重置。
\end{itemize}

使用此接口发送的消息被所有相关的执行级别接收。
也就是说,如果实现了 EL2 和 EL3,那么由客户中的 Rich OS 发送的消息必须被 hypervisor 接收。
如果 hypervisor 发送了消息,则该消息必须被安全固件接收,然后与 Trusted OS 协调。
这允许每个操作系统确定是否需要上下文保存。

要了解更多信息,请参阅 Power State Coordination Interface (PSCI) 规范。


\subsection{big.LITTLE 技术}\label{sec:big-little-tec}

现代软件堆栈对移动系统提出了相互冲突的要求。
一方面,对于像游戏这样的任务,需要非常高的性能,而另一方面,需要对能源储备进行节约,以满足像音频播放这样的低强度应用的需求。
传统上,很难设计出一种能够既具有高峰值性能又具有高能效性的单一处理器设计。
这意味着大量的能源被浪费,因为高性能内核会被用于低强度任务,导致电池寿命缩短。
性能本身受到核心可以持续运行的热限制的影响。

ARM 的 big.LITTLE 技术通过将高性能的 big 核与能效高的 LITTLE 核结合起来,解决了这个问题。
big.LITTLE 是异构处理系统的一个例子。
这种系统通常包括几种不同类型的处理器,具有不同的微体系结构,如通用处理器和专用 ASIC。

big.LITTLE 将异构性推向更高一层,因为它包含了具有不同微体系结构但具有兼容指令集体系结构的通用处理器。
一个经常与这种系统一起使用的术语是异构多处理(HMP)。
与不对称多处理(AMP)不同的是,HMP 系统中的所有处理器都是完全一致的,并且运行相同的操作系统映像。
软件可以根据性能要求在 big 或 LITTLE 处理器上运行(或两者兼有)。
当需要峰值性能时,软件可以移至仅在 big 处理器上运行。
对于普通任务,软件可以在 LITTLE 处理器上运行得很好。
通过这种组合,big.LITTLE 提供了一种解决方案,能够在系统的热限制范围内,以最大的能效性提供最新移动设备所需的峰值性能。

\subsubsection{big.LITTLE 系统结构}

在 big.LITTLE 系统中,两种类型的核心都是完全缓存一致的,并且共享相同的指令集架构(ISA)。
相同的应用程序二进制文件可以在任一核心上不经修改地运行。
处理器内部微架构的差异使它们能够提供 big.LITTLE 概念所需的不同功耗和性能特性。
这些通常由操作系统管理。

big.LITTLE 软件模型要求在 big 和 LITTLE 集群之间透明且高效地传输数据。
硬件一致性使这一点对软件来说是透明的。
集群之间的一致性由缓存一致性互连(如第 14 章描述的 ARM CoreLink CCI-400)提供。
如果没有硬件一致性,big 和 LITTLE 核心之间的数据传输将始终通过主内存进行,这会很慢且不节能。
此外,这将需要复杂的缓存管理软件来实现 big 和 LITTLE 集群之间的数据一致性。

此外,这样的系统还需要一个共享的中断控制器,如 GIC-400,使中断可以在集群中的任意核心之间迁移。
所有核心可以通过分布式中断控制器(如 CoreLink GIC-400)互相发送信号。
任务切换通常完全由操作系统调度程序处理,对应用软件是不可见的。
下图展示了一个示例系统。

\Figure[caption={典型 big.LITTLE 系统}, label={fig:typical-big-little-sys}, width=0.6]{typical-big-little-sys}

\subsubsection{big.LITTLE 配置}

多种 big.LITTLE 配置是可能的,上图中使用 Cortex-A57 内核作为 big 集群,Cortex-A53 内核作为 LITTLE 集群,但其他配置也是可能的。

LITTLE 集群能够处理大多数低强度任务,如音频播放、网页滚动、操作系统事件和其他始终在线、始终连接的任务。
因此,软件堆栈可能会一直停留在 LITTLE 集群上,直到运行像游戏或视频处理这样强度较大的任务。

big 集群可以用于高负载任务,例如某些高性能游戏图形。
网页渲染是另一个常见的例子。
这两种集群类型的结合提供了节能的机会,并满足了移动设备上应用程序堆栈日益增长的性能需求。

\subsubsection{big.LITTLE 中的软件执行模型}

big.LITTLE 主要有两种执行模型:

\begin{description}
  \item[{迁移(Migration)}] \hfill \\
    迁移模型是 DVFS 等电源性能管理技术的自然扩展(如动态电压和频率调节,参见第~\ref{sec:pm-state-coordination-if} 章节的动态电压和频率调节)。
    迁移模型有两种类型:

    \begin{itemize}
      \item 集群迁移
      \item CPU 迁移
    \end{itemize}

    迁移操作类似于 DVFS 操作点转换。
    核心的 DVFS 曲线上的操作点根据负载变化进行遍历。
    当当前核心(或集群)达到最高操作点时,如果软件堆栈需要更高的性能,就会执行核心(或集群)迁移操作。
    然后,执行在另一核心(或集群)上继续,遍历该核心(或集群)的操作点。
    当不需要高性能时,执行可以切换回去。

  \item[{全局任务调度}] \hfill \\
    在全局任务调度(参见第~\ref{sec:global-task-sched} 的全局任务调度)中,操作系统任务调度器了解 big 和 LITTLE 核心之间的计算能力差异。
    调度器跟踪每个软件线程的性能需求,并根据这些信息决定使用哪种类型的核心。
    未使用的核心可以关闭。
    与迁移模型相比,这种方法有许多优点。
\end{description}

\BlockDesc{集群迁移}

在任何时刻,只有一个集群(big 或 LITTLE)是活跃的,除非在集群上下文切换到另一个集群期间会有短暂的例外。
为了实现最佳的功耗和性能效率,软件堆栈主要运行在节能的 LITTLE 集群上,仅在需要高性能的短时间内运行在 big 集群上。
此模型要求两个集群中的核心数量相同。

该模型在处理不平衡的软件工作负载时表现不佳,即在一个集群内的核心上负载显著不同的工作负载。
在这种情况下,集群迁移会导致完全切换到 big 集群,即使并非所有核心都需要那种性能水平。
因此,集群迁移不如其他方法受欢迎。

\BlockDesc{CPU 迁移}

在这种模型中,每个 big 核心与一个 LITTLE 核心配对。
在任何时候,每对中的只有一个核心是活跃的,非活跃的核心被关闭。
根据当前的负载情况选择活跃核心。
使用图 16 - 2(第 16 - 5 页)中的例子,操作系统看到的是四个逻辑核心。
每个逻辑核心在物理上可以是 big 核心或 LITTLE 核心。
这个选择由动态电压和频率调节(DVFS)驱动。
该模型要求两个集群中的核心数量相同。

\Figure[caption={CPU 迁移}, label={fig:cpu-migration}, width=0.9]{cpu-migration}

系统主动监控每个核心的负载情况。
当负载较高时,执行上下文被移动到 big 核心;
相反,当负载较低时,执行被移动到 LITTLE 核心。
在任何时候,每对中的只有一个核心可以是活跃的。
当负载从出站核心(负载离开的核心)移动到入站核心(负载到达的核心)时,前者被关闭。
该模型允许在任何时候混合使用 big 和 LITTLE 核心。

\BlockDesc{全局任务调度}\label{sec:global-task-sched}

通过 big.LITTLE 技术的发展,ARM 已经演化出了一系列软件模型,从各种迁移模型到全局任务调度(GTS),后者构成了所有未来 big.LITTLE 技术发展的基础。
ARM 对 GTS 的实现称为 big.LITTLE 多处理(MP)。

在这种模型中,操作系统的任务调度器能够感知 big 核心和 LITTLE 核心之间计算能力的差异。
利用统计数据,调度器跟踪每个软件线程的性能需求,并根据这些信息决定使用哪种类型的核心。
这种模型可以在具有任意数量核心的 big.LITTLE 系统上工作。
如图 16 - 3(第 16 - 5 页)所示。
这种方法相较于迁移模型有许多优势,例如:

\begin{itemize}
\item
  系统可以有不同数量的 big 和 LITTLE 核心。
\item
  与迁移模型不同,任何时候可以有任意数量的核心处于活跃状态。
  如果需要峰值性能,这可以增加可用的最大计算能力。
\item
  可以将 big 集群隔离,用于专门处理密集线程,而轻量线程在 LITTLE 集群上运行。
  这使得重计算任务可以更快完成,因为没有额外的后台线程干扰。
\item
  可以单独将中断目标设定为 big 或 LITTLE 核心。
\end{itemize}

\subsubsection{big.LITTLE MP}

对于 Linux 内核上的 big.LITTLE 多处理(MP),基本要求是调度器决定软件线程何时可以在 LITTLE 核心或 big 核心上运行。
调度器通过将软件线程的跟踪负载与可调负载阈值进行比较来完成这一任务,如下图所示,包括上迁移阈值和下迁移阈值。

\Figure[caption={迁移阈值}, label={fig:migration-thresholds}, width=1]{migration-thresholds}

当一个分配给 LITTLE 核心的线程的跟踪负载平均值超过上迁移阈值时,该线程被视为符合迁移到 big 核心的条件。
相反地,当一个分配给 big 核心的线程的负载平均值下降到低于下迁移阈值时,它被视为符合迁移到 LITTLE 核心的条件。
在big.LITTLE MP 中,这些基本规则决定了任务在 big 核心和 LITTLE 核心之间的迁移。
在集群内部,标准的 Linux 调度器负载均衡适用。
这试图在一个集群的所有核心之间保持负载平衡。

该模型通过根据核心当前的频率调整跟踪负载指标来进行优化。
当核心以一半的速度运行时,正在运行的任务以一半的速度累积跟踪负载,如果核心以全速运行,该任务的跟踪负载将以全速累积。
这使得 big.LITTLE MP 和 DVFS 管理可以协同工作。

big.LITTLE MP 使用多种机制来确定何时在 big 核心和 LITTLE 核心之间迁移任务:

\BlockDesc{fork 迁移}

这种机制在使用 fork 系统调用创建新的软件线程时起作用。
在这一点上,显然没有可用的历史负载信息。
系统默认将新线程分配到 big 核心,假设轻量线程通过唤醒迁移迅速迁移到 LITTLE 核心。

fork 迁移使要求严格的任务受益而不会造成昂贵的开销。
低强度和持续的线程,如 Android 系统服务,在创建时仅移动到 big 核心,随后迅速移动到更适合的 LITTLE 核心。
明显要求持续性性能的线程不会因为被强制在 LITTLE 核心上启动而产生不利影响。
偶尔运行但倾向于需要性能的线程受益于在 big 集群上启动,并根据需要继续在那里运行。

\BlockDesc{Wake 迁移}

当先前处于空闲状态的任务准备运行时,调度器必须决定哪个集群执行该任务。
为了在 big 和 LITTLE 之间做出选择,big.LITTLE MP 使用了任务的跟踪负载历史。
通常情况下,假设任务在同一集群上恢复运行。
对于正在睡眠的任务,负载指标不会更新。
因此,在调度器在唤醒时检查任务的负载指标之前,在选择要在哪个集群上执行任务时,该指标的值与任务上次运行时的值相同。
这一特性意味着足够繁忙的任务总是倾向于在 big 核心上唤醒。
例如,音频播放任务周期性地繁忙。
但这通常是一个不苛刻的任务,因此总体负载可能很容易适应 LITTLE 核心。
要改变集群,任务必须实际修改其行为。

\Figure[caption={在大核上唤醒迁移}, label={fig:wake-mig-on-big-core}, width=1]{wake-mig-on-big-core}

\Figure[caption={在小核上唤醒迁移}, label={fig:wake-mig-on-lit-core}, width=1]{wake-mig-on-lit-core}

如果一个任务修改了其行为,并且负载指标越过了上迁移或下迁移阈值中的任何一个,那么该任务可以被分配到一个不同的集群。
图~\ref{fig:wake-mig-on-big-core} 和图~\ref{fig:wake-mig-on-lit-core} 说明了这个过程。
有一些规则被定义,以确保 big 核心通常只运行一个高强度的线程,并将其运行到完成,因此向上迁移只会发生在空闲的 big 核心上。
当向下迁移时,这个规则不适用,可以将多个软件线程分配给一个 LITTLE 核心。

\BlockDesc{强制迁移}

强制迁移处理了长时间运行的软件线程不休眠或很少休眠的问题。
调度器定期检查每个 LITTLE 核心上正在运行的当前线程。
如果跟踪的负载超过上迁移阈值,则任务被转移到一个 big 核心上,如下图所示。

\Figure[caption={强制迁移}, label={fig:force-mig}, width=0.8]{force-mig}

\BlockDesc{Idle pull 迁移}

空闲拉动(Idle pull)迁移旨在充分利用活跃的 big 核心。
当一个 big 核心没有任务可运行时,会检查所有 LITTLE 核心,看看 LITTLE 核心上当前运行的任务是否具有比上迁移阈值更高的负载指标。
这样的任务可以立即迁移到空闲的 big 核心上。
如果没有找到合适的任务,那么该 big 核心可以被关闭电源。
这种技术确保了 big 核心在运行时始终承担系统中最密集的任务,并将它们运行到完成。

\BlockDesc{Offload 迁移}

离载(Offload)迁移要求禁用正常的调度器负载平衡。
这样做的缺点是,长时间运行的线程可能会集中在 big 核心上,导致 LITTLE 核心处于空闲和低利用状态。
在这种情况下,通过利用所有核心,可以明显改善整体系统性能。

离载迁移定期将线程向下迁移到 LITTLE 核心,以利用未使用的计算容量。
以这种方式向下迁移的线程仍然是在下一次调度机会时超过阈值的向上迁移的候选项。

\subsection{安全}

\subsection{调试}

\subsection{ARMv8 模型}



\section{RISC-V}
\section{RISC-V}

\subsection{Privileged Architecture}

\subsection{Unprivileged ISA}


\section{内存层级}

\section{每个程序员都应该知道的内存知识}

关于内存相关的知识,强烈建议阅读
\href{https://people.freebsd.org/~lstewart/articles/cpumemory.pdf}{What Every Programmer Should Know About Memory}。


% nvim: set ts=2 sts=2 sw=2 et:
