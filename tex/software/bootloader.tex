\chapter{Bootloader}

Bootloader(引导加载程序)是系统启动过程中的关键组件,负责初始化硬件、加载操作系统内核,并完成从硬件上电到操作系统运行的过渡。
在嵌入式系统和通用计算系统中,Bootloader 的设计和实现直接影响系统的启动速度、安全性和可靠性。
本章深入探讨 Bootloader 的工作原理、主流实现以及开发实践。

\section{Bootloader 基础}

\subsection{Bootloader 的作用}

\BlockDesc{Bootloader}

Bootloader 是系统启动的第一个软件组件,其主要职责包括:

\begin{enumerate}
  \item \textbf{硬件初始化}
    \begin{itemize}
      \item 初始化 CPU、内存控制器
      \item 配置时钟系统
      \item 初始化存储设备(Flash、eMMC、SD 卡等)
      \item 初始化串口、网络等外设
    \end{itemize}

  \item \textbf{加载操作系统}
    \begin{itemize}
      \item 从存储设备读取内核镜像
      \item 加载设备树(Device Tree)或 ACPI 表
      \item 加载初始 RAM 文件系统(initramfs/initrd)
      \item 设置内核启动参数
    \end{itemize}

  \item \textbf{系统配置}
    \begin{itemize}
      \item 设置内存布局
      \item 配置中断向量表
      \item 设置安全启动参数
      \item 提供用户交互界面(可选)
    \end{itemize}

  \item \textbf{系统维护}
    \begin{itemize}
      \item 固件更新
      \item 系统恢复
      \item 诊断和调试
    \end{itemize}
\end{enumerate}

\subsection{系统启动流程}

\BlockDesc{系统启动流程}

现代计算机系统的启动是一个多阶段的过程,不同架构和系统有不同的实现,但基本流程相似:

\subsubsection{通用启动流程}

\begin{enumerate}
  \item \textbf{硬件上电(Power-On)}
    \begin{itemize}
      \item CPU 从固定地址(Reset Vector)开始执行
      \item 在 x86 架构中,CPU 从 0xFFFFFFF0 开始执行
      \item 在 ARM 架构中,通常从 0x00000000 或 0xFFFF0000 开始执行
      \item 在 RISC-V 架构中,从设备树或配置指定的地址开始执行
    \end{itemize}

  \item \textbf{ROM Bootloader(第一阶段)}
    \begin{itemize}
      \item CPU 内置的只读引导代码
      \item 初始化最基本的硬件(时钟、内存控制器)
      \item 从预定义位置加载下一阶段 Bootloader
      \item 在 x86 系统中,这是 BIOS 或 UEFI 固件
      \item 在 ARM 系统中,可能是 BootROM 或 ARM Trusted Firmware
    \end{itemize}

  \item \textbf{主 Bootloader(第二阶段)}
    \begin{itemize}
      \item 从存储设备加载的 Bootloader
      \item 在 PC 系统中,通常是 GRUB、LILO 等
      \item 在嵌入式系统中,通常是 U-Boot、barebox 等
      \item 提供更丰富的功能和用户界面
    \end{itemize}

  \item \textbf{内核加载}
    \begin{itemize}
      \item Bootloader 加载内核镜像到内存
      \item 设置内核启动参数
      \item 跳转到内核入口点
    \end{itemize}

  \item \textbf{内核初始化}
    \begin{itemize}
      \item 内核初始化各种子系统
      \item 挂载根文件系统
      \item 启动第一个用户空间进程(init)
    \end{itemize}

  \item \textbf{用户空间启动}
    \begin{itemize}
      \item init 进程启动
      \item 执行启动脚本
      \item 启动系统服务
      \item 进入正常运行状态
    \end{itemize}
\end{enumerate}

\subsubsection{ARM 架构启动流程}

ARM 架构的启动流程具有特殊性,特别是在 ARMv8-A 架构中:

\begin{enumerate}
  \item \textbf{EL3 阶段(Secure Monitor)}
    \begin{itemize}
      \item CPU 从 EL3(最高特权级)开始执行
      \item ARM Trusted Firmware(ATF)在此阶段运行
      \item 初始化安全世界(Secure World)和非安全世界(Non-Secure World)
      \item 配置系统安全状态
    \end{itemize}

  \item \textbf{EL2 阶段(Hypervisor,可选)}
    \begin{itemize}
      \item 如果启用虚拟化,在此阶段初始化 Hypervisor
      \item 配置虚拟化相关寄存器
    \end{itemize}

  \item \textbf{EL1 阶段(Kernel)}
    \begin{itemize}
      \item 操作系统内核在 EL1 运行
      \item U-Boot 或其他 Bootloader 在此阶段运行
      \item 加载并启动 Linux 内核
    \end{itemize}

  \item \textbf{EL0 阶段(User Space)}
    \begin{itemize}
      \item 用户空间应用程序运行
    \end{itemize}
\end{enumerate}

\subsection{Bootloader 的分类}

根据应用场景和功能,Bootloader 可以分为以下几类:

\begin{enumerate}
  \item \textbf{PC 系统 Bootloader}
    \begin{itemize}
      \item GRUB:GNU GRand Unified Bootloader,Linux 系统最常用
      \item LILO:Linux Loader,较老的 Bootloader
      \item Windows Boot Manager:Windows 系统的 Bootloader
      \item rEFInd:UEFI 启动管理器
    \end{itemize}

  \item \textbf{嵌入式系统 Bootloader}
    \begin{itemize}
      \item U-Boot:Universal Bootloader,最流行的嵌入式 Bootloader
      \item barebox:轻量级嵌入式 Bootloader
      \item RedBoot:基于 eCos 的 Bootloader
      \item Das U-Boot:U-Boot 的早期名称
    \end{itemize}

  \item \textbf{固件级 Bootloader}
    \begin{itemize}
      \item UEFI:统一可扩展固件接口
      \item BIOS:传统 PC 固件
      \item ARM Trusted Firmware:ARM 安全固件
      \item Coreboot:开源固件项目
    \end{itemize}

  \item \textbf{专用 Bootloader}
    \begin{itemize}
      \item 各种 SoC 厂商提供的专用 Bootloader
      \item 手机 Bootloader(如 Android Bootloader)
      \item 路由器 Bootloader(如 CFE、U-Boot)
    \end{itemize}
\end{enumerate}

\section{U-Boot}

\BlockDesc{U-Boot}

U-Boot(Universal Bootloader)是嵌入式系统中最广泛使用的开源 Bootloader。
它支持多种 CPU 架构(ARM、x86、MIPS、PowerPC、RISC-V 等)和大量硬件平台。

\subsection{U-Boot 概述}

U-Boot 最初由 Wolfgang Denk 开发,是 Das U-Boot 项目的延续。
它提供了丰富的功能,包括:

\begin{itemize}
  \item 多架构支持:ARM、ARM64、x86、x86\_64、MIPS、PowerPC、RISC-V 等
  \item 网络支持:TFTP、NFS、DHCP、HTTP 等
  \item 文件系统支持:FAT、ext2/3/4、UBIFS、JFFS2 等
  \item 设备树支持:完整的设备树解析和传递
  \item 安全启动:支持 UEFI Secure Boot、Verified Boot
  \item 脚本支持:U-Boot 脚本语言
  \item 交互式命令行:类似 Shell 的命令行界面
\end{itemize}

\subsection{U-Boot 架构}

U-Boot 采用模块化设计,主要组件包括:

\subsubsection{启动阶段}

U-Boot 的启动分为多个阶段:

\begin{enumerate}
  \item \textbf{SPL(Secondary Program Loader)}
    \begin{itemize}
      \item 第一阶段 Bootloader,通常很小(< 64KB)
      \item 初始化 DDR 内存
      \item 加载主 U-Boot 到内存
      \item 在内存受限的系统中使用
    \end{itemize}

  \item \textbf{TPL(Tertiary Program Loader)}
    \begin{itemize}
      \item 在某些 SoC 中使用,在 SPL 之前运行
      \item 初始化更基础的硬件
    \end{itemize}

  \item \textbf{主 U-Boot}
    \begin{itemize}
      \item 完整的 U-Boot 功能
      \item 提供命令行界面
      \item 加载内核和文件系统
    \end{itemize}
\end{enumerate}

\subsubsection{核心组件}

\begin{enumerate}
  \item \textbf{板级支持包(BSP)}
    \begin{itemize}
      \item 板级特定的初始化代码
      \item 位于 \texttt{board/} 目录
      \item 包含硬件初始化、内存配置等
    \end{itemize}

  \item \textbf{驱动框架}
    \begin{itemize}
      \item 统一的驱动模型(DM, Driver Model)
      \item 支持设备树绑定
      \item 自动设备发现和初始化
    \end{itemize}

  \item \textbf{命令系统}
    \begin{itemize}
      \item 可扩展的命令框架
      \item 每个命令是独立的模块
      \item 支持命令别名和脚本
    \end{itemize}

  \item \textbf{环境变量系统}
    \begin{itemize}
      \item 存储在 Flash 或 eMMC 的特定区域
      \item 用于配置启动参数
      \item 支持默认值和用户自定义值
    \end{itemize}
\end{enumerate}

\subsection{U-Boot 编译和配置}

\subsubsection{配置系统}

U-Boot 使用 Kconfig 系统进行配置,类似于 Linux 内核:

\begin{lstlisting}[language=bash, caption={U-Boot 配置示例}]
# 查看支持的板级配置
ls configs/ | grep rpi

# 配置 Raspberry Pi 4
make rpi_4_defconfig

# 交互式配置
make menuconfig

# 编译
make -j$(nproc)
\end{lstlisting}

\subsubsection{关键配置选项}

\begin{enumerate}
  \item \textbf{架构相关}
    \begin{itemize}
      \item \texttt{CONFIG\_ARM}:ARM 架构支持
      \item \texttt{CONFIG\_ARM64}:ARM64 架构支持
      \item \texttt{CONFIG\_CPU}:CPU 类型
    \end{itemize}

  \item \textbf{存储支持}
    \begin{itemize}
      \item \texttt{CONFIG\_MMC}:MMC/SD 卡支持
      \item \texttt{CONFIG\_USB\_STORAGE}:USB 存储支持
      \item \texttt{CONFIG\_SATA}:SATA 硬盘支持
      \item \texttt{CONFIG\_NVME}:NVMe SSD 支持
    \end{itemize}

  \item \textbf{网络支持}
    \begin{itemize}
      \item \texttt{CONFIG\_NET}:网络功能
      \item \texttt{CONFIG\_CMD\_TFTP}:TFTP 命令
      \item \texttt{CONFIG\_CMD\_DHCP}:DHCP 客户端
      \item \texttt{CONFIG\_CMD\_NFS}:NFS 支持
    \end{itemize}

  \item \textbf{文件系统}
    \begin{itemize}
      \item \texttt{CONFIG\_FS\_EXT4}:ext4 文件系统
      \item \texttt{CONFIG\_FS\_FAT}:FAT 文件系统
      \item \texttt{CONFIG\_FS\_UBIFS}:UBIFS 文件系统
    \end{itemize}

  \item \textbf{设备树}
    \begin{itemize}
      \item \texttt{CONFIG\_OF\_LIBFDT}:设备树库支持
      \item \texttt{CONFIG\_OF\_CONTROL}:设备树控制
    \end{itemize}
\end{enumerate}

\subsection{U-Boot 命令}

U-Boot 提供了丰富的命令,用于系统配置、调试和启动:

\subsubsection{信息查询命令}

\begin{lstlisting}[language=bash, caption={U-Boot 信息查询命令}]
# 显示环境变量
printenv

# 显示内存信息
bdinfo

# 显示版本信息
version

# 显示设备树
fdt print

# 显示帮助
help
help <command>
\end{lstlisting}

\subsubsection{环境变量命令}

\begin{lstlisting}[language=bash, caption={U-Boot 环境变量操作}]
# 设置环境变量
setenv bootargs 'console=ttyS0,115200 root=/dev/mmcblk0p2'

# 保存环境变量到 Flash
saveenv

# 删除环境变量
env delete bootdelay

# 重置为默认值
env default -a
\end{lstlisting}

\subsubsection{存储设备命令}

\begin{lstlisting}[language=bash, caption={U-Boot 存储设备操作}]
# 列出存储设备
mmc list
usb storage

# 读取/写入内存
md.b 0x80000000 0x100    # 显示内存内容
mw.l 0x80000000 0x12345678 0x100  # 写入内存

# 从存储设备读取
mmc read 0x80000000 0x2000 0x1000  # 从 MMC 读取
fatload mmc 0:1 0x80000000 zImage  # 从 FAT 分区加载文件
ext4load mmc 0:2 0x80000000 /boot/zImage  # 从 ext4 分区加载
\end{lstlisting}

\subsubsection{网络命令}

\begin{lstlisting}[language=bash, caption={U-Boot 网络操作}]
# 配置网络
setenv ipaddr 192.168.1.100
setenv serverip 192.168.1.1
setenv netmask 255.255.255.0

# 使用 DHCP
dhcp

# TFTP 下载
tftp 0x80000000 zImage
tftp 0x82000000 dtb

# NFS 启动
nfs 0x80000000 192.168.1.1:/tftpboot/zImage
\end{lstlisting}

\subsubsection{启动命令}

\begin{lstlisting}[language=bash, caption={U-Boot 启动命令}]
# 启动内核
bootm 0x80000000 - 0x82000000  # 地址、initrd、设备树

# 从文件系统启动
bootz 0x80000000 0x81000000 0x82000000  # zImage、initrd、dtb

# 执行脚本
source 0x80000000
run bootcmd
\end{lstlisting}

\subsection{U-Boot 脚本}

U-Boot 支持脚本语言,可以自动化启动过程:

\begin{lstlisting}[language=bash, caption={U-Boot 启动脚本示例}]
# 设置启动参数
setenv bootargs 'console=ttyS0,115200 root=/dev/mmcblk0p2 rw rootwait'

# 定义启动命令
setenv bootcmd 'mmc dev 0; \
                fatload mmc 0:1 0x80000000 zImage; \
                fatload mmc 0:1 0x82000000 dtb; \
                bootz 0x80000000 - 0x82000000'

# 网络启动脚本
setenv netboot 'dhcp; \
                tftp 0x80000000 zImage; \
                tftp 0x82000000 dtb; \
                setenv bootargs console=ttyS0,115200 root=/dev/nfs nfsroot=192.168.1.1:/nfsroot ip=192.168.1.100; \
                bootz 0x80000000 - 0x82000000'

# 保存并执行
saveenv
run bootcmd
\end{lstlisting}

\subsection{U-Boot 设备树支持}

U-Boot 完全支持设备树(Device Tree),可以:

\begin{enumerate}
  \item \textbf{解析设备树}
    \begin{itemize}
      \item 在编译时或运行时加载设备树
      \item 验证设备树格式
      \item 提取硬件配置信息
    \end{itemize}

  \item \textbf{修改设备树}
    \begin{itemize}
      \item 在运行时修改设备树节点
      \item 添加或删除设备
      \item 修改设备属性
    \end{itemize}

  \item \textbf{传递设备树}
    \begin{itemize}
      \item 将设备树传递给内核
      \item 设置设备树地址
      \item 验证设备树完整性
    \end{itemize}
\end{enumerate}

设备树操作示例:

\begin{lstlisting}[language=bash, caption={U-Boot 设备树操作}]
# 显示设备树
fdt print
fdt print /cpus
fdt print /memory

# 修改设备树
fdt set /chosen bootargs "console=ttyS0,115200"
fdt set /memory reg <0x80000000 0x40000000>

# 添加节点
fdt mknode /test-node
fdt set /test-node compatible "test,device"
fdt set /test-node reg <0x10000000 0x1000>

# 保存设备树
fdt save 0x82000000
\end{lstlisting}

\subsection{U-Boot 开发实践}

\subsubsection{添加新命令}

在 U-Boot 中添加新命令的步骤:

\begin{enumerate}
  \item 创建命令源文件(如 \texttt{cmd\_mycmd.c})
  \item 实现命令处理函数
  \item 注册命令到 U-Boot 命令系统
\end{enumerate}

示例代码:

\begin{lstlisting}[language=C, caption={U-Boot 自定义命令示例}]
#include <common.h>
#include <command.h>

static int do_mycmd(cmd_tbl_t *cmdtp, int flag, int argc, char * const argv[])
{
    if (argc < 2) {
        printf("Usage: mycmd <arg>\n");
        return CMD_RET_USAGE;
    }

    printf("My command executed with arg: %s\n", argv[1]);
    return CMD_RET_SUCCESS;
}

U_BOOT_CMD(
    mycmd,    // 命令名
    2,        // 最大参数数
    1,        // 可重复执行
    do_mycmd, // 处理函数
    "My custom command",           // 简短描述
    "mycmd <arg> - Execute my command"  // 详细帮助
);
\end{lstlisting}

\subsubsection{添加板级支持}

添加新板级支持的步骤:

\begin{enumerate}
  \item 创建板级目录(如 \texttt{board/mycompany/myboard/})
  \item 实现板级初始化函数
  \item 创建配置文件
  \item 添加设备树文件
\end{enumerate}

\section{UEFI}

\BlockDesc{UEFI}

UEFI(Unified Extensible Firmware Interface,统一可扩展固件接口)是现代 PC 和服务器系统的主流固件标准,替代了传统的 BIOS。

\subsection{UEFI 概述}

UEFI 由 UEFI Forum 制定和维护,提供了:

\begin{itemize}
  \item 标准化的固件接口
  \item 模块化设计
  \item 网络和图形支持
  \item 安全启动(Secure Boot)
  \item 更大的磁盘支持(> 2TB)
  \item 更快的启动速度
\end{itemize}

\subsection{UEFI 架构}

\subsubsection{启动流程}

UEFI 启动流程:

\begin{enumerate}
  \item \textbf{SEC 阶段(Security Phase)}
    \begin{itemize}
      \item CPU 初始化
      \item 临时 RAM 初始化
      \item 验证固件完整性
    \end{itemize}

  \item \textbf{PEI 阶段(Pre-EFI Initialization)}
    \begin{itemize}
      \item 内存初始化
      \item CPU 初始化
      \item 芯片组初始化
      \item 传递信息到 DXE 阶段
    \end{itemize}

  \item \textbf{DXE 阶段(Driver Execution Environment)}
    \begin{itemize}
      \item 加载和执行驱动程序
      \item 初始化所有硬件
      \item 提供 UEFI 服务
    \end{itemize}

  \item \textbf{BS 阶段(Boot Services)}
    \begin{itemize}
      \item 提供启动时服务
      \item 加载操作系统加载器
      \item 传递控制权给操作系统
    \end{itemize}

  \item \textbf{RT 阶段(Runtime Services)}
    \begin{itemize}
      \item 操作系统运行时的 UEFI 服务
      \item 系统管理功能
    \end{itemize}
\end{enumerate}

\subsubsection{UEFI 服务}

UEFI 提供两类服务:

\begin{enumerate}
  \item \textbf{启动服务(Boot Services)}
    \begin{itemize}
      \item \texttt{EFI\_BOOT\_SERVICES.AllocatePages}:分配内存页
      \item \texttt{EFI\_BOOT\_SERVICES.FreePages}:释放内存页
      \item \texttt{EFI\_BOOT\_SERVICES.InstallProtocolInterface}:安装协议接口
      \item \texttt{EFI\_BOOT\_SERVICES.LoadImage}:加载可执行镜像
      \item \texttt{EFI\_BOOT\_SERVICES.StartImage}:启动镜像
      \item \texttt{EFI\_BOOT\_SERVICES.ExitBootServices}:退出启动服务
    \end{itemize}

  \item \textbf{运行时服务(Runtime Services)}
    \begin{itemize}
      \item \texttt{EFI\_RUNTIME\_SERVICES.GetTime}:获取系统时间
      \item \texttt{EFI\_RUNTIME\_SERVICES.SetTime}:设置系统时间
      \item \texttt{EFI\_RUNTIME\_SERVICES.GetVariable}:获取 UEFI 变量
      \item \texttt{EFI\_RUNTIME\_SERVICES.SetVariable}:设置 UEFI 变量
      \item \texttt{EFI\_RUNTIME\_SERVICES.ResetSystem}:重置系统
    \end{itemize}
\end{enumerate}

\subsection{UEFI 协议}

UEFI 使用协议(Protocol)机制提供功能接口:

\begin{enumerate}
  \item \textbf{设备协议}
    \begin{itemize}
      \item \texttt{EFI\_BLOCK\_IO\_PROTOCOL}:块设备接口
      \item \texttt{EFI\_SIMPLE\_FILE\_SYSTEM\_PROTOCOL}:文件系统接口
      \item \texttt{EFI\_GRAPHICS\_OUTPUT\_PROTOCOL}:图形输出接口
      \item \texttt{EFI\_SIMPLE\_TEXT\_INPUT\_PROTOCOL}:文本输入接口
    \end{itemize}

  \item \textbf{网络协议}
    \begin{itemize}
      \item \texttt{EFI\_MANAGED\_NETWORK\_PROTOCOL}:网络管理
      \item \texttt{EFI\_IP4\_CONFIG\_PROTOCOL}:IPv4 配置
      \item \texttt{EFI\_TCP4\_PROTOCOL}:TCP/IP 协议
    \end{itemize}

  \item \textbf{系统协议}
    \begin{itemize}
      \item \texttt{EFI\_LOADED\_IMAGE\_PROTOCOL}:已加载镜像信息
      \item \texttt{EFI\_DEVICE\_PATH\_PROTOCOL}:设备路径
      \item \texttt{EFI\_SIMPLE\_NETWORK\_PROTOCOL}:简单网络接口
    \end{itemize}
\end{enumerate}

\subsection{UEFI 应用程序开发}

UEFI 应用程序使用 EDK II(EFI Development Kit II)开发:

\begin{lstlisting}[language=C, caption={UEFI 应用程序示例}]
#include <Uefi.h>
#include <Library/UefiLib.h>
#include <Library/UefiBootServicesTableLib.h>

EFI_STATUS
EFIAPI
UefiMain (
  IN EFI_HANDLE        ImageHandle,
  IN EFI_SYSTEM_TABLE  *SystemTable
  )
{
  EFI_STATUS  Status;

  // 打印信息
  Print(L"Hello from UEFI Application!\n");

  // 获取系统时间
  EFI_TIME Time;
  Status = gRT->GetTime(&Time, NULL);
  if (!EFI_ERROR(Status)) {
    Print(L"Current time: %04d-%02d-%02d %02d:%02d:%02d\n",
          Time.Year, Time.Month, Time.Day,
          Time.Hour, Time.Minute, Time.Second);
  }

  // 等待按键
  Print(L"Press any key to exit...\n");
  gBS->WaitForEvent(1, &gST->ConIn->WaitForKey, NULL);

  return EFI_SUCCESS;
}
\end{lstlisting}

\subsection{UEFI Secure Boot}

\BlockDesc{UEFI Secure Boot}

UEFI Secure Boot 是 UEFI 规范的安全功能,用于防止恶意软件在系统启动时加载。

工作原理:
\begin{enumerate}
  \item 使用公钥加密体系
  \item 固件包含平台密钥(Platform Key, PK)
  \item 密钥交换密钥(Key Exchange Key, KEK)用于更新签名数据库
  \item 允许的签名数据库(db)包含受信任的签名
  \item 禁止的签名数据库(dbx)包含被撤销的签名
\end{enumerate}

启动验证流程:
\begin{enumerate}
  \item 固件验证 Bootloader 的签名
  \item Bootloader 验证内核的签名
  \item 内核验证驱动和模块的签名
  \item 任何验证失败都会阻止加载
\end{enumerate}

\section{GRUB}

\BlockDesc{GRUB}

GRUB(GRand Unified Bootloader)是 GNU 项目开发的 Bootloader,广泛用于 Linux 系统。

\subsection{GRUB 概述}

GRUB 的特点:
\begin{itemize}
  \item 支持多种文件系统(ext2/3/4、FAT、NTFS、XFS 等)
  \item 支持多种操作系统(Linux、Windows、BSD 等)
  \item 提供菜单界面和命令行界面
  \item 支持网络启动
  \item 支持主题和图形界面
  \item 支持加密和密码保护
\end{itemize}

\subsection{GRUB 架构}

GRUB 2 的架构:

\begin{enumerate}
  \item \textbf{Core Image}
    \begin{itemize}
      \item \texttt{core.img}:GRUB 核心镜像
      \item 包含基本功能:磁盘访问、文件系统驱动
      \item 通常安装在 MBR 或 GPT 分区
    \end{itemize}

  \item \textbf{Modules}
    \begin{itemize}
      \item 模块化设计,按需加载
      \item 文件系统模块:\texttt{ext2.mod}、\texttt{fat.mod} 等
      \item 命令模块:\texttt{linux.mod}、\texttt{chainloader.mod} 等
      \item 主题和图形模块
    \end{itemize}

  \item \textbf{Configuration}
    \begin{itemize}
      \item \texttt{/boot/grub/grub.cfg}:主配置文件
      \item 由 \texttt{grub-mkconfig} 生成
      \item 支持脚本语言
    \end{itemize}
\end{enumerate}

\subsection{GRUB 配置}

\subsubsection{配置文件结构}

GRUB 配置文件示例:

\begin{lstlisting}[language=bash, caption={GRUB 配置文件示例}]
# 全局设置
set default=0
set timeout=5
set gfxmode=1920x1080

# 加载主题
loadfont unicode
insmod gfxterm
insmod gfxmenu
insmod theme
set theme=/boot/grub/themes/mytheme/theme.txt

# 菜单项
menuentry 'Linux 5.10' {
    set root='hd0,gpt2'
    linux /vmlinuz-5.10 root=/dev/sda2 ro quiet
    initrd /initrd.img-5.10
}

menuentry 'Linux 5.10 (Recovery)' {
    set root='hd0,gpt2'
    linux /vmlinuz-5.10 root=/dev/sda2 ro single
    initrd /initrd.img-5.10
}

menuentry 'Windows 10' {
    set root='hd0,gpt1'
    chainloader /EFI/Microsoft/Boot/bootmgfw.efi
}
\end{lstlisting}

\subsubsection{常用配置选项}

\begin{enumerate}
  \item \textbf{显示设置}
    \begin{itemize}
      \item \texttt{set default=N}:默认启动项
      \item \texttt{set timeout=T}:超时时间(秒)
      \item \texttt{set gfxmode=WxH}:图形模式分辨率
    \end{itemize}

  \item \textbf{内核参数}
    \begin{itemize}
      \item \texttt{root=}:根文件系统设备
      \item \texttt{ro/rw}:只读/读写挂载
      \item \texttt{quiet}:减少启动信息输出
      \item \texttt{splash}:显示启动画面
      \item \texttt{init=}:指定 init 程序
    \end{itemize}

  \item \textbf{模块加载}
    \begin{itemize}
      \item \texttt{insmod <module>}:加载模块
      \item \texttt{set root=}:设置根设备
    \end{itemize}
\end{enumerate}

\subsection{GRUB 命令}

GRUB 提供交互式命令行:

\begin{lstlisting}[language=bash, caption={GRUB 命令示例}]
# 帮助
help
help <command>

# 设备操作
ls                    # 列出所有设备
ls (hd0,gpt2)/        # 列出分区内容
set root=(hd0,gpt2)   # 设置根设备

# 文件系统操作
cat (hd0,gpt2)/etc/fstab  # 查看文件
insmod ext2          # 加载文件系统模块

# 启动相关
linux /vmlinuz root=/dev/sda2
initrd /initrd.img
boot

# 网络启动
set net_default_server=192.168.1.1
linux (tftp)/vmlinuz root=/dev/nfs nfsroot=192.168.1.1:/nfsroot
initrd (tftp)/initrd.img
boot
\end{lstlisting}

\subsection{GRUB 安装和更新}

\subsubsection{安装 GRUB}

\begin{lstlisting}[language=bash, caption={GRUB 安装步骤}]
# BIOS 系统
grub-install /dev/sda
grub-mkconfig -o /boot/grub/grub.cfg

# UEFI 系统
grub-install --target=x86_64-efi --efi-directory=/boot/efi --bootloader-id=GRUB
grub-mkconfig -o /boot/grub/grub.cfg

# 指定设备
grub-install --root-directory=/mnt /dev/sda
\end{lstlisting}

\subsubsection{修复 GRUB}

当 GRUB 损坏或无法启动时:

\begin{lstlisting}[language=bash, caption={GRUB 修复}]
# 使用 Live CD/USB 启动
# 挂载根文件系统
mount /dev/sda2 /mnt
mount /dev/sda1 /mnt/boot/efi  # UEFI 系统

# 挂载必要的文件系统
mount --bind /dev /mnt/dev
mount --bind /proc /mnt/proc
mount --bind /sys /mnt/sys

# chroot 到系统
chroot /mnt

# 重新安装 GRUB
grub-install /dev/sda
grub-mkconfig -o /boot/grub/grub.cfg
\end{lstlisting}

\section{ARM Trusted Firmware}

\BlockDesc{ARM Trusted Firmware}

ARM Trusted Firmware(ATF)是 ARM 官方提供的安全固件参考实现,用于 ARMv8-A 架构系统。

\subsection{ATF 概述}

ATF 提供:
\begin{itemize}
  \item 安全世界(Secure World)固件
  \item 非安全世界(Non-Secure World)引导
  \item 可信执行环境(TEE)支持
  \item PSCI(Power State Coordination Interface)实现
  \item 安全启动支持
\end{itemize}

\subsection{ATF 启动流程}

ATF 在 EL3 运行,启动流程:

\begin{enumerate}
  \item \textbf{BL1(Boot Loader Stage 1)}
    \begin{itemize}
      \item 在 ROM 或 Trusted Boot ROM 中运行
      \item 初始化安全世界
      \item 加载 BL2 到安全 RAM
    \end{itemize}

  \item \textbf{BL2(Boot Loader Stage 2)}
    \begin{itemize}
      \item 在安全 RAM 中运行
      \item 加载 BL31(EL3 固件)
      \item 加载 BL32(可选,Secure Payload)
      \item 加载 BL33(Non-Secure Bootloader,如 U-Boot)
    \end{itemize}

  \item \textbf{BL31(Runtime Firmware)}
    \begin{itemize}
      \item EL3 运行时固件
      \item 提供安全监控调用(SMC)
      \item 处理安全世界和非安全世界之间的切换
    \end{itemize}

  \item \textbf{BL32(Optional Secure Payload)}
    \begin{itemize}
      \item 可信执行环境(如 OP-TEE)
      \item 在安全世界运行
    \end{itemize}

  \item \textbf{BL33(Non-Secure Firmware)}
    \begin{itemize}
      \item 非安全世界的 Bootloader(如 U-Boot)
      \item 加载和启动操作系统
    \end{itemize}
\end{enumerate}

\subsection{ATF 配置和编译}

\begin{lstlisting}[language=bash, caption={ATF 编译示例}]
# 设置工具链
export CROSS_COMPILE=aarch64-linux-gnu-

# 配置平台
make PLAT=rpi4

# 编译
make all

# 输出文件
# build/rpi4/release/bl1.bin
# build/rpi4/release/bl2.bin
# build/rpi4/release/bl31.bin
\end{lstlisting}

\section{设备树(Device Tree)}

\BlockDesc{设备树}

设备树(Device Tree)是一种描述硬件的数据结构,用于在 Bootloader 和内核之间传递硬件配置信息。

\subsection{设备树基础}

设备树解决了什么问题:
\begin{itemize}
  \item 避免在内核中硬编码硬件信息
  \item 支持同一内核在不同硬件上运行
  \item 简化内核移植工作
  \item 提供标准的硬件描述格式
\end{itemize}

设备树组成:
\begin{enumerate}
  \item \textbf{设备树源文件(.dts)}
    \begin{itemize}
      \item 人类可读的设备树描述
      \item 使用设备树语法编写
    \end{itemize}

  \item \textbf{设备树头文件(.dtsi)}
    \begin{itemize}
      \item 可重用的设备树片段
      \item 通过 \texttt{\#include} 包含
    \end{itemize}

  \item \textbf{设备树二进制文件(.dtb)}
    \begin{itemize}
      \item 编译后的设备树
      \item 由 Bootloader 加载并传递给内核
    \end{itemize}
\end{enumerate}

\subsection{设备树语法}

\subsubsection{基本结构}

\begin{lstlisting}[language=dts, caption={设备树基本结构}]
/dts-v1/;

/ {
    compatible = "mycompany,myboard";
    model = "My Board";
    #address-cells = <1>;
    #size-cells = <1>;

    cpus {
        #address-cells = <1>;
        #size-cells = <0>;

        cpu@0 {
            compatible = "arm,cortex-a53";
            device_type = "cpu";
            reg = <0x0>;
        };
    };

    memory@80000000 {
        device_type = "memory";
        reg = <0x80000000 0x40000000>;
    };

    uart0: serial@ff000000 {
        compatible = "ns16550a";
        reg = <0xff000000 0x1000>;
        interrupts = <0 10 4>;
        clock-frequency = <115200>;
    };
};
\end{lstlisting}

\subsubsection{常用属性}

\begin{enumerate}
  \item \textbf{compatible}
    \begin{itemize}
      \item 设备兼容性字符串
      \item 格式:\texttt{"manufacturer,model"}
      \item 用于匹配设备驱动
    \end{itemize}

  \item \textbf{reg}
    \begin{itemize}
      \item 设备寄存器地址和大小
      \item 格式:\texttt{reg = <address size>;}
      \item 需要配合 \texttt{\#address-cells} 和 \texttt{\#size-cells}
    \end{itemize}

  \item \textbf{interrupts}
    \begin{itemize}
      \item 中断号定义
      \item 格式:\texttt{interrupts = <irq type level>;}
    \end{itemize}

  \item \textbf{clocks}
    \begin{itemize}
      \item 时钟引用
      \item 使用 \texttt{\&clock\_ref} 引用时钟节点
    \end{itemize}

  \item \textbf{status}
    \begin{itemize}
      \item 设备状态
      \item 值:\texttt{"okay"}、\texttt{"disabled"}、\texttt{"reserved"}
    \end{itemize}
\end{enumerate}

\subsection{设备树编译}

设备树使用 DTC(Device Tree Compiler)编译:

\begin{lstlisting}[language=bash, caption={设备树编译}]
# 编译设备树
dtc -I dts -O dtb -o myboard.dtb myboard.dts

# 反编译设备树
dtc -I dtb -O dts -o myboard.dts myboard.dtb

# 查看设备树信息
fdtdump myboard.dtb
\end{lstlisting}

\subsection{设备树在启动过程中的作用}

\begin{enumerate}
  \item \textbf{Bootloader 阶段}
    \begin{itemize}
      \item U-Boot 可以加载设备树
      \item 可以在运行时修改设备树
      \item 设置设备树地址供内核使用
    \end{itemize}

  \item \textbf{内核阶段}
    \begin{itemize}
      \item 内核解析设备树
      \item 根据设备树信息初始化硬件
      \item 匹配设备驱动
    \end{itemize}
\end{enumerate}

\section{安全启动}

\BlockDesc{安全启动}

安全启动(Secure Boot)是防止恶意软件在系统启动时加载的安全机制。

\subsection{安全启动原理}

安全启动使用公钥加密体系:

\begin{enumerate}
  \item \textbf{密钥层次}
    \begin{itemize}
      \item 平台密钥(Platform Key, PK):根密钥
      \item 密钥交换密钥(Key Exchange Key, KEK):用于更新数据库
      \item 签名数据库(db):允许的签名列表
      \item 禁止数据库(dbx):被撤销的签名列表
    \end{itemize}

  \item \textbf{验证流程}
    \begin{itemize}
      \item 每个可执行文件都有数字签名
      \item 启动时验证签名
      \item 验证失败则拒绝加载
    \end{itemize}

  \item \textbf{信任链}
    \begin{itemize}
      \item 固件验证 Bootloader
      \item Bootloader 验证内核
      \item 内核验证模块和驱动
    \end{itemize}
\end{enumerate}

\subsection{UEFI Secure Boot}

UEFI Secure Boot 实现:

\begin{enumerate}
  \item \textbf{密钥管理}
    \begin{itemize}
      \item 使用 \texttt{efibootmgr} 管理密钥
      \item 密钥存储在 NVRAM 中
      \item 支持自定义密钥
    \end{itemize}

  \item \textbf{签名工具}
    \begin{itemize}
      \item \texttt{sbsign}:签名 EFI 可执行文件
      \item \texttt{pesign}:PE 文件签名工具
    \end{itemize}

  \item \textbf{配置}
    \begin{itemize}
      \item 在 UEFI 设置中启用 Secure Boot
      \item 安装平台密钥
      \item 配置签名数据库
    \end{itemize}
\end{enumerate}

\subsection{Verified Boot}

Verified Boot(验证启动)是 Android 系统使用的安全启动机制:

\begin{enumerate}
  \item \textbf{AVB(Android Verified Boot)}
    \begin{itemize}
      \item 使用 dm-verity 验证分区完整性
      \item 支持 A/B 分区更新
      \item 使用哈希树验证数据块
    \end{itemize}

  \item \textbf{实现}
    \begin{itemize}
      \item Bootloader 验证 boot 分区
      \item 内核验证 system 分区
      \item 使用哈希树和签名
    \end{itemize}
\end{enumerate}

\section{Bootloader 开发实践}

\subsection{调试技巧}

\subsubsection{串口调试}

串口是最常用的 Bootloader 调试方法:

\begin{enumerate}
  \item \textbf{配置串口}
    \begin{itemize}
      \item 设置波特率(通常 115200)
      \item 配置数据位、停止位、校验位
      \item 使用 \texttt{printf} 输出调试信息
    \end{itemize}

  \item \textbf{调试工具}
    \begin{itemize}
      \item \texttt{minicom}、\texttt{picocom}:串口终端
      \item \texttt{screen}:简单的串口工具
      \item \texttt{cu}:连接工具
    \end{itemize}
\end{enumerate}

\subsubsection{网络调试}

网络调试适用于支持网络的 Bootloader:

\begin{enumerate}
  \item \textbf{TFTP 调试}
    \begin{itemize}
      \item 通过 TFTP 下载文件
      \item 快速测试新版本
      \item 无需重新烧录 Flash
    \end{itemize}

  \item \textbf{网络控制台}
    \begin{itemize}
      \item 通过网络发送调试信息
      \item 使用 UDP 或 TCP
    \end{itemize}
\end{enumerate}

\subsubsection{JTAG 调试}

JTAG 提供底层调试能力:

\begin{enumerate}
  \item \textbf{硬件调试器}
    \begin{itemize}
      \item J-Link、OpenOCD 等
      \item 支持断点、单步执行
      \item 查看和修改寄存器、内存
    \end{itemize}

  \item \textbf{使用场景}
    \begin{itemize}
      \item 早期启动阶段调试
      \item 硬件初始化问题
      \item 崩溃分析
    \end{itemize}
\end{enumerate}

\subsection{性能优化}

\subsubsection{启动时间优化}

\begin{enumerate}
  \item \textbf{减少初始化}
    \begin{itemize}
      \item 只初始化必要的硬件
      \item 延迟非关键初始化
      \item 使用快速启动模式
    \end{itemize}

  \item \textbf{优化加载}
    \begin{itemize}
      \item 使用压缩镜像
      \item 并行加载多个文件
      \item 预加载常用数据
    \end{itemize}

  \item \textbf{减少延迟}
    \begin{itemize}
      \item 减少超时时间
      \item 跳过不必要的检查
      \item 优化存储访问
    \end{itemize}
\end{enumerate}

\subsubsection{内存优化}

\begin{enumerate}
  \item \textbf{代码优化}
    \begin{itemize}
      \item 移除未使用的功能
      \item 使用链接时优化(LTO)
      \item 优化数据结构大小
    \end{itemize}

  \item \textbf{运行时优化}
    \begin{itemize}
      \item 使用栈而不是堆
      \item 重用缓冲区
      \item 及时释放不需要的内存
    \end{itemize}
\end{enumerate}

\subsection{常见问题与解决方案}

\subsubsection{启动失败}

\begin{enumerate}
  \item \textbf{问题诊断}
    \begin{itemize}
      \item 检查串口输出
      \item 查看错误信息
      \item 使用调试器检查寄存器状态
    \end{itemize}

  \item \textbf{常见原因}
    \begin{itemize}
      \item 内存初始化失败
      \item 存储设备访问错误
      \item 设备树配置错误
      \item 内核镜像损坏
    \end{itemize}
\end{enumerate}

\subsubsection{环境变量丢失}

\begin{enumerate}
  \item \textbf{原因}
    \begin{itemize}
      \item Flash 擦写错误
      \item 环境变量区域损坏
      \item 电源异常
    \end{itemize}

  \item \textbf{解决}
    \begin{itemize}
      \item 使用默认环境变量
      \item 从备份恢复
      \item 重新配置环境变量
    \end{itemize}
\end{enumerate}

\section{嵌入式系统启动流程}

\subsection{典型嵌入式系统启动}

以 ARM 嵌入式系统为例,完整启动流程:

\begin{enumerate}
  \item \textbf{ROM Bootloader}
    \begin{itemize}
      \item SoC 内置的 BootROM
      \item 初始化基本硬件
      \item 从预定义位置加载下一阶段
    \end{itemize}

  \item \textbf{SPL/U-Boot}
    \begin{itemize}
      \item 初始化 DDR 内存
      \item 初始化存储设备
      \item 加载主 U-Boot
    \end{itemize}

  \item \textbf{主 U-Boot}
    \begin{itemize}
      \item 完整功能初始化
      \item 加载设备树
      \item 加载内核镜像
      \item 启动内核
    \end{itemize}

  \item \textbf{Linux 内核}
    \begin{itemize}
      \item 解析设备树
      \item 初始化子系统
      \item 挂载根文件系统
      \item 启动 init 进程
    \end{itemize}

  \item \textbf{用户空间}
    \begin{itemize}
      \item init 进程启动
      \item 执行启动脚本
      \item 启动应用程序
    \end{itemize}
\end{enumerate}

\subsection{存储布局}

典型的嵌入式系统存储布局:

\begin{lstlisting}[language=bash, caption={典型存储布局}]
# MMC/eMMC 分区布局
/dev/mmcblk0p1:  Bootloader (U-Boot)      - 1MB
/dev/mmcblk0p2:  Environment variables    - 128KB
/dev/mmcblk0p3:  Device Tree              - 64KB
/dev/mmcblk0p4:  Kernel                    - 10MB
/dev/mmcblk0p5:  Initramfs                  - 5MB
/dev/mmcblk0p6:  Root filesystem          - 剩余空间
\end{lstlisting}

\subsection{多阶段启动}

为了适应不同的存储和内存限制,嵌入式系统通常使用多阶段启动:

\begin{enumerate}
  \item \textbf{阶段 1:ROM Bootloader}
    \begin{itemize}
      \item 在 ROM 中,不可修改
      \item 最小功能:加载下一阶段
    \end{itemize}

  \item \textbf{阶段 2:SPL}
    \begin{itemize}
      \item 小尺寸(< 64KB)
      \item 初始化 DDR
      \item 加载主 Bootloader
    \end{itemize}

  \item \textbf{阶段 3:主 Bootloader}
    \begin{itemize}
      \item 完整功能
      \item 加载内核和文件系统
    \end{itemize}
\end{enumerate}

Bootloader 是系统启动的关键组件,理解其工作原理和实现细节对于嵌入式系统开发至关重要。
选择合适的 Bootloader、正确配置启动参数、实现安全启动机制,都是确保系统可靠启动的重要环节。
随着硬件和软件技术的发展,Bootloader 的功能也在不断扩展,包括更好的安全支持、更快的启动速度、更强的硬件兼容性等。
