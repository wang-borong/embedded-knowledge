\chapter{数据结构与算法}

数据结构与算法是计算机科学的核心基础,也是嵌入式系统开发中不可或缺的知识。
本章系统介绍各种数据结构的实现原理、应用场景和性能特征,以及常用算法的设计思想、实现方法和复杂度分析。
掌握这些知识对于编写高效、可靠的嵌入式软件至关重要。

\section{算法复杂度分析}

在讨论具体的数据结构和算法之前,我们需要建立评估算法性能的标准。
算法复杂度分析帮助我们理解算法在不同输入规模下的性能表现,是选择合适算法的重要依据。

\subsection{时间复杂度}

\BlockDesc{时间复杂度}

时间复杂度(Time Complexity)描述算法执行时间随输入规模增长的趋势。
我们使用大 O 记号(Big-O notation)来表示时间复杂度,它描述了算法在最坏情况下的渐近行为。

常见的时间复杂度:

\begin{enumerate}
  \item \textbf{O(1) - 常数时间}
    \begin{itemize}
      \item 无论输入规模多大,执行时间都是常数
      \item 例如:数组索引访问、哈希表查找(平均情况)
    \end{itemize}

  \item \textbf{O(log n) - 对数时间}
    \begin{itemize}
      \item 执行时间随输入规模的对数增长
      \item 例如:二分查找、平衡二叉树的查找
    \end{itemize}

  \item \textbf{O(n) - 线性时间}
    \begin{itemize}
      \item 执行时间与输入规模成正比
      \item 例如:遍历数组、链表查找
    \end{itemize}

  \item \textbf{O(n log n) - 线性对数时间}
    \begin{itemize}
      \item 执行时间介于线性和平方之间
      \item 例如:归并排序、快速排序(平均情况)、堆排序
    \end{itemize}

  \item \textbf{O(n²) - 平方时间}
    \begin{itemize}
      \item 执行时间与输入规模的平方成正比
      \item 例如:冒泡排序、选择排序、插入排序、嵌套循环
    \end{itemize}

  \item \textbf{O(2ⁿ) - 指数时间}
    \begin{itemize}
      \item 执行时间随输入规模指数增长
      \item 例如:递归计算斐波那契数列(朴素实现)、穷举搜索
    \end{itemize}

  \item \textbf{O(n!) - 阶乘时间}
    \begin{itemize}
      \item 执行时间随输入规模阶乘增长
      \item 例如:生成所有排列、旅行商问题的暴力解法
    \end{itemize}
\end{enumerate}

\subsection{空间复杂度}

\BlockDesc{空间复杂度}

空间复杂度(Space Complexity)描述算法所需内存空间随输入规模增长的趋势。
同样使用大 O 记号表示。

空间复杂度包括:
\begin{itemize}
  \item \textbf{输入空间}:存储输入数据所需的空间
  \item \textbf{辅助空间}:算法执行过程中额外使用的空间
  \item 通常我们关注的是辅助空间复杂度
\end{itemize}

示例:
\begin{enumerate}
  \item \textbf{O(1)}:原地排序算法(如快速排序的某些实现)
  \item \textbf{O(n)}:需要额外数组的归并排序
  \item \textbf{O(log n)}:递归调用栈的深度(如快速排序的递归实现)
\end{enumerate}

\subsection{复杂度分析技巧}

\begin{enumerate}
  \item \textbf{最坏情况分析}
    \begin{itemize}
      \item 分析算法在最坏输入下的性能
      \item 保证算法在任何情况下都不会超过这个界限
    \end{itemize}

  \item \textbf{平均情况分析}
    \begin{itemize}
      \item 分析算法在随机输入下的期望性能
      \item 需要知道输入数据的概率分布
    \end{itemize}

  \item \textbf{最好情况分析}
    \begin{itemize}
      \item 分析算法在最优输入下的性能
      \item 通常用于理解算法的下界
    \end{itemize}

  \item \textbf{摊还分析}
    \begin{itemize}
      \item 分析一系列操作的平均成本
      \item 适用于动态数组扩容、哈希表重哈希等场景
    \end{itemize}
\end{enumerate}

\section{数据结构}

数据结构是组织和存储数据的方式,不同的数据结构适用于不同的应用场景。
选择合适的数据结构可以显著提高程序的效率和可维护性。

\subsection{数组}

\BlockDesc{数组}

数组(Array)是最基本的数据结构,是一组连续内存空间存储的相同类型元素的集合。

特点:
\begin{itemize}
  \item 支持随机访问,时间复杂度 O(1)
  \item 插入和删除操作需要移动元素,时间复杂度 O(n)
  \item 内存连续,缓存友好
  \item 大小固定(静态数组)或可动态调整(动态数组)
\end{itemize}

动态数组实现要点:
\begin{enumerate}
  \item 初始分配一定大小的内存空间
  \item 当空间不足时,分配更大的空间(通常是原来的 2 倍)
  \item 将原有数据复制到新空间
  \item 释放旧空间
  \item 摊还分析:单次插入操作的平均时间复杂度为 O(1)
\end{enumerate}

C 语言实现示例:

\begin{lstlisting}[language=C, caption={动态数组实现}]
typedef struct {
    int *data;
    size_t size;
    size_t capacity;
} DynamicArray;

DynamicArray* da_create(size_t initial_capacity) {
    DynamicArray *da = malloc(sizeof(DynamicArray));
    da->data = malloc(initial_capacity * sizeof(int));
    da->size = 0;
    da->capacity = initial_capacity;
    return da;
}

void da_push(DynamicArray *da, int value) {
    if (da->size >= da->capacity) {
        da->capacity *= 2;
        da->data = realloc(da->data, da->capacity * sizeof(int));
    }
    da->data[da->size++] = value;
}

int da_get(DynamicArray *da, size_t index) {
    if (index >= da->size) return -1; // 错误处理
    return da->data[index];
}
\end{lstlisting}

\subsection{链表}

\BlockDesc{链表}

链表(Linked List)是一种动态数据结构,通过指针将节点连接起来。
每个节点包含数据和指向下一个节点的指针。

特点:
\begin{itemize}
  \item 动态大小,无需预先分配内存
  \item 插入和删除操作时间复杂度 O(1)(已知节点位置)
  \item 不支持随机访问,查找需要 O(n)
  \item 内存不连续,缓存性能较差
\end{itemize}

\subsubsection{单向链表}

单向链表(Singly Linked List)的每个节点只有一个指向下一个节点的指针。

\begin{lstlisting}[language=C, caption={单向链表实现}]
typedef struct Node {
    int data;
    struct Node *next;
} Node;

typedef struct {
    Node *head;
    size_t size;
} LinkedList;

LinkedList* list_create() {
    LinkedList *list = malloc(sizeof(LinkedList));
    list->head = NULL;
    list->size = 0;
    return list;
}

void list_insert(LinkedList *list, int value) {
    Node *new_node = malloc(sizeof(Node));
    new_node->data = value;
    new_node->next = list->head;
    list->head = new_node;
    list->size++;
}

Node* list_find(LinkedList *list, int value) {
    Node *current = list->head;
    while (current != NULL) {
        if (current->data == value) {
            return current;
        }
        current = current->next;
    }
    return NULL;
}

void list_delete(LinkedList *list, int value) {
    if (list->head == NULL) return;

    if (list->head->data == value) {
        Node *temp = list->head;
        list->head = list->head->next;
        free(temp);
        list->size--;
        return;
    }

    Node *current = list->head;
    while (current->next != NULL) {
        if (current->next->data == value) {
            Node *temp = current->next;
            current->next = current->next->next;
            free(temp);
            list->size--;
            return;
        }
        current = current->next;
    }
}
\end{lstlisting}

\subsubsection{双向链表}

双向链表(Doubly Linked List)的每个节点有两个指针,分别指向前一个和后一个节点。

优势:
\begin{itemize}
  \item 可以从两个方向遍历
  \item 删除节点时不需要前驱节点
  \item 支持 O(1) 时间复杂度的前向和后向遍历
\end{itemize}

\subsubsection{循环链表}

循环链表(Circular Linked List)的尾节点指向头节点,形成一个环。

应用场景:
\begin{itemize}
  \item 轮询调度算法
  \item 实现队列(使用循环链表可以避免头尾指针的边界处理)
  \item 约瑟夫问题
\end{itemize}

\subsection{栈}

\BlockDesc{栈}

栈(Stack)是一种后进先出(LIFO, Last In First Out)的数据结构。
只允许在栈顶进行插入和删除操作。

操作:
\begin{itemize}
  \item \texttt{push}:将元素压入栈顶,O(1)
  \item \texttt{pop}:弹出栈顶元素,O(1)
  \item \texttt{top/peek}:查看栈顶元素,O(1)
  \item \texttt{empty}:判断栈是否为空,O(1)
\end{itemize}

应用场景:
\begin{enumerate}
  \item \textbf{函数调用栈}
    \begin{itemize}
      \item 保存函数调用的返回地址和局部变量
      \item 支持递归调用
    \end{itemize}

  \item \textbf{表达式求值}
    \begin{itemize}
      \item 中缀表达式转后缀表达式
      \item 后缀表达式求值
      \item 括号匹配检查
    \end{itemize}

  \item \textbf{回溯算法}
    \begin{itemize}
      \item 深度优先搜索(DFS)
      \item 迷宫求解
    \end{itemize}

  \item \textbf{撤销操作}
    \begin{itemize}
      \item 文本编辑器的撤销功能
      \item 浏览器的后退功能
    \end{itemize}
\end{enumerate}

实现方式:
\begin{enumerate}
  \item \textbf{数组实现}
    \begin{itemize}
      \item 使用数组和栈顶指针
      \item 需要处理数组扩容问题
    \end{itemize}

  \item \textbf{链表实现}
    \begin{itemize}
      \item 使用链表头作为栈顶
      \item 动态大小,无需扩容
    \end{itemize}
\end{enumerate}

\begin{lstlisting}[language=C, caption={栈的数组实现}]
typedef struct {
    int *data;
    size_t top;
    size_t capacity;
} Stack;

Stack* stack_create(size_t capacity) {
    Stack *s = malloc(sizeof(Stack));
    s->data = malloc(capacity * sizeof(int));
    s->top = 0;
    s->capacity = capacity;
    return s;
}

void stack_push(Stack *s, int value) {
    if (s->top >= s->capacity) {
        // 扩容处理
        s->capacity *= 2;
        s->data = realloc(s->data, s->capacity * sizeof(int));
    }
    s->data[s->top++] = value;
}

int stack_pop(Stack *s) {
    if (s->top == 0) {
        return -1; // 栈空
    }
    return s->data[--s->top];
}

int stack_top(Stack *s) {
    if (s->top == 0) {
        return -1;
    }
    return s->data[s->top - 1];
}
\end{lstlisting}

\subsection{队列}

\BlockDesc{队列}

队列(Queue)是一种先进先出(FIFO, First In First Out)的数据结构。
元素从队尾入队,从队头出队。

操作:
\begin{itemize}
  \item \texttt{enqueue}:将元素加入队尾,O(1)
  \item \texttt{dequeue}:从队头移除元素,O(1)
  \item \texttt{front}:查看队头元素,O(1)
  \item \texttt{empty}:判断队列是否为空,O(1)
\end{itemize}

应用场景:
\begin{enumerate}
  \item \textbf{任务调度}
    \begin{itemize}
      \item 操作系统的进程调度
      \item 打印队列
    \end{itemize}

  \item \textbf{广度优先搜索(BFS)}
    \begin{itemize}
      \item 图的遍历
      \item 最短路径问题
    \end{itemize}

  \item \textbf{消息队列}
    \begin{itemize}
      \item 生产者-消费者模型
      \item 异步任务处理
    \end{itemize}

  \item \textbf{缓冲区}
    \begin{itemize}
      \item 网络数据包缓冲
      \item I/O 缓冲区
    \end{itemize}
\end{enumerate}

\subsubsection{循环队列}

循环队列(Circular Queue)使用固定大小的数组,通过模运算实现循环使用。

优势:
\begin{itemize}
  \item 避免数组实现的假溢出问题
  \item 充分利用数组空间
  \item 实现简单高效
\end{itemize}

实现要点:
\begin{itemize}
  \item 使用 \texttt{(front + 1) \% capacity} 计算下一个位置
  \item 区分队列满和队列空的状态(使用计数器或预留一个空位)
\end{itemize}

\begin{lstlisting}[language=C, caption={循环队列实现}]
typedef struct {
    int *data;
    size_t front;
    size_t rear;
    size_t size;
    size_t capacity;
} CircularQueue;

CircularQueue* queue_create(size_t capacity) {
    CircularQueue *q = malloc(sizeof(CircularQueue));
    q->data = malloc(capacity * sizeof(int));
    q->front = 0;
    q->rear = 0;
    q->size = 0;
    q->capacity = capacity;
    return q;
}

void queue_enqueue(CircularQueue *q, int value) {
    if (q->size >= q->capacity) {
        return; // 队列满
    }
    q->data[q->rear] = value;
    q->rear = (q->rear + 1) % q->capacity;
    q->size++;
}

int queue_dequeue(CircularQueue *q) {
    if (q->size == 0) {
        return -1; // 队列空
    }
    int value = q->data[q->front];
    q->front = (q->front + 1) % q->capacity;
    q->size--;
    return value;
}
\end{lstlisting}

\subsubsection{双端队列}

双端队列(Deque, Double-Ended Queue)允许在两端进行插入和删除操作。

操作:
\begin{itemize}
  \item \texttt{push\_front}:在队头插入
  \item \texttt{push\_back}:在队尾插入
  \item \texttt{pop\_front}:从队头删除
  \item \texttt{pop\_back}:从队尾删除
\end{itemize}

\subsection{树}

\BlockDesc{树}

树(Tree)是一种层次化的数据结构,由节点和边组成。
每个节点有零个或多个子节点,除了根节点外,每个节点都有一个父节点。

术语:
\begin{itemize}
  \item \textbf{节点(Node)}:树的基本单元
  \item \textbf{根节点(Root)}:没有父节点的节点
  \item \textbf{叶子节点(Leaf)}:没有子节点的节点
  \item \textbf{内部节点(Internal Node)}:有子节点的节点
  \item \textbf{深度(Depth)}:从根到节点的路径长度
  \item \textbf{高度(Height)}:从节点到最远叶子节点的路径长度
  \item \textbf{度(Degree)}:节点的子节点数量
  \item \textbf{层(Level)}:根节点为第 0 层,每向下一层加 1
\end{itemize}

\subsubsection{二叉树}

二叉树(Binary Tree)是每个节点最多有两个子节点的树结构。

性质:
\begin{itemize}
  \item 第 $i$ 层最多有 $2^i$ 个节点
  \item 深度为 $k$ 的二叉树最多有 $2^{k+1} - 1$ 个节点
  \item 对于有 $n$ 个节点的二叉树,最小高度为 $\lfloor \log_2 n \rfloor$
\end{itemize}

遍历方式:
\begin{enumerate}
  \item \textbf{前序遍历(Pre-order)}
    \begin{itemize}
      \item 访问顺序:根 → 左子树 → 右子树
      \item 应用:复制树、前缀表达式
    \end{itemize}

  \item \textbf{中序遍历(In-order)}
    \begin{itemize}
      \item 访问顺序:左子树 → 根 → 右子树
      \item 应用:二叉搜索树的有序输出
    \end{itemize}

  \item \textbf{后序遍历(Post-order)}
    \begin{itemize}
      \item 访问顺序:左子树 → 右子树 → 根
      \item 应用:删除树、后缀表达式
    \end{itemize}

  \item \textbf{层序遍历(Level-order)}
    \begin{itemize}
      \item 按层从上到下、从左到右访问
      \item 使用队列实现
    \end{itemize}
\end{enumerate}

\begin{lstlisting}[language=C, caption={二叉树节点定义和遍历}]
typedef struct TreeNode {
    int data;
    struct TreeNode *left;
    struct TreeNode *right;
} TreeNode;

// 前序遍历
void preorder_traverse(TreeNode *root) {
    if (root == NULL) return;
    printf("%d ", root->data);
    preorder_traverse(root->left);
    preorder_traverse(root->right);
}

// 中序遍历
void inorder_traverse(TreeNode *root) {
    if (root == NULL) return;
    inorder_traverse(root->left);
    printf("%d ", root->data);
    inorder_traverse(root->right);
}

// 后序遍历
void postorder_traverse(TreeNode *root) {
    if (root == NULL) return;
    postorder_traverse(root->left);
    postorder_traverse(root->right);
    printf("%d ", root->data);
}

// 层序遍历
void levelorder_traverse(TreeNode *root) {
    if (root == NULL) return;

    CircularQueue *q = queue_create(100);
    queue_enqueue(q, (int)root);

    while (q->size > 0) {
        TreeNode *node = (TreeNode*)queue_dequeue(q);
        printf("%d ", node->data);

        if (node->left != NULL) {
            queue_enqueue(q, (int)node->left);
        }
        if (node->right != NULL) {
            queue_enqueue(q, (int)node->right);
        }
    }
}
\end{lstlisting}

\subsubsection{二叉搜索树}

二叉搜索树(Binary Search Tree, BST)是一种特殊的二叉树,满足:
\begin{itemize}
  \item 左子树所有节点的值小于根节点的值
  \item 右子树所有节点的值大于根节点的值
  \item 左右子树也都是二叉搜索树
\end{itemize}

操作复杂度:
\begin{itemize}
  \item 查找:O(h),h 为树的高度
  \item 插入:O(h)
  \item 删除:O(h)
  \item 最坏情况(退化为链表):O(n)
  \item 最好情况(平衡树):O(log n)
\end{itemize}

\begin{lstlisting}[language=C, caption={二叉搜索树操作}]
TreeNode* bst_search(TreeNode *root, int key) {
    if (root == NULL || root->data == key) {
        return root;
    }
    if (key < root->data) {
        return bst_search(root->left, key);
    } else {
        return bst_search(root->right, key);
    }
}

TreeNode* bst_insert(TreeNode *root, int key) {
    if (root == NULL) {
        TreeNode *node = malloc(sizeof(TreeNode));
        node->data = key;
        node->left = node->right = NULL;
        return node;
    }

    if (key < root->data) {
        root->left = bst_insert(root->left, key);
    } else if (key > root->data) {
        root->right = bst_insert(root->right, key);
    }
    return root;
}

TreeNode* bst_delete(TreeNode *root, int key) {
    if (root == NULL) return root;

    if (key < root->data) {
        root->left = bst_delete(root->left, key);
    } else if (key > root->data) {
        root->right = bst_delete(root->right, key);
    } else {
        // 找到要删除的节点
        if (root->left == NULL) {
            TreeNode *temp = root->right;
            free(root);
            return temp;
        } else if (root->right == NULL) {
            TreeNode *temp = root->left;
            free(root);
            return temp;
        }

        // 有两个子节点,找到右子树的最小值
        TreeNode *temp = root->right;
        while (temp->left != NULL) {
            temp = temp->left;
        }
        root->data = temp->data;
        root->right = bst_delete(root->right, temp->data);
    }
    return root;
}
\end{lstlisting}

\subsubsection{平衡二叉树}

平衡二叉树(Balanced Binary Tree)通过旋转操作保持树的平衡,避免退化为链表。

\paragraph{AVL 树}

AVL 树是最早的自平衡二叉搜索树,得名于其发明者 Adelson-Velsky 和 Landis。

性质:
\begin{itemize}
  \item 任意节点的左右子树高度差不超过 1
  \item 所有操作的时间复杂度为 O(log n)
\end{itemize}

旋转操作:
\begin{enumerate}
  \item \textbf{左旋(Left Rotation)}
    \begin{itemize}
      \item 当右子树过高时使用
      \item 将右子节点提升为新的根节点
    \end{itemize}

  \item \textbf{右旋(Right Rotation)}
    \begin{itemize}
      \item 当左子树过高时使用
      \item 将左子节点提升为新的根节点
    \end{itemize}

  \item \textbf{左右旋(Left-Right Rotation)}
    \begin{itemize}
      \item 先对左子树左旋,再对根节点右旋
    \end{itemize}

  \item \textbf{右左旋(Right-Left Rotation)}
    \begin{itemize}
      \item 先对右子树右旋,再对根节点左旋
    \end{itemize}
\end{enumerate}

\paragraph{红黑树}

红黑树(Red-Black Tree)是另一种自平衡二叉搜索树,在 Linux 内核、C++ STL 的 map/set 中广泛使用。

性质:
\begin{enumerate}
  \item 每个节点要么是红色,要么是黑色
  \item 根节点是黑色
  \item 每个叶子节点(NIL)是黑色
  \item 如果一个节点是红色,则它的两个子节点都是黑色
  \item 从任意节点到其每个叶子的所有路径都包含相同数目的黑色节点
\end{enumerate}

优势:
\begin{itemize}
  \item 插入和删除操作的平均和最坏情况都是 O(log n)
  \item 旋转操作比 AVL 树少,性能更优
  \item 实现相对简单
\end{itemize}

\subsubsection{B 树和 B+ 树}

B 树(B-Tree)是一种多路平衡搜索树,主要用于数据库和文件系统。

特点:
\begin{itemize}
  \item 每个节点可以有多个子节点(通常数百到数千个)
  \item 所有叶子节点在同一层
  \item 节点中的关键字按升序排列
  \item 降低树的高度,减少磁盘 I/O 次数
\end{itemize}

B+ 树是 B 树的变种:
\begin{itemize}
  \item 所有数据都存储在叶子节点
  \item 叶子节点之间用指针连接,支持范围查询
  \item 内部节点只存储关键字,不存储数据
  \item 广泛应用于数据库索引
\end{itemize}

\subsubsection{Trie 树}

Trie 树(前缀树,Prefix Tree)是一种专门用于字符串检索的树结构。

特点:
\begin{itemize}
  \item 每个节点代表一个字符
  \item 从根到节点的路径表示一个字符串
  \item 支持前缀匹配和字符串查找
  \item 查找时间复杂度 O(m),m 为字符串长度
\end{itemize}

应用场景:
\begin{itemize}
  \item 字符串检索
  \item 自动补全
  \item IP 路由表
  \item 拼写检查
\end{itemize}

\subsection{堆}

\BlockDesc{堆}

堆(Heap)是一种特殊的完全二叉树,满足堆序性质。

堆的类型:
\begin{enumerate}
  \item \textbf{最大堆(Max Heap)}
    \begin{itemize}
      \item 父节点的值大于等于所有子节点的值
      \item 根节点是最大值
    \end{itemize}

  \item \textbf{最小堆(Min Heap)}
    \begin{itemize}
      \item 父节点的值小于等于所有子节点的值
      \item 根节点是最小值
    \end{itemize}
\end{enumerate}

性质:
\begin{itemize}
  \item 完全二叉树,可以用数组实现
  \item 对于节点 $i$,父节点为 $\lfloor (i-1)/2 \rfloor$,左子节点为 $2i+1$,右子节点为 $2i+2$
  \item 插入和删除的时间复杂度为 O(log n)
  \item 查找最大值/最小值的时间复杂度为 O(1)
\end{itemize}

应用场景:
\begin{enumerate}
  \item \textbf{优先队列}
    \begin{itemize}
      \item 任务调度
      \item 事件驱动系统
    \end{itemize}

  \item \textbf{堆排序}
    \begin{itemize}
      \item 时间复杂度 O(n log n)
      \item 原地排序算法
    \end{itemize}

  \item \textbf{Top-K 问题}
    \begin{itemize}
      \item 找出最大的 K 个元素
      \item 使用最小堆维护 K 个元素
    \end{itemize}

  \item \textbf{合并 K 个有序链表}
    \begin{itemize}
      \item 使用最小堆维护每个链表的当前节点
    \end{itemize}
\end{enumerate}

\begin{lstlisting}[language=C, caption={最大堆实现}]
typedef struct {
    int *data;
    size_t size;
    size_t capacity;
} MaxHeap;

MaxHeap* heap_create(size_t capacity) {
    MaxHeap *h = malloc(sizeof(MaxHeap));
    h->data = malloc(capacity * sizeof(int));
    h->size = 0;
    h->capacity = capacity;
    return h;
}

void heapify_up(MaxHeap *h, size_t index) {
    while (index > 0) {
        size_t parent = (index - 1) / 2;
        if (h->data[parent] >= h->data[index]) {
            break;
        }
        // 交换
        int temp = h->data[parent];
        h->data[parent] = h->data[index];
        h->data[index] = temp;
        index = parent;
    }
}

void heapify_down(MaxHeap *h, size_t index) {
    while (1) {
        size_t left = 2 * index + 1;
        size_t right = 2 * index + 2;
        size_t largest = index;

        if (left < h->size && h->data[left] > h->data[largest]) {
            largest = left;
        }
        if (right < h->size && h->data[right] > h->data[largest]) {
            largest = right;
        }

        if (largest == index) break;

        // 交换
        int temp = h->data[index];
        h->data[index] = h->data[largest];
        h->data[largest] = temp;
        index = largest;
    }
}

void heap_insert(MaxHeap *h, int value) {
    if (h->size >= h->capacity) {
        h->capacity *= 2;
        h->data = realloc(h->data, h->capacity * sizeof(int));
    }
    h->data[h->size] = value;
    heapify_up(h, h->size);
    h->size++;
}

int heap_extract_max(MaxHeap *h) {
    if (h->size == 0) return -1;

    int max = h->data[0];
    h->data[0] = h->data[h->size - 1];
    h->size--;
    heapify_down(h, 0);
    return max;
}
\end{lstlisting}

\subsection{哈希表}

\BlockDesc{哈希表}

哈希表(Hash Table)是一种根据键(Key)直接访问内存存储位置的数据结构。
通过哈希函数将键映射到数组索引,实现 O(1) 平均时间复杂度的查找、插入和删除。

哈希函数:
\begin{itemize}
  \item 将任意大小的输入映射到固定大小的输出
  \item 理想情况下,不同的键映射到不同的索引
  \item 实际中可能出现哈希冲突
\end{itemize}

常见哈希函数:
\begin{enumerate}
  \item \textbf{除法哈希法}
    \begin{itemize}
      \item $h(k) = k \bmod m$
      \item $m$ 通常选择质数
    \end{itemize}

  \item \textbf{乘法哈希法}
    \begin{itemize}
      \item $h(k) = \lfloor m(kA \bmod 1) \rfloor$
      \item $A$ 通常选择 $(\sqrt{5} - 1)/2 \approx 0.618$
    \end{itemize}

  \item \textbf{字符串哈希}
    \begin{itemize}
      \item 多项式哈希:$h(s) = \sum_{i=0}^{n-1} s[i] \cdot a^{n-1-i} \bmod m$
      \item 常用的 $a$ 值:31, 131, 1313
    \end{itemize}
\end{enumerate}

冲突处理方法:
\begin{enumerate}
  \item \textbf{链地址法(Chaining)}
    \begin{itemize}
      \item 每个桶存储一个链表
      \item 冲突的元素添加到链表中
      \item 实现简单,但需要额外的指针空间
    \end{itemize}

  \item \textbf{开放寻址法(Open Addressing)}
    \begin{itemize}
      \item 所有元素都存储在数组中
      \item 冲突时寻找下一个可用位置
      \item 方法包括:线性探测、二次探测、双重哈希
    \end{itemize}

  \item \textbf{再哈希法}
    \begin{itemize}
      \item 使用第二个哈希函数计算新的位置
      \item 减少聚集现象
    \end{itemize}
\end{enumerate}

负载因子:
\begin{itemize}
  \item $\alpha = n/m$,其中 $n$ 是元素数量,$m$ 是桶的数量
  \item 负载因子过高会导致性能下降
  \item 通常当 $\alpha > 0.75$ 时进行扩容
  \item 扩容时需要重新哈希所有元素
\end{itemize}

\begin{lstlisting}[language=C, caption={哈希表实现(链地址法)}]
#define HASH_TABLE_SIZE 1000

typedef struct HashNode {
    int key;
    int value;
    struct HashNode *next;
} HashNode;

typedef struct {
    HashNode **buckets;
    size_t size;
} HashTable;

HashTable* hash_create() {
    HashTable *ht = malloc(sizeof(HashTable));
    ht->buckets = calloc(HASH_TABLE_SIZE, sizeof(HashNode*));
    ht->size = 0;
    return ht;
}

size_t hash_function(int key) {
    return key % HASH_TABLE_SIZE;
}

void hash_insert(HashTable *ht, int key, int value) {
    size_t index = hash_function(key);
    HashNode *node = ht->buckets[index];

    // 检查是否已存在
    while (node != NULL) {
        if (node->key == key) {
            node->value = value;
            return;
        }
        node = node->next;
    }

    // 插入新节点
    HashNode *new_node = malloc(sizeof(HashNode));
    new_node->key = key;
    new_node->value = value;
    new_node->next = ht->buckets[index];
    ht->buckets[index] = new_node;
    ht->size++;
}

int hash_get(HashTable *ht, int key) {
    size_t index = hash_function(key);
    HashNode *node = ht->buckets[index];

    while (node != NULL) {
        if (node->key == key) {
            return node->value;
        }
        node = node->next;
    }
    return -1; // 未找到
}

void hash_delete(HashTable *ht, int key) {
    size_t index = hash_function(key);
    HashNode *node = ht->buckets[index];
    HashNode *prev = NULL;

    while (node != NULL) {
        if (node->key == key) {
            if (prev == NULL) {
                ht->buckets[index] = node->next;
            } else {
                prev->next = node->next;
            }
            free(node);
            ht->size--;
            return;
        }
        prev = node;
        node = node->next;
    }
}
\end{lstlisting}

\subsection{位图}

\BlockDesc{位图}

位图(Bitmap)是一种使用位数组来表示集合的数据结构。
每个位表示一个元素是否存在,非常适合表示稠密集合。

特点:
\begin{itemize}
  \item 空间效率高:每个元素只占用 1 位
  \item 操作快速:位运算实现,时间复杂度 O(1)
  \item 适合表示整数集合
  \item 内存连续,缓存友好
\end{itemize}

应用场景:
\begin{enumerate}
  \item \textbf{集合操作}
    \begin{itemize}
      \item 并集、交集、差集
      \item 成员查询
    \end{itemize}

  \item \textbf{去重}
    \begin{itemize}
      \item 标记已访问的元素
      \item 找出重复元素
    \end{itemize}

  \item \textbf{布隆过滤器}
    \begin{itemize}
      \item 快速判断元素是否可能存在
      \item 用于缓存、数据库等场景
    \end{itemize}

  \item \textbf{内存管理}
    \begin{itemize}
      \item 标记内存页的使用情况
      \item 标记文件系统中的块使用情况
    \end{itemize}
\end{enumerate}

\begin{lstlisting}[language=C, caption={位图实现}]
typedef struct {
    unsigned char *bits;
    size_t size;  // 位数量
} Bitmap;

Bitmap* bitmap_create(size_t size) {
    Bitmap *bm = malloc(sizeof(Bitmap));
    bm->size = size;
    size_t bytes = (size + 7) / 8;  // 向上取整
    bm->bits = calloc(bytes, sizeof(unsigned char));
    return bm;
}

void bitmap_set(Bitmap *bm, size_t index) {
    if (index >= bm->size) return;
    size_t byte_index = index / 8;
    size_t bit_index = index % 8;
    bm->bits[byte_index] |= (1 << bit_index);
}

void bitmap_clear(Bitmap *bm, size_t index) {
    if (index >= bm->size) return;
    size_t byte_index = index / 8;
    size_t bit_index = index % 8;
    bm->bits[byte_index] &= ~(1 << bit_index);
}

int bitmap_get(Bitmap *bm, size_t index) {
    if (index >= bm->size) return 0;
    size_t byte_index = index / 8;
    size_t bit_index = index % 8;
    return (bm->bits[byte_index] >> bit_index) & 1;
}
\end{lstlisting}

\subsection{图}

\BlockDesc{图}

图(Graph)是一种由节点(顶点)和边组成的数据结构,用于表示实体之间的关系。

术语:
\begin{itemize}
  \item \textbf{顶点(Vertex)}:图中的节点
  \item \textbf{边(Edge)}:连接两个顶点的线
  \item \textbf{有向图(Directed Graph)}:边有方向
  \item \textbf{无向图(Undirected Graph)}:边无方向
  \item \textbf{权重(Weight)}:边的权值
  \item \textbf{度(Degree)}:与顶点相连的边的数量
  \item \textbf{路径(Path)}:从顶点 $u$ 到 $v$ 的顶点序列
  \item \textbf{环(Cycle)}:起点和终点相同的路径
  \item \textbf{连通图(Connected Graph)}:任意两个顶点之间都有路径
\end{itemize}

\subsubsection{图的表示}

\begin{enumerate}
  \item \textbf{邻接矩阵(Adjacency Matrix)}
    \begin{itemize}
      \item 使用二维数组表示
      \item $matrix[i][j] = 1$ 表示存在边 $(i, j)$
      \item 对于有权图,存储权重;无权重时存储 1 或 0
      \item 空间复杂度 O($V^2$)
      \item 查询边是否存在:O(1)
      \item 遍历所有邻接顶点:O(V)
    \end{itemize}

  \item \textbf{邻接表(Adjacency List)}
    \begin{itemize}
      \item 每个顶点维护一个链表,存储其邻接顶点
      \item 空间复杂度 O(V + E)
      \item 查询边是否存在:O(degree(v))
      \item 遍历所有邻接顶点:O(degree(v))
      \item 适合稀疏图
    \end{itemize}

  \item \textbf{边列表(Edge List)}
    \begin{itemize}
      \item 直接存储所有边的列表
      \item 空间复杂度 O(E)
      \item 适合某些特定算法(如 Kruskal 最小生成树算法)
    \end{itemize}
\end{enumerate}

\begin{lstlisting}[language=C, caption={图的邻接表表示}]
typedef struct EdgeNode {
    int vertex;
    int weight;
    struct EdgeNode *next;
} EdgeNode;

typedef struct {
    EdgeNode **adj_list;
    int num_vertices;
    int num_edges;
    int directed;
} Graph;

Graph* graph_create(int num_vertices, int directed) {
    Graph *g = malloc(sizeof(Graph));
    g->num_vertices = num_vertices;
    g->num_edges = 0;
    g->directed = directed;
    g->adj_list = calloc(num_vertices, sizeof(EdgeNode*));
    return g;
}

void graph_add_edge(Graph *g, int from, int to, int weight) {
    EdgeNode *edge = malloc(sizeof(EdgeNode));
    edge->vertex = to;
    edge->weight = weight;
    edge->next = g->adj_list[from];
    g->adj_list[from] = edge;
    g->num_edges++;

    if (!g->directed) {
        // 无向图,添加反向边
        EdgeNode *reverse_edge = malloc(sizeof(EdgeNode));
        reverse_edge->vertex = from;
        reverse_edge->weight = weight;
        reverse_edge->next = g->adj_list[to];
        g->adj_list[to] = reverse_edge;
    }
}
\end{lstlisting}

\subsubsection{图的遍历}

\begin{enumerate}
  \item \textbf{深度优先搜索(DFS)}
    \begin{itemize}
      \item 尽可能深地搜索图的分支
      \item 使用栈(递归或显式栈)实现
      \item 时间复杂度:O(V + E)
      \item 应用:拓扑排序、强连通分量、路径查找
    \end{itemize}

  \item \textbf{广度优先搜索(BFS)}
    \begin{itemize}
      \item 逐层遍历图
      \item 使用队列实现
      \item 时间复杂度:O(V + E)
      \item 应用:最短路径(无权图)、层次遍历、连通性检测
    \end{itemize}
\end{enumerate}

\begin{lstlisting}[language=C, caption={图的深度优先搜索}]
void dfs_recursive(Graph *g, int vertex, int *visited) {
    visited[vertex] = 1;
    printf("%d ", vertex);

    EdgeNode *edge = g->adj_list[vertex];
    while (edge != NULL) {
        if (!visited[edge->vertex]) {
            dfs_recursive(g, edge->vertex, visited);
        }
        edge = edge->next;
    }
}

void dfs_iterative(Graph *g, int start) {
    int *visited = calloc(g->num_vertices, sizeof(int));
    Stack *s = stack_create(g->num_vertices);

    stack_push(s, start);
    visited[start] = 1;

    while (s->top > 0) {
        int vertex = stack_pop(s);
        printf("%d ", vertex);

        EdgeNode *edge = g->adj_list[vertex];
        while (edge != NULL) {
            if (!visited[edge->vertex]) {
                visited[edge->vertex] = 1;
                stack_push(s, edge->vertex);
            }
            edge = edge->next;
        }
    }

    free(visited);
}
\end{lstlisting}

\begin{lstlisting}[language=C, caption={图的广度优先搜索}]
void bfs(Graph *g, int start) {
    int *visited = calloc(g->num_vertices, sizeof(int));
    CircularQueue *q = queue_create(g->num_vertices);

    queue_enqueue(q, start);
    visited[start] = 1;

    while (q->size > 0) {
        int vertex = queue_dequeue(q);
        printf("%d ", vertex);

        EdgeNode *edge = g->adj_list[vertex];
        while (edge != NULL) {
            if (!visited[edge->vertex]) {
                visited[edge->vertex] = 1;
                queue_enqueue(q, edge->vertex);
            }
            edge = edge->next;
        }
    }

    free(visited);
}
\end{lstlisting}

\section{算法}

算法是解决问题的步骤和方法。
好的算法不仅要正确,还要高效。
本节介绍各种算法设计思想和经典算法。

\subsection{排序算法}

排序是计算机科学中最基础的操作之一。
不同的排序算法有不同的时间复杂度和适用场景。

\subsubsection{比较排序}

\begin{enumerate}
  \item \textbf{冒泡排序(Bubble Sort)}
    \begin{itemize}
      \item 时间复杂度:O(n²)
      \item 空间复杂度:O(1)
      \item 稳定排序
      \item 实现简单,但效率低
    \end{itemize}

  \item \textbf{选择排序(Selection Sort)}
    \begin{itemize}
      \item 时间复杂度:O(n²)
      \item 空间复杂度:O(1)
      \item 不稳定排序
      \item 每次选择最小元素放到前面
    \end{itemize}

  \item \textbf{插入排序(Insertion Sort)}
    \begin{itemize}
      \item 时间复杂度:O(n²),最好情况 O(n)
      \item 空间复杂度:O(1)
      \item 稳定排序
      \item 适合小规模数据或基本有序的数据
    \end{itemize}

  \item \textbf{归并排序(Merge Sort)}
    \begin{itemize}
      \item 时间复杂度:O(n log n)
      \item 空间复杂度:O(n)
      \item 稳定排序
      \item 分治思想,适合链表排序
    \end{itemize}

  \item \textbf{快速排序(Quick Sort)}
    \begin{itemize}
      \item 平均时间复杂度:O(n log n)
      \item 最坏时间复杂度:O(n²)
      \item 空间复杂度:O(log n)(递归栈)
      \item 不稳定排序
      \item 实际应用中性能优秀
    \end{itemize}

  \item \textbf{堆排序(Heap Sort)}
    \begin{itemize}
      \item 时间复杂度:O(n log n)
      \item 空间复杂度:O(1)
      \item 不稳定排序
      \item 不需要额外空间,适合内存受限场景
    \end{itemize}
\end{enumerate}

\begin{lstlisting}[language=C, caption={快速排序实现}]
int partition(int arr[], int low, int high) {
    int pivot = arr[high];
    int i = low - 1;

    for (int j = low; j < high; j++) {
        if (arr[j] < pivot) {
            i++;
            int temp = arr[i];
            arr[i] = arr[j];
            arr[j] = temp;
        }
    }

    int temp = arr[i + 1];
    arr[i + 1] = arr[high];
    arr[high] = temp;
    return i + 1;
}

void quicksort(int arr[], int low, int high) {
    if (low < high) {
        int pi = partition(arr, low, high);
        quicksort(arr, low, pi - 1);
        quicksort(arr, pi + 1, high);
    }
}
\end{lstlisting}

\begin{lstlisting}[language=C, caption={归并排序实现}]
void merge(int arr[], int left, int mid, int right) {
    int n1 = mid - left + 1;
    int n2 = right - mid;

    int *L = malloc(n1 * sizeof(int));
    int *R = malloc(n2 * sizeof(int));

    for (int i = 0; i < n1; i++) {
        L[i] = arr[left + i];
    }
    for (int j = 0; j < n2; j++) {
        R[j] = arr[mid + 1 + j];
    }

    int i = 0, j = 0, k = left;
    while (i < n1 && j < n2) {
        if (L[i] <= R[j]) {
            arr[k++] = L[i++];
        } else {
            arr[k++] = R[j++];
        }
    }

    while (i < n1) arr[k++] = L[i++];
    while (j < n2) arr[k++] = R[j++];

    free(L);
    free(R);
}

void mergesort(int arr[], int left, int right) {
    if (left < right) {
        int mid = left + (right - left) / 2;
        mergesort(arr, left, mid);
        mergesort(arr, mid + 1, right);
        merge(arr, left, mid, right);
    }
}
\end{lstlisting}

\subsubsection{非比较排序}

\begin{enumerate}
  \item \textbf{计数排序(Counting Sort)}
    \begin{itemize}
      \item 时间复杂度:O(n + k),k 为数据范围
      \item 空间复杂度:O(k)
      \item 稳定排序
      \item 适用于数据范围较小的整数排序
    \end{itemize}

  \item \textbf{基数排序(Radix Sort)}
    \begin{itemize}
      \item 时间复杂度:O(d(n + k)),d 为位数,k 为基数
      \item 空间复杂度:O(n + k)
      \item 稳定排序
      \item 适用于整数或字符串排序
    \end{itemize}

  \item \textbf{桶排序(Bucket Sort)}
    \begin{itemize}
      \item 平均时间复杂度:O(n + k)
      \item 空间复杂度:O(n + k)
      \item 稳定排序
      \item 适用于数据分布均匀的情况
    \end{itemize}
\end{enumerate}

\subsection{搜索算法}

\subsubsection{线性搜索}

线性搜索(Linear Search)是最简单的搜索算法,逐个检查每个元素。

\begin{itemize}
  \item 时间复杂度:O(n)
  \item 空间复杂度:O(1)
  \item 适用于无序数组或链表
\end{itemize}

\subsubsection{二分搜索}

二分搜索(Binary Search)要求数据有序,通过不断缩小搜索范围来查找目标。

\begin{itemize}
  \item 时间复杂度:O(log n)
  \item 空间复杂度:O(1)(迭代)或 O(log n)(递归)
  \item 要求数据有序
  \item 适用于有序数组
\end{itemize}

\begin{lstlisting}[language=C, caption={二分搜索实现}]
int binary_search(int arr[], int size, int target) {
    int left = 0, right = size - 1;

    while (left <= right) {
        int mid = left + (right - left) / 2;

        if (arr[mid] == target) {
            return mid;
        } else if (arr[mid] < target) {
            left = mid + 1;
        } else {
            right = mid - 1;
        }
    }

    return -1; // 未找到
}

// 查找第一个大于等于 target 的位置
int lower_bound(int arr[], int size, int target) {
    int left = 0, right = size;

    while (left < right) {
        int mid = left + (right - left) / 2;
        if (arr[mid] < target) {
            left = mid + 1;
        } else {
            right = mid;
        }
    }

    return left;
}
\end{lstlisting}

\subsection{递归}

\BlockDesc{递归}

递归(Recursion)是一种函数调用自身的编程技巧。
递归算法通常更简洁,但需要注意栈溢出和重复计算的问题。

递归的要素:
\begin{enumerate}
  \item \textbf{基准情况(Base Case)}
    \begin{itemize}
      \item 递归终止的条件
      \item 避免无限递归
    \end{itemize}

  \item \textbf{递归关系(Recurrence Relation)}
    \begin{itemize}
      \item 将问题分解为更小的子问题
      \item 子问题的解组合成原问题的解
    \end{itemize}
\end{enumerate}

递归的优缺点:
\begin{itemize}
  \item 优点:代码简洁、思路清晰、符合问题本质
  \item 缺点:可能栈溢出、重复计算、性能开销
\end{itemize}

优化技巧:
\begin{enumerate}
  \item \textbf{尾递归优化}
    \begin{itemize}
      \item 递归调用是函数的最后一个操作
      \item 编译器可以优化为迭代
    \end{itemize}

  \item \textbf{记忆化(Memoization)}
    \begin{itemize}
      \item 缓存已计算的结果
      \item 避免重复计算
      \item 例如:斐波那契数列的动态规划解法
    \end{itemize}

  \item \textbf{迭代转换}
    \begin{itemize}
      \item 将递归算法转换为迭代算法
      \item 使用栈模拟递归调用
    \end{itemize}
\end{enumerate}

\begin{lstlisting}[language=C, caption={递归示例:斐波那契数列}]
// 朴素递归,时间复杂度 O(2^n)
int fibonacci_naive(int n) {
    if (n <= 1) return n;
    return fibonacci_naive(n - 1) + fibonacci_naive(n - 2);
}

// 记忆化递归,时间复杂度 O(n)
int fibonacci_memo(int n, int *memo) {
    if (n <= 1) return n;
    if (memo[n] != -1) return memo[n];
    memo[n] = fibonacci_memo(n - 1, memo) + fibonacci_memo(n - 2, memo);
    return memo[n];
}

// 迭代版本,时间复杂度 O(n),空间复杂度 O(1)
int fibonacci_iterative(int n) {
    if (n <= 1) return n;
    int a = 0, b = 1;
    for (int i = 2; i <= n; i++) {
        int temp = a + b;
        a = b;
        b = temp;
    }
    return b;
}
\end{lstlisting}

\subsection{分治}

\BlockDesc{分治}

分治(Divide and Conquer)是一种重要的算法设计思想,将问题分解为多个子问题,递归求解,然后合并结果。

分治算法的步骤:
\begin{enumerate}
  \item \textbf{分解(Divide)}:将问题分解为规模更小的子问题
  \item \textbf{解决(Conquer)}:递归求解子问题
  \item \textbf{合并(Combine)}:将子问题的解合并为原问题的解
\end{enumerate}

经典应用:
\begin{enumerate}
  \item \textbf{归并排序}
    \begin{itemize}
      \item 分解:将数组分为两半
      \item 解决:递归排序两半
      \item 合并:合并两个有序数组
    \end{itemize}

  \item \textbf{快速排序}
    \begin{itemize}
      \item 分解:选择基准,分割数组
      \item 解决:递归排序两部分
      \item 合并:无需合并(原地排序)
    \end{itemize}

  \item \textbf{二分搜索}
    \begin{itemize}
      \item 分解:比较中间元素
      \item 解决:在左半或右半递归搜索
      \item 合并:直接返回结果
    \end{itemize}

  \item \textbf{大整数乘法}
    \begin{itemize}
      \item 使用 Karatsuba 算法
      \item 时间复杂度:$O(n^log₂³) ≈ O(n^1.585)$
    \end{itemize}

  \item \textbf{矩阵乘法}
    \begin{itemize}
      \item Strassen 算法
      \item 时间复杂度:$O(n^log₂⁷) ≈ O(n^2.807)$
    \end{itemize}

  \item \textbf{最近点对问题}
    \begin{itemize}
      \item 分解:将点集分为两半
      \item 解决:递归求解两半的最近点对
      \item 合并:考虑跨分割线的点对
    \end{itemize}
\end{enumerate}

分治算法的时间复杂度通常可以用主定理(Master Theorem)分析:

对于递归式 $T(n) = aT(n/b) + f(n)$,其中 $a \geq 1$,$b > 1$:
\begin{itemize}
  \item 如果 $f(n) = O(n^{\log_b a - \epsilon})$,则 $T(n) = \Theta(n^{\log_b a})$
  \item 如果 $f(n) = \Theta(n^{\log_b a})$,则 $T(n) = \Theta(n^{\log_b a} \log n)$
  \item 如果 $f(n) = \Omega(n^{\log_b a + \epsilon})$,则 $T(n) = \Theta(f(n))$
\end{itemize}

\subsection{贪心算法}

\BlockDesc{贪心算法}

贪心算法(Greedy Algorithm)在每一步都做出当前看起来最优的选择,希望最终得到全局最优解。

特点:
\begin{itemize}
  \item 局部最优选择
  \item 不回溯,一旦做出选择就不再改变
  \item 效率高,但可能不是全局最优
\end{itemize}

适用条件:
\begin{enumerate}
  \item \textbf{贪心选择性质}
    \begin{itemize}
      \item 可以通过局部最优选择达到全局最优
    \end{itemize}

  \item \textbf{最优子结构}
    \begin{itemize}
      \item 问题的最优解包含子问题的最优解
    \end{itemize}
\end{enumerate}

经典应用:
\begin{enumerate}
  \item \textbf{活动选择问题}
    \begin{itemize}
      \item 选择最多的互不重叠的活动
      \item 贪心策略:每次选择结束时间最早的活动
    \end{itemize}

  \item \textbf{霍夫曼编码}
    \begin{itemize}
      \item 构造最优前缀编码
      \item 贪心策略:每次合并频率最低的两个节点
    \end{itemize}

  \item \textbf{最小生成树}
    \begin{itemize}
      \item Kruskal 算法:按边权重排序,依次加入不形成环的边
      \item Prim 算法:从任意顶点开始,每次加入距离当前树最近的顶点
    \end{itemize}

  \item \textbf{最短路径(Dijkstra 算法)}
    \begin{itemize}
      \item 单源最短路径问题(非负权重)
      \item 贪心策略:每次选择距离源点最近的未访问顶点
    \end{itemize}

  \item \textbf{区间调度问题}
    \begin{itemize}
      \item 选择最多的不重叠区间
      \item 贪心策略:按结束时间排序,依次选择
    \end{itemize}

  \item \textbf{找零问题}
    \begin{itemize}
      \item 用最少的硬币找零
      \item 贪心策略:每次选择面额最大的硬币(仅适用于某些硬币系统)
    \end{itemize}
\end{enumerate}

\begin{lstlisting}[language=C, caption={Dijkstra 最短路径算法}]
#define INF 1000000
#define MAX_V 1000

void dijkstra(Graph *g, int start, int *dist) {
    int *visited = calloc(g->num_vertices, sizeof(int));

    // 初始化距离
    for (int i = 0; i < g->num_vertices; i++) {
        dist[i] = INF;
    }
    dist[start] = 0;

    for (int count = 0; count < g->num_vertices - 1; count++) {
        // 找到未访问的距离最小的顶点
        int u = -1;
        int min_dist = INF;
        for (int v = 0; v < g->num_vertices; v++) {
            if (!visited[v] && dist[v] < min_dist) {
                min_dist = dist[v];
                u = v;
            }
        }

        if (u == -1) break;
        visited[u] = 1;

        // 更新邻接顶点的距离
        EdgeNode *edge = g->adj_list[u];
        while (edge != NULL) {
            int v = edge->vertex;
            if (!visited[v] && dist[u] + edge->weight < dist[v]) {
                dist[v] = dist[u] + edge->weight;
            }
            edge = edge->next;
        }
    }

    free(visited);
}
\end{lstlisting}

\subsection{动态规划}

\BlockDesc{动态规划}

动态规划(Dynamic Programming, DP)通过将问题分解为重叠子问题,并存储子问题的解来避免重复计算。

特点:
\begin{itemize}
  \item 最优子结构:问题的最优解包含子问题的最优解
  \item 重叠子问题:递归过程中会重复计算相同的子问题
  \item 记忆化:存储已计算的子问题的解
\end{itemize}

实现方式:
\begin{enumerate}
  \item \textbf{自顶向下(Top-Down)}
    \begin{itemize}
      \item 递归 + 记忆化
      \item 代码更直观
      \item 可能栈溢出
    \end{itemize}

  \item \textbf{自底向上(Bottom-Up)}
    \begin{itemize}
      \item 迭代,从基础情况开始构建
      \item 空间效率更高
      \item 通常更快
    \end{itemize}
\end{enumerate}

经典问题:
\begin{enumerate}
  \item \textbf{斐波那契数列}
    \begin{itemize}
      \item 状态转移方程:$F(n) = F(n-1) + F(n-2)$
      \item 时间复杂度:O(n)
    \end{itemize}

  \item \textbf{最长公共子序列(LCS)}
    \begin{itemize}
      \item 状态转移方程:
      \begin{align*}
        dp[i][j] = \begin{cases}
          dp[i-1][j-1] + 1 & \text{if } s1[i] == s2[j] \\
          \max(dp[i-1][j], dp[i][j-1]) & \text{otherwise}
        \end{cases}
      \end{align*}
      \item 时间复杂度:O(mn)
    \end{itemize}

  \item \textbf{最长递增子序列(LIS)}
    \begin{itemize}
      \item 状态转移方程:$dp[i] = \max(dp[j]) + 1$,其中 $j < i$ 且 $arr[j] < arr[i]$
      \item 时间复杂度:O(n²),可用二分优化到 O(n log n)
    \end{itemize}

  \item \textbf{0-1 背包问题}
    \begin{itemize}
      \item 状态转移方程:
      \begin{align*}
        dp[i][w] = \max(dp[i-1][w], dp[i-1][w-weight[i]] + value[i])
      \end{align*}
      \item 时间复杂度:O(nW)
    \end{itemize}

  \item \textbf{编辑距离}
    \begin{itemize}
      \item 状态转移方程:
      \begin{align*}
        dp[i][j] = \min\begin{cases}
          dp[i-1][j] + 1 & \text{删除} \\
          dp[i][j-1] + 1 & \text{插入} \\
          dp[i-1][j-1] + cost & \text{替换}
        \end{cases}
      \end{align*}
      \item 时间复杂度:O(mn)
    \end{itemize}

  \item \textbf{矩阵链乘法}
    \begin{itemize}
      \item 状态转移方程:$dp[i][j] = \min(dp[i][k] + dp[k+1][j] + p[i-1]p[k]p[j])$
      \item 时间复杂度:O(n³)
    \end{itemize}
\end{enumerate}

\begin{lstlisting}[language=C, caption={0-1 背包问题动态规划解法}]
int knapsack_01(int weights[], int values[], int n, int capacity) {
    int **dp = malloc((n + 1) * sizeof(int*));
    for (int i = 0; i <= n; i++) {
        dp[i] = calloc(capacity + 1, sizeof(int));
    }

    for (int i = 1; i <= n; i++) {
        for (int w = 1; w <= capacity; w++) {
            if (weights[i-1] <= w) {
                dp[i][w] = (dp[i-1][w] >
                           dp[i-1][w-weights[i-1]] + values[i-1]) ?
                           dp[i-1][w] :
                           dp[i-1][w-weights[i-1]] + values[i-1];
            } else {
                dp[i][w] = dp[i-1][w];
            }
        }
    }

    int result = dp[n][capacity];

    // 释放内存
    for (int i = 0; i <= n; i++) {
        free(dp[i]);
    }
    free(dp);

    return result;
}

// 空间优化版本(一维数组)
int knapsack_01_optimized(int weights[], int values[], int n, int capacity) {
    int *dp = calloc(capacity + 1, sizeof(int));

    for (int i = 0; i < n; i++) {
        for (int w = capacity; w >= weights[i]; w--) {
            if (dp[w] < dp[w - weights[i]] + values[i]) {
                dp[w] = dp[w - weights[i]] + values[i];
            }
        }
    }

    int result = dp[capacity];
    free(dp);
    return result;
}
\end{lstlisting}

\subsection{回溯算法}

\BlockDesc{回溯算法}

回溯算法(Backtracking)通过尝试所有可能的解,并在发现当前路径不可能得到解时回退,寻找问题的解。

特点:
\begin{itemize}
  \item 系统性搜索:尝试所有可能的解
  \item 剪枝:提前终止不可能得到解的分支
  \item 递归实现:使用递归进行深度优先搜索
\end{itemize}

应用场景:
\begin{enumerate}
  \item \textbf{N 皇后问题}
    \begin{itemize}
      \item 在 N×N 棋盘上放置 N 个皇后,使得它们互不攻击
      \item 回溯:尝试每一行的每个位置,检查冲突
    \end{itemize}

  \item \textbf{数独求解}
    \begin{itemize}
      \item 填充数独棋盘
      \item 回溯:尝试每个空格的每个数字,检查有效性
    \end{itemize}

  \item \textbf{全排列}
    \begin{itemize}
      \item 生成所有可能的排列
      \item 回溯:交换元素,递归,然后恢复
    \end{itemize}

  \item \textbf{子集生成}
    \begin{itemize}
      \item 生成集合的所有子集
      \item 回溯:对每个元素选择包含或不包含
    \end{itemize}

  \item \textbf{组合问题}
    \begin{itemize}
      \item 从 n 个元素中选择 k 个
      \item 回溯:选择当前元素,递归,然后取消选择
    \end{itemize}

  \item \textbf{路径查找}
    \begin{itemize}
      \item 迷宫求解
      \item 回溯:尝试每个方向,标记已访问,递归,然后取消标记
    \end{itemize}
\end{enumerate}

\begin{lstlisting}[language=C, caption={N 皇后问题回溯解法}]
int is_safe(int board[], int row, int col, int n) {
    for (int i = 0; i < row; i++) {
        // 检查同一列
        if (board[i] == col) return 0;
        // 检查对角线
        if (abs(board[i] - col) == abs(i - row)) return 0;
    }
    return 1;
}

int solve_n_queens(int board[], int row, int n) {
    if (row == n) {
        // 找到解,打印或计数
        return 1;
    }

    int count = 0;
    for (int col = 0; col < n; col++) {
        if (is_safe(board, row, col, n)) {
            board[row] = col;  // 放置皇后
            count += solve_n_queens(board, row + 1, n);
            // 回溯:不需要显式恢复,因为会被覆盖
        }
    }
    return count;
}
\end{lstlisting}

\subsection{字符串算法}

字符串处理是编程中的常见任务,本节介绍一些重要的字符串算法。

\subsubsection{字符串匹配}

\begin{enumerate}
  \item \textbf{朴素匹配(Naive)}
    \begin{itemize}
      \item 逐个比较模式串和文本串
      \item 时间复杂度:O(mn),m 为模式串长度,n 为文本串长度
    \end{itemize}

  \item \textbf{KMP 算法}
    \begin{itemize}
      \item 利用已匹配信息,避免重复比较
      \item 时间复杂度:O(m + n)
      \item 核心:构建部分匹配表(前缀函数)
    \end{itemize}

  \item \textbf{Boyer-Moore 算法}
    \begin{itemize}
      \item 从右到左匹配,利用坏字符和好后缀规则
      \item 平均时间复杂度:O(n/m),最坏 O(mn)
      \item 实际应用中非常高效
    \end{itemize}

  \item \textbf{Rabin-Karp 算法}
    \begin{itemize}
      \item 使用哈希函数比较字符串
      \item 平均时间复杂度:O(m + n)
      \item 适合多模式匹配
    \end{itemize}
\end{enumerate}

\begin{lstlisting}[language=C, caption={KMP 算法实现}]
void build_prefix_table(char *pattern, int *prefix, int m) {
    prefix[0] = 0;
    int len = 0;
    int i = 1;

    while (i < m) {
        if (pattern[i] == pattern[len]) {
            len++;
            prefix[i] = len;
            i++;
        } else {
            if (len != 0) {
                len = prefix[len - 1];
            } else {
                prefix[i] = 0;
                i++;
            }
        }
    }
}

int kmp_search(char *text, char *pattern) {
    int n = strlen(text);
    int m = strlen(pattern);

    int *prefix = malloc(m * sizeof(int));
    build_prefix_table(pattern, prefix, m);

    int i = 0, j = 0;
    while (i < n) {
        if (text[i] == pattern[j]) {
            i++;
            j++;
        }

        if (j == m) {
            free(prefix);
            return i - j;  // 找到匹配
        } else if (i < n && text[i] != pattern[j]) {
            if (j != 0) {
                j = prefix[j - 1];
            } else {
                i++;
            }
        }
    }

    free(prefix);
    return -1;  // 未找到
}
\end{lstlisting}

\subsubsection{其他字符串算法}

\begin{enumerate}
  \item \textbf{最长公共子串}
    \begin{itemize}
      \item 使用动态规划或后缀数组
      \item 时间复杂度:O(mn) 或 O(m + n)
    \end{itemize}

  \item \textbf{最长回文子串}
    \begin{itemize}
      \item Manacher 算法:O(n)
      \item 动态规划:O(n²)
    \end{itemize}

  \item \textbf{字符串编辑距离}
    \begin{itemize}
      \item Levenshtein 距离
      \item 动态规划:O(mn)
    \end{itemize}
\end{enumerate}

\subsection{图算法进阶}

\subsubsection{最短路径算法}

\begin{enumerate}
  \item \textbf{Dijkstra 算法}
    \begin{itemize}
      \item 单源最短路径(非负权重)
      \item 时间复杂度:O((V + E) log V)(使用优先队列)
      \item 贪心算法
    \end{itemize}

  \item \textbf{Bellman-Ford 算法}
    \begin{itemize}
      \item 单源最短路径(允许负权重,检测负环)
      \item 时间复杂度:O(VE)
      \item 动态规划思想
    \end{itemize}

  \item \textbf{Floyd-Warshall 算法}
    \begin{itemize}
      \item 所有顶点对之间的最短路径
      \item 时间复杂度:O(V³)
      \item 动态规划思想
    \end{itemize}

  \item \textbf{A* 算法}
    \begin{itemize}
      \item 启发式搜索,用于路径查找
      \item 使用启发函数估计到目标的距离
      \item 广泛应用于游戏开发和路径规划
    \end{itemize}
\end{enumerate}

\subsubsection{最小生成树算法}

\begin{enumerate}
  \item \textbf{Kruskal 算法}
    \begin{itemize}
      \item 按边权重排序,依次加入不形成环的边
      \item 使用并查集检测环
      \item 时间复杂度:O(E log E)
    \end{itemize}

  \item \textbf{Prim 算法}
    \begin{itemize}
      \item 从任意顶点开始,每次加入距离当前树最近的顶点
      \item 使用优先队列
      \item 时间复杂度:O((V + E) log V)
    \end{itemize}
\end{enumerate}

\subsubsection{拓扑排序}

拓扑排序(Topological Sort)对有向无环图(DAG)的顶点进行排序,使得对于每条边 (u, v),u 在排序中出现在 v 之前。

应用:
\begin{itemize}
  \item 任务调度
  \item 编译顺序确定
  \item 依赖关系解析
\end{itemize}

实现方法:
\begin{enumerate}
  \item \textbf{DFS 方法}
    \begin{itemize}
      \item 深度优先搜索,记录完成时间
      \item 按完成时间逆序排列
      \item 时间复杂度:O(V + E)
    \end{itemize}

  \item \textbf{Kahn 算法}
    \begin{itemize}
      \item 使用入度数组
      \item 每次选择入度为 0 的顶点
      \item 时间复杂度:O(V + E)
    \end{itemize}
\end{enumerate}

\subsection{高级数据结构}

\subsubsection{并查集}

并查集(Union-Find)是一种树型数据结构,用于处理一些不交集的合并及查询问题。

操作:
\begin{itemize}
  \item \texttt{find}:查找元素所属的集合
  \item \texttt{union}:合并两个集合
  \item 时间复杂度:接近 O(1)(使用路径压缩和按秩合并)
\end{itemize}

应用:
\begin{itemize}
  \item 连通性检测
  \item Kruskal 最小生成树算法
  \item 等价类划分
\end{itemize}

\begin{lstlisting}[language=C, caption={并查集实现}]
typedef struct {
    int *parent;
    int *rank;
    int n;
} UnionFind;

UnionFind* uf_create(int n) {
    UnionFind *uf = malloc(sizeof(UnionFind));
    uf->parent = malloc(n * sizeof(int));
    uf->rank = calloc(n, sizeof(int));
    uf->n = n;

    for (int i = 0; i < n; i++) {
        uf->parent[i] = i;
    }
    return uf;
}

int uf_find(UnionFind *uf, int x) {
    if (uf->parent[x] != x) {
        uf->parent[x] = uf_find(uf, uf->parent[x]);  // 路径压缩
    }
    return uf->parent[x];
}

void uf_union(UnionFind *uf, int x, int y) {
    int root_x = uf_find(uf, x);
    int root_y = uf_find(uf, y);

    if (root_x == root_y) return;

    // 按秩合并
    if (uf->rank[root_x] < uf->rank[root_y]) {
        uf->parent[root_x] = root_y;
    } else if (uf->rank[root_x] > uf->rank[root_y]) {
        uf->parent[root_y] = root_x;
    } else {
        uf->parent[root_y] = root_x;
        uf->rank[root_x]++;
    }
}
\end{lstlisting}

\subsubsection{线段树}

线段树(Segment Tree)是一种二叉树,用于处理区间查询和更新操作。

特点:
\begin{itemize}
  \item 支持区间查询:O(log n)
  \item 支持单点更新:O(log n)
  \item 支持区间更新:O(log n)(使用懒标记)
  \item 空间复杂度:O(n)
\end{itemize}

应用:
\begin{itemize}
  \item 区间最值查询
  \item 区间和查询
  \item 区间更新
  \item 逆序对统计
\end{itemize}

\subsubsection{树状数组}

树状数组(Binary Indexed Tree, BIT)是一种支持单点更新和前缀和查询的数据结构。

特点:
\begin{itemize}
  \item 单点更新:O(log n)
  \item 前缀和查询:O(log n)
  \item 空间复杂度:O(n)
  \item 实现简单,常数小
\end{itemize}

应用:
\begin{itemize}
  \item 动态前缀和
  \item 逆序对统计
  \item 区间更新、单点查询(差分数组)
\end{itemize}

\section{算法设计技巧总结}

\subsection{问题分析}

在解决算法问题时,应该:
\begin{enumerate}
  \item \textbf{理解问题}
    \begin{itemize}
      \item 明确输入输出
      \item 理解约束条件
      \item 识别边界情况
    \end{itemize}

  \item \textbf{分析复杂度要求}
    \begin{itemize}
      \item 数据规模
      \item 时间限制
      \item 空间限制
    \end{itemize}

  \item \textbf{选择合适的数据结构}
    \begin{itemize}
      \item 根据操作特点选择
      \item 考虑时间和空间权衡
    \end{itemize}

  \item \textbf{设计算法}
    \begin{itemize}
      \item 识别问题类型
      \item 选择合适的算法思想
      \item 考虑优化空间
    \end{itemize}
\end{enumerate}

\subsection{常见问题模式}

\begin{enumerate}
  \item \textbf{滑动窗口}
    \begin{itemize}
      \item 维护一个窗口,在数组上滑动
      \item 用于子数组、子串问题
      \item 时间复杂度:O(n)
    \end{itemize}

  \item \textbf{双指针}
    \begin{itemize}
      \item 使用两个指针遍历数组
      \item 用于有序数组、链表问题
      \item 时间复杂度:O(n)
    \end{itemize}

  \item \textbf{前缀和}
    \begin{itemize}
      \item 预处理前缀和数组
      \item 快速计算区间和
      \item 时间复杂度:O(1) 查询
    \end{itemize}

  \item \textbf{差分数组}
    \begin{itemize}
      \item 用于区间更新
      \item 单点查询
      \item 时间复杂度:O(1) 更新
    \end{itemize}

  \item \textbf{单调栈/队列}
    \begin{itemize}
      \item 维护单调性
      \item 用于寻找下一个更大/更小元素
      \item 时间复杂度:O(n)
    \end{itemize}
\end{enumerate}

\subsection{优化技巧}

\begin{enumerate}
  \item \textbf{空间换时间}
    \begin{itemize}
      \item 使用哈希表加速查找
      \item 预处理数据
      \item 缓存计算结果
    \end{itemize}

  \item \textbf{时间换空间}
    \begin{itemize}
      \item 使用位运算压缩空间
      \item 原地算法
      \item 延迟计算
    \end{itemize}

  \item \textbf{算法优化}
    \begin{itemize}
      \item 减少不必要的计算
      \item 使用更高效的数据结构
      \item 利用问题特性
    \end{itemize}

  \item \textbf{实现优化}
    \begin{itemize}
      \item 减少函数调用开销
      \item 优化内存访问模式
      \item 使用位运算
      \item 编译器优化选项
    \end{itemize}
\end{enumerate}

数据结构与算法是编程的基础,掌握这些知识对于编写高效、可靠的嵌入式软件至关重要。
在实际开发中,需要根据具体场景选择合适的数据结构和算法,在时间、空间和代码复杂度之间做出权衡。
持续学习和实践是提高算法能力的关键。
