\section{Linux 内核}

\subsection{启动}

关于 Linux 的启动分析(ARMv8 架构),我们专门在另外一个文档撰写。
所以在此不再浪费篇幅。

\subsection{内存管理}

\subsection{task}

\subsection{网络}

\subsection{内核开发}

\subsection{驱动架构及子系统}
\subsubsection{UART}

\subsubsection{Timer}

\subsubsection{WDT}

\subsubsection{DMA}

\paragraph{DMA 引擎控制器}

如果您要构建一个 DMA 控制器驱动,那么需要查阅这个文档
\url{https://www.kernel.org/doc/Documentation/dmaengine/provider.txt}。

\paragraph{DMA 客户端和接口使用}

如果您需要在设备驱动中使用 DMA,那么以下文档是您必须要理解的。

\begin{itemize}
  \item \href{https://www.kernel.org/doc/Documentation/DMA-API.txt}{DMA 映射接口介绍}
  \item \href{https://www.kernel.org/doc/Documentation/DMA-API-HOWTO.txt}{DMA 动态映射接口指南}
  \item \href{https://www.kernel.org/doc/Documentation/dmaengine/client.txt}{DMA 引擎 API 指南}
\end{itemize}

如果对上述的文档介绍比较迷糊,那么可以参考内核驱动代码的实际例子。
比如,可以参考 pl011 串口驱动中对 DMA 的使用(drivers/tty/serial/amba-pl011.c)。

根据 DMA 映射接口相关文档的介绍,一般 DMA 映射有:

\begin{enumerate}
  \item 使用大型 DMA-coherent 缓冲区。
    \begin{lstcode}
      void *
      dma_alloc_coherent(struct device *dev, size_t size,
          dma_addr_t *dma_handle, gfp_t flag);
      void
      dma_free_coherent(struct device *dev, size_t size, void *cpu_addr,
          dma_addr_t dma_handle);
    \end{lstcode}
  \item 使用小型 DMA-coherent 缓冲区。
    \begin{lstcode}
      struct dma_pool *
      dma_pool_create(const char *name, struct device *dev,
          size_t size, size_t align, size_t alloc);
      void *
      dma_pool_zalloc(struct dma_pool *pool, gfp_t mem_flags,
          dma_addr_t *handle);
      void *
      dma_pool_alloc(struct dma_pool *pool, gfp_t gfp_flags,
          dma_addr_t *dma_handle);
      void
      dma_pool_free(struct dma_pool *pool, void *vaddr,
          dma_addr_t addr);
      void
      dma_pool_destroy(struct dma_pool *pool);
    \end{lstcode}
  \item DMA 地址限制
    \begin{lstcode}
      int
      dma_set_mask_and_coherent(struct device *dev, u64 mask);
      int
      dma_set_mask(struct device *dev, u64 mask);
      int
      dma_set_coherent_mask(struct device *dev, u64 mask);
      u64
      dma_get_required_mask(struct device *dev);
      size_t
      dma_max_mapping_size(struct device *dev);
      unsigned long
      dma_get_merge_boundary(struct device *dev);
    \end{lstcode}
  \item 流式 DMA 映射
    \begin{lstcode}
      dma_addr_t
      dma_map_single(struct device *dev, void *cpu_addr, size_t size,
          enum dma_data_direction direction)
    \end{lstcode}
    以下是 direction 枚举类型:
    \begin{stblr}
      {direction 枚举类型说明}
      {dir-enum}
      {ll}
      \hline[1pt]
      方向 & 说明 \\
      \hline
      DMA\_NONE            & no direction (used for debugging) \\
      DMA\_TO\_DEVICE      & data is going from the memory to the device \\
      DMA\_FROM\_DEVICE    & data is coming from the device to the memory \\
      DMA\_BIDIRECTIONAL  & direction isn't known \\
      \hline[1pt]
    \end{stblr}
\end{enumerate}

根据 DMA 引擎 API 指南中的接口介绍,使用 DMA 传输功能我们需要完成以下步骤:

\begin{enumerate}
  \item 申请一个 DMA slave channel。
    使用到的接口为:

    \lstinline!struct dma_chan *dma_request_chan(struct device *dev, const char *name);!

  \item 向控制器和 slave 设置特定参数。
    接口为:

    \lstinline!int dmaengine_slave_config(struct dma_chan *chan, struct dma_slave_config *config);!

  \item 获取一个事务描述符。
    由于有三种不同的传输模式,所以有以下三种接口:
    \begin{lstlisting}
      struct dma_async_tx_descriptor *dmaengine_prep_slave_sg(
        struct dma_chan *chan, struct scatterlist *sgl,
        unsigned int sg_len, enum dma_data_direction direction,
        unsigned long flags);

      struct dma_async_tx_descriptor *dmaengine_prep_dma_cyclic(
        struct dma_chan *chan, dma_addr_t buf_addr, size_t buf_len,
        size_t period_len, enum dma_data_direction direction);

      struct dma_async_tx_descriptor *dmaengine_prep_interleaved_dma(
        struct dma_chan *chan, struct dma_interleaved_template *xt,
        unsigned long flags);
    \end{lstlisting}
  \item 提交事务。
    描述符准备到位后将其提交给 DMA 引擎驱动的等待队列。

    \lstinline!dma_cookie_t dmaengine_submit(struct dma_async_tx_descriptor *desc);!

  \item 发起等待的 DMA 请求并且等待获取回调通知。
    接口为:

    \lstinline!void dma_async_issue_pending(struct dma_chan *chan);!
\end{enumerate}

以上只是简单的提炼,具体细节可以查阅相关文档。

\paragraph{scatterlist}

关于 scatterlist,可以参阅 \href{https://lwn.net/Articles/234617/}{LWN 介绍文章},下面是该文章的翻译。

\begin{quote}

高性能 I/O 通常涉及使用直接内存访问(DMA)操作。
通过 DMA,I/O 设备可以在不干扰 CPU 的情况下直接与主内存交换数据。
在最简单的 DMA 形式中,控制器被赋予一个指向内存区域的指针,给定长度,并指示其进行操作。
处理器随后可以忘记这个操作,直到设备发出工作完成的信号。

然而,这种简单的视图有一个缺点,即它假设要传输的数据在内存中是连续存储的。
当I/O 缓冲区在内核空间时,内核通常可以安排它在物理上是连续的——尽管随着缓冲区大小的增加,这变得越来越困难。
如果缓冲区在用户空间,则保证它分散在物理内存中。
因此,如果 DMA 操作能够处理被分割成多个不同部分的缓冲区,那将是非常理想的。

事实上,对于任何具有合理能力的外围设备,缓冲区都可以这样分割。
对这种缓冲区进行操作的术语是“散 / 聚集 I/O”;
Linux 早已很好地支持散 / 聚集。
Linux 设备驱动程序的 DMA 章节详细介绍了散 / 聚集。
简而言之,驱动程序首先填充一个散列表结构数组,这在 i386 架构上看起来像这样:

\begin{lstlisting}[language=C]
struct scatterlist {
    struct page *page;
    unsigned int    offset;
    dma_addr_t  dma_address;
    unsigned int    length;
};
\end{lstlisting}

对于每个段,page 指针指示该段在内存中的位置,offset 指示数据在页面内的起始位置,length 是段的长度。
列表填充完成后,驱动程序调用:

\begin{lstlisting}[language=C]
int dma_map_sg(struct device *dev, struct scatterlist *sg, int nents,
               enum dma_data_direction direction);
\end{lstlisting}

此操作至少会填充每个散列表条目的 dma\_address 字段,该地址可以提供给外围设备。
然而,它可能做的不止这些:物理上连续的页面可能会合并为单个散列表条目,或者系统的 I/O 内存管理单元可能会被编程,使得从设备的角度看,列表的部分(或全部)在虚拟上是连续的。
所有这些——包括 struct scatterlist 的确切形式——都是架构相关的,但散 / 聚集接口的设置使得驱动程序无需担心架构细节。

最近,散 / 聚集接口中的一个特定限制显现出来。
由于各种原因,散列表必须适应单个页面;
这一限制对可以表示的段数设置了上限。
在启用高内存的 i386 架构上,struct scatterlist 需要 20 字节,这限制了散列表为 204 个条目。
如果每个散列表条目指向一个完整页面,DMA 传输的最大大小将约为 800KB。
在x86 - 64 系统上,情况更糟:结构体使用 40 字节,将最大长度减半。

在某些情况下,需要更大的 I/O 操作。
块I/O 子系统是其中之一,但肯定还有其他情况:高分辨率视频捕获设备就是一个例子。
散列表长度的限制是激励开发人员工作的因素之一,他们致力于支持更大的块大小。
通过增加有效页面大小,他们能够增加最大 I/O 操作大小。

然而,增加页面大小并不是唯一可行的方法;
另一种方法是简单地增加散列表的长度。
多页面连续的散列表实际上并不可行,但可以链式连接单页面散列表。
Jens Axboe 一直在研究这种方法;
截至本文撰写时,他的散列表链式连接补丁已经是第六版。

链式连接通过重载页面指针在页面中的最后一个散列表条目来完成。
设置最低有效位以指示该条目实际上是链条链接,而不是另一个要传输的段。
对驱动程序来说,这个改变几乎是透明的。
在当前内核中,遍历散列表的代码通常看起来像这样:

\begin{lstlisting}[language=C]
struct scatterlist *sg = &the_scatterlist[0];

for (i = 0; i < nentries; i++) {
    program_io_operation(sg);
    sg++;
}
\end{lstlisting}

当使用链式连接时,简单地通过数组递增将不再有效。
所以 Jens 添加了一个简单的 sg\_next() 宏,当需要时跟随链条链接。
因此,上述 sg++ 行变成如下所示:

\begin{lstlisting}[language=C]
sg = sg_next(sg);
\end{lstlisting}

由于需要对驱动程序进行更改,除非确定驱动程序已准备好,否则不应使用链式散列表。
Jens 的补丁修复了许多驱动程序,特别是在块子系统中。
即便如此,最大 I/O 大小必须由管理员通过 sysfs 文件显式提高,然后链式连接才会启用。
一旦启用,多兆字节的 I/O 操作就变得可能。
无需进行侵入性的内存管理更改。
\end{quote}

\subsubsection{I2C/I3C}

\subsubsection{SPI/QSPI}

\subsubsection{CAN}

\subsubsection{Ethernet}

\subsubsection{USB}

\subsubsection{PCI/PCIe}

\subsection{文件系统}

\subsubsection{ext2 简介}
