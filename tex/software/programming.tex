\chapter{编程}

编程是嵌入式系统开发的核心技能,涉及从底层汇编到高级语言的多个层次。
本章系统介绍 ARMv8 和 RISC-V 汇编语言、ELF 程序结构、C 语言编程(硬件层和用户层)、
软件架构思想,以及 Rust 和 C++ 等现代编程语言。

\section{ARMv8 汇编}

ARMv8 架构是 ARM 公司推出的 64 位架构,广泛应用于移动设备、服务器和嵌入式系统。
理解 ARMv8 汇编对于底层系统编程、性能优化和调试至关重要。

\subsection{ARMv8 基础}

\BlockDesc{ARMv8 架构}

ARMv8 架构的主要特点:
\begin{itemize}
  \item 支持 64 位和 32 位执行状态(AArch64 和 AArch32)
  \item 31 个 64 位通用寄存器(X0-X30)
  \item 独立的程序计数器(PC)和栈指针(SP)
  \item 改进的异常模型和虚拟化支持
  \item 增强的 SIMD 和浮点支持
\end{itemize}

\subsubsection{寄存器}

AArch64 寄存器组织:

\begin{enumerate}
  \item \textbf{通用寄存器}
    \begin{itemize}
      \item X0-X30:31 个 64 位通用寄存器
      \item W0-W30:32 位视图(X 寄存器的低 32 位)
      \item X0-X7:参数和返回值寄存器
      \item X8:间接结果位置寄存器
      \item X9-X15:临时寄存器
      \item X16-X17:IP0/IP1(内部过程调用)
      \item X18:平台寄存器
      \item X19-X28:被调用者保存寄存器
      \item X29:帧指针(FP)
      \item X30:链接寄存器(LR)
    \end{itemize}

  \item \textbf{特殊寄存器}
    \begin{itemize}
      \item SP:栈指针(Stack Pointer)
      \item PC:程序计数器(Program Counter)
      \item PSTATE:程序状态寄存器
    \end{itemize}

  \item \textbf{SIMD 和浮点寄存器}
    \begin{itemize}
      \item V0-V31:128 位 SIMD/浮点寄存器
      \item 可以访问为 8 位、16 位、32 位、64 位或 128 位
    \end{itemize}
\end{enumerate}

\subsubsection{指令格式}

ARMv8 指令的基本格式:

\begin{lstlisting}[language={[ARM]Assembler}]
  opcode  dest, src1, src2
\end{lstlisting}

指令特点:
\begin{itemize}
  \item 固定长度:所有指令都是 32 位
  \item 加载-存储架构:只有加载和存储指令访问内存
  \item 三地址格式:大多数指令有两个源操作数和一个目标操作数
\end{itemize}

\subsection{数据传送指令}

\BlockDesc{数据传送}

\begin{enumerate}
  \item \textbf{MOV}
    \begin{itemize}
      \item 寄存器间移动数据
      \item 立即数加载(受限制)
      \item 语法:\texttt{MOV Xd, Xm} 或 \texttt{MOV Xd, \#imm}
    \end{itemize}

  \item \textbf{LDR/STR}
    \begin{itemize}
      \item 加载/存储寄存器
      \item 支持多种寻址模式
      \item 语法:\texttt{LDR Xd, [Xn, \#offset]} 或 \texttt{STR Xd, [Xn, \#offset]}
    \end{itemize}

  \item \textbf{LDP/STP}
    \begin{itemize}
      \item 加载/存储一对寄存器
      \item 用于函数调用约定
      \item 语法:\texttt{LDP Xd1, Xd2, [Xn, \#offset]}
    \end{itemize}
\end{enumerate}

示例代码:
\begin{lstlisting}[language={[ARM]Assembler}]
  /* 加载立即数到寄存器 */
  mov x0, #42

  /* 从内存加载 */
  ldr x1, [x2]           /* x1 = *x2 */
  ldr x1, [x2, #8]       /* x1 = *(x2 + 8) */
  ldr x1, [x2, x3]       /* x1 = *(x2 + x3) */
  ldr x1, [x2, x3, lsl #3] /* x1 = *(x2 + (x3 << 3)) */

  /* 存储到内存 */
  str x0, [x1]           /* *x1 = x0 */
  str x0, [x1, #-8]!     /* *(--x1) = x0 (预索引) */
  str x0, [x1], #8      /* *(x1++) = x0 (后索引) */
\end{lstlisting}

\subsection{算术和逻辑指令}

\BlockDesc{算术运算}

\begin{enumerate}
  \item \textbf{ADD/SUB}
    \begin{itemize}
      \item 加法和减法
      \item 支持扩展操作数(移位、扩展)
      \item 语法:\texttt{ADD Xd, Xn, Xm} 或 \texttt{ADD Xd, Xn, \#imm}
    \end{itemize}

  \item \textbf{AND/ORR/EOR}
    \begin{itemize}
      \item 按位与、或、异或
      \item 语法:\texttt{AND Xd, Xn, Xm}
    \end{itemize}

  \item \textbf{MUL/UMULH/SMULH}
    \begin{itemize}
      \item 乘法(低 64 位)
      \item 无符号/有符号乘法(高 64 位)
    \end{itemize}

  \item \textbf{SDIV/UDIV}
    \begin{itemize}
      \item 有符号/无符号除法
    \end{itemize}
\end{enumerate}

示例代码:
\begin{lstlisting}[language={[ARM]Assembler}]
  /* 加法 */
  add x0, x1, x2         /* x0 = x1 + x2 */
  add x0, x1, #100      /* x0 = x1 + 100 */
  add x0, x1, x2, lsl #2 /* x0 = x1 + (x2 << 2) */

  /* 减法 */
  sub x0, x1, x2        /* x0 = x1 - x2 */

  /* 逻辑运算 */
  and x0, x1, x2        /* x0 = x1 & x2 */
  orr x0, x1, x2        /* x0 = x1 | x2 */
  eor x0, x1, x2        /* x0 = x1 ^ x2 */

  /* 乘法 */
  mul x0, x1, x2        /* x0 = x1 * x2 (低64位) */
\end{lstlisting}

\subsection{控制流指令}

\BlockDesc{分支和跳转}

\begin{enumerate}
  \item \textbf{B/BL}
    \begin{itemize}
      \item B:无条件分支
      \item BL:带链接的分支(函数调用)
      \item 语法:\texttt{B label} 或 \texttt{BL label}
    \end{itemize}

  \item \textbf{条件分支}
    \begin{itemize}
      \item B.EQ:相等时分支
      \item B.NE:不等时分支
      \item B.GT:大于时分支
      \item B.LT:小于时分支
      \item 等等
    \end{itemize}

  \item \textbf{BR/BLR/RET}
    \begin{itemize}
      \item BR:寄存器间接分支
      \item BLR:带链接的寄存器间接分支
      \item RET:从函数返回
    \end{itemize}
\end{enumerate}

示例代码:
\begin{lstlisting}[language={[ARM]Assembler}]
  /* 无条件跳转 */
  b label

  /* 条件分支 */
  cmp x0, x1
  b.eq equal          /* if (x0 == x1) goto equal */
  b.gt greater        /* if (x0 > x1) goto greater */

  /* 函数调用 */
  bl function         /* 调用函数,LR 保存返回地址 */

  /* 函数返回 */
  ret                 /* 返回到 LR 指向的地址 */
\end{lstlisting}

\subsection{比较和条件执行}

\BlockDesc{条件码}

条件码标志位(NZCV):
\begin{itemize}
  \item N:负数标志
  \item Z:零标志
  \item C:进位标志
  \item V:溢出标志
\end{itemize}

比较指令:
\begin{lstlisting}[language={[ARM]Assembler}]
  cmp x0, x1          /* 比较 x0 和 x1,设置标志位 */
  cmn x0, x1          /* 比较 x0 和 -x1 */
  tst x0, x1          /* 测试 x0 & x1 */
\end{lstlisting}

条件选择指令(CSEL):
\begin{lstlisting}[language={[ARM]Assembler}]
  csel x0, x1, x2, eq /* if (Z==1) x0=x1 else x0=x2 */
\end{lstlisting}

\subsection{函数调用约定}

\BlockDesc{ARMv8 ABI}

ARMv8 函数调用约定(AAPCS64):

\begin{enumerate}
  \item \textbf{参数传递}
    \begin{itemize}
      \item X0-X7:整数参数
      \item V0-V7:浮点/SIMD 参数
      \item 更多参数通过栈传递
    \end{itemize}

  \item \textbf{返回值}
    \begin{itemize}
      \item X0:整数返回值
      \item V0:浮点返回值
      \item 大结构通过 X8 间接返回
    \end{itemize}

  \item \textbf{被调用者保存寄存器}
    \begin{itemize}
      \item X19-X28:必须由被调用者保存
      \item V8-V15:SIMD/浮点寄存器
    \end{itemize}

  \item \textbf{调用者保存寄存器}
    \begin{itemize}
      \item X0-X7, X9-X15:调用者负责保存
      \item V0-V7:SIMD/浮点寄存器
    \end{itemize}
\end{enumerate}

函数调用示例:
\begin{lstlisting}[language={[ARM]Assembler}]
  /* 函数调用 */
  function:
    /* 保存被调用者保存寄存器 */
    stp x29, x30, [sp, #-16]!  /* 保存 FP 和 LR */
    mov x29, sp                 /* 设置帧指针 */

    /* 函数体 */
    add x0, x0, x1

    /* 恢复并返回 */
    ldp x29, x30, [sp], #16     /* 恢复 FP 和 LR */
    ret
\end{lstlisting}

\section{RISC-V 汇编}

RISC-V 是一个开源的指令集架构(ISA),设计简洁、模块化,适合从嵌入式系统到超级计算机的各种应用。

\subsection{RISC-V 基础}

\BlockDesc{RISC-V 架构}

RISC-V 的主要特点:
\begin{itemize}
  \item 开源:BSD 许可证,可自由使用
  \item 模块化:基础指令集 + 可选扩展
  \item 简洁:指令数量少,易于实现
  \item 可扩展:支持自定义指令
  \item 支持 32 位、64 位和 128 位架构
\end{itemize}

\subsubsection{寄存器}

RISC-V 寄存器组织(RV64):

\begin{enumerate}
  \item \textbf{通用寄存器}
    \begin{itemize}
      \item x0:零寄存器(始终为 0)
      \item x1:返回地址(RA)
      \item x2:栈指针(SP)
      \item x3:全局指针(GP)
      \item x4:线程指针(TP)
      \item x5-x7:临时寄存器(t0-t2)
      \item x8:帧指针(FP/S0)
      \item x9:保存寄存器(S1)
      \item x10-x11:参数/返回值(a0-a1)
      \item x12-x17:参数寄存器(a2-a7)
      \item x18-x27:保存寄存器(s2-s11)
      \item x28-x31:临时寄存器(t3-t6)
    \end{itemize}

  \item \textbf{程序计数器(PC)}
    \begin{itemize}
      \item 独立的 PC 寄存器
      \item 不能直接访问,通过指令间接操作
    \end{itemize}
\end{enumerate}

\subsubsection{指令格式}

RISC-V 指令格式:

\begin{enumerate}
  \item \textbf{R 类型}:寄存器-寄存器操作
    \begin{itemize}
      \item 格式:\texttt{opcode rd, rs1, rs2}
      \item 示例:\texttt{ADD x1, x2, x3}
    \end{itemize}

  \item \textbf{I 类型}:立即数操作
    \begin{itemize}
      \item 格式:\texttt{opcode rd, rs1, imm}
      \item 示例:\texttt{ADDI x1, x2, 100}
    \end{itemize}

  \item \textbf{S 类型}:存储指令
    \begin{itemize}
      \item 格式:\texttt{opcode rs2, imm(rs1)}
      \item 示例:\texttt{SW x1, 8(x2)}
    \end{itemize}

  \item \textbf{B 类型}:条件分支
    \begin{itemize}
      \item 格式:\texttt{opcode rs1, rs2, offset}
      \item 示例:\texttt{BEQ x1, x2, label}
    \end{itemize}

  \item \textbf{U 类型}:高位立即数
    \begin{itemize}
      \item 格式:\texttt{opcode rd, imm}
      \item 示例:\texttt{LUI x1, 0x12345}
    \end{itemize}

  \item \textbf{J 类型}:无条件跳转
    \begin{itemize}
      \item 格式:\texttt{opcode rd, offset}
      \item 示例:\texttt{JAL x1, label}
    \end{itemize}
\end{enumerate}

\subsection{数据传送指令}

\BlockDesc{加载和存储}

\begin{enumerate}
  \item \textbf{加载指令}
    \begin{itemize}
      \item LB:加载字节(有符号扩展)
      \item LH:加载半字(有符号扩展)
      \item LW:加载字(有符号扩展)
      \item LD:加载双字(RV64)
      \item LBU/LHU/LWU:无符号加载
    \end{itemize}

  \item \textbf{存储指令}
    \begin{itemize}
      \item SB:存储字节
      \item SH:存储半字
      \item SW:存储字
      \item SD:存储双字(RV64)
    \end{itemize}
\end{enumerate}

示例代码:
\begin{lstlisting}[language={[RISC-V]Assembler}]
  /* 加载 */
  lb  x1, 0(x2)        /* x1 = *(int8_t*)(x2 + 0) */
  lw  x1, 4(x2)        /* x1 = *(int32_t*)(x2 + 4) */
  ld  x1, 8(x2)        /* x1 = *(int64_t*)(x2 + 8) */

  /* 存储 */
  sb  x1, 0(x2)        /* *(int8_t*)(x2 + 0) = x1 */
  sw  x1, 4(x2)        /* *(int32_t*)(x2 + 4) = x1 */
  sd  x1, 8(x2)        /* *(int64_t*)(x2 + 8) = x1 */
\end{lstlisting}

\subsection{算术和逻辑指令}

\BlockDesc{算术运算}

\begin{enumerate}
  \item \textbf{加法}
    \begin{itemize}
      \item ADD:寄存器加法
      \item ADDI:立即数加法
      \item SUB:减法
    \end{itemize}

  \item \textbf{逻辑运算}
    \begin{itemize}
      \item AND/OR/XOR:按位与/或/异或
      \item ANDI/ORI/XORI:立即数逻辑运算
    \end{itemize}

  \item \textbf{移位}
    \begin{itemize}
      \item SLL/SRL/SRA:逻辑左移/右移、算术右移
      \item SLLI/SRLI/SRAI:立即数移位
    \end{itemize}

  \item \textbf{比较}
    \begin{itemize}
      \item SLT:设置小于(有符号)
      \item SLTU:设置小于(无符号)
      \item SLTI/SLTIU:立即数比较
    \end{itemize}
\end{enumerate}

示例代码:
\begin{lstlisting}[language={[RISC-V]Assembler}]
  /* 算术运算 */
  add  x1, x2, x3      /* x1 = x2 + x3 */
  addi x1, x2, 100     /* x1 = x2 + 100 */
  sub  x1, x2, x3      /* x1 = x2 - x3 */

  /* 逻辑运算 */
  and  x1, x2, x3      /* x1 = x2 & x3 */
  or   x1, x2, x3      /* x1 = x2 | x3 */
  xor  x1, x2, x3      /* x1 = x2 ^ x3 */

  /* 移位 */
  sll  x1, x2, x3      /* x1 = x2 << x3 */
  srl  x1, x2, x3      /* x1 = x2 >> x3 (逻辑) */
  sra  x1, x2, x3      /* x1 = x2 >> x3 (算术) */
\end{lstlisting}

\subsection{控制流指令}

\BlockDesc{分支和跳转}

\begin{enumerate}
  \item \textbf{无条件跳转}
    \begin{itemize}
      \item JAL:跳转并链接(函数调用)
      \item JALR:寄存器间接跳转并链接
    \end{itemize}

  \item \textbf{条件分支}
    \begin{itemize}
      \item BEQ:相等时分支
      \item BNE:不等时分支
      \item BLT:小于时分支(有符号)
      \item BGE:大于等于时分支(有符号)
      \item BLTU:小于时分支(无符号)
      \item BGEU:大于等于时分支(无符号)
    \end{itemize}
\end{enumerate}

示例代码:
\begin{lstlisting}[language={[RISC-V]Assembler}]
  /* 无条件跳转 */
  jal x1, label        /* 跳转到 label,返回地址存入 x1 */

  /* 条件分支 */
  beq x1, x2, label    /* if (x1 == x2) goto label */
  bne x1, x2, label    /* if (x1 != x2) goto label */
  blt x1, x2, label    /* if (x1 < x2) goto label */
  bge x1, x2, label    /* if (x1 >= x2) goto label */

  /* 函数返回 */
  jalr x0, 0(x1)       /* 返回到 x1 指向的地址 */
\end{lstlisting}

\subsection{函数调用约定}

\BlockDesc{RISC-V ABI}

RISC-V 函数调用约定:

\begin{enumerate}
  \item \textbf{参数传递}
    \begin{itemize}
      \item a0-a7(x10-x17):整数参数
      \item fa0-fa7(f10-f17):浮点参数
      \item 更多参数通过栈传递
    \end{itemize}

  \item \textbf{返回值}
    \begin{itemize}
      \item a0(x10):整数返回值
      \item fa0(f10):浮点返回值
    \end{itemize}

  \item \textbf{被调用者保存寄存器}
    \begin{itemize}
      \item s0-s11(x8-x9, x18-x27):必须保存
      \item fs0-fs11(f8-f9, f18-f27):浮点寄存器
    \end{itemize}

  \item \textbf{调用者保存寄存器}
    \begin{itemize}
      \item t0-t6(x5-x7, x28-x31):临时寄存器
      \item ft0-ft11(f0-f7, f28-f31):浮点临时寄存器
    \end{itemize}
\end{enumerate}

函数调用示例:
\begin{lstlisting}[language={[RISC-V]Assembler}]
  /* 函数调用 */
  function:
    /* 保存被调用者保存寄存器 */
    addi sp, sp, -16
    sd   ra, 8(sp)      /* 保存返回地址 */
    sd   s0, 0(sp)      /* 保存帧指针 */
    addi s0, sp, 16     /* 设置帧指针 */

    /* 函数体 */
    add  a0, a0, a1

    /* 恢复并返回 */
    ld   s0, 0(sp)
    ld   ra, 8(sp)
    addi sp, sp, 16
    jalr x0, 0(ra)      /* 返回 */
\end{lstlisting}

\section{ELF 程序结构}

ELF(Executable and Linkable Format)是 Unix/Linux 系统上标准的可执行文件和目标文件格式。
理解 ELF 格式对于程序链接、加载、调试和逆向工程至关重要。

\subsection{ELF 概述}

\BlockDesc{ELF 文件格式}

ELF 文件类型:
\begin{itemize}
  \item \textbf{可重定位文件(Relocatable)}:.o 文件,包含代码和数据,可链接成可执行文件或共享库
  \item \textbf{可执行文件(Executable)}:可直接运行的程序
  \item \textbf{共享目标文件(Shared Object)}:.so 文件,动态链接库
  \item \textbf{核心转储文件(Core Dump)}:程序崩溃时的内存映像
\end{itemize}

ELF 文件结构:
\begin{itemize}
  \item \textbf{ELF 头部}:文件的基本信息
  \item \textbf{程序头表}:可执行文件使用,描述段(Segment)
  \item \textbf{节头表}:描述节(Section)
  \item \textbf{节数据}:实际的代码和数据
\end{itemize}

\subsection{ELF 头部}

\BlockDesc{ELF Header}

ELF 头部结构(64 位):
\begin{lstlisting}[language=C]
typedef struct {
    unsigned char e_ident[16];  // ELF 标识
    uint16_t      e_type;       // 文件类型
    uint16_t      e_machine;    // 目标架构
    uint32_t      e_version;    // 版本
    uint64_t      e_entry;      // 入口点地址
    uint64_t      e_phoff;      // 程序头表偏移
    uint64_t      e_shoff;      // 节头表偏移
    uint32_t      e_flags;      // 处理器特定标志
    uint16_t      e_ehsize;     // ELF 头部大小
    uint16_t      e_phentsize;  // 程序头表项大小
    uint16_t      e_phnum;      // 程序头表项数量
    uint16_t      e_shentsize;  // 节头表项大小
    uint16_t      e_shnum;      // 节头表项数量
    uint16_t      e_shstrndx;   // 节名字符串表索引
} Elf64_Ehdr;
\end{lstlisting}

\lstinline!e_ident! 字段(ELF 魔数):
\begin{itemize}
  \item [0x00-0x03]:\texttt{0x7F, 'E', 'L', 'F'}
  \item [0x04]:类别(1=32位, 2=64位)
  \item [0x05]:字节序(1=小端, 2=大端)
  \item [0x06]:ELF 版本
  \item [0x07-0x0F]:填充字节
\end{itemize}

\subsection{程序头表}

\BlockDesc{Program Header}

程序头表描述段(Segment),用于程序加载。

程序头结构:
\begin{lstlisting}[language=C]
typedef struct {
    uint32_t p_type;    // 段类型
    uint32_t p_flags;    // 段标志
    uint64_t p_offset;   // 文件偏移
    uint64_t p_vaddr;    // 虚拟地址
    uint64_t p_paddr;    // 物理地址
    uint64_t p_filesz;   // 文件大小
    uint64_t p_memsz;    // 内存大小
    uint64_t p_align;    // 对齐
} Elf64_Phdr;
\end{lstlisting}

段类型(\lstinline!p_type!):
\begin{itemize}
  \item \texttt{PT\_LOAD}:可加载段(代码和数据)
  \item \texttt{PT\_DYNAMIC}:动态链接信息
  \item \texttt{PT\_INTERP}:解释器路径(如 /lib/ld-linux.so)
  \item \texttt{PT\_NOTE}:注释信息
  \item \texttt{PT\_GNU\_STACK}:栈权限
\end{itemize}

段标志(\lstinline!p_flags!):
\begin{itemize}
  \item \texttt{PF\_X}:可执行
  \item \texttt{PF\_W}:可写
  \item \texttt{PF\_R}:可读
\end{itemize}

\subsection{节头表}

\BlockDesc{Section Header}

节头表描述节(Section),用于链接和调试。

节头结构:
\begin{lstlisting}[language=C]
typedef struct {
    uint32_t sh_name;      // 节名索引
    uint32_t sh_type;      // 节类型
    uint64_t sh_flags;     // 节标志
    uint64_t sh_addr;      // 虚拟地址
    uint64_t sh_offset;    // 文件偏移
    uint64_t sh_size;      // 节大小
    uint32_t sh_link;      // 链接到其他节
    uint32_t sh_info;      // 附加信息
    uint64_t sh_addralign; // 对齐
    uint64_t sh_entsize;   // 表项大小(如果有)
} Elf64_Shdr;
\end{lstlisting}

重要节:
\begin{itemize}
  \item \texttt{.text}:代码段
  \item \texttt{.data}:已初始化数据
  \item \texttt{.bss}:未初始化数据
  \item \texttt{.rodata}:只读数据
  \item \texttt{.symtab}:符号表
  \item \texttt{.strtab}:字符串表
  \item \texttt{.rela.xxx}:重定位表
  \item \texttt{.dynamic}:动态链接信息
  \item \texttt{.got}:全局偏移表
  \item \texttt{.plt}:过程链接表
\end{itemize}

\subsection{符号表}

\BlockDesc{Symbol Table}

符号表用于存储函数和变量的信息。

符号结构:
\begin{lstlisting}[language=C]
typedef struct {
    uint32_t st_name;   // 符号名索引
    unsigned char st_info; // 类型和绑定
    unsigned char st_other;
    uint16_t st_shndx;  // 节索引
    uint64_t st_value;  // 符号值
    uint64_t st_size;   // 符号大小
} Elf64_Sym;
\end{lstlisting}

符号类型(\lstinline!st_info!):
\begin{itemize}
  \item \texttt{STT\_NOTYPE}:未指定类型
  \item \texttt{STT\_OBJECT}:数据对象
  \item \texttt{STT\_FUNC}:函数
  \item \texttt{STT\_SECTION}:节
  \item \texttt{STT\_FILE}:源文件名
\end{itemize}

符号绑定(\lstinline!st_info!):
\begin{itemize}
  \item \texttt{STB\_LOCAL}:局部符号
  \item \texttt{STB\_GLOBAL}:全局符号
  \item \texttt{STB\_WEAK}:弱符号
\end{itemize}

\subsection{重定位}

\BlockDesc{Relocation}

重定位用于在链接时修正地址引用。

重定位项结构:
\begin{lstlisting}[language=C]
typedef struct {
    uint64_t r_offset;  // 重定位位置
    uint64_t r_info;    // 类型和符号索引
    int64_t  r_addend;  // 加数
} Elf64_Rela;
\end{lstlisting}

重定位类型(常见):
\begin{itemize}
  \item \texttt{R\_X86\_64\_64}:64 位绝对地址
  \item \texttt{R\_X86\_64\_PC32}:32 位 PC 相对地址
  \item \texttt{R\_X86\_64\_GOTPCREL}:GOT 相对地址
  \item \texttt{R\_AARCH64\_ABS64}:ARM64 64 位绝对地址
  \item \texttt{R\_AARCH64\_CALL26}:ARM64 26 位调用地址
\end{itemize}

\subsection{动态链接}

\BlockDesc{Dynamic Linking}

动态链接允许程序在运行时加载共享库。

动态段(.dynamic):
\begin{lstlisting}[language=C]
typedef struct {
    int64_t d_tag;   // 标签类型
    uint64_t d_val;   // 值或指针
} Elf64_Dyn;
\end{lstlisting}

重要标签:
\begin{itemize}
  \item \texttt{DT\_NEEDED}:需要的共享库
  \item \texttt{DT\_SYMTAB}:动态符号表
  \item \texttt{DT\_STRTAB}:动态字符串表
  \item \texttt{DT\_INIT}:初始化函数
  \item \texttt{DT\_FINI}:终止函数
  \item \texttt{DT\_GOT}:全局偏移表
  \item \texttt{DT\_PLT}:过程链接表
\end{itemize}

全局偏移表(GOT):
\begin{itemize}
  \item 存储外部符号的地址
  \item 在程序加载时由动态链接器填充
  \item 支持位置无关代码(PIC)
\end{itemize}

过程链接表(PLT):
\begin{itemize}
  \item 用于延迟绑定(lazy binding)
  \item 第一次调用时解析符号地址
  \item 后续调用直接跳转
\end{itemize}

\section{硬件层 C 编程}

硬件层 C 编程涉及直接操作硬件寄存器、中断处理、内存映射等底层操作。
这在嵌入式系统开发中至关重要。

\subsection{基础}

\BlockDesc{硬件层编程特点}

硬件层 C 编程的特点:
\begin{itemize}
  \item 直接访问硬件寄存器
  \item 无操作系统支持(或最小支持)
  \item 对内存和时序要求严格
  \item 需要理解硬件架构
  \item 代码可移植性较低
\end{itemize}

\subsubsection{寄存器操作}

\BlockDesc{寄存器访问}

在 C 语言中访问硬件寄存器:

\begin{enumerate}
  \item \textbf{使用指针}
    \begin{lstlisting}[language=C]
      // 定义寄存器地址
      #define UART_BASE 0x10000000

      // 定义寄存器结构
      typedef struct {
          volatile uint32_t DR;      // 数据寄存器
          volatile uint32_t SR;      // 状态寄存器
          volatile uint32_t BRR;     // 波特率寄存器
      } uart_regs_t;

      // 映射到地址
      uart_regs_t *uart = (uart_regs_t *)UART_BASE;

      // 访问寄存器
      uart->BRR = 0x68;              // 设置波特率
      while (!(uart->SR & 0x80));    // 等待发送就绪
      uart->DR = 'A';                // 发送字符
    \end{lstlisting}

  \item \textbf{使用 volatile}
    \begin{itemize}
      \item \texttt{volatile} 关键字防止编译器优化
      \item 确保每次访问都实际读取/写入寄存器
      \item 对硬件寄存器访问至关重要
    \end{itemize}

  \item \textbf{内存屏障}
    \begin{lstlisting}[language=C]
      // 内存屏障确保内存访问顺序
      __sync_synchronize();          // GCC 内置函数
      __dsb();                       // ARM 数据同步屏障
      __dmb();                       // ARM 数据内存屏障
    \end{lstlisting}
\end{enumerate}

\subsubsection{位操作}

\BlockDesc{位操作}

硬件编程中经常需要操作寄存器的特定位:

\begin{lstlisting}[language=C]
// 设置位
#define SET_BIT(reg, bit)    ((reg) |= (1 << (bit)))
#define CLEAR_BIT(reg, bit)  ((reg) &= ~(1 << (bit)))
#define TOGGLE_BIT(reg, bit) ((reg) ^= (1 << (bit)))
#define READ_BIT(reg, bit)   (((reg) >> (bit)) & 1)

// 设置位域
#define SET_FIELD(reg, field, val) \
    ((reg) = ((reg) & ~(field##_MASK)) | ((val) << (field##_SHIFT)))

// 示例
SET_BIT(GPIO->ODR, 5);              // 设置 GPIO5
CLEAR_BIT(GPIO->ODR, 5);            // 清除 GPIO5
\end{lstlisting}

\subsubsection{中断处理}

\BlockDesc{中断处理}

中断处理函数的特点:
\begin{itemize}
  \item 执行时间短
  \item 不能阻塞
  \item 通常使用中断标志位通知主程序
\end{itemize}

中断处理示例:
\begin{lstlisting}[language=C]
// 中断服务例程
void __attribute__((interrupt)) UART_IRQHandler(void)
{
    if (UART->ISR & UART_ISR_RXNE) {
        // 接收数据
        uint8_t data = UART->RDR;
        rx_buffer[rx_index++] = data;
    }

    if (UART->ISR & UART_ISR_TXE) {
        // 发送数据
        if (tx_index < tx_length) {
            UART->TDR = tx_buffer[tx_index++];
        } else {
            UART->CR1 &= ~UART_CR1_TXEIE; // 禁用发送中断
        }
    }
}

// 启用中断
void uart_init(void)
{
    // 配置 UART
    // ...

    // 启用接收中断
    UART->CR1 |= UART_CR1_RXNEIE;

    // 在 NVIC 中启用中断
    NVIC_EnableIRQ(UART_IRQn);
}
\end{lstlisting}

\subsubsection{内存映射}

\BlockDesc{内存映射}

硬件资源通常映射到特定的内存地址:

\begin{lstlisting}[language=C]
// 内存映射 I/O 地址
#define PERIPH_BASE    0x40000000UL
#define APB1PERIPH_BASE (PERIPH_BASE + 0x00000000UL)
#define APB2PERIPH_BASE (PERIPH_BASE + 0x00010000UL)

// 外设基地址
#define GPIOA_BASE     (APB2PERIPH_BASE + 0x00000800UL)
#define GPIOB_BASE     (APB2PERIPH_BASE + 0x00000C00UL)
#define UART1_BASE     (APB2PERIPH_BASE + 0x00003800UL)

// 寄存器结构
typedef struct {
    volatile uint32_t CRL;
    volatile uint32_t CRH;
    volatile uint32_t IDR;
    volatile uint32_t ODR;
    volatile uint32_t BSRR;
    volatile uint32_t BRR;
    volatile uint32_t LCKR;
} GPIO_TypeDef;

// 外设指针
#define GPIOA ((GPIO_TypeDef *)GPIOA_BASE)
#define GPIOB ((GPIO_TypeDef *)GPIOB_BASE)
\end{lstlisting}

\subsection{MCU 与 RTOS}

微控制器(MCU)通常资源有限,实时操作系统(RTOS)提供了任务调度、同步和通信机制。

\subsubsection{MCU 编程特点}

\BlockDesc{MCU 编程}

MCU 编程的特点:
\begin{itemize}
  \item 资源受限:有限的 RAM 和 Flash
  \item 低功耗:需要优化功耗
  \item 实时性:需要快速响应
  \item 单线程:通常只有一个 CPU 核心
\end{itemize}

\subsubsection{RTOS 基础}

\BlockDesc{实时操作系统}

RTOS 提供的基本功能:
\begin{itemize}
  \item 任务调度:多任务并发执行
  \item 同步机制:信号量、互斥锁、消息队列
  \item 时间管理:延时、定时器
  \item 内存管理:动态内存分配
\end{itemize}

\subsubsection{FreeRTOS 示例}

FreeRTOS 是流行的开源 RTOS:

\begin{lstlisting}[language=C]
#include "FreeRTOS.h"
#include "task.h"
#include "queue.h"
#include "semphr.h"

// 任务函数
void vTask1(void *pvParameters)
{
    for (;;) {
        // 任务代码
        vTaskDelay(pdMS_TO_TICKS(1000)); // 延时 1 秒
    }
}

void vTask2(void *pvParameters)
{
    for (;;) {
        // 任务代码
        vTaskDelay(pdMS_TO_TICKS(500)); // 延时 500ms
    }
}

// 创建任务
int main(void)
{
    // 硬件初始化
    // ...

    // 创建任务
    xTaskCreate(vTask1, "Task1", configMINIMAL_STACK_SIZE,
                NULL, 1, NULL);
    xTaskCreate(vTask2, "Task2", configMINIMAL_STACK_SIZE,
                NULL, 1, NULL);

    // 启动调度器
    vTaskStartScheduler();

    // 不会执行到这里
    return 0;
}
\end{lstlisting}

\subsubsection{任务同步}

使用信号量和队列进行任务同步:

\begin{lstlisting}[language=C]
// 创建信号量
SemaphoreHandle_t xSemaphore = xSemaphoreCreateBinary();

// 创建队列
QueueHandle_t xQueue = xQueueCreate(10, sizeof(int));

// 任务 1:发送数据
void vSenderTask(void *pvParameters)
{
    int value = 0;
    for (;;) {
        xQueueSend(xQueue, &value, portMAX_DELAY);
        value++;
        vTaskDelay(pdMS_TO_TICKS(1000));
    }
}

// 任务 2:接收数据
void vReceiverTask(void *pvParameters)
{
    int received;
    for (;;) {
        if (xQueueReceive(xQueue, &received, portMAX_DELAY)) {
            // 处理接收到的数据
        }
    }
}
\end{lstlisting}

\section{软件架构思想}

良好的软件架构是构建可维护、可扩展、可靠软件系统的关键。
本节介绍重要的软件架构原则和模式。

\subsection{设计原则}

\BlockDesc{SOLID 原则}

SOLID 是面向对象设计的五个基本原则:

\begin{enumerate}
  \item \textbf{单一职责原则(SRP)}
    \begin{itemize}
      \item 一个类应该只有一个改变的理由
      \item 每个类只负责一个功能
      \item 提高内聚性,降低耦合
    \end{itemize}

  \item \textbf{开闭原则(OCP)}
    \begin{itemize}
      \item 对扩展开放,对修改关闭
      \item 通过接口和抽象类实现扩展
      \item 减少对现有代码的修改
    \end{itemize}

  \item \textbf{里氏替换原则(LSP)}
    \begin{itemize}
      \item 子类应该能够替换父类
      \item 子类不应该改变父类的行为
      \item 保证多态的正确性
    \end{itemize}

  \item \textbf{接口隔离原则(ISP)}
    \begin{itemize}
      \item 客户端不应该依赖它不需要的接口
      \item 接口应该小而专一
      \item 避免接口污染
    \end{itemize}

  \item \textbf{依赖倒置原则(DIP)}
    \begin{itemize}
      \item 高层模块不应该依赖低层模块
      \item 两者都应该依赖抽象
      \item 降低模块间的耦合
    \end{itemize}
\end{enumerate}

\subsection{设计模式}

\BlockDesc{常用设计模式}

\begin{enumerate}
  \item \textbf{单例模式(Singleton)}
    \begin{itemize}
      \item 确保一个类只有一个实例
      \item 提供全局访问点
      \item 适用于配置管理、日志等
    \end{itemize}

  \item \textbf{工厂模式(Factory)}
    \begin{itemize}
      \item 创建对象而不指定具体类
      \item 封装对象创建逻辑
      \item 支持多态和扩展
    \end{itemize}

  \item \textbf{观察者模式(Observer)}
    \begin{itemize}
      \item 定义对象间一对多的依赖关系
      \item 当一个对象状态改变时,所有依赖者都得到通知
      \item 适用于事件驱动系统
    \end{itemize}

  \item \textbf{策略模式(Strategy)}
    \begin{itemize}
      \item 定义一系列算法,封装它们
      \item 使它们可以互换
      \item 算法独立于使用它的客户端
    \end{itemize}

  \item \textbf{适配器模式(Adapter)}
    \begin{itemize}
      \item 将一个类的接口转换成客户希望的接口
      \item 使原本不兼容的类可以一起工作
      \item 适用于系统集成
    \end{itemize}
\end{enumerate}

\subsection{分层架构}

\BlockDesc{分层架构}

分层架构将系统划分为多个层次:

\begin{enumerate}
  \item \textbf{表示层}
    \begin{itemize}
      \item 用户界面
      \item 处理用户输入和输出
    \end{itemize}

  \item \textbf{业务逻辑层}
    \begin{itemize}
      \item 核心业务规则
      \item 处理业务逻辑
    \end{itemize}

  \item \textbf{数据访问层}
    \begin{itemize}
      \item 数据库访问
      \item 数据持久化
    \end{itemize}

  \item \textbf{硬件抽象层(HAL)}
    \begin{itemize}
      \item 硬件接口抽象
      \item 便于移植和测试
    \end{itemize}
\end{enumerate}

\subsection{模块化设计}

\BlockDesc{模块化}

模块化设计的原则:
\begin{itemize}
  \item 高内聚:模块内部元素紧密相关
  \item 低耦合:模块间依赖最小化
  \item 接口清晰:定义明确的接口
  \item 可测试:模块可以独立测试
\end{itemize}

\section{用户层 C 编程}

用户层 C 编程是在操作系统之上进行的应用开发,可以利用操作系统提供的丰富功能。

\subsection{系统编程基础}

\BlockDesc{系统调用}

系统调用是用户程序请求操作系统服务的接口:

\begin{lstlisting}[language=C]
#include <unistd.h>
#include <sys/types.h>
#include <fcntl.h>

// 文件操作
int fd = open("file.txt", O_RDONLY);
read(fd, buffer, size);
write(fd, buffer, size);
close(fd);

// 进程操作
pid_t pid = fork();
if (pid == 0) {
    // 子进程
    execve("/bin/ls", argv, envp);
} else {
    // 父进程
    wait(NULL);
}
\end{lstlisting}

\subsection{动态链接}

\BlockDesc{动态链接}

动态链接允许程序在运行时加载共享库,而不是在编译时静态链接。

\subsubsection{动态链接的优势}

\begin{itemize}
  \item 节省内存:多个程序共享同一库代码
  \item 便于更新:更新库无需重新编译程序
  \item 减少可执行文件大小
  \item 支持插件机制
\end{itemize}

\subsubsection{动态链接过程}

\begin{enumerate}
  \item \textbf{程序启动}
    \begin{itemize}
      \item 动态链接器(ld.so)加载程序
      \item 解析依赖的共享库
      \item 加载共享库到内存
    \end{itemize}

  \item \textbf{符号解析}
    \begin{itemize}
      \item 解析未定义的符号
      \item 在共享库中查找符号
      \item 建立符号到地址的映射
    \end{itemize}

  \item \textbf{重定位}
    \begin{itemize}
      \item 修正代码中的地址引用
      \item 更新 GOT(全局偏移表)
      \item 设置 PLT(过程链接表)
    \end{itemize}

  \item \textbf{延迟绑定}
    \begin{itemize}
      \item 第一次调用时解析符号
      \item 后续调用直接跳转
      \item 提高启动速度
    \end{itemize}
\end{enumerate}

\subsubsection{动态链接 API}

\begin{lstlisting}[language=C]
#include <dlfcn.h>

// 打开共享库
void *handle = dlopen("libexample.so", RTLD_LAZY);
if (!handle) {
    fprintf(stderr, "Error: %s\n", dlerror());
    return 1;
}

// 获取符号地址
void (*func)(int) = (void (*)(int))dlsym(handle, "function_name");
if (!func) {
    fprintf(stderr, "Error: %s\n", dlerror());
    dlclose(handle);
    return 1;
}

// 调用函数
func(42);

// 关闭共享库
dlclose(handle);
\end{lstlisting}

\subsubsection{共享库创建}

\begin{lstlisting}[language=C]
// example.c
#include <stdio.h>

void hello(void)
{
    printf("Hello from shared library!\n");
}

int add(int a, int b)
{
    return a + b;
}
\end{lstlisting}

编译共享库:
\begin{lstlisting}[language=bash]
# 编译为位置无关代码(PIC)
gcc -fPIC -shared -o libexample.so example.c

# 链接时使用
gcc -o program program.c -L. -lexample
\end{lstlisting}

\subsection{内存管理}

\BlockDesc{用户空间内存管理}

用户空间内存管理的特点:
\begin{itemize}
  \item 使用虚拟内存
  \item 由操作系统管理
  \item 可以使用 malloc/free
  \item 支持内存映射文件
\end{itemize}

\subsubsection{动态内存分配}

\begin{lstlisting}[language=C]
#include <stdlib.h>

// 分配内存
void *ptr = malloc(1024);
if (ptr == NULL) {
    // 处理错误
}

// 重新分配
ptr = realloc(ptr, 2048);

// 释放内存
free(ptr);
ptr = NULL; // 防止悬空指针
\end{lstlisting}

\subsubsection{内存映射}

\begin{lstlisting}[language=C]
#include <sys/mman.h>
#include <fcntl.h>
#include <unistd.h>

// 打开文件
int fd = open("file.dat", O_RDWR);

// 映射到内存
void *addr = mmap(NULL, file_size, PROT_READ | PROT_WRITE,
                   MAP_SHARED, fd, 0);
if (addr == MAP_FAILED) {
    // 处理错误
}

// 使用映射的内存
// ...

// 取消映射
munmap(addr, file_size);
close(fd);
\end{lstlisting}

\section{Rust 编程}

Rust 是一种系统编程语言,专注于安全性、并发性和性能。
它通过所有权系统在编译时防止内存安全问题。

\subsection{Rust 基础}

\BlockDesc{Rust 语言特点}

Rust 的主要特点:
\begin{itemize}
  \item 内存安全:编译时检查,无需垃圾回收
  \item 零成本抽象:高级特性不带来运行时开销
  \item 并发安全:防止数据竞争
  \item 模式匹配:强大的模式匹配系统
  \item 类型系统:强大的类型系统和类型推断
\end{itemize}

\subsubsection{所有权系统}

\BlockDesc{所有权}

Rust 的所有权规则:
\begin{enumerate}
  \item 每个值都有一个所有者
  \item 同一时间只能有一个所有者
  \item 当所有者离开作用域时,值被丢弃
\end{enumerate}

示例:
\begin{lstlisting}[language=Rust]
fn main() {
    let s = String::from("hello");  // s 拥有字符串
    takes_ownership(s);              // s 的所有权被移动
    // println!("{}", s);            // 错误!s 不再有效

    let x = 5;                        // x 是 i32,实现了 Copy
    makes_copy(x);                    // x 被复制
    println!("{}", x);               // x 仍然有效
}

fn takes_ownership(some_string: String) {
    println!("{}", some_string);
} // some_string 离开作用域,内存被释放

fn makes_copy(some_integer: i32) {
    println!("{}", some_integer);
} // some_integer 离开作用域,但 i32 是 Copy 类型
\end{lstlisting}

\subsubsection{借用}

\BlockDesc{借用}

借用允许使用值而不获取所有权:

\begin{lstlisting}[language=Rust]
fn main() {
    let s1 = String::from("hello");
    let len = calculate_length(&s1);  // 借用 s1
    println!("The length of '{}' is {}.", s1, len);  // s1 仍然有效
}

fn calculate_length(s: &String) -> usize {
    s.len()
} // s 离开作用域,但因为它没有所有权,所以不会释放内存
\end{lstlisting}

借用规则:
\begin{itemize}
  \item 同一时间可以有多个不可变引用
  \item 同一时间只能有一个可变引用
  \item 引用必须始终有效
\end{itemize}

\subsubsection{错误处理}

\BlockDesc{Result 类型}

Rust 使用 Result 类型处理错误:

\begin{lstlisting}[language=Rust]
use std::fs::File;
use std::io::ErrorKind;

fn main() {
    let f = File::open("hello.txt");

    let f = match f {
        Ok(file) => file,
        Err(error) => match error.kind() {
            ErrorKind::NotFound => match File::create("hello.txt") {
                Ok(fc) => fc,
                Err(e) => panic!("Problem creating the file: {:?}", e),
            },
            other_error => {
                panic!("Problem opening the file: {:?}", other_error)
            }
        },
    };
}
\end{lstlisting}

使用 ? 操作符简化错误处理:

\begin{lstlisting}[language=Rust]
use std::fs::File;
use std::io;
use std::io::Read;

fn read_username_from_file() -> Result<String, io::Error> {
    let mut f = File::open("hello.txt")?;
    let mut s = String::new();
    f.read_to_string(&mut s)?;
    Ok(s)
}
\end{lstlisting}

\subsection{并发编程}

\BlockDesc{并发}

Rust 的并发模型:

\begin{lstlisting}[language=Rust]
use std::thread;
use std::sync::mpsc;

fn main() {
    let (tx, rx) = mpsc::channel();

    thread::spawn(move || {
        let val = String::from("hi");
        tx.send(val).unwrap();
    });

    let received = rx.recv().unwrap();
    println!("Got: {}", received);
}
\end{lstlisting}

使用 Mutex 保护共享数据:

\begin{lstlisting}[language=Rust]
use std::sync::{Arc, Mutex};
use std::thread;

fn main() {
    let counter = Arc::new(Mutex::new(0));
    let mut handles = vec![];

    for _ in 0..10 {
        let counter = Arc::clone(&counter);
        let handle = thread::spawn(move || {
            let mut num = counter.lock().unwrap();
            *num += 1;
        });
        handles.push(handle);
    }

    for handle in handles {
        handle.join().unwrap();
    }

    println!("Result: {}", *counter.lock().unwrap());
}
\end{lstlisting}

\section{C++ 编程}

C++ 是一种多范式编程语言,支持面向对象、泛型和函数式编程。
它是系统编程和性能关键应用的重要语言。

\subsection{C++ 基础}

\BlockDesc{C++ 语言特点}

C++ 的主要特点:
\begin{itemize}
  \item 多范式:支持多种编程风格
  \item 零成本抽象:高级特性不带来运行时开销
  \item 直接内存控制:可以精确控制内存
  \item 丰富的标准库:STL 提供强大的数据结构和算法
  \item 模板元编程:编译时计算和代码生成
\end{itemize}

\subsubsection{现代 C++ 特性}

\BlockDesc{C++11/14/17/20}

现代 C++ 的重要特性:

\begin{enumerate}
  \item \textbf{自动类型推断}
    \begin{lstlisting}[language=C++]
      auto x = 42;                    // int
      auto y = 3.14;                  // double
      auto z = std::make_shared<int>(10);
    \end{lstlisting}

  \item \textbf{范围 for 循环}
    \begin{lstlisting}[language=C++]
      std::vector<int> vec = {1, 2, 3, 4, 5};
      for (auto& item : vec) {
          std::cout << item << std::endl;
      }
    \end{lstlisting}

  \item \textbf{智能指针}
    \begin{lstlisting}[language=C++]
      // 自动内存管理
      std::unique_ptr<int> ptr1 = std::make_unique<int>(42);
      std::shared_ptr<int> ptr2 = std::make_shared<int>(42);
      std::weak_ptr<int> ptr3 = ptr2;
    \end{lstlisting}

  \item \textbf{Lambda 表达式}
    \begin{lstlisting}[language=C++]
      auto add = [](int a, int b) { return a + b; };
      int result = add(3, 4);

      // 捕获变量
      int x = 10;
      auto lambda = [x](int y) { return x + y; };
    \end{lstlisting}

  \item \textbf{移动语义}
    \begin{lstlisting}[language=C++]
      std::vector<int> vec1 = {1, 2, 3};
      std::vector<int> vec2 = std::move(vec1);  // 移动,不复制
    \end{lstlisting}
\end{enumerate}

\subsubsection{面向对象编程}

\BlockDesc{类和对象}

\begin{lstlisting}[language=C++]
class Rectangle {
private:
    double width;
    double height;

public:
    // 构造函数
    Rectangle(double w, double h) : width(w), height(h) {}

    // 析构函数
    ~Rectangle() = default;

    // 拷贝构造函数
    Rectangle(const Rectangle& other) = default;

    // 移动构造函数
    Rectangle(Rectangle&& other) noexcept = default;

    // 成员函数
    double area() const {
        return width * height;
    }

    // 操作符重载
    Rectangle& operator=(const Rectangle& other) = default;
};

// 使用
Rectangle rect(10.0, 5.0);
double a = rect.area();
\end{lstlisting}

\subsubsection{模板编程}

\BlockDesc{模板}

函数模板:
\begin{lstlisting}[language=C++]
template<typename T>
T max(T a, T b) {
    return (a > b) ? a : b;
}

// 使用
int m1 = max(3, 5);
double m2 = max(3.14, 2.71);
\end{lstlisting}

类模板:
\begin{lstlisting}[language=C++]
template<typename T>
class Stack {
private:
    std::vector<T> elements;

public:
    void push(const T& item) {
        elements.push_back(item);
    }

    void pop() {
        if (!elements.empty()) {
            elements.pop_back();
        }
    }

    T top() const {
        return elements.back();
    }
};

// 使用
Stack<int> intStack;
intStack.push(42);
\end{lstlisting}

\subsubsection{STL 容器和算法}

\BlockDesc{STL}

常用容器:
\begin{lstlisting}[language=C++]
#include <vector>
#include <list>
#include <map>
#include <unordered_map>
#include <set>

// 向量
std::vector<int> vec = {1, 2, 3, 4, 5};

// 列表
std::list<int> lst = {1, 2, 3};

// 映射
std::map<std::string, int> map;
map["one"] = 1;
map["two"] = 2;

// 无序映射(哈希表)
std::unordered_map<std::string, int> umap;
umap["one"] = 1;
\end{lstlisting}

STL 算法:
\begin{lstlisting}[language=C++]
#include <algorithm>
#include <numeric>

std::vector<int> vec = {3, 1, 4, 1, 5, 9, 2, 6};

// 排序
std::sort(vec.begin(), vec.end());

// 查找
auto it = std::find(vec.begin(), vec.end(), 5);

// 计数
int count = std::count(vec.begin(), vec.end(), 1);

// 累加
int sum = std::accumulate(vec.begin(), vec.end(), 0);

// 变换
std::transform(vec.begin(), vec.end(), vec.begin(),
               [](int x) { return x * 2; });
\end{lstlisting}

\subsubsection{并发编程}

\BlockDesc{多线程}

C++11 标准库提供线程支持:

\begin{lstlisting}[language=C++]
#include <thread>
#include <mutex>
#include <condition_variable>
#include <future>

// 创建线程
std::thread t([]() {
    std::cout << "Hello from thread!" << std::endl;
});
t.join();

// 互斥锁
std::mutex mtx;
std::lock_guard<std::mutex> lock(mtx);
// 临界区代码

// 条件变量
std::condition_variable cv;
cv.wait(lock, []() { return condition; });
cv.notify_one();

// 异步执行
auto future = std::async(std::launch::async, []() {
    return compute();
});
int result = future.get();
\end{lstlisting}

\subsection{RAII 和智能指针}

\BlockDesc{RAII}

RAII(Resource Acquisition Is Initialization)是 C++ 的重要编程范式:

\begin{lstlisting}[language=C++]
class File {
private:
    FILE* file;

public:
    File(const char* filename) {
        file = fopen(filename, "r");
        if (!file) {
            throw std::runtime_error("Failed to open file");
        }
    }

    ~File() {
        if (file) {
            fclose(file);
        }
    }

    // 禁用拷贝
    File(const File&) = delete;
    File& operator=(const File&) = delete;

    // 允许移动
    File(File&& other) noexcept : file(other.file) {
        other.file = nullptr;
    }
};

// 使用
{
    File f("data.txt");
    // 使用文件
} // 文件自动关闭
\end{lstlisting}

智能指针自动管理内存:
\begin{lstlisting}[language=C++]
#include <memory>

// unique_ptr:独占所有权
std::unique_ptr<int> ptr1 = std::make_unique<int>(42);

// shared_ptr:共享所有权
std::shared_ptr<int> ptr2 = std::make_shared<int>(42);
std::shared_ptr<int> ptr3 = ptr2;  // 引用计数 = 2

// weak_ptr:不增加引用计数
std::weak_ptr<int> ptr4 = ptr2;
\end{lstlisting}

\subsection{异常处理}

\BlockDesc{异常}

C++ 异常处理:

\begin{lstlisting}[language=C++]
#include <stdexcept>

void function() {
    throw std::runtime_error("Something went wrong");
}

int main() {
    try {
        function();
    } catch (const std::runtime_error& e) {
        std::cerr << "Error: " << e.what() << std::endl;
    } catch (...) {
        std::cerr << "Unknown error" << std::endl;
    }
    return 0;
}
\end{lstlisting}

\subsection{性能优化}

\BlockDesc{性能优化}

C++ 性能优化技巧:

\begin{enumerate}
  \item \textbf{避免不必要的拷贝}
    \begin{itemize}
      \item 使用引用传递
      \item 使用移动语义
      \item 返回值优化(RVO)
    \end{itemize}

  \item \textbf{内联函数}
    \begin{lstlisting}[language=C++]
      inline int add(int a, int b) {
          return a + b;
      }
    \end{lstlisting}

  \item \textbf{预分配内存}
    \begin{lstlisting}[language=C++]
      std::vector<int> vec;
      vec.reserve(1000);  // 预分配容量
    \end{lstlisting}

  \item \textbf{使用 constexpr}
    \begin{lstlisting}[language=C++]
      constexpr int factorial(int n) {
          return (n <= 1) ? 1 : n * factorial(n - 1);
      }
    \end{lstlisting}
\end{enumerate}
