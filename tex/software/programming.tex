\chapter{编程}

\section{GNU toolchain}

当今的编程工具离不开 GNU toolchain,可以说没有 GNU 工具链计算机编程不会发展的如此丰富多彩。

GNU 工具链是 GNU 项目生成的广泛的编程工具集合。
它在 Linux 开发和嵌入式系统软件以及一些 BSD 系统开发上扮演着至关重要的角色。
因此,我们要在首节就对 GNU 工具链进行介绍。

GNU 工具链由以下工具组成:

\begin{itemize}
  \item GNU Autotools(构建系统)——GNU 软件打包工具
  \item GNU Binutils——可执行代码开发工具
  \item GNU Bison——Yacc 兼容的解析生成工具
  \item GNU C Library——GNU 工程实现的标准标准 C 库
  \item GNU Compiler Collection——支持不同编程语言的自由开源的编译器集
  \item GNU Debugger——代码级调试器
  \item GNU m4——通用宏处理器
  \item GNU make——自动化软件构建工具
\end{itemize}

\subsection{简介}

\subsubsection{GNU Autotools}

Autotools 由 GNU 工具 Autoconf、Automake 和 Libtool 组成。
与它经常一起使用的其他相关工具包括 GNU make、GNU gettext、pkg-config 和 GNU 编译器套件(GCC)。

\begin{description}
  \item[Autoconf] \hfill \\
    Autoconf 基于 configure.ac 文件的内容生成一个 configure 脚本,该文件描述了特定源代码体系的特征。
    运行 configure 脚本时,它会扫描构建环境并生成一个次级的 config.status 脚本,然后将其他输入文件(通常是 Makefile.in)转换为适合该构建环境的输出文件(Makefile)。
    最后,make 程序使用 Makefile 从源代码生成可执行程序。

    Autotools 的复杂性反映了源代码可能被构建的各种情况。

    \begin{itemize}
      \item 如果更改了源代码文件,则只需重新运行 make,它只会重新编译受更改影响的源代码部分。
      \item 如果.in 文件已更改,则只需重新运行 config.status 和 make。
      \item 如果将源代码体系复制到另一台计算机上,则只需重新运行 configure(它运行 config.status)和 make。
        (因此,使用 Autotools 的源代码通常在不包含 configure 生成的文件的情况下进行分发。)
      \item 如果源代码体系发生更根本性的变化,则需要更改 configure.ac 和.in 文件,并且还需要遵循所有后续步骤。
    \end{itemize}

    为了处理文件,autoconf 使用 GNU 实现的 m4 宏系统。

    Autoconf 附带了几个辅助程序,例如 autoheader,用于帮助管理 C 头文件;
    autoscan,可以为 Autoconf 创建一个初始输入文件;
    以及 ifnames,可以列出程序中使用的 C 预处理器标识符。

  \item[Automake] \hfill \\
    Automake 帮助创建可移植的 Makefile,这些 Makefile 又会被 make 实用程序处理。
    它以 Makefile.am 作为输入,并将其转换为 Makefile.in,该文件由 configure 脚本使用以生成 Makefile 输出文件。
    它还执行自动依赖项跟踪;
    每当编译源文件时,依赖项列表(例如 C 头文件)都会被记录下来。
    稍后,每次运行 make 并且依赖项似乎已更改时,相关文件将被重新构建。

  \item[Libtool] \hfill \\
    Libtool 帮助在各种类 Unix 操作系统上管理静态和动态库的创建。
    Libtool 通过抽象化库创建过程来实现这一点,隐藏了各种系统之间的差异(例如 Linux 系统与 Solaris)。
\end{description}

\subsubsection{GNU Binutils}

\subsubsection{GNU Bison}

\subsubsection{GNU C Library}

\subsubsection{GNU Compiler Collection}

\subsubsection{GNU Debugger}

\subsubsection{GNU m4}

\subsubsection{GNU make}

\subsection{使用教程}

\subsection{交叉编译链制作}

\subsubsection{crosstool-ng}

\section{ARMv8 汇编}

\section{RISC-V 汇编}

\section{ELF 程序结构}

\section{bare metal 编程}

\subsection{GNU 编译器选项}

\begin{itemize}
  \item 编译器选项 \lstinline!-ffreestanding!
  \item 链接器选项 \lstinline!-nostdlib!
\end{itemize}

\section{硬件层 C 编程}
\subsection{基础}
\subsection{MCU 与 RTOS}

\section{软件架构思想}

\section{用户层 C 编程}
\subsection{动态链接}

\section{每个程序员都应该知道的内存知识}

关于内存相关的知识,强烈建议阅读
\href{https://people.freebsd.org/~lstewart/articles/cpumemory.pdf}{What Every Programmer Should Know About Memory}。

\section{Rust 编程}

\section{C++ 编程}
