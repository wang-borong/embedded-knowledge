RISC-V 是一个开源的指令集架构(ISA),由加州大学伯克利分校开发。RISC-V 具有模块化、可扩展、无专利限制等特点,在学术界和工业界都得到了广泛关注。

\subsection{Privileged Architecture}

RISC-V 特权架构定义了系统级的功能,包括异常处理、内存管理、中断处理等。

\subsubsection{特权级别}

RISC-V 定义了多个特权级别(模式):

\begin{enumerate}
  \item \textbf{用户模式(User Mode,U-Mode)}
    \begin{itemize}
      \item 最低特权级别
      \item 运行用户应用程序
      \item 受限的指令和资源访问
    \end{itemize}

  \item \textbf{监管者模式(Supervisor Mode,S-Mode)}
    \begin{itemize}
      \item 运行操作系统内核
      \item 管理虚拟内存
      \item 处理系统调用和异常
    \end{itemize}

  \item \textbf{机器模式(Machine Mode,M-Mode)}
    \begin{itemize}
      \item 最高特权级别
      \item 运行固件和引导代码
      \item 最低级别的硬件访问
    \end{itemize}

  \item \textbf{Hypervisor 模式(H-Mode)}
    \begin{itemize}
      \item 可选的特权级别
      \item 用于虚拟化
      \item 在 S-Mode 和 M-Mode 之间
    \end{itemize}
\end{enumerate}

\subsubsection{控制状态寄存器(CSR)}

RISC-V 使用控制状态寄存器来管理系统状态:

\paragraph{机器模式 CSR}

\begin{itemize}
  \item \textbf{mstatus}:机器状态寄存器
  \item \textbf{misa}:机器 ISA 寄存器
  \item \textbf{medeleg}:机器异常委托寄存器
  \item \textbf{mideleg}:机器中断委托寄存器
  \item \textbf{mie}:机器中断使能寄存器
  \item \textbf{mtvec}:机器陷阱向量基址寄存器
  \item \textbf{mepc}:机器异常程序计数器
  \item \textbf{mcause}:机器异常原因寄存器
  \item \textbf{mtval}:机器陷阱值寄存器
  \item \textbf{mip}:机器中断挂起寄存器
\end{itemize}

\paragraph{监管者模式 CSR}

\begin{itemize}
  \item \textbf{sstatus}:监管者状态寄存器
  \item \textbf{sedeleg}:监管者异常委托寄存器
  \item \textbf{sideleg}:监管者中断委托寄存器
  \item \textbf{sie}:监管者中断使能寄存器
  \item \textbf{stvec}:监管者陷阱向量基址寄存器
  \item \textbf{sepc}:监管者异常程序计数器
  \item \textbf{scause}:监管者异常原因寄存器
  \item \textbf{stval}:监管者陷阱值寄存器
  \item \textbf{sip}:监管者中断挂起寄存器
  \item \textbf{satp}:监管者地址转换和保护寄存器
\end{itemize}

\subsubsection{异常和中断}

\paragraph{异常类型}

RISC-V 定义了多种异常类型:
\begin{itemize}
  \item 指令地址未对齐
  \item 指令访问错误
  \item 非法指令
  \item 断点
  \item 加载地址未对齐
  \item 加载访问错误
  \item 存储地址未对齐
  \item 存储访问错误
  \item 用户模式环境调用
  \item 监管者模式环境调用
  \item 机器模式环境调用
  \item 指令页错误
  \item 加载页错误
  \item 存储页错误
\end{itemize}

\paragraph{中断类型}

RISC-V 支持多种中断:
\begin{itemize}
  \item 软件中断(Software Interrupt)
  \item 定时器中断(Timer Interrupt)
  \item 外部中断(External Interrupt)
\end{itemize}

\subsubsection{内存管理}

\paragraph{Sv32(32 位虚拟内存)}

\begin{itemize}
  \item 32 位虚拟地址空间
  \item 4 KB 页大小
  \item 两级页表
\end{itemize}

\paragraph{Sv39(39 位虚拟内存)}

\begin{itemize}
  \item 39 位虚拟地址空间(512 GB)
  \item 4 KB 页大小
  \item 三级页表
  \item 64 位系统常用
\end{itemize}

\paragraph{Sv48(48 位虚拟内存)}

\begin{itemize}
  \item 48 位虚拟地址空间(256 TB)
  \item 4 KB 页大小
  \item 四级页表
  \item 大型系统使用
\end{itemize}

\paragraph{页表项(PTE)}

RISC-V 页表项格式:
\begin{itemize}
  \item V(Valid):有效位
  \item R(Read):读权限
  \item W(Write):写权限
  \item X(Execute):执行权限
  \item U(User):用户模式可访问
  \item G(Global):全局映射
  \item A(Accessed):已访问
  \item D(Dirty):已修改
  \item RSW:保留给软件使用
  \item PPN:物理页号
\end{itemize}

\subsection{Unprivileged ISA}

RISC-V 非特权 ISA 定义了用户程序可以使用的指令。

\subsubsection{基本指令集}

\paragraph{RV32I(32 位基础指令集)}

\begin{itemize}
  \item 32 个通用寄存器(x0--x31)
  \item x0 硬连线为 0
  \item 基本整数运算指令
  \item 加载/存储指令
  \item 分支和跳转指令
  \item 系统调用指令(ECALL)
\end{itemize}

\paragraph{RV64I(64 位基础指令集)}

\begin{itemize}
  \item 扩展 RV32I 到 64 位
  \item 64 位寄存器
  \item 64 位地址空间
  \item 额外的 64 位运算指令
\end{itemize}

\subsubsection{标准扩展}

\paragraph{M 扩展(乘法/除法)}

\begin{itemize}
  \item MUL:乘法
  \item MULH:高位乘法
  \item DIV:有符号除法
  \item DIVU:无符号除法
  \item REM:有符号取余
  \item REMU:无符号取余
\end{itemize}

\paragraph{A 扩展(原子操作)}

\begin{itemize}
  \item LR(Load Reserved):保留加载
  \item SC(Store Conditional):条件存储
  \item AMO(Atomic Memory Operation):原子内存操作
  \item 用于实现锁和同步原语
\end{itemize}

\paragraph{F 扩展(单精度浮点)}

\begin{itemize}
  \item 32 位浮点运算
  \item 32 个浮点寄存器(f0--f31)
  \item IEEE 754 标准
\end{itemize}

\paragraph{D 扩展(双精度浮点)}

\begin{itemize}
  \item 64 位浮点运算
  \item 扩展 F 扩展
\end{itemize}

\paragraph{C 扩展(压缩指令)}

\begin{itemize}
  \item 16 位指令编码
  \item 减少代码大小
  \item 提高代码密度
\end{itemize}

\subsubsection{寄存器约定}

\paragraph{通用寄存器用途}

\begin{itemize}
  \item x0:零寄存器(硬连线为 0)
  \item x1(ra):返回地址
  \item x2(sp):栈指针
  \item x3(gp):全局指针
  \item x4(tp):线程指针
  \item x5--x7(t0--t2):临时寄存器
  \item x8(s0/fp):保存寄存器/帧指针
  \item x9(s1):保存寄存器
  \item x10--x11(a0--a1):函数参数/返回值
  \item x12--x17(a2--a7):函数参数
  \item x18--x27(s2--s11):保存寄存器
  \item x28--x31(t3--t6):临时寄存器
\end{itemize}

\subsubsection{调用约定}

\paragraph{参数传递}

\begin{itemize}
  \item 前 8 个整数参数通过 a0--a7 传递
  \item 前 8 个浮点参数通过 f0--f7 传递
  \item 更多参数通过栈传递
\end{itemize}

\paragraph{返回值}

\begin{itemize}
  \item 整数返回值通过 a0 传递
  \item 浮点返回值通过 f0 传递
  \item 大结构体通过内存传递
\end{itemize}

\paragraph{调用者保存 vs 被调用者保存}

\begin{itemize}
  \item 临时寄存器(t0--t6):调用者保存
  \item 保存寄存器(s0--s11):被调用者保存
  \item 参数寄存器(a0--a7):调用者保存
\end{itemize}

\section{内存层级}

现代计算机系统采用多级内存层次结构,以平衡性能、容量和成本。

\subsection{内存层次结构}

\subsubsection{层次结构}

典型的内存层次结构(从快到慢,从小到大):

\begin{enumerate}
  \item \textbf{寄存器}
    \begin{itemize}
      \item 最快,容量最小(几十到几百字节)
      \item CPU 内部,访问延迟:0 周期
    \end{itemize}

  \item \textbf{L1 缓存}
    \begin{itemize}
      \item 很快,容量小(32--64 KB)
      \item CPU 内部,访问延迟:1--3 周期
    \end{itemize}

  \item \textbf{L2 缓存}
    \begin{itemize}
      \item 较快,容量中等(256 KB--1 MB)
      \item CPU 内部或外部,访问延迟:10--20 周期
    \end{itemize}

  \item \textbf{L3 缓存}
    \begin{itemize}
      \item 中等速度,容量较大(几 MB 到几十 MB)
      \item 多核共享,访问延迟:20--40 周期
    \end{itemize}

  \item \textbf{主内存(DRAM)}
    \begin{itemize}
      \item 较慢,容量大(几 GB 到几 TB)
      \item 访问延迟:100--300 周期
    \end{itemize}

  \item \textbf{辅助存储(SSD/HDD)}
    \begin{itemize}
      \item 最慢,容量最大(几百 GB 到几十 TB)
      \item 访问延迟:数万到数百万周期
    \end{itemize}
\end{enumerate}

\subsubsection{访问时间对比}

\begin{table}[H]
  \centering
  \caption{不同存储层次的典型访问时间}
  \begin{tabular}{|l|l|l|}
    \hline
    存储类型 & 容量 & 访问时间 \\
    \hline
    寄存器 & 几十字节 & 0 周期 \\
    L1 缓存 & 32--64 KB & 1--3 周期 \\
    L2 缓存 & 256 KB--1 MB & 10--20 周期 \\
    L3 缓存 & 几 MB--几十 MB & 20--40 周期 \\
    主内存 & 几 GB--几 TB & 100--300 周期 \\
    SSD & 几百 GB--几 TB & 数万周期 \\
    HDD & 几 TB--几十 TB & 数百万周期 \\
    \hline
  \end{tabular}
\end{table}

\subsection{缓存性能}

\subsubsection{缓存命中率}

缓存命中率是衡量缓存性能的重要指标:

\begin{equation}
\text{命中率} = \frac{\text{缓存命中次数}}{\text{总访问次数}} \label{eq:cache-hit-rate}
\end{equation}

\subsubsection{平均访问时间}

平均内存访问时间:

\begin{equation}
T_{avg} = T_{hit} \times H + T_{miss} \times (1 - H) \label{eq:avg-access-time}
\end{equation}

其中:
\begin{itemize}
  \item $T_{hit}$:缓存命中时间
  \item $T_{miss}$:缓存未命中时间(包括访问内存的时间)
  \item $H$:缓存命中率
\end{itemize}

\subsubsection{缓存效率}

缓存效率取决于:
\begin{itemize}
  \item 命中率
  \item 缓存大小
  \item 替换策略
  \item 预取策略
\end{itemize}

\subsection{内存带宽}

\subsubsection{理论带宽}

内存的理论带宽:

\begin{equation}
\text{带宽} = \text{数据宽度} \times \text{频率} \times \text{通道数} \label{eq:memory-bandwidth}
\end{equation}

\subsubsection{实际带宽}

实际带宽受多种因素影响:
\begin{itemize}
  \item 访问模式(顺序 vs 随机)
  \item 缓存效率
  \item 内存控制器效率
  \item 系统负载
\end{itemize}

\section{每个程序员都应该知道的内存知识}

内存是计算机系统中最关键的资源之一,理解内存的工作原理对于编写高效、正确的程序至关重要。本节总结程序员需要了解的关键内存知识。

\subsection{内存的基本概念}

\subsubsection{物理内存 vs 虚拟内存}

\begin{enumerate}
  \item \textbf{物理内存}
    \begin{itemize}
      \item 实际的硬件内存(DRAM)
      \item 有限的容量
      \item 直接由硬件访问
    \end{itemize}

  \item \textbf{虚拟内存}
    \begin{itemize}
      \item 程序看到的地址空间
      \item 可以比物理内存大
      \item 通过 MMU 映射到物理内存
    \end{itemize}
\end{enumerate}

\subsubsection{内存对齐}

\paragraph{对齐的重要性}

\begin{itemize}
  \item 未对齐的访问可能导致性能下降
  \item 某些架构不支持未对齐访问
  \item 对齐到自然边界可以提高性能
\end{itemize}

\paragraph{对齐规则}

\begin{itemize}
  \item 基本类型通常对齐到其大小
  \item 结构体对齐到最大成员的对齐要求
  \item 可以使用 \texttt{\_\_attribute\_\_((packed))} 取消对齐(不推荐)
\end{itemize}

\subsection{缓存友好的编程}

\subsubsection{数据访问模式}

\paragraph{顺序访问}

\begin{itemize}
  \item 顺序访问内存效率高
  \item 充分利用空间局部性
  \item 预取效果好
\end{itemize}

\paragraph{随机访问}

\begin{itemize}
  \item 随机访问效率低
  \item 缓存命中率低
  \item 应尽量避免
\end{itemize}

\subsubsection{数据结构设计}

\paragraph{紧凑的数据结构}

\begin{itemize}
  \item 减少内存占用
  \item 提高缓存利用率
  \item 减少缓存未命中
\end{itemize}

\paragraph{数据布局}

\begin{itemize}
  \item 将经常一起访问的数据放在一起
  \item 避免伪共享
  \item 考虑缓存行大小(通常 64 字节)
\end{itemize}

\subsubsection{循环优化}

\paragraph{循环顺序}

\begin{itemize}
  \item 按行访问优于按列访问(对于二维数组)
  \item 考虑内存布局(行主序 vs 列主序)
\end{itemize}

\paragraph{循环展开}

\begin{itemize}
  \item 减少循环开销
  \item 提高指令级并行
  \item 但可能增加代码大小
\end{itemize}

\subsection{多核编程的内存考虑}

\subsubsection{伪共享}

\paragraph{问题}

\begin{itemize}
  \item 不同线程访问同一缓存行的不同数据
  \item 导致不必要的缓存一致性开销
  \item 严重影响性能
\end{itemize}

\paragraph{解决方法}

\begin{itemize}
  \item 数据对齐到缓存行边界
  \item 使用填充(Padding)分离数据
  \item 将共享数据改为私有数据
\end{itemize}

\subsubsection{内存屏障}

\begin{itemize}
  \item 确保内存操作的顺序
  \item 防止编译器优化导致的问题
  \item 保证多线程程序的正确性
\end{itemize}

\subsection{内存分配}

\subsubsection{栈 vs 堆}

\begin{enumerate}
  \item \textbf{栈}
    \begin{itemize}
      \item 自动管理
      \item 速度快
      \item 容量有限
      \item 局部变量
    \end{itemize}

  \item \textbf{堆}
    \begin{itemize}
      \item 手动管理(或垃圾回收)
      \item 速度较慢
      \item 容量大
      \item 动态分配
    \end{itemize}
\end{enumerate}

\subsubsection{内存分配器}

\begin{itemize}
  \item 理解分配器的行为
  \item 避免频繁分配/释放
  \item 使用对象池减少分配
  \item 考虑内存碎片
\end{itemize}

\subsection{性能分析工具}

\subsubsection{缓存分析}

\begin{itemize}
  \item perf:Linux 性能分析工具
  \item Cachegrind:Valgrind 的缓存分析工具
  \item Intel VTune:Intel 性能分析工具
\end{itemize}

\subsubsection{内存分析}

\begin{itemize}
  \item Valgrind:内存错误检测
  \item AddressSanitizer:快速内存错误检测
  \item 内存分析器:分析内存使用模式
\end{itemize}

\subsection{推荐阅读}

关于内存相关的知识,强烈建议阅读
\href{https://people.freebsd.org/~lstewart/articles/cpumemory.pdf}{What Every Programmer Should Know About Memory}。

这篇经典文章详细介绍了:
\begin{itemize}
  \item 内存硬件的工作原理
  \item CPU 缓存的组织和性能
  \item 虚拟内存系统
  \item NUMA 系统
  \item 编程实践和优化技巧
\end{itemize}

理解这些内容对于编写高性能程序至关重要。

