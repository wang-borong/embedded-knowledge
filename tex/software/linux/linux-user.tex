\chapter{Linux 应用开发}

\section{文件及 IO}

一切皆文件,所有在 Linux/UNIX 系统上打开的文件都用文件描述符表示,并且描述符是非零的。
当我们创建或打开一个文件,内核会返回一个文件描述符给到当前进程。
当我们读取或写入数据到一个文件中,我们便使用相应的文件描述符指定这个文件。

在类 UNIX 系统中约定俗成,文件描述符 0 代表一个进程的标准输入,1 代表标准输出,2 代表标准错误输出。
为了兼容性,应该使用 STDIN\_FILENO、STDOUT\_FILENO 和 STDERR\_FILENO 来代替 0、1 和 2。

文件描述符的范围从 0 到 OPEN\_MAX - 1。
Linux 系统对文件描述符的限制数量是 1024(soft limit),hard limit 为 524288。
然而我们可以通过修改 /etc/security/limits.conf 来修改类似的 limit。

\BlockDesc{打开文件}

\begin{lstlisting}[language=C]
  #include <fcntl.h>
  int open(const char *path, int oflag, ... /* mode_t mode */ );
  int openat(int fd, const char *path, int oflag, ... /* mode_t mode */ );
  /* Both return: file descriptor if OK, −1 on error */
\end{lstlisting}

\section{网络}

\section{进程}

\section{线程}

\section{异常处理}
