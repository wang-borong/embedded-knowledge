\chapter{OS 原理与 Linux}

操作系统(Operating System, OS)是计算机系统中最重要的系统软件,它管理计算机硬件资源,为应用程序提供运行环境。
本章将深入探讨操作系统的基本原理,以及 Linux 内核的实现细节和应用开发技术。

\section{OS 原理}

操作系统是计算机硬件和应用程序之间的中间层,它提供了资源管理、进程管理、内存管理、文件系统、网络通信等核心功能。
理解操作系统原理对于深入理解 Linux 系统至关重要。

\subsection{内存管理}

内存管理是操作系统的核心功能之一,负责管理物理内存和虚拟内存,为进程提供独立的内存空间。

\subsubsection{内存管理的基本概念}

\BlockDesc{物理地址与虚拟地址}

物理地址(Physical Address)是内存硬件实际使用的地址,直接对应内存条上的物理位置。
虚拟地址(Virtual Address)是进程使用的地址空间,每个进程都有自己独立的虚拟地址空间。

虚拟内存机制的主要优势:
\begin{itemize}
  \item \textbf{地址空间隔离}:每个进程拥有独立的虚拟地址空间,进程之间不会相互干扰
  \item \textbf{内存保护}:通过页表权限位实现读写保护,防止进程访问未授权的内存区域
  \item \textbf{内存扩展}:通过交换机制,可以使用磁盘空间扩展可用内存
  \item \textbf{简化内存管理}:进程可以使用连续的虚拟地址,而物理内存可以是碎片化的
\end{itemize}

\subsubsection{分页机制}

分页(Paging)是现代操作系统最常用的内存管理机制。虚拟地址空间和物理地址空间都被划分为固定大小的页(Page),通常是 4KB。

页表(Page Table)用于将虚拟地址映射到物理地址。页表项(Page Table Entry, PTE)包含以下信息:
\begin{itemize}
  \item 物理页框号(Physical Frame Number, PFN)
  \item 访问权限位(读、写、执行)
  \item 存在位(Present bit):指示页面是否在内存中
  \item 修改位(Dirty bit):指示页面是否被修改
  \item 访问位(Accessed bit):用于页面置换算法
\end{itemize}

多级页表用于减少页表占用的内存空间。例如,在 x86-64 架构中,使用四级页表:
\begin{itemize}
  \item Page Global Directory (PGD)
  \item Page Upper Directory (PUD)
  \item Page Middle Directory (PMD)
  \item Page Table (PT)
\end{itemize}

\subsubsection{页面置换算法}

当物理内存不足时,操作系统需要将某些页面换出到磁盘,为新页面腾出空间。页面置换算法的目标是减少缺页中断(Page Fault)的次数。

常用的页面置换算法:

\begin{enumerate}
  \item \textbf{先进先出(FIFO)}
    \begin{itemize}
      \item 选择最早进入内存的页面进行置换
      \item 实现简单,但可能置换掉经常使用的页面
      \item 存在 Belady 异常:增加物理页框数可能增加缺页次数
    \end{itemize}

  \item \textbf{最优置换(Optimal)}
    \begin{itemize}
      \item 选择未来最长时间不会被访问的页面
      \item 理论上最优,但需要预知未来的访问模式,实际无法实现
      \item 主要用于评估其他算法的性能
    \end{itemize}

  \item \textbf{最近最少使用(LRU)}
    \begin{itemize}
      \item 选择最长时间未被访问的页面
      \item 基于时间局部性原理,性能接近最优算法
      \item 实现需要维护访问时间信息,开销较大
    \end{itemize}

  \item \textbf{时钟算法(Clock)}
    \begin{itemize}
      \item LRU 的近似实现,使用访问位(Reference bit)
      \item 将页面组织成环形队列,指针按顺序扫描
      \item 如果访问位为 1,清除并继续;如果为 0,选择该页面置换
    \end{itemize}

  \item \textbf{工作集算法(Working Set)}
    \begin{itemize}
      \item 基于工作集理论:进程在时间窗口 $\Delta$ 内访问的页面集合
      \item 只保留工作集中的页面在内存中
      \item 适应程序的局部性特征
    \end{itemize}
\end{enumerate}

\subsubsection{内存分配算法}

操作系统需要管理空闲内存块,为进程分配和回收内存。常用的内存分配算法:

\begin{enumerate}
  \item \textbf{首次适应(First Fit)}
    \begin{itemize}
      \item 从内存起始位置开始查找,选择第一个满足大小的空闲块
      \item 实现简单,速度快
      \item 可能在前端产生大量小碎片
    \end{itemize}

  \item \textbf{最佳适应(Best Fit)}
    \begin{itemize}
      \item 选择满足要求且最小的空闲块
      \item 减少空间浪费,但可能产生大量难以利用的小碎片
    \end{itemize}

  \item \textbf{最坏适应(Worst Fit)}
    \begin{itemize}
      \item 选择最大的空闲块
      \item 避免产生小碎片,但可能浪费大块内存
    \end{itemize}

  \item \textbf{伙伴系统(Buddy System)}
    \begin{itemize}
      \item 将内存划分为 $2^n$ 大小的块
      \item 分配时如果块太大,递归分割;回收时合并伙伴块
      \item Linux 内核使用伙伴系统管理物理页框
    \end{itemize}

  \item \textbf{Slab 分配器}
    \begin{itemize}
      \item 为频繁分配和释放的小对象提供缓存
      \item 减少内存碎片和分配开销
      \item Linux 内核使用 Slab/Slub/Slob 分配器
    \end{itemize}
\end{enumerate}

\subsection{进程与线程}

进程(Process)是程序的一次执行,是操作系统进行资源分配和调度的基本单位。
线程(Thread)是进程内的执行单元,是 CPU 调度的基本单位。

\subsubsection{进程的概念与状态}

\BlockDesc{进程}

进程是正在执行的程序的实例,包括:
\begin{itemize}
  \item 程序代码(文本段)
  \item 数据段(全局变量、静态变量)
  \item 堆(动态分配的内存)
  \item 栈(局部变量、函数调用信息)
  \item 进程控制块(Process Control Block, PCB)
\end{itemize}

进程的状态转换:

\begin{enumerate}
  \item \textbf{新建(New)}:进程正在被创建
  \item \textbf{就绪(Ready)}:进程已准备好运行,等待 CPU
  \item \textbf{运行(Running)}:进程正在 CPU 上执行
  \item \textbf{阻塞(Blocked/Waiting)}:进程等待某个事件(如 I/O 完成)
  \item \textbf{终止(Terminated)}:进程执行完毕或被终止
\end{enumerate}

进程控制块(PCB)包含进程的所有信息:
\begin{itemize}
  \item 进程标识符(PID)
  \item 进程状态
  \item 程序计数器(PC)
  \item CPU 寄存器
  \item 内存管理信息(页表、内存限制)
  \item 打开的文件列表
  \item I/O 状态信息
  \item 调度信息(优先级、调度队列)
\end{itemize}

\subsubsection{线程模型}

线程是轻量级的进程,同一进程内的线程共享地址空间和资源,但拥有独立的栈和寄存器。

线程的优势:
\begin{itemize}
  \item 创建和切换开销小
  \item 线程间通信简单(共享内存)
  \item 适合并行计算和 I/O 密集型应用
\end{itemize}

线程的实现方式:

\begin{enumerate}
  \item \textbf{用户级线程(User-Level Threads)}
    \begin{itemize}
      \item 线程管理在用户空间完成
      \item 内核不知道线程的存在
      \item 一个线程阻塞会导致整个进程阻塞
      \item 切换开销小,但不能利用多核 CPU
    \end{itemize}

  \item \textbf{内核级线程(Kernel-Level Threads)}
    \begin{itemize}
      \item 线程由内核管理
      \item 一个线程阻塞不影响其他线程
      \item 可以利用多核 CPU
      \item 切换需要系统调用,开销较大
    \end{itemize}

  \item \textbf{混合模型(Hybrid Model)}
    \begin{itemize}
      \item 结合用户级和内核级线程的优点
      \item 多个用户级线程映射到少量内核级线程
      \item 如 N:1、1:1、M:N 模型
    \end{itemize}
\end{enumerate}

\subsubsection{调度}

进程调度是操作系统的核心功能,决定哪个进程在何时使用 CPU。

\BlockDesc{调度算法}

调度算法的评价指标:
\begin{itemize}
  \item CPU 利用率:CPU 忙碌时间的百分比
  \item 吞吐量:单位时间内完成的进程数
  \item 周转时间:从提交到完成的时间
  \item 等待时间:进程在就绪队列中等待的时间
  \item 响应时间:从提交到首次响应的时间
\end{itemize}

常见的调度算法:

\begin{enumerate}
  \item \textbf{先来先服务(FCFS, First-Come First-Served)}
    \begin{itemize}
      \item 按照进程到达的顺序执行
      \item 实现简单,公平
      \item 短进程可能被长进程阻塞(护航效应)
      \item 平均等待时间可能较长
    \end{itemize}

  \item \textbf{最短作业优先(SJF, Shortest Job First)}
    \begin{itemize}
      \item 选择执行时间最短的进程
      \item 可以证明是最优的(最小平均等待时间)
      \item 需要预知执行时间,实际难以实现
      \item 可能导致长进程饥饿
    \end{itemize}

  \item \textbf{最短剩余时间优先(SRTF, Shortest Remaining Time First)}
    \begin{itemize}
      \item SJF 的可抢占版本
      \item 当新进程到达时,如果剩余时间更短,抢占当前进程
      \item 平均等待时间最小,但可能导致频繁切换
    \end{itemize}

  \item \textbf{时间片轮转(RR, Round-Robin)}
    \begin{itemize}
      \item 每个进程分配一个时间片(Time Quantum)
      \item 时间片用完后,进程回到就绪队列末尾
      \item 适合交互式系统,响应时间有保障
      \item 时间片大小需要权衡:太小导致频繁切换,太大退化为 FCFS
    \end{itemize}

  \item \textbf{优先级调度(Priority Scheduling)}
    \begin{itemize}
      \item 每个进程分配一个优先级
      \item 优先执行高优先级进程
      \item 可以是抢占式或非抢占式
      \item 可能导致低优先级进程饥饿,需要动态调整优先级
    \end{itemize}

  \item \textbf{多级队列(Multilevel Queue)}
    \begin{itemize}
      \item 将进程分为多个队列,每个队列有自己的调度算法
      \item 队列间可以设置优先级或时间片分配
      \item 适合不同类型的进程(前台交互、后台批处理)
    \end{itemize}

  \item \textbf{多级反馈队列(Multilevel Feedback Queue)}
    \begin{itemize}
      \item 允许进程在队列间移动
      \item CPU 密集型进程会移到低优先级队列
      \item I/O 密集型进程会移到高优先级队列
      \item 结合了多种算法的优点,适应不同类型的进程
    \end{itemize}
\end{enumerate}

Linux 内核使用完全公平调度器(CFS, Completely Fair Scheduler),基于虚拟运行时间(vruntime)实现公平调度。

\subsubsection{IPC}

进程间通信(Inter-Process Communication, IPC)是进程之间交换数据和同步的机制。

\BlockDesc{进程间通信方式}

\begin{enumerate}
  \item \textbf{管道(Pipe)}
    \begin{itemize}
      \item 半双工通信,数据单向流动
      \item 只能在有亲缘关系的进程间使用
      \item 使用系统调用 \texttt{pipe()} 创建
      \item 容量有限(通常 64KB),写满会阻塞
    \end{itemize}

  \item \textbf{命名管道(FIFO)}
    \begin{itemize}
      \item 通过文件系统中的特殊文件实现
      \item 可以在无亲缘关系的进程间使用
      \item 使用 \texttt{mkfifo()} 创建
    \end{itemize}

  \item \textbf{消息队列(Message Queue)}
    \begin{itemize}
      \item 消息的有序链表,存储在内核中
      \item 支持不同类型和优先级的消息
      \item 进程可以异步发送和接收消息
      \item System V IPC 或 POSIX 消息队列
    \end{itemize}

  \item \textbf{共享内存(Shared Memory)}
    \begin{itemize}
      \item 多个进程映射到同一块物理内存
      \item 最快的 IPC 方式,无需内核参与数据传输
      \item 需要同步机制(信号量、互斥锁)防止竞争
      \item System V 共享内存或 POSIX 共享内存
    \end{itemize}

  \item \textbf{信号量(Semaphore)}
    \begin{itemize}
      \item 用于进程同步和互斥
      \item 可以是二进制信号量(互斥锁)或计数信号量
      \item P 操作(wait)和 V 操作(signal)
      \item System V 信号量或 POSIX 信号量
    \end{itemize}

  \item \textbf{信号(Signal)}
    \begin{itemize}
      \item 异步通知机制,用于进程间通信和异常处理
      \item 发送信号给进程,触发信号处理函数
      \item 标准信号(1-31)和实时信号(34-64)
      \item 使用 \texttt{kill()}、\texttt{sigqueue()} 发送
    \end{itemize}

  \item \textbf{套接字(Socket)}
    \begin{itemize}
      \item 网络通信接口,也可用于本地进程间通信
      \item 支持 TCP、UDP、Unix Domain Socket
      \item 最灵活的 IPC 方式,可用于不同主机间的通信
    \end{itemize}
\end{enumerate}

\subsubsection{同步}

多进程/多线程环境下,需要同步机制来协调对共享资源的访问,防止竞争条件(Race Condition)。

\BlockDesc{同步原语}

\begin{enumerate}
  \item \textbf{互斥锁(Mutex)}
    \begin{itemize}
      \item 保证同一时刻只有一个线程访问临界区
      \item 获取锁的线程可以进入,其他线程阻塞
      \item 必须由获取锁的线程释放
      \item 防止死锁:避免嵌套锁、按固定顺序获取锁
    \end{itemize}

  \item \textbf{信号量(Semaphore)}
    \begin{itemize}
      \item 计数信号量:允许多个线程同时访问资源
      \item 二进制信号量:等价于互斥锁
      \item P 操作:等待信号量,计数减 1
      \item V 操作:释放信号量,计数加 1
    \end{itemize}

  \item \textbf{条件变量(Condition Variable)}
    \begin{itemize}
      \item 用于线程等待某个条件成立
      \item 必须与互斥锁配合使用
      \item \texttt{wait()}:释放锁并等待条件
      \item \texttt{signal()} 或 \texttt{broadcast()}:唤醒等待的线程
    \end{itemize}

  \item \textbf{读写锁(Read-Write Lock)}
    \begin{itemize}
      \item 允许多个读者同时访问,但写者独占
      \item 适合读多写少的场景
      \item 读者优先或写者优先策略
    \end{itemize}

  \item \textbf{自旋锁(Spinlock)}
    \begin{itemize}
      \item 线程在等待锁时忙等待(busy-waiting)
      \item 适合锁持有时间很短的场景
      \item 避免上下文切换的开销,但浪费 CPU
    \end{itemize}

  \item \textbf{屏障(Barrier)}
    \begin{itemize}
      \item 同步多个线程,等待所有线程到达屏障点
      \item 用于并行计算的同步点
    \end{itemize}
\end{enumerate}

经典同步问题:

\begin{enumerate}
  \item \textbf{生产者-消费者问题}
    \begin{itemize}
      \item 生产者向缓冲区写入数据,消费者从缓冲区读取
      \item 需要互斥访问缓冲区,以及条件变量协调生产/消费
    \end{itemize}

  \item \textbf{读者-写者问题}
    \begin{itemize}
      \item 多个读者可以同时读取,但写者需要独占访问
      \item 使用读写锁或信号量实现
    \end{itemize}

  \item \textbf{哲学家就餐问题}
    \begin{itemize}
      \item 演示死锁和资源竞争问题
      \item 解决方案:限制同时就餐的哲学家数量、按顺序获取筷子
    \end{itemize}
\end{enumerate}

\subsection{网络}

网络子系统是操作系统的重要组成部分,负责实现网络协议栈,提供网络通信功能。

\subsubsection{网络协议栈}

\BlockDesc{TCP/IP 协议栈}

TCP/IP 协议栈采用分层模型,每层负责不同的功能:

\begin{enumerate}
  \item \textbf{应用层(Application Layer)}
    \begin{itemize}
      \item HTTP、FTP、SSH、DNS 等应用协议
      \item 使用套接字接口与传输层交互
    \end{itemize}

  \item \textbf{传输层(Transport Layer)}
    \begin{itemize}
      \item TCP:面向连接、可靠传输、流量控制、拥塞控制
      \item UDP:无连接、不可靠、低延迟
      \item 端口号用于区分不同的应用
    \end{itemize}

  \item \textbf{网络层(Network Layer)}
    \begin{itemize}
      \item IP 协议:路由和转发
      \item ICMP:错误报告和诊断
      \item ARP:地址解析协议,将 IP 地址映射到 MAC 地址
    \end{itemize}

  \item \textbf{数据链路层(Data Link Layer)}
    \begin{itemize}
      \item 以太网(Ethernet)协议
      \item MAC 地址、帧格式、错误检测
    \end{itemize}

  \item \textbf{物理层(Physical Layer)}
    \begin{itemize}
      \item 物理传输介质和信号编码
    \end{itemize}
\end{enumerate}

\subsubsection{TCP 协议详解}

TCP(Transmission Control Protocol)提供可靠的、面向连接的字节流服务。

TCP 的特点:
\begin{itemize}
  \item \textbf{可靠性}:通过确认、重传、校验和保证数据正确传输
  \item \textbf{面向连接}:建立连接(三次握手)和释放连接(四次挥手)
  \item \textbf{流量控制}:通过滑动窗口机制控制发送速率
  \item \textbf{拥塞控制}:通过拥塞窗口避免网络过载
  \item \textbf{全双工}:支持双向数据传输
\end{itemize}

TCP 连接建立(三次握手):
\begin{enumerate}
  \item 客户端发送 SYN,序列号为 $x$
  \item 服务器响应 SYN-ACK,序列号为 $y$,确认号为 $x+1$
  \item 客户端发送 ACK,确认号为 $y+1$
\end{enumerate}

TCP 连接释放(四次挥手):
\begin{enumerate}
  \item 主动关闭方发送 FIN
  \item 被动关闭方发送 ACK
  \item 被动关闭方发送 FIN
  \item 主动关闭方发送 ACK,进入 TIME\_WAIT 状态
\end{enumerate}

TCP 拥塞控制算法:
\begin{itemize}
  \item \textbf{慢启动(Slow Start)}:拥塞窗口指数增长
  \item \textbf{拥塞避免(Congestion Avoidance)}:拥塞窗口线性增长
  \item \textbf{快速重传(Fast Retransmit)}:收到 3 个重复 ACK 立即重传
  \item \textbf{快速恢复(Fast Recovery)}:快速重传后的拥塞窗口调整
\end{itemize}

\subsubsection{UDP 协议}

UDP(User Datagram Protocol)提供无连接的、不可靠的数据报服务。

UDP 的特点:
\begin{itemize}
  \item 无连接:无需建立连接,直接发送数据
  \item 不可靠:不保证数据到达,不保证顺序
  \item 低延迟:无连接建立和确认开销
  \item 简单:协议头部小,处理开销低
\end{itemize}

UDP 适用于:
\begin{itemize}
  \item 实时应用(视频、音频流)
  \item DNS 查询
  \item 广播和多播
  \item 对延迟敏感的应用
\end{itemize}

\subsubsection{TCP/IP 协议栈总结}

TCP/IP 协议栈是互联网的基础,理解其工作原理对于网络编程和系统优化至关重要。

关键概念:
\begin{itemize}
  \item \textbf{封装}:数据从上层向下传递时,每层添加头部信息
  \item \textbf{解封装}:数据从下层向上传递时,每层移除头部信息
  \item \textbf{端口号}:16 位整数,用于标识应用进程
  \item \textbf{套接字}:IP 地址和端口号的组合,唯一标识网络连接
  \item \textbf{路由}:根据目标 IP 地址选择转发路径
\end{itemize}

\subsection{文件系统}

文件系统是操作系统用于组织和管理存储设备上数据的方法。

\subsubsection{文件系统的基本概念}

\BlockDesc{文件系统}

文件系统的主要功能:
\begin{itemize}
  \item 文件的组织和管理
  \item 目录结构
  \item 文件访问控制
  \item 磁盘空间管理
  \item 数据一致性保证
\end{itemize}

文件系统的层次结构:
\begin{itemize}
  \item \textbf{逻辑文件系统}:处理用户请求,管理目录结构
  \item \textbf{文件组织模块}:管理文件逻辑块和物理块的映射
  \item \textbf{基本文件系统}:访问物理块
  \item \textbf{I/O 控制}:设备驱动程序和中断处理
\end{itemize}

\subsubsection{文件分配方法}

\begin{enumerate}
  \item \textbf{连续分配(Contiguous Allocation)}
    \begin{itemize}
      \item 文件占用连续的磁盘块
      \item 访问速度快,支持顺序和随机访问
      \item 容易产生外部碎片,文件扩展困难
    \end{itemize}

  \item \textbf{链接分配(Linked Allocation)}
    \begin{itemize}
      \item 文件由链表连接的不连续块组成
      \item 无外部碎片,文件扩展容易
      \item 只支持顺序访问,指针占用空间,可靠性差
    \end{itemize}

  \item \textbf{索引分配(Indexed Allocation)}
    \begin{itemize}
      \item 使用索引块存储文件所有块的指针
      \item 支持随机访问,无外部碎片
      \item 索引块占用空间,大文件需要多级索引
    \end{itemize}
\end{enumerate}

\subsubsection{目录结构}

\begin{enumerate}
  \item \textbf{单级目录}:所有文件在一个目录中,简单但查找效率低
  \item \textbf{两级目录}:为每个用户创建独立目录
  \item \textbf{树形目录}:层次化结构,支持子目录
  \item \textbf{无环图目录}:支持文件共享,通过链接实现
\end{enumerate}

\subsubsection{磁盘调度算法}

磁盘 I/O 是系统性能的瓶颈,调度算法可以减少寻道时间。

\begin{enumerate}
  \item \textbf{FCFS}:按请求顺序服务,简单但效率低
  \item \textbf{SSTF(最短寻道时间优先)}:选择距离当前磁道最近的请求
  \item \textbf{SCAN(电梯算法)}:磁头在一个方向移动,处理所有请求后反向
  \item \textbf{C-SCAN}:SCAN 的改进,只在一个方向服务请求
  \item \textbf{LOOK}:SCAN 的改进,到达最后一个请求后立即反向
  \item \textbf{C-LOOK}:LOOK 的改进,只在一个方向服务
\end{enumerate}

\subsection{安全管理}

操作系统安全涉及访问控制、身份认证、加密、审计等多个方面。

\subsubsection{访问控制}

\BlockDesc{访问控制模型}

\begin{enumerate}
  \item \textbf{自主访问控制(DAC, Discretionary Access Control)}
    \begin{itemize}
      \item 文件所有者决定访问权限
      \item Linux 的文件权限:所有者、组、其他用户
      \item 权限类型:读(r)、写(w)、执行(x)
    \end{itemize}

  \item \textbf{强制访问控制(MAC, Mandatory Access Control)}
    \begin{itemize}
      \item 系统策略决定访问权限,用户无法修改
      \item SELinux、AppArmor 等实现
      \item 基于安全标签和策略规则
    \end{itemize}

  \item \textbf{基于角色的访问控制(RBAC, Role-Based Access Control)}
    \begin{itemize}
      \item 用户被分配角色,角色拥有权限
      \item 简化权限管理
    \end{itemize}
\end{enumerate}

\subsubsection{身份认证}

\begin{itemize}
  \item \textbf{密码认证}:用户名和密码
  \item \textbf{公钥认证}:SSH 密钥对
  \item \textbf{多因素认证}:结合密码、令牌、生物特征
  \item \textbf{单点登录(SSO)}:一次认证访问多个系统
\end{itemize}

\subsubsection{加密}

\begin{itemize}
  \item \textbf{对称加密}:AES、DES,加密解密使用相同密钥
  \item \textbf{非对称加密}:RSA、ECC,公钥加密私钥解密
  \item \textbf{哈希函数}:SHA-256、MD5,用于数据完整性验证
  \item \textbf{数字签名}:使用私钥签名,公钥验证
\end{itemize}

\subsection{系统调用}

系统调用(System Call)是用户程序请求操作系统服务的接口。

\BlockDesc{系统调用}

系统调用的作用:
\begin{itemize}
  \item 提供用户空间和内核空间的接口
  \item 保护系统资源,防止用户程序直接访问硬件
  \item 提供统一的系统服务接口
\end{itemize}

系统调用的实现机制:
\begin{enumerate}
  \item 用户程序调用库函数(如 \texttt{open()})
  \item 库函数设置系统调用号和参数
  \item 触发软中断或使用 \texttt{syscall} 指令进入内核
  \item 内核根据系统调用号查找处理函数
  \item 执行系统调用,返回结果
  \item 返回到用户空间
\end{enumerate}

常见的系统调用类别:
\begin{itemize}
  \item 进程控制:\texttt{fork}、\texttt{exec}、\texttt{exit}、\texttt{wait}
  \item 文件操作:\texttt{open}、\texttt{read}、\texttt{write}、\texttt{close}
  \item 设备管理:\texttt{ioctl}、\texttt{read}、\texttt{write}
  \item 信息维护:\texttt{getpid}、\texttt{alarm}、\texttt{sleep}
  \item 通信:\texttt{pipe}、\texttt{shmget}、\texttt{msgget}
\end{itemize}

\section{Linux 内核}

Linux 内核是 Linux 操作系统的核心,负责管理系统资源、提供系统调用接口、实现各种内核子系统。
本节深入探讨 Linux 内核的架构和实现细节。

\subsection{启动}

Linux 系统的启动是一个复杂的过程,涉及硬件初始化、引导加载程序、内核启动、用户空间初始化等多个阶段。

\subsubsection{启动流程分析}

\BlockDesc{Linux 启动流程}

Linux 系统的启动流程:

\begin{enumerate}
  \item \textbf{BIOS/UEFI 阶段}
    \begin{itemize}
      \item 硬件上电,CPU 从固定地址开始执行
      \item BIOS/UEFI 进行硬件自检(POST)
      \item 查找可启动设备(硬盘、USB、网络等)
      \item 加载并执行引导加载程序(Boot Loader)
    \end{itemize}

  \item \textbf{引导加载程序阶段}
    \begin{itemize}
      \item GRUB(Grand Unified Bootloader)是最常用的引导加载程序
      \item 加载内核镜像和初始 RAM 文件系统(initramfs)
      \item 设置内核启动参数
      \item 跳转到内核入口点
    \end{itemize}

  \item \textbf{内核初始化阶段}
    \begin{itemize}
      \item 内核入口:\texttt{start\_kernel()}
      \item 初始化各种内核子系统:
        \begin{itemize}
          \item 内存管理子系统
          \item 进程调度器
          \item 中断处理系统
          \item 设备驱动框架
          \item 文件系统
        \end{itemize}
      \item 挂载根文件系统
      \item 启动第一个用户空间进程(init)
    \end{itemize}

  \item \textbf{用户空间初始化}
    \begin{itemize}
      \item init 进程(PID 1)启动
      \item 传统系统使用 SysV init 或 systemd
      \item 执行启动脚本,启动各种系统服务
      \item 进入多用户模式或图形界面
    \end{itemize}
\end{enumerate}

内核启动的关键函数调用链:
\begin{lstlisting}[language=C]
start_kernel()
  -> setup_arch()
  -> setup_command_line()
  -> setup_per_cpu_areas()
  -> smp_prepare_boot_cpu()
  -> sched_init()
  -> preempt_disable()
  -> init_IRQ()
  -> init_timers()
  -> hrtimers_init()
  -> softirq_init()
  -> timekeeping_init()
  -> time_init()
  -> console_init()
  -> lockdep_init()
  -> cgroup_init()
  -> page_alloc_init()
  -> vfs_caches_init()
  -> signals_init()
  -> proc_root_init()
  -> rest_init()
    -> kernel_thread(kernel_init)
    -> kernel_thread(kthreadd)
\end{lstlisting}

\subsection{内存管理}

Linux 内核的内存管理子系统负责管理物理内存和虚拟内存,为内核和用户进程提供内存分配服务。

\subsubsection{物理内存管理}

Linux 使用伙伴系统(Buddy System)管理物理页框。

伙伴系统的特点:
\begin{itemize}
  \item 将物理内存划分为 $2^n$ 大小的块(order 0-10,对应 4KB 到 4MB)
  \item 分配时如果块太大,递归分割成两个伙伴块
  \item 释放时如果伙伴块也是空闲的,合并成更大的块
  \item 减少内存碎片,提高分配效率
\end{itemize}

每个内存区域(Zone)维护多个空闲链表,对应不同的 order:
\begin{itemize}
  \item \texttt{ZONE\_DMA}:DMA 可访问的内存(< 16MB)
  \item \texttt{ZONE\_DMA32}:32 位 DMA 可访问的内存(< 4GB)
  \item \texttt{ZONE\_NORMAL}:正常映射的内存
  \item \texttt{ZONE\_HIGHMEM}:高端内存(32 位系统)
\end{itemize}

\subsubsection{虚拟内存管理}

Linux 使用多级页表实现虚拟地址到物理地址的转换。

页表结构(x86-64):
\begin{itemize}
  \item 四级页表:PGD、PUD、PMD、PT
  \item 每级页表有 512 个表项(9 位索引)
  \item 页表项大小:8 字节
  \item 支持 4KB、2MB、1GB 页面大小
\end{itemize}

内存映射:
\begin{itemize}
  \item \textbf{匿名映射}:用于堆、栈等,不与文件关联
  \item \textbf{文件映射}:将文件映射到虚拟地址空间
  \item \textbf{共享映射}:多个进程共享同一物理页面
  \item \textbf{私有映射}:写时复制(Copy-on-Write)
\end{itemize}

\subsubsection{内存分配接口}

内核提供多种内存分配接口:

\begin{enumerate}
  \item \textbf{页分配器}
    \begin{itemize}
      \item \texttt{alloc\_pages()}:分配一个或多个物理页
      \item \texttt{get\_free\_page()}:分配一个物理页并返回虚拟地址
      \item 用于大块内存分配
    \end{itemize}

  \item \textbf{Slab 分配器}
    \begin{itemize}
      \item \texttt{kmalloc()}:分配小块内存(< 128KB)
      \item \texttt{kfree()}:释放内存
      \item 使用 Slab 缓存减少碎片和分配开销
    \end{itemize}

  \item \textbf{vmalloc}
    \begin{itemize}
      \item \texttt{vmalloc()}:分配虚拟地址连续但物理地址可能不连续的内存
      \item 用于需要大块连续虚拟地址的场景
      \item 开销较大,因为需要建立页表映射
    \end{itemize}
\end{enumerate}

\subsubsection{页面回收}

当系统内存不足时,内核需要回收页面。

页面回收策略:
\begin{itemize}
  \item \textbf{直接回收}:分配内存时如果内存不足,立即尝试回收
  \item \textbf{kswapd}:后台线程定期回收页面
  \item \textbf{OOM Killer}:内存严重不足时,杀死占用内存最多的进程
\end{itemize}

页面回收算法:
\begin{itemize}
  \item 优先回收不活跃的匿名页面(交换到 swap)
  \item 优先回收文件缓存页面(可以直接丢弃)
  \item 使用 LRU 链表管理页面
\end{itemize}

\subsection{task}

Linux 内核使用 task\_struct 结构体表示进程和线程,统一称为任务(Task)。

\subsubsection{task\_struct 结构}

\BlockDesc{task\_struct}

task\_struct 是 Linux 内核中最重要的数据结构之一,包含任务的所有信息:

\begin{lstlisting}[language=C]
struct task_struct {
    volatile long state;        // 进程状态
    void *stack;                // 内核栈
    unsigned int flags;         // 进程标志
    int prio;                   // 动态优先级
    int static_prio;            // 静态优先级
    struct mm_struct *mm;       // 内存描述符
    struct mm_struct *active_mm; // 活动内存描述符
    pid_t pid;                  // 进程 ID
    pid_t tgid;                 // 线程组 ID
    struct task_struct *parent; // 父进程
    struct list_head children;  // 子进程链表
    struct files_struct *files; // 打开的文件
    struct fs_struct *fs;       // 文件系统信息
    // ... 更多字段
};
\end{lstlisting}

进程状态:
\begin{itemize}
  \item \texttt{TASK\_RUNNING}:可运行状态
  \item \texttt{TASK\_INTERRUPTIBLE}:可中断睡眠
  \item \texttt{TASK\_UNINTERRUPTIBLE}:不可中断睡眠
  \item \texttt{TASK\_STOPPED}:停止状态
  \item \texttt{TASK\_TRACED}:被跟踪状态
  \item \texttt{EXIT\_ZOMBIE}:僵尸状态
  \item \texttt{EXIT\_DEAD}:死亡状态
\end{itemize}

\subsubsection{进程创建}

Linux 使用 \texttt{fork()} 系统调用创建新进程。

\texttt{fork()} 的实现:
\begin{enumerate}
  \item 分配新的 task\_struct
  \item 复制父进程的内存空间(写时复制)
  \item 复制父进程的文件描述符
  \item 设置新的 PID
  \item 将新进程加入运行队列
\end{enumerate}

写时复制(Copy-on-Write, COW):
\begin{itemize}
  \item 子进程共享父进程的物理页面
  \item 页面标记为只读
  \item 当任一进程写入时,触发页错误,分配新页面并复制数据
  \item 减少内存复制开销
\end{itemize}

\subsubsection{进程调度}

Linux 使用完全公平调度器(CFS)进行进程调度。

CFS 的核心思想:
\begin{itemize}
  \item 维护虚拟运行时间(vruntime)
  \item 选择 vruntime 最小的进程运行
  \item 保证所有进程公平分享 CPU 时间
\end{itemize}

调度实体(sched\_entity):
\begin{itemize}
  \item 每个任务有一个调度实体
  \item 使用红黑树组织,vruntime 作为键
  \item 最左侧节点是下一个要运行的进程
\end{itemize}

调度策略:
\begin{itemize}
  \item \texttt{SCHED\_NORMAL}:普通进程,使用 CFS
  \item \texttt{SCHED\_FIFO}:实时进程,先进先出
  \item \texttt{SCHED\_RR}:实时进程,时间片轮转
  \item \texttt{SCHED\_BATCH}:批处理进程
  \item \texttt{SCHED\_IDLE}:低优先级进程
\end{itemize}

\subsubsection{中断和异常处理}

中断是硬件或软件发出的信号,要求 CPU 立即处理。

中断类型:
\begin{itemize}
  \item \textbf{硬件中断}:由硬件设备产生(IRQ)
  \item \textbf{软件中断}:由软件产生(系统调用、异常)
  \item \textbf{异常}:CPU 执行指令时产生的错误(页错误、除零等)
\end{itemize}

中断处理流程:
\begin{enumerate}
  \item 保存当前上下文(寄存器)
  \item 切换到内核栈
  \item 调用中断处理函数
  \item 执行中断服务例程(ISR)
  \item 恢复上下文,返回用户空间
\end{enumerate}

中断下半部机制:
\begin{itemize}
  \item \textbf{软中断(Softirq)}:可重入,同一 CPU 上串行执行
  \item \textbf{任务队列(Tasklet)}:基于软中断,不可重入
  \item \textbf{工作队列(Workqueue)}:在进程上下文中执行,可以睡眠
\end{itemize}

\subsection{网络}

Linux 网络子系统实现了完整的 TCP/IP 协议栈,提供高性能的网络通信能力。

\subsubsection{网络协议栈架构}

Linux 网络协议栈采用分层设计:

\begin{enumerate}
  \item \textbf{套接字层(Socket Layer)}
    \begin{itemize}
      \item 提供 BSD 套接字接口
      \item 管理套接字数据结构
      \item 处理系统调用(socket、bind、listen、accept 等)
    \end{itemize}

  \item \textbf{传输层(Transport Layer)}
    \begin{itemize}
      \item TCP:可靠传输、流量控制、拥塞控制
      \item UDP:无连接数据报传输
      \item 管理连接状态、端口号
    \end{itemize}

  \item \textbf{网络层(Network Layer)}
    \begin{itemize}
      \item IP 协议:路由和转发
      \item 路由表管理
      \item 分片和重组
    \end{itemize}

  \item \textbf{数据链路层(Data Link Layer)}
    \begin{itemize}
      \item 网络设备驱动
      \item 以太网协议处理
      \item MAC 地址管理
    \end{itemize}
\end{enumerate}

\subsubsection{套接字缓冲区(sk\_buff)}

\BlockDesc{sk\_buff}

sk\_buff 是 Linux 网络子系统的核心数据结构,用于在网络层之间传递数据包。

sk\_buff 的特点:
\begin{itemize}
  \item 包含数据包的所有信息(头部、数据、元数据)
  \item 支持头部添加和删除(push/pull)
  \item 支持数据包分片和合并
  \item 引用计数,支持零拷贝
\end{itemize}

sk\_buff 的主要字段:
\begin{lstlisting}[language=C]
struct sk_buff {
    struct sk_buff *next;      // 下一个缓冲区
    struct sk_buff *prev;      // 上一个缓冲区
    struct sock *sk;           // 所属套接字
    unsigned int len;          // 数据长度
    unsigned int data_len;     // 分片数据长度
    __u16 transport_header;   // 传输层头部偏移
    __u16 network_header;      // 网络层头部偏移
    __u16 mac_header;          // MAC 层头部偏移
    unsigned char *head;       // 数据区起始地址
    unsigned char *data;       // 当前数据指针
    unsigned char *tail;       // 数据区结束地址
    unsigned char *end;        // 缓冲区结束地址
};
\end{lstlisting}

\subsubsection{网络设备}

网络设备(net\_device)表示一个网络接口。

网络设备的操作:
\begin{itemize}
  \item \texttt{open()}:打开设备
  \item \texttt{stop()}:停止设备
  \item \texttt{hard\_start\_xmit()}:发送数据包
  \item \texttt{get\_stats()}:获取统计信息
\end{itemize}

网络设备驱动需要:
\begin{itemize}
  \item 初始化硬件
  \item 注册中断处理函数
  \item 实现数据包发送和接收
  \item 处理网络设备状态变化
\end{itemize}

\subsection{驱动架构及子系统}

Linux 内核采用统一的设备驱动模型,提供标准化的驱动接口和框架。

\subsubsection{设备模型}

Linux 设备模型的核心概念:

\begin{enumerate}
  \item \textbf{设备(Device)}
    \begin{itemize}
      \item 表示系统中的硬件设备
      \item 包含设备信息、驱动指针、父设备等
    \end{itemize}

  \item \textbf{驱动(Driver)}
    \begin{itemize}
      \item 实现设备的操作接口
      \item 包含 probe、remove、suspend、resume 等函数
    \end{itemize}

  \item \textbf{总线(Bus)}
    \begin{itemize}
      \item 连接设备和驱动
      \item 管理设备枚举和驱动匹配
      \item 如 PCI、USB、I2C、SPI 等
    \end{itemize}

  \item \textbf{类(Class)}
    \begin{itemize}
      \item 将功能相似的设备归类
      \item 在 /sys/class 下创建统一接口
    \end{itemize}
\end{enumerate}

\subsubsection{UART}

UART(Universal Asynchronous Receiver/Transmitter)是异步串行通信接口。

UART 驱动架构:
\begin{itemize}
  \item \textbf{TTY 子系统}:提供统一的终端接口
  \item \textbf{串口核心层}:管理串口设备和驱动
  \item \textbf{串口驱动层}:实现具体硬件的操作
\end{itemize}

UART 驱动需要实现:
\begin{itemize}
  \item 波特率设置
  \item 数据位、停止位、校验位配置
  \item 发送和接收数据
  \item 中断处理(发送完成、接收数据)
  \item 流控制(RTS/CTS、XON/XOFF)
\end{itemize}

\subsubsection{Timer}

定时器是内核的重要功能,用于延迟执行、超时处理、周期性任务等。

Linux 定时器类型:
\begin{enumerate}
  \item \textbf{高精度定时器(hrtimer)}
    \begin{itemize}
      \item 纳秒级精度
      \item 基于红黑树组织
      \item 用于需要高精度的场景
    \end{itemize}

  \item \textbf{传统定时器(timer\_list)}
    \begin{itemize}
      \item 基于 jiffies
      \item 精度较低(通常 1ms 或 10ms)
      \item 用于一般延迟和超时
    \end{itemize}

  \item \textbf{工作队列定时器(delayed\_work)}
    \begin{itemize}
      \item 在进程上下文中执行
      \item 可以睡眠和阻塞
    \end{itemize}
\end{enumerate}

定时器实现:
\begin{lstlisting}[language=C]
struct timer_list {
    struct hlist_node entry;
    unsigned long expires;      // 到期时间
    void (*function)(unsigned long); // 回调函数
    unsigned long data;         // 回调参数
    u32 flags;
};
\end{lstlisting}

\subsubsection{WDT}

看门狗定时器(Watchdog Timer, WDT)用于检测系统是否正常运行。

看门狗工作原理:
\begin{itemize}
  \item 硬件定时器定期递减计数
  \item 软件需要定期"喂狗"(重置计数器)
  \item 如果超时未喂狗,系统复位或触发中断
\end{itemize}

Linux 看门狗框架:
\begin{itemize}
  \item 统一的看门狗接口(/dev/watchdog)
  \item 支持软件看门狗和硬件看门狗
  \item 自动喂狗机制(如果启用)
  \item 看门狗设备驱动实现具体硬件操作
\end{itemize}

\subsubsection{I2C/I3C}

I2C(Inter-Integrated Circuit)是两线制串行总线,用于连接低速外设。

I2C 总线特点:
\begin{itemize}
  \item 两线:SDA(数据线)和 SCL(时钟线)
  \item 多主多从,支持仲裁
  \item 7 位或 10 位地址
  \item 标准模式(100kHz)、快速模式(400kHz)、高速模式(3.4MHz)
\end{itemize}

Linux I2C 子系统:
\begin{itemize}
  \item \textbf{I2C 核心}:总线管理、设备注册
  \item \textbf{I2C 适配器驱动}:实现具体 I2C 控制器
  \item \textbf{I2C 设备驱动}:实现具体 I2C 设备
\end{itemize}

I3C(Improved Inter-Integrated Circuit)是 I2C 的改进版本:
\begin{itemize}
  \item 更高的速度(最高 12.5MHz)
  \item 支持动态地址分配
  \item 向后兼容 I2C
  \item 支持带内中断(IBI)
\end{itemize}

\subsubsection{SPI/QSPI}

SPI(Serial Peripheral Interface)是同步串行接口,用于连接高速外设。

SPI 总线特点:
\begin{itemize}
  \item 四线:MOSI、MISO、SCK、CS
  \item 全双工通信
  \item 主从模式,支持多从设备
  \item 时钟速度可达几十 MHz
\end{itemize}

Linux SPI 子系统:
\begin{itemize}
  \item \textbf{SPI 核心}:总线管理和消息传递
  \item \textbf{SPI 控制器驱动}:实现具体 SPI 控制器
  \item \textbf{SPI 设备驱动}:实现具体 SPI 设备
\end{itemize}

QSPI(Quad SPI)是 SPI 的扩展:
\begin{itemize}
  \item 使用 4 条数据线(IO0-IO3)提高速度
  \item 常用于 Flash 存储器
  \item 支持单线、双线、四线模式
\end{itemize}

\subsubsection{RS232/RS422/RS485}

这些是串行通信标准,通常通过 UART 实现。

\begin{enumerate}
  \item \textbf{RS232}
    \begin{itemize}
      \item 点对点通信
      \item 单端信号,传输距离短(< 15 米)
      \item 常用电压:-15V 到 +15V
    \end{itemize}

  \item \textbf{RS422}
    \begin{itemize}
      \item 差分信号,抗干扰能力强
      \item 点对多点(一主多从)
      \item 传输距离可达 1200 米
    \end{itemize}

  \item \textbf{RS485}
    \begin{itemize}
      \item 差分信号,与 RS422 类似
      \item 支持多主多从
      \item 需要终端电阻
      \item 工业现场总线常用
    \end{itemize}
\end{enumerate}

在 Linux 中,这些协议通常通过 UART 驱动和相应的电平转换芯片实现。

\subsubsection{CAN}

CAN(Controller Area Network)是用于汽车和工业控制的总线协议。

CAN 总线特点:
\begin{itemize}
  \item 多主总线,支持仲裁
  \item 基于消息的通信,无地址概念
  \item 错误检测和恢复机制
  \item 实时性要求高
\end{itemize}

Linux CAN 子系统:
\begin{itemize}
  \item \textbf{SocketCAN}:基于套接字的 CAN 接口
  \item \textbf{CAN 核心}:协议实现和消息路由
  \item \textbf{CAN 设备驱动}:实现具体 CAN 控制器
\end{itemize}

CAN 消息格式:
\begin{itemize}
  \item 标准帧(11 位 ID)或扩展帧(29 位 ID)
  \item 数据长度:0-8 字节
  \item 包含 CRC 校验
\end{itemize}

\subsubsection{Ethernet}

以太网是局域网的主要技术,Linux 提供了完整的以太网支持。

以太网驱动架构:
\begin{itemize}
  \item \textbf{网络设备接口}:net\_device
  \item \textbf{网络协议栈}:处理以太网帧
  \item \textbf{以太网驱动}:实现具体网卡硬件操作
\end{itemize}

以太网驱动需要实现:
\begin{itemize}
  \item 网卡初始化和配置
  \item 数据包发送(hard\_start\_xmit)
  \item 数据包接收(中断处理)
  \item 统计信息收集
  \item 链路状态管理
\end{itemize}

网络接口操作:
\begin{itemize}
  \item \texttt{ifconfig}:配置网络接口
  \item \texttt{ethtool}:查看和配置网卡参数
  \item \texttt{ip}:现代网络配置工具
\end{itemize}

\subsubsection{USB}

USB(Universal Serial Bus)是通用的串行总线标准。

USB 架构:
\begin{itemize}
  \item \textbf{USB 主机控制器}:管理 USB 总线
  \item \textbf{USB 设备}:连接到总线的设备
  \item \textbf{USB 驱动}:设备驱动和主机控制器驱动
\end{itemize}

Linux USB 子系统:
\begin{itemize}
  \item \textbf{USB 核心}:总线管理、设备枚举
  \item \textbf{USB 主机控制器驱动}:UHCI、OHCI、EHCI、XHCI
  \item \textbf{USB 设备驱动}:实现具体 USB 设备功能
  \item \textbf{USB Gadget 框架}:USB 设备模式支持
\end{itemize}

USB 设备类型:
\begin{itemize}
  \item HID(人机接口设备):键盘、鼠标
  \item Mass Storage:U 盘、移动硬盘
  \item Serial:USB 转串口
  \item Network:USB 网卡
  \item Audio:USB 音频设备
\end{itemize}

\subsubsection{PCI/PCIe}

PCI(Peripheral Component Interconnect)和 PCIe(PCI Express)是计算机扩展总线标准。

PCI/PCIe 特点:
\begin{itemize}
  \item 即插即用(Plug and Play)
  \item 自动资源分配(I/O 地址、中断)
  \item 支持热插拔(PCIe)
  \item 高速数据传输
\end{itemize}

Linux PCI 子系统:
\begin{itemize}
  \item \textbf{PCI 核心}:总线枚举、资源配置
  \item \textbf{PCI 驱动}:实现具体 PCI 设备功能
  \item \textbf{PCIe 支持}:高级特性(MSI、MSI-X、热插拔)
\end{itemize}

PCI 配置空间:
\begin{itemize}
  \item 256 字节配置空间
  \item 包含设备 ID、厂商 ID、资源需求
  \item 系统自动分配资源
\end{itemize}

\subsection{文件系统}

Linux 支持多种文件系统,从传统的 ext2 到现代的 btrfs、xfs 等。

\subsubsection{虚拟文件系统(VFS)}

\BlockDesc{VFS}

VFS 是 Linux 文件系统的抽象层,提供统一的文件系统接口。

VFS 的核心数据结构:
\begin{itemize}
  \item \textbf{super\_block}:文件系统超级块
  \item \textbf{inode}:文件索引节点
  \item \textbf{dentry}:目录项缓存
  \item \textbf{file}:打开的文件
\end{itemize}

VFS 的操作接口:
\begin{itemize}
  \item \textbf{super\_operations}:文件系统级操作
  \item \textbf{inode\_operations}:inode 操作
  \item \textbf{file\_operations}:文件操作(read、write 等)
  \item \textbf{dentry\_operations}:目录项操作
\end{itemize}

\subsubsection{ext2 简介}

ext2(Second Extended Filesystem)是 Linux 的传统文件系统。

ext2 的磁盘布局:
\begin{itemize}
  \item \textbf{引导块}:第一个块,包含引导程序
  \item \textbf{超级块}:文件系统元数据
  \item \textbf{块组描述符表}:描述每个块组
  \item \textbf{块位图}:记录数据块使用情况
  \item \textbf{inode 位图}:记录 inode 使用情况
  \item \textbf{inode 表}:存储 inode
  \item \textbf{数据块}:存储文件数据
\end{itemize}

ext2 的特点:
\begin{itemize}
  \item 使用索引节点(inode)管理文件
  \item 支持 12 个直接块、1 个间接块、1 个二级间接块、1 个三级间接块
  \item 最大文件大小:4TB(块大小 4KB)
  \item 最大文件系统大小:16TB
\end{itemize}

ext3 和 ext4 的改进:
\begin{itemize}
  \item ext3:添加日志功能,提高可靠性
  \item ext4:扩展性改进、延迟分配、多块分配等
\end{itemize}

\subsubsection{其他文件系统}

\begin{enumerate}
  \item \textbf{btrfs}
    \begin{itemize}
      \item 写时复制(COW)文件系统
      \item 支持快照、压缩、子卷
      \item 内置 RAID 和校验和
    \end{itemize}

  \item \textbf{xfs}
    \begin{itemize}
      \item 高性能文件系统
      \item 适合大文件和大文件系统
      \item 延迟分配、在线碎片整理
    \end{itemize}

  \item \textbf{tmpfs}
    \begin{itemize}
      \item 基于内存的文件系统
      \item 用于 /tmp、/dev/shm
      \item 数据存储在 RAM 中
    \end{itemize}

  \item \textbf{procfs 和 sysfs}
    \begin{itemize}
      \item 虚拟文件系统,不占用磁盘空间
      \item procfs:进程和系统信息
      \item sysfs:设备和驱动信息
    \end{itemize}
\end{enumerate}

\section{Linux 应用开发}

Linux 应用开发涉及系统编程、网络编程、多线程编程等多个方面。
本节介绍 Linux 应用开发的核心技术和最佳实践。

\subsection{文件及 IO}

一切皆文件,所有在 Linux/UNIX 系统上打开的文件都用文件描述符表示,并且描述符是非零的。
当我们创建或打开一个文件,内核会返回一个文件描述符给到当前进程。
当我们读取或写入数据到一个文件中,我们便使用相应的文件描述符指定这个文件。

在类 UNIX 系统中约定俗成,文件描述符 0 代表一个进程的标准输入,1 代表标准输出,2 代表标准错误输出。
为了兼容性,应该使用 STDIN\_FILENO、STDOUT\_FILENO 和 STDERR\_FILENO 来代替 0、1 和 2。

文件描述符的范围从 0 到 OPEN\_MAX - 1。
Linux 系统对文件描述符的限制数量是 1024(soft limit),hard limit 为 524288。
然而我们可以通过修改 /etc/security/limits.conf 来修改类似的 limit。

\BlockDesc{打开文件}

\begin{lstlisting}[language=C]
#include <fcntl.h>
int open(const char *path, int oflag, ... /* mode_t mode */ );
int openat(int fd, const char *path, int oflag, ... /* mode_t mode */ );
/* Both return: file descriptor if OK, −1 on error */
\end{lstlisting}

\subsubsection{文件操作}

\BlockDesc{读取和写入}

\begin{lstlisting}[language=C]
#include <unistd.h>
ssize_t read(int fd, void *buf, size_t nbytes);
ssize_t write(int fd, const void *buf, size_t nbytes);
/* Both return: number of bytes read/written if OK, −1 on error */
\end{lstlisting}

\texttt{read()} 和 \texttt{write()} 是基本的文件 I/O 函数:
\begin{itemize}
  \item \texttt{read()} 从文件描述符读取数据
  \item \texttt{write()} 向文件描述符写入数据
  \item 返回值是实际读取/写入的字节数
  \item 可能返回的值小于请求的字节数(部分读取/写入)
\end{itemize}

\BlockDesc{文件定位}

\begin{lstlisting}[language=C]
#include <unistd.h>
off_t lseek(int fd, off_t offset, int whence);
/* Returns: new file offset if OK, −1 on error */
\end{lstlisting}

\texttt{lseek()} 用于改变文件的当前偏移量:
\begin{itemize}
  \item \texttt{SEEK\_SET}:从文件开始偏移
  \item \texttt{SEEK\_CUR}:从当前位置偏移
  \item \texttt{SEEK\_END}:从文件末尾偏移
\end{itemize}

\subsubsection{文件描述符操作}

\BlockDesc{复制文件描述符}

\begin{lstlisting}[language=C]
#include <unistd.h>
int dup(int fd);
int dup2(int fd, int fd2);
/* Both return: new file descriptor if OK, −1 on error */
\end{lstlisting}

\texttt{dup()} 和 \texttt{dup2()} 用于复制文件描述符:
\begin{itemize}
  \item \texttt{dup()} 返回最小的未使用文件描述符
  \item \texttt{dup2()} 将 \texttt{fd} 复制到 \texttt{fd2},如果 \texttt{fd2} 已打开则先关闭
  \item 复制的文件描述符共享同一个文件表项
\end{itemize}

\subsubsection{文件状态}

\BlockDesc{获取文件信息}

\begin{lstlisting}[language=C]
#include <sys/stat.h>
int stat(const char *restrict pathname, struct stat *restrict buf);
int fstat(int fd, struct stat *buf);
int lstat(const char *restrict pathname, struct stat *restrict buf);
/* All return: 0 if OK, −1 on error */
\end{lstlisting}

\texttt{stat} 结构体包含文件的详细信息:
\begin{itemize}
  \item 文件类型和权限
  \item 文件大小
  \item 时间戳(访问、修改、状态改变)
  \item 设备号和 inode 号
\end{itemize}

\subsubsection{目录操作}

\BlockDesc{目录操作}

\begin{lstlisting}[language=C]
#include <dirent.h>
DIR *opendir(const char *pathname);
struct dirent *readdir(DIR *dp);
int closedir(DIR *dp);
\end{lstlisting}

目录操作函数:
\begin{itemize}
  \item \texttt{opendir()} 打开目录,返回 DIR 指针
  \item \texttt{readdir()} 读取目录项,返回 dirent 结构
  \item \texttt{closedir()} 关闭目录
\end{itemize}

\subsubsection{IO 多路复用}

\BlockDesc{select/poll/epoll}

当需要同时监控多个文件描述符时,可以使用 IO 多路复用。

\texttt{select()}:
\begin{lstlisting}[language=C]
#include <sys/select.h>
int select(int maxfdp1, fd_set *restrict readfds,
           fd_set *restrict writefds, fd_set *restrict exceptfds,
           struct timeval *restrict tvptr);
/* Returns: count of ready descriptors, 0 on timeout, −1 on error */
\end{lstlisting}

\texttt{select()} 的特点:
\begin{itemize}
  \item 监控多个文件描述符
  \item 有文件描述符数量限制(FD\_SETSIZE,通常 1024)
  \item 每次调用需要重新设置文件描述符集合
  \item 效率较低,适合少量文件描述符
\end{itemize}

\texttt{poll()}:
\begin{lstlisting}[language=C]
#include <poll.h>
int poll(struct pollfd fdarray[], nfds_t nfds, int timeout);
/* Returns: count of ready descriptors, 0 on timeout, −1 on error */
\end{lstlisting}

\texttt{poll()} 的改进:
\begin{itemize}
  \item 没有文件描述符数量限制
  \item 使用 pollfd 数组,更灵活
  \item 仍然需要遍历所有文件描述符
\end{itemize}

\texttt{epoll()}(Linux 特有):
\begin{lstlisting}[language=C]
#include <sys/epoll.h>
int epoll_create(int size);
int epoll_ctl(int epfd, int op, int fd, struct epoll_event *event);
int epoll_wait(int epfd, struct epoll_event *events, int maxevents, int timeout);
\end{lstlisting}

\texttt{epoll()} 的优势:
\begin{itemize}
  \item 高效:只返回就绪的文件描述符
  \item 支持边缘触发(ET)和水平触发(LT)模式
  \item 适合大量文件描述符的场景
  \item Linux 高性能网络服务器的首选
\end{itemize}

\subsection{网络}

网络编程是 Linux 应用开发的重要部分,涉及套接字编程、协议处理等。

\subsubsection{套接字基础}

\BlockDesc{套接字}

套接字(Socket)是网络通信的端点,提供进程间网络通信的接口。

套接字类型:
\begin{itemize}
  \item \textbf{流套接字(SOCK\_STREAM)}:TCP,面向连接,可靠传输
  \item \textbf{数据报套接字(SOCK\_DGRAM)}:UDP,无连接,不可靠
  \item \textbf{原始套接字(SOCK\_RAW)}:直接访问网络层
\end{itemize}

\BlockDesc{创建套接字}

\begin{lstlisting}[language=C]
#include <sys/socket.h>
int socket(int domain, int type, int protocol);
/* Returns: file descriptor if OK, −1 on error */
\end{lstlisting}

套接字域(domain):
\begin{itemize}
  \item \texttt{AF\_INET}:IPv4
  \item \texttt{AF\_INET6}:IPv6
  \item \texttt{AF\_UNIX}:Unix 域套接字(本地通信)
\end{itemize}

\subsubsection{TCP 服务器编程}

TCP 服务器编程的基本步骤:

\begin{enumerate}
  \item 创建套接字
  \item 绑定地址和端口
  \item 监听连接
  \item 接受连接
  \item 收发数据
  \item 关闭连接
\end{enumerate}

示例代码:
\begin{lstlisting}[language=C]
#include <sys/socket.h>
#include <netinet/in.h>
#include <arpa/inet.h>

// 创建套接字
int sockfd = socket(AF_INET, SOCK_STREAM, 0);

// 设置地址结构
struct sockaddr_in servaddr;
memset(&servaddr, 0, sizeof(servaddr));
servaddr.sin_family = AF_INET;
servaddr.sin_addr.s_addr = htonl(INADDR_ANY);
servaddr.sin_port = htons(8080);

// 绑定
bind(sockfd, (struct sockaddr *)&servaddr, sizeof(servaddr));

// 监听
listen(sockfd, 5);

// 接受连接
struct sockaddr_in cliaddr;
socklen_t len = sizeof(cliaddr);
int connfd = accept(sockfd, (struct sockaddr *)&cliaddr, &len);

// 收发数据
char buf[1024];
read(connfd, buf, sizeof(buf));
write(connfd, buf, strlen(buf));

// 关闭
close(connfd);
close(sockfd);
\end{lstlisting}

\subsubsection{TCP 客户端编程}

TCP 客户端编程的基本步骤:

\begin{enumerate}
  \item 创建套接字
  \item 连接到服务器
  \item 收发数据
  \item 关闭连接
\end{enumerate}

示例代码:
\begin{lstlisting}[language=C]
// 创建套接字
int sockfd = socket(AF_INET, SOCK_STREAM, 0);

// 设置服务器地址
struct sockaddr_in servaddr;
memset(&servaddr, 0, sizeof(servaddr));
servaddr.sin_family = AF_INET;
servaddr.sin_port = htons(8080);
inet_pton(AF_INET, "127.0.0.1", &servaddr.sin_addr);

// 连接
connect(sockfd, (struct sockaddr *)&servaddr, sizeof(servaddr));

// 收发数据
write(sockfd, "Hello", 5);
read(sockfd, buf, sizeof(buf));

// 关闭
close(sockfd);
\end{lstlisting}

\subsubsection{UDP 编程}

UDP 是无连接的,直接发送和接收数据报。

UDP 服务器:
\begin{lstlisting}[language=C]
int sockfd = socket(AF_INET, SOCK_DGRAM, 0);

struct sockaddr_in servaddr;
servaddr.sin_family = AF_INET;
servaddr.sin_addr.s_addr = htonl(INADDR_ANY);
servaddr.sin_port = htons(8080);

bind(sockfd, (struct sockaddr *)&servaddr, sizeof(servaddr));

struct sockaddr_in cliaddr;
socklen_t len = sizeof(cliaddr);
char buf[1024];

// 接收数据
recvfrom(sockfd, buf, sizeof(buf), 0,
         (struct sockaddr *)&cliaddr, &len);

// 发送数据
sendto(sockfd, buf, strlen(buf), 0,
       (struct sockaddr *)&cliaddr, len);
\end{lstlisting}

\subsubsection{高级网络编程}

\BlockDesc{非阻塞 IO}

设置套接字为非阻塞模式:
\begin{lstlisting}[language=C]
int flags = fcntl(sockfd, F_GETFL, 0);
fcntl(sockfd, F_SETFL, flags | O_NONBLOCK);
\end{lstlisting}

非阻塞 IO 的特点:
\begin{itemize}
  \item 操作立即返回,不会阻塞
  \item 需要配合 IO 多路复用使用
  \item 适合高并发场景
\end{itemize}

\BlockDesc{地址重用}

允许地址重用,避免 TIME\_WAIT 状态的影响:
\begin{lstlisting}[language=C]
int opt = 1;
setsockopt(sockfd, SOL_SOCKET, SO_REUSEADDR, &opt, sizeof(opt));
\end{lstlisting}

\subsection{进程}

进程管理是 Linux 系统编程的核心内容。

\subsubsection{进程创建}

\BlockDesc{fork}

\begin{lstlisting}[language=C]
#include <unistd.h>
pid_t fork(void);
/* Returns: 0 in child, process ID of child in parent, −1 on error */
\end{lstlisting}

\texttt{fork()} 创建新进程:
\begin{itemize}
  \item 子进程是父进程的副本
  \item 子进程获得父进程数据空间、堆和栈的副本
  \item 父子进程共享代码段
  \item 返回值:子进程返回 0,父进程返回子进程 PID
\end{itemize}

\BlockDesc{exec}

\begin{lstlisting}[language=C]
#include <unistd.h>
int execl(const char *pathname, const char *arg0, ... /* (char *)0 */ );
int execv(const char *pathname, char *const argv[]);
int execle(const char *pathname, const char *arg0, ... /* (char *)0, char *const envp[] */ );
int execve(const char *pathname, char *const argv[], char *const envp[]);
int execlp(const char *filename, const char *arg0, ... /* (char *)0 */ );
int execvp(const char *filename, char *const argv[]);
/* All return: −1 on error, no return on success */
\end{lstlisting}

\texttt{exec} 函数族用于执行新程序:
\begin{itemize}
  \item 替换当前进程的映像
  \item 新程序从 \texttt{main()} 开始执行
  \item 进程 ID 不变
  \item 通常与 \texttt{fork()} 配合使用
\end{itemize}

\subsubsection{进程终止}

\BlockDesc{进程退出}

\begin{lstlisting}[language=C]
#include <stdlib.h>
void exit(int status);
void _Exit(int status);

#include <unistd.h>
void _exit(int status);
\end{lstlisting}

进程终止方式:
\begin{itemize}
  \item \texttt{exit()}:正常终止,执行清理函数
  \item \texttt{\_exit()} 和 \texttt{\_Exit()}:立即终止,不执行清理
  \item \texttt{return}:从 \texttt{main()} 返回
\end{itemize}

\BlockDesc{等待子进程}

\begin{lstlisting}[language=C]
#include <sys/wait.h>
pid_t wait(int *statloc);
pid_t waitpid(pid_t pid, int *statloc, int options);
/* Both return: process ID if OK, 0 or −1 on error */
\end{lstlisting}

\texttt{wait()} 和 \texttt{waitpid()} 用于等待子进程终止:
\begin{itemize}
  \item \texttt{wait()} 等待任意子进程
  \item \texttt{waitpid()} 可以等待特定子进程
  \item 防止僵尸进程的产生
\end{itemize}

\subsubsection{进程间通信}

\BlockDesc{管道}

\begin{lstlisting}[language=C]
#include <unistd.h>
int pipe(int fd[2]);
/* Returns: 0 if OK, −1 on error */
\end{lstlisting}

管道用于有亲缘关系的进程间通信:
\begin{itemize}
  \item \texttt{fd[0]} 用于读,\texttt{fd[1]} 用于写
  \item 半双工通信
  \item 数据在内核缓冲区中
\end{itemize}

\BlockDesc{共享内存}

\begin{lstlisting}[language=C]
#include <sys/shm.h>
int shmget(key_t key, size_t size, int flag);
void *shmat(int shmid, const void *addr, int flag);
int shmdt(const void *addr);
int shmctl(int shmid, int cmd, struct shmid_ds *buf);
\end{lstlisting}

共享内存是最快的 IPC 方式:
\begin{itemize}
  \item \texttt{shmget()} 创建或获取共享内存段
  \item \texttt{shmat()} 附加共享内存到进程地址空间
  \item \texttt{shmdt()} 分离共享内存
  \item 需要同步机制防止竞争
\end{itemize}

\subsubsection{信号}

\BlockDesc{信号处理}

\begin{lstlisting}[language=C]
#include <signal.h>
void (*signal(int signo, void (*func)(int)))(int);
/* Returns: previous disposition of signal if OK, SIG_ERR on error */
\end{lstlisting}

信号是异步通知机制:
\begin{itemize}
  \item 标准信号:SIGINT、SIGTERM、SIGKILL 等
  \item 实时信号:SIGRTMIN 到 SIGRTMAX
  \item 可以忽略、捕获或使用默认处理
\end{itemize}

\BlockDesc{发送信号}

\begin{lstlisting}[language=C]
#include <signal.h>
int kill(pid_t pid, int signo);
int raise(int signo);
/* Both return: 0 if OK, −1 on error */
\end{lstlisting}

\subsection{线程}

Linux 使用 POSIX 线程(pthread)库进行多线程编程。

\subsubsection{线程创建和终止}

\BlockDesc{创建线程}

\begin{lstlisting}[language=C]
#include <pthread.h>
int pthread_create(pthread_t *restrict tidp,
                   const pthread_attr_t *restrict attr,
                   void *(*start_rtn)(void *), void *restrict arg);
/* Returns: 0 if OK, error number on failure */
\end{lstlisting}

\texttt{pthread\_create()} 创建新线程:
\begin{itemize}
  \item \texttt{tidp} 返回线程 ID
  \item \texttt{attr} 指定线程属性
  \item \texttt{start\_rtn} 是线程函数
  \item \texttt{arg} 是传递给线程函数的参数
\end{itemize}

\BlockDesc{等待线程}

\begin{lstlisting}[language=C]
#include <pthread.h>
int pthread_join(pthread_t thread, void **rval_ptr);
/* Returns: 0 if OK, error number on error */
\end{lstlisting}

\texttt{pthread\_join()} 等待线程终止:
\begin{itemize}
  \item 阻塞调用线程,直到目标线程终止
  \item \texttt{rval\_ptr} 接收线程返回值
  \item 防止线程成为僵尸线程
\end{itemize}

\subsubsection{线程同步}

\BlockDesc{互斥锁}

\begin{lstlisting}[language=C]
#include <pthread.h>
int pthread_mutex_init(pthread_mutex_t *restrict mutex,
                       const pthread_mutexattr_t *restrict attr);
int pthread_mutex_destroy(pthread_mutex_t *mutex);
int pthread_mutex_lock(pthread_mutex_t *mutex);
int pthread_mutex_trylock(pthread_mutex_t *mutex);
int pthread_mutex_unlock(pthread_mutex_t *mutex);
/* All return: 0 if OK, error number on error */
\end{lstlisting}

互斥锁用于保护临界区:
\begin{itemize}
  \item \texttt{pthread\_mutex\_lock()} 获取锁,阻塞直到可用
  \item \texttt{pthread\_mutex\_trylock()} 尝试获取锁,立即返回
  \item \texttt{pthread\_mutex\_unlock()} 释放锁
\end{itemize}

\BlockDesc{条件变量}

\begin{lstlisting}[language=C]
#include <pthread.h>
int pthread_cond_init(pthread_cond_t *restrict cond,
                      const pthread_condattr_t *restrict attr);
int pthread_cond_destroy(pthread_cond_t *cond);
int pthread_cond_wait(pthread_cond_t *restrict cond,
                      pthread_mutex_t *restrict mutex);
int pthread_cond_signal(pthread_cond_t *cond);
int pthread_cond_broadcast(pthread_cond_t *cond);
/* All return: 0 if OK, error number on error */
\end{lstlisting}

条件变量用于线程等待条件:
\begin{itemize}
  \item \texttt{pthread\_cond\_wait()} 释放互斥锁并等待条件
  \item \texttt{pthread\_cond\_signal()} 唤醒一个等待的线程
  \item \texttt{pthread\_cond\_broadcast()} 唤醒所有等待的线程
  \item 必须与互斥锁配合使用
\end{itemize}

\BlockDesc{读写锁}

\begin{lstlisting}[language=C]
#include <pthread.h>
int pthread_rwlock_init(pthread_rwlock_t *restrict rwlock,
                        const pthread_rwlockattr_t *restrict attr);
int pthread_rwlock_rdlock(pthread_rwlock_t *rwlock);
int pthread_rwlock_wrlock(pthread_rwlock_t *rwlock);
int pthread_rwlock_unlock(pthread_rwlock_t *rwlock);
/* All return: 0 if OK, error number on error */
\end{lstlisting}

读写锁允许多个读者或一个写者:
\begin{itemize}
  \item \texttt{pthread\_rwlock\_rdlock()} 获取读锁
  \item \texttt{pthread\_rwlock\_wrlock()} 获取写锁
  \item 适合读多写少的场景
\end{itemize}

\subsection{异常处理}

异常处理是程序健壮性的重要保障。

\subsubsection{错误处理}

\BlockDesc{errno}

系统调用失败时,错误码存储在 \texttt{errno} 中:
\begin{lstlisting}[language=C]
#include <errno.h>
extern int errno;
\end{lstlisting}

常见错误码:
\begin{itemize}
  \item \texttt{EAGAIN}:资源暂时不可用
  \item \texttt{EBADF}:无效的文件描述符
  \item \texttt{EINTR}:系统调用被信号中断
  \item \texttt{EINVAL}:无效参数
  \item \texttt{ENOMEM}:内存不足
  \item \texttt{EPERM}:权限不足
\end{itemize}

\BlockDesc{错误处理函数}

\begin{lstlisting}[language=C]
#include <string.h>
char *strerror(int errnum);

#include <stdio.h>
void perror(const char *msg);
\end{lstlisting}

\texttt{strerror()} 和 \texttt{perror()} 用于打印错误信息。

\subsubsection{信号处理}

\BlockDesc{信号处理函数}

\begin{lstlisting}[language=C]
#include <signal.h>
int sigaction(int signo, const struct sigaction *restrict act,
              struct sigaction *restrict oact);
/* Returns: 0 if OK, −1 on error */
\end{lstlisting}

\texttt{sigaction()} 是更可靠的信号处理接口:
\begin{itemize}
  \item 可以指定信号处理函数
  \item 可以设置信号掩码
  \item 可以指定信号处理标志
\end{itemize}

\subsubsection{资源限制}

\BlockDesc{资源限制}

\begin{lstlisting}[language=C]
#include <sys/resource.h>
int getrlimit(int resource, struct rlimit *rlptr);
int setrlimit(int resource, const struct rlimit *rlptr);
/* Both return: 0 if OK, −1 on error */
\end{lstlisting}

资源限制类型:
\begin{itemize}
  \item \texttt{RLIMIT\_CPU}:CPU 时间限制
  \item \texttt{RLIMIT\_DATA}:数据段大小
  \item \texttt{RLIMIT\_FSIZE}:文件大小
  \item \texttt{RLIMIT\_NOFILE}:打开文件数
  \item \texttt{RLIMIT\_STACK}:栈大小
\end{itemize}

\subsubsection{调试和诊断}

\BlockDesc{调试工具}

常用的调试和诊断工具:
\begin{itemize}
  \item \textbf{gdb}:GNU 调试器,用于调试程序
  \item \textbf{strace}:跟踪系统调用
  \item \textbf{ltrace}:跟踪库函数调用
  \item \textbf{valgrind}:内存错误检测
  \item \textbf{perf}:性能分析工具
  \item \textbf{tcpdump}:网络包捕获
\end{itemize}

\subsubsection{日志记录}

\BlockDesc{系统日志}

\begin{lstlisting}[language=C]
#include <syslog.h>
void openlog(const char *ident, int option, int facility);
void syslog(int priority, const char *format, ...);
void closelog(void);
\end{lstlisting}

系统日志用于记录程序运行信息:
\begin{itemize}
  \item 优先级:LOG\_EMERG、LOG\_ERR、LOG\_WARNING、LOG\_INFO 等
  \item 设施:LOG\_USER、LOG\_DAEMON、LOG\_LOCAL0-7 等
  \item 日志通常写入 /var/log/syslog 或 /var/log/messages
\end{itemize}

\subsection{系统性能优化}

系统性能优化是 Linux 应用开发的重要方面,涉及 CPU、内存、I/O、网络等多个维度。

\subsubsection{CPU 性能优化}

\begin{enumerate}
  \item \textbf{CPU 亲和性}
    \begin{itemize}
      \item 将进程绑定到特定 CPU 核心
      \item 减少 CPU 缓存失效
      \item 使用 \texttt{sched\_setaffinity()} 设置
    \end{itemize}

  \item \textbf{进程优先级}
    \begin{itemize}
      \item 使用 \texttt{nice()} 调整进程优先级
      \item 实时进程使用 \texttt{sched\_setscheduler()}
      \item 平衡系统负载和响应性
    \end{itemize}

  \item \textbf{CPU 频率调节}
    \begin{itemize}
      \item 使用 cpufreq 子系统
      \item 性能模式 vs 节能模式
      \item 动态频率调节(DVFS)
    \end{itemize}
\end{enumerate}

\subsubsection{内存性能优化}

\begin{enumerate}
  \item \textbf{内存对齐}
    \begin{itemize}
      \item 数据结构对齐到缓存行边界
      \item 减少 false sharing
      \item 使用 \texttt{\_\_attribute\_\_((aligned))}
    \end{itemize}

  \item \textbf{内存预取}
    \begin{itemize}
      \item 使用 \texttt{prefetch()} 预取数据
      \item 减少缓存未命中
      \item 需要根据访问模式调整
    \end{itemize}

  \item \textbf{大页内存}
    \begin{itemize}
      \item 使用 2MB 或 1GB 大页
      \item 减少 TLB 未命中
      \item 适合大内存应用
    \end{itemize}
\end{enumerate}

\subsubsection{I/O 性能优化}

\begin{enumerate}
  \item \textbf{异步 I/O}
    \begin{itemize}
      \item 使用 \texttt{aio\_read()}、\texttt{aio\_write()}
      \item 避免阻塞等待 I/O 完成
      \item 提高并发性能
    \end{itemize}

  \item \textbf{直接 I/O}
    \begin{itemize}
      \item 使用 \texttt{O\_DIRECT} 标志
      \item 绕过页缓存,直接访问磁盘
      \item 适合大文件顺序访问
    \end{itemize}

  \item \textbf{零拷贝}
    \begin{itemize}
      \item 使用 \texttt{sendfile()}、\texttt{splice()}
      \item 减少数据拷贝次数
      \item 提高网络和文件传输性能
    \end{itemize}
\end{enumerate}

\subsubsection{网络性能优化}

\begin{enumerate}
  \item \textbf{套接字选项}
    \begin{itemize}
      \item \texttt{SO\_REUSEADDR}:地址重用
      \item \texttt{SO\_REUSEPORT}:端口重用(负载均衡)
      \item \texttt{SO\_RCVBUF}、\texttt{SO\_SNDBUF}:缓冲区大小
    \end{itemize}

  \item \textbf{TCP 优化}
    \begin{itemize}
      \item 调整 TCP 窗口大小
      \item 启用 TCP\_NODELAY(禁用 Nagle 算法)
      \item 使用 TCP\_CORK 批量发送
    \end{itemize}

  \item \textbf{epoll 优化}
    \begin{itemize}
      \item 使用边缘触发(ET)模式
      \item 一次性处理所有就绪事件
      \item 减少系统调用次数
    \end{itemize}
\end{enumerate}

\subsection{内核模块开发}

内核模块是 Linux 内核的扩展,可以在运行时动态加载和卸载。

\subsubsection{模块基础}

\BlockDesc{内核模块}

内核模块的特点:
\begin{itemize}
  \item 可以动态加载和卸载
  \item 运行在内核空间
  \item 可以访问内核符号和资源
  \item 需要遵循内核编程规范
\end{itemize}

简单的内核模块示例:
\begin{lstlisting}[language=C]
#include <linux/init.h>
#include <linux/module.h>
#include <linux/kernel.h>

MODULE_LICENSE("GPL");
MODULE_AUTHOR("Your Name");
MODULE_DESCRIPTION("A simple kernel module");

static int __init hello_init(void)
{
    printk(KERN_INFO "Hello, kernel!\n");
    return 0;
}

static void __exit hello_exit(void)
{
    printk(KERN_INFO "Goodbye, kernel!\n");
}

module_init(hello_init);
module_exit(hello_exit);
\end{lstlisting}

\subsubsection{字符设备驱动}

字符设备是最简单的设备类型,提供字节流接口。

字符设备驱动的基本结构:
\begin{lstlisting}[language=C]
#include <linux/fs.h>
#include <linux/cdev.h>

static int major;
static struct cdev my_cdev;

static int my_open(struct inode *inode, struct file *file)
{
    // 打开设备
    return 0;
}

static ssize_t my_read(struct file *file, char __user *buf,
                       size_t count, loff_t *pos)
{
    // 读取数据
    return 0;
}

static ssize_t my_write(struct file *file, const char __user *buf,
                        size_t count, loff_t *pos)
{
    // 写入数据
    return count;
}

static struct file_operations my_fops = {
    .owner = THIS_MODULE,
    .open = my_open,
    .read = my_read,
    .write = my_write,
};

static int __init my_init(void)
{
    // 分配设备号
    alloc_chrdev_region(&major, 0, 1, "mydevice");

    // 初始化字符设备
    cdev_init(&my_cdev, &my_fops);
    cdev_add(&my_cdev, major, 1);

    return 0;
}

static void __exit my_exit(void)
{
    cdev_del(&my_cdev);
    unregister_chrdev_region(major, 1);
}

module_init(my_init);
module_exit(my_exit);
\end{lstlisting}

\subsubsection{内核同步}

内核模块中需要使用同步机制保护共享资源。

\begin{enumerate}
  \item \textbf{自旋锁}
    \begin{itemize}
      \item \texttt{spinlock\_t}:自旋锁类型
      \item \texttt{spin\_lock()}、\texttt{spin\_unlock()}
      \item 适合中断上下文和短时间持有
    \end{itemize}

  \item \textbf{互斥锁}
    \begin{itemize}
      \item \texttt{struct mutex}:互斥锁类型
      \item \texttt{mutex\_lock()}、\texttt{mutex\_unlock()}
      \item 可以睡眠,适合进程上下文
    \end{itemize}

  \item \textbf{信号量}
    \begin{itemize}
      \item \texttt{struct semaphore}:信号量类型
      \item \texttt{down()}、\texttt{up()}
      \item 支持计数信号量
    \end{itemize}

  \item \textbf{完成量}
    \begin{itemize}
      \item \texttt{struct completion}:完成量类型
      \item \texttt{wait\_for\_completion()}、\texttt{complete()}
      \item 用于等待某个事件完成
    \end{itemize}
\end{enumerate}

\subsubsection{内存管理}

内核模块中的内存分配:

\begin{enumerate}
  \item \textbf{kmalloc/kfree}
    \begin{itemize}
      \item 分配连续物理内存
      \item 类似用户空间的 malloc/free
      \item 使用 GFP\_KERNEL 或 GFP\_ATOMIC 标志
    \end{itemize}

  \item \textbf{vmalloc/vfree}
    \begin{itemize}
      \item 分配虚拟地址连续的内存
      \item 物理地址可能不连续
      \item 开销较大,适合大块内存
    \end{itemize}

  \item \textbf{页分配器}
    \begin{itemize}
      \item \texttt{alloc\_pages()}:分配物理页
      \item \texttt{get\_free\_page()}:分配并映射到虚拟地址
      \item 用于大块内存分配
    \end{itemize}
\end{enumerate}

\subsection{系统监控和调试}

系统监控和调试是维护 Linux 系统的重要技能。

\subsubsection{系统监控工具}

\begin{enumerate}
  \item \textbf{top/htop}
    \begin{itemize}
      \item 实时显示进程和系统资源使用情况
      \item CPU、内存、负载等信息
      \item 交互式操作,可以排序和杀死进程
    \end{itemize}

  \item \textbf{vmstat}
    \begin{itemize}
      \item 显示虚拟内存统计信息
      \item CPU、内存、I/O、中断等统计
      \item 可以定期采样
    \end{itemize}

  \item \textbf{iostat}
    \begin{itemize}
      \item 显示 I/O 统计信息
      \item 磁盘读写速度、利用率、等待时间
      \item 帮助诊断 I/O 瓶颈
    \end{itemize}

  \item \textbf{netstat/ss}
    \begin{itemize}
      \item 显示网络连接和统计信息
      \item 监听端口、连接状态、路由表
      \item ss 是 netstat 的现代替代品
    \end{itemize}

  \item \textbf{sar}
    \begin{itemize}
      \item 系统活动报告工具
      \item 可以记录和回放历史数据
      \item 全面的系统性能分析
    \end{itemize}
\end{enumerate}

\subsubsection{性能分析}

\begin{enumerate}
  \item \textbf{perf}
    \begin{itemize}
      \item Linux 性能分析工具
      \item 支持硬件性能计数器
      \item 可以分析 CPU、内存、缓存等性能
    \end{itemize}

  \item \textbf{strace}
    \begin{itemize}
      \item 跟踪系统调用
      \item 显示程序调用的所有系统调用
      \item 帮助理解程序行为和调试问题
    \end{itemize}

  \item \textbf{ltrace}
    \begin{itemize}
      \item 跟踪库函数调用
      \item 显示程序调用的动态库函数
      \item 帮助分析程序依赖
    \end{itemize}

  \item \textbf{valgrind}
    \begin{itemize}
      \item 内存错误检测工具
      \item 检测内存泄漏、越界访问等
      \item 性能分析(callgrind)
    \end{itemize}
\end{enumerate}

\subsubsection{内核调试}

\begin{enumerate}
  \item \textbf{printk}
    \begin{itemize}
      \item 内核日志输出函数
      \item 不同日志级别(KERN\_DEBUG、KERN\_INFO 等)
      \item 通过 dmesg 查看
    \end{itemize}

  \item \textbf{kgdb}
    \begin{itemize}
      \item 内核调试器
      \item 可以远程调试内核
      \item 需要特殊配置和硬件支持
    \end{itemize}

  \item \textbf{ftrace}
    \begin{itemize}
      \item 内核函数跟踪框架
      \item 可以跟踪函数调用、中断、调度等
      \item 通过 /sys/kernel/debug/tracing 访问
    \end{itemize}

  \item \textbf{SystemTap}
    \begin{itemize}
      \item 动态跟踪工具
      \item 可以跟踪内核和用户空间
      \item 使用脚本语言编写跟踪程序
    \end{itemize}
\end{enumerate}

\subsection{容器技术}

容器技术是现代 Linux 系统的重要特性,提供了轻量级的虚拟化方案。

\subsubsection{容器基础}

\BlockDesc{容器}

容器的特点:
\begin{itemize}
  \item 轻量级:共享主机内核,无需完整操作系统
  \item 隔离:进程、网络、文件系统隔离
  \item 可移植:一次构建,到处运行
  \item 快速启动:秒级启动时间
\end{itemize}

容器的核心技术:
\begin{itemize}
  \item \textbf{命名空间(Namespace)}:提供资源隔离
  \item \textbf{控制组(Cgroup)}:资源限制和统计
  \item \textbf{联合文件系统(UnionFS)}:分层文件系统
  \item \textbf{容器运行时}:管理容器的生命周期
\end{itemize}

\subsubsection{命名空间}

Linux 命名空间类型:
\begin{enumerate}
  \item \textbf{PID 命名空间}:进程 ID 隔离
  \item \textbf{网络命名空间}:网络接口、路由表隔离
  \item \textbf{挂载命名空间}:文件系统挂载点隔离
  \item \textbf{UTS 命名空间}:主机名和域名隔离
  \item \textbf{IPC 命名空间}:System V IPC 隔离
  \item \textbf{用户命名空间}:用户和组 ID 隔离
\end{enumerate}

\subsubsection{控制组(Cgroup)}

Cgroup 用于限制和统计进程组使用的资源。

Cgroup v2 的主要控制器:
\begin{itemize}
  \item \textbf{CPU}:CPU 使用限制和权重
  \item \textbf{Memory}:内存使用限制
  \item \textbf{IO}:I/O 带宽限制
  \item \textbf{PIDs}:进程数量限制
\end{itemize}

Cgroup 文件系统通常挂载在 /sys/fs/cgroup。

\subsubsection{Docker}

Docker 是最流行的容器平台。

Docker 的核心概念:
\begin{itemize}
  \item \textbf{镜像(Image)}:只读的模板,用于创建容器
  \item \textbf{容器(Container)}:镜像的运行实例
  \item \textbf{仓库(Repository)}:存储镜像的地方
  \item \textbf{Dockerfile}:用于构建镜像的脚本
\end{itemize}

Docker 的基本命令:
\begin{itemize}
  \item \texttt{docker build}:构建镜像
  \item \texttt{docker run}:运行容器
  \item \texttt{docker ps}:列出容器
  \item \texttt{docker images}:列出镜像
  \item \texttt{docker stop/start}:停止/启动容器
\end{itemize}
