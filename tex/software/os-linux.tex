\chapter{OS 原理与 Linux}

\section{OS 原理}

\subsection{内存管理}

\subsection{进程与线程}

\subsubsection{调度}

调度算法:
\begin{itemize}
  \item Fist-Come, Fist-Served
  \item Shortest-Job-First
  \item Round-Robin
  \item Priority
  \item Multilevel Queue
  \item Multilevel Feedback Queue
\end{itemize}

\subsubsection{IPC}

\subsubsection{同步}

\subsection{网络}
\subsubsection{TCP/IP 协议栈总结}

\subsection{文件系统}

\subsection{安全管理}

\subsection{系统调用}

\section{Linux 内核}

\subsection{启动}
\subsubsection{启动流程分析}

\subsection{内存管理}

\subsection{task}

\subsection{网络}

\subsection{驱动架构及子系统}
\subsubsection{UART}

\subsubsection{Timer}

\subsubsection{WDT}

\subsubsection{I2C/I3C}

\subsubsection{SPI/QSPI}

\subsubsection{RS232/RS422/RS485}

\subsubsection{CAN}

\subsubsection{Ethernet}

\subsubsection{USB}

\subsubsection{PCI/PCIe}

\subsection{文件系统}

\subsubsection{ext2 简介}

\section{Linux 应用开发}

\subsection{文件及 IO}

一切皆文件,所有在 Linux/UNIX 系统上打开的文件都用文件描述符表示,并且描述符是非零的。
当我们创建或打开一个文件,内核会返回一个文件描述符给到当前进程。
当我们读取或写入数据到一个文件中,我们便使用相应的文件描述符指定这个文件。

在类 UNIX 系统中约定俗成,文件描述符 0 代表一个进程的标准输入,1 代表标准输出,2 代表标准错误输出。
为了兼容性,应该使用 STDIN\_FILENO、STDOUT\_FILENO 和 STDERR\_FILENO 来代替 0、1 和 2。

文件描述符的范围从 0 到 OPEN\_MAX - 1。
Linux 系统对文件描述符的限制数量是 1024(soft limit),hard limit 为 524288。
然而我们可以通过修改 /etc/security/limits.conf 来修改类似的 limit。

\BlockDesc{打开文件}

\begin{lstlisting}[language=C]
  #include <fcntl.h>
  int open(const char *path, int oflag, ... /* mode_t mode */ );
  int openat(int fd, const char *path, int oflag, ... /* mode_t mode */ );
  /* Both return: file descriptor if OK, −1 on error */
\end{lstlisting}

\subsection{网络}

\subsection{进程}

\subsection{线程}

\subsection{异常处理}
