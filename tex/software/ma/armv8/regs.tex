\subsection{寄存器}

AArch64 64 位通用寄存器(X0-X30),也可以只使用低 32 位(W0-W30)用于 A32 状态。
从 W 寄存器读取时,忽略相应 X 寄存器高 32 位,并保持其它不变。
写入 W 寄存器时,将 X 寄存器的高 32 位设置为零。
也就是说,将 0xFFFFFFFF 写入 W0 会将 X0 设置为 0x00000000FFFFFFFF。

另外还有一些特殊寄存器:
\begin{itemize}
  \item Zero 寄存器,注意并没有所谓的 31 号寄存器(X31/W31),编号为 31 的寄存器就是零寄存器。
    当访问零寄存器时,所有写操作都被忽略,所有读操作返回 0。
  \item PC 寄存器(Program Counter)。
  \item SP 寄存器(SP/WSP)。
    注意 A64 下的 SP 并不加前缀 X。
  \item SPSR 寄存器(Program Status Register)。
    SPSR 保存着异常发生之前的 PSTATE 的值,用于在异常返回时恢复 PSTATE 的值。
  \item ELR 寄存器(Exception Link Register)。
    保存异常返回地址。
\end{itemize}

只有 EL1 和更高的异常级存在 SPSR 和 ELR。

在 ARMv8 体系结构中,选择使用栈指针寄存器在一定程度上与异常级别是分开的。
默认情况下,发生异常时会选择目标异常级别的 SP\_ELn 作为栈指针。
例如,当触发到 EL1 的异常时,就会选择 SP\_EL1 作为栈指针。
每个异常级别都有自己的栈指针,SP\_EL0、SP\_EL1、SP\_EL2 和 SP\_EL3。
EL0 永远只能访问 SP\_EL0。

\begin{stblr}
  {AArch64 SP 选项}
  {a64-sp-opt}
  {cc}
  \hline[1pt]
  Exception Level & Options \\
  \hline
  EL0 & EL0t \\
  EL1 & EL1t, EL1h \\
  EL2 & EL2t, EL2h \\
  EL3 & EL3t, EL3h \\
  \hline[1pt]
\end{stblr}

后缀 t 表示选择 SP\_EL0,h 表示选择 SP\_ELn。

大多数指令无法使用 SP 寄存器,但是有一些形式的算术指令可以操作 SP,例如,ADD 指令可以读写当前的栈指针以调整函数中的栈指针。

原来的 ARMv7 指令集的一个特性是 R15 作为程序计数器(PC),并作为一个通用寄存器使用。
PC 寄存器的使用带来了一些编程技巧,但它为编译器和复杂的流水线的设计引入了复杂性。
在 ARMv8 中删除了对 PC 的直接访问,使返回预测更容易,并简化了 ABI 规范。

PC 永远不能作为一个命名的寄存器来访问。
但是,可以在某些指令中隐式的使用 PC,如 PC 相对加载和地址生成。
PC 不能被指定为数据处理或加载指令的目的操作数。

下表总结了 SPSR 各 bit 的含义:
\begin{ltblr}[caption={AArch64 SPSR bit 位含义}, label={tbl:a64_spsr}]
  {colspec={c>{\centering\arraybackslash}X}}
    \hline[1pt]
    bit & 含义 \\
    \hline
    N & 负数标志位,如果结果为负数,则 N=1;
    如果结果为非负数,则 N=0。\\
    Z & 零标志位,如果结果为零,Z=1,否则 Z=0。\\
    C & 进位标志位\\
    V & 溢出标志位\\
    SS & 软件步进标志位,表示当一个异常发生时,软件步进是否开启\\
    IL & 非法执行状态位\\
    D & 程序状态调试掩码,在异常发生时的异常级别下,来自监视点、断点和软件单步调试事件中的调试异常是否被屏蔽。\\
    A & SError(系统错误)掩码位\\
    I & IRQ 掩码位\\
    F & FIQ 掩码位\\
    M$[4]$ & 异常发生时的执行状态,0 表示 AArch64\\
    M$[3:0]$ & 异常发生时的 mode 或异常级别\\
    \hline[1pt]
\end{ltblr}

AArch64 没有直接与 ARMv7 当前程序状态寄存器 (CPSR) 等价的寄存器。
在 AArch64 中,传统 CPSR 的组件作为可以独立访问的字段提供。
这些状态被统称为处理器状态 (PSTATE)。

\begin{ltblr}[caption={AArch64 PSTATE field}, label={tbl:a64_pstate}]
  {colspec={c>{\centering\arraybackslash}X}}
    \hline[1pt]
    bit & 含义 \\
    \hline
    N & Negative condition flag. \\
    Z & Zero condition flag. \\
    C & Carry condition flag. \\
    V & oVerflow condition flag. \\
    D & Debug mask bit. \\
    A & SError mask bit. \\
    I & IRQ mask bit. \\
    F & FIQ mask bit. \\
    SS & Software Step bit. \\
    IL & Illegal execution state bit. \\
    EL (2) & Exception level. \\
    nRW & Execution state 
          0 = 64-bit
          1 = 32-bit \\
    SP & Stack Pointer selector.
          0 = SP\_EL0
          1 = SP\_ELn \\
    \hline[1pt]
\end{ltblr}

在 AArch64 中,您可以通过执行 ERET 指令从一个异常中返回,那么 SPSR\_ELn 被复制到 PSTATE 中。
包括恢复 ALU 标志、执行状态、异常级别和处理器分支。
并将从 ELR\_ELn 中的地址开始继续执行。

PSTATE.\{N, Z, C, V\} 字段可以在 EL0 级别访问。
其他的字段可以在 EL1 或更高级别访问,但是这些字段在 EL0 级别未定义。

\subsubsection{系统寄存器}

在 AArch64 中,系统配置通过系统寄存器进行控制,并使用 MSR 和 MRS 指令进行访问。
简化了 ARMv7 架构通过协处理器 CP15 来操作系统寄存器的方式。

高异常级下可以访问本异常级和低异常级的系统寄存器。
EL0 异常级具有最低的权限,并且只有极少数的系统系统器可以在 EL0 下访问,例如:CTR\_EL0。

详细的系统寄存器列表请参见 Arm® Architecture Reference Manual for A-profile architecture \cite{armrefman}。

\paragraph*{系统控制寄存器 SCTLR}

系统控制寄存器是一个很重要的系统寄存器,用于控制内存、配置系统能力和提供处理器核的状态信息。
这个寄存器在 EL0 下拥有更多的可获取 bit,更高的异常级则更少。

% npx bit-field -i <json> --fontsize=9 > figures/<bit_field>.svg
\Figure[caption={SCTLR BIT FIELD EL1}, label={fig:sctlr-bit-field-el1}, width=1]{sctlr-bit-field-el1}
\Figure[caption={SCTLR BIT FIELD EL2/3}, label={fig:sctlr-bit-field-el23}, width=1]{sctlr-bit-field-el23}

\begin{description}
  \item[UCI] 设置该域将使能 AArch64 下的 EL0 异常级对 DC CVAU、DC CIVAC、DC CVAC 和 IC IVAU 指令的访问权限。
  \item[EE] 控制异常 Endianness。
    \begin{description}
      \item[0] 小端 
      \item[1] 大端
    \end{description}
  \item[EOE] EL0 下 Explicit data 访问字节序
    \begin{description}
      \item[0] 小端 
      \item[1] 大端
    \end{description}
  \item[WXN] 写权限下应用不可执行权限 XN(eXecute Never)
    \begin{description}
      \item[0] 可写区域可执行
      \item[1] 可写区域强制不可执行
    \end{description}
  \item[nTWE] 不陷入 WFE,此标志为 1 表示 WFE 作为普通指令执行。
  \item[nTWI] 不陷入 WFI, 此标志为 1 表示 WFI 作为普通指令执行
  \item[UCT] 此标志为 1 时,开启 AArch64 的 EL0 下对 CTR\_EL0 寄存器访问权限。
  \item[DZE] EL0 下对 DC AVA 指令的访问权限。
    \begin{description}
      \item[0] 禁止访问
      \item[1] 允许访问
    \end{description}
  \item[I] 开启指令缓存,这是在 EL0 和 EL1 下的指令缓存的启用位。
    对可缓存的正常内存的指令访问被缓存。
  \item[UMA] 用户中断屏蔽控制,EL0 运行在 AArch64 状态下有效。
  \item[SED] 控制 AArch64 状态下的 EL0 是否可以使用 SETEND 指令。
    \begin{description}
      \item[0] 可用
      \item[1] 禁用
    \end{description}
  \item[ITD] 禁止 IT 指令
    \begin{description}
      \item[0] IT 指令有效
      \item[1] IT 指令被当作 16 位指令。
        仅另外 16 位指令或 32 位指令的头 16 位可以使用,这依赖于实现。
    \end{description}
  \item[CP15BEN] CP15 barrier 使能。
    如果实现了,它是 AArch32 CP15 DMB、DSB 和 ISB barrier 操作的使能位
  \item[SA0] EL0 的栈对齐检查使能位
  \item[SA] 栈对齐检查使能位
  \item[C] 数据 cache 使能。
    EL0 和 EL1 的数据访问使能位。
    cacheable 普通内存的数据访问都被缓存。
  \item[A] 对齐检查使能位。
  \item[M] 使能 MMU。
\end{description}

\begin{probsolu}[title={Problem and Solution \theprob}]{
    手册上写的 Explicit data access 怎么理解?}

  “Explicit data access”是指在编程中明确地访问数据的操作。
  这意味着程序员直接指定要访问的数据和操作,而不依赖于编译器或运行时系统的隐式处理。

  这种方式通常用于对内存中的数据进行读取、写入或者其他操作。
  与之相对应的是隐式访问方式,其中编译器或者运行时系统负责管理数据的访问。
  在隐式访问中,程序员通常只需指定数据的名称,而无需关心数据的存储位置或者具体的访问方式。

\end{probsolu}

\begin{remark}
  The caches in the processor must be invalidated before caching of data and instructions is
enabled in any of the Exception levels.
\end{remark}

\paragraph*{大小端的设置}

每个异常级别的数据的大小端都被单独控制。
对于 EL3,EL2 和 EL1,通过 SCTLR\_ELn.EE 设置大小端。
EL1 中其他位,SCTLR\_EL1.E0E 控制 EL0 的数据大小端的设置。
在 AArch64 执行状态中,数据访问可以为 LE 或 BE,但指令的获取通常为 LE。

\subsubsection{NEON 和浮点}

除了通用寄存器之外,ARMv8 还有 32 个 128 位浮点寄存器,标记为 V0-V31。
32 个寄存器用于保存标量浮点指令的浮点操作数,以及 NEON 操作的标量操作数和向量操作数。

操作数使用 H (Half)、S (Single) 和 D (Double) 分别用来指定 V 寄存器的低 16、32 和 64 位。

\begin{itemize}
  \item
    \textbf{v0-v7}: 用于传递浮点参数和返回值。
  \item
    \textbf{v8-v15}: 临时寄存器,用于浮点和SIMD操作。
  \item
    \textbf{v16-v31}: 被调用者保存的寄存器,需要在函数调用之间保持其值。
\end{itemize}
