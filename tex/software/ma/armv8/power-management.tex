\subsection{电源管理}

许多 ARM 系统是移动设备,并由电池供电。
在这些系统中,优化电力使用和总能耗是一个关键的设计约束。
程序员通常会花费大量时间来尝试节省这类系统的电池寿命。

即使在不使用电池的系统中,节能也是一个需要关注的问题。
例如,您可能希望通过减少电费、出于环保原因或为了尽量减少设备产生的热量来最小化能源使用。

ARM 内核内置了许多旨在减少电力使用的硬件设计方法。
能源使用可以分为两个组成部分:

\begin{description}
    \item[静态功耗] \hfill \\
    静态功耗,也常被称为泄漏功耗,当内核逻辑或 RAM 模块通电时就会发生。
    一般来说,泄漏电流与总硅面积成正比,这意味着芯片越大,泄漏就越高。
    随着制造工艺几何尺寸的减小,来自泄漏的功耗比例显著增加。
    \item[{动态功耗}] \hfill \\
    动态功耗是由于晶体管切换引起的,取决于内核时钟速度和每个周期内改变状态的晶体管数量。
    显然,时钟速度越高、内核越复杂,功耗就越高。

    具备电源管理功能的操作系统会动态改变内核的电源状态,平衡当前工作负载与可用计算能力,同时尽量使用最少的电量。
    这些技术中有些会动态切换内核的开关状态,或者将其置于静止状态,使其不再进行计算,从而消耗极少的电量。
    主要的这些技术示例包括:
    \begin{itemize}
      \item
        第~\ref{sec:pm-idle-man} 节的空闲管理。
      \item
        第~\ref{sec:pm-dyn-vol-freq-scale} 节的动态电压和频率调节。
    \end{itemize}
\end{description}

\subsubsection{空闲管理}\label{sec:pm-idle-man}

当一个内核处于空闲状态时,操作系统电源管理(OSPM)会将其转换为低功耗状态。
通常会有多个状态可供选择,每个状态有不同的进入和退出延迟,以及不同的功耗水平。
所使用的状态通常取决于内核再次需要的速度。
任何时候可用的电源状态也可能取决于 SoC 中其他组件的活动情况,而不仅仅是内核本身。
每个状态都由进入该状态时时钟门控或电源门控的组件集合来定义。

从低功耗状态切换到运行状态所需的时间被称为唤醒延迟。
在更深的低功耗状态中,唤醒延迟更长。
虽然空闲电源管理是由内核上的线程行为驱动的,OSPM 可以将平台置于影响许多其他组件的状态,而不仅仅是内核本身。
如果一个集群中的最后一个内核变为空闲,OSPM 可以选择影响整个集群的电源状态。
同样,如果 SoC 中的最后一个内核变为空闲,OSPM 可以选择影响整个 SoC 的电源状态。
选择也会受到系统中其他组件使用情况的驱动。
一个典型的例子是在所有内核和其他总线主控设备空闲时,将内存置于自刷新模式。

OSPM 必须提供必要的电源管理软件基础设施,以确定正确的状态选择。
在空闲管理中,当一个内核或集群被置于低功耗状态时,它可以在任何时候通过内核唤醒事件被重新激活。
也就是说,一个可以从低功耗状态唤醒内核的事件,例如中断。
不需要 OSPM 发出明确的命令来使内核或集群重新运行。
OSPM 认为受影响的内核或多个内核在任何时候都是可用的,即使它们当前处于低功耗状态。

\BlockDesc{电源和时钟}

减少能耗的一种方法是断电,这会同时消除动态和静态电流(有时称为电源门控),或者停止内核的时钟,这仅能消除动态功耗,称为时钟门控。

ARM 内核通常支持几种不同级别的电源管理,如下:

\begin{itemize}
\item
  待机模式(Standby)。
\item
  保持模式(Retention)。
\item
  关机模式(Power down)。
\item
  休眠模式(Dormant mode)。
\item
  热插拔(Hotplug)。
\end{itemize}

对于某些操作,在断电前后需要保存和恢复状态。
执行这些保存和恢复操作所需的时间以及这项额外工作消耗的电力在软件选择适当的电源管理活动时是一个重要因素。

包含内核的 SoC 设备可以有其他低功耗状态,如 STOP 和 Deep sleep。
这些状态指的是电源管理软件可以控制硬件锁相环(PLL)和电压调节器的能力。

\BlockDesc{待机模式}

在待机模式(Standby)下,内核保持通电,但其大部分时钟被停止或时钟门控。
这意味着内核的大多数部分处于静态状态,唯一的电力消耗来自于漏电流和少量用于监测唤醒条件的逻辑电路的时钟。

该模式通过 WFI(等待中断)或 WFE(等待事件)指令进入。
ARM 建议在 WFI 或 WFE 指令之前使用数据同步屏障(DSB)指令,以确保待处理的内存事务在改变状态之前完成。
如果调试通道处于活动状态,它将保持活跃。
内核会停止执行,直到检测到唤醒事件。
唤醒条件取决于进入模式的指令。
对于 WFI,一个中断或外部调试请求会唤醒内核。
对于 WFE,则有若干指定事件可以触发唤醒,包括集群中另一个内核执行 SEV 指令。

在多核系统中,来自嗅探控制单元(SCU)的请求也可以唤醒时钟以进行缓存一致性操作。
这意味着处于待机状态的内核的缓存与其他内核的缓存保持一致(但处于待机状态的内核不一定执行下一条指令)。
核心重置总是会迫使内核退出待机状态。

各种形式的动态时钟门控也可以在硬件中实现。
例如,当检测到空闲状态时,SCU、GIC、定时器、指令流水线或
NEON 块可以自动进行时钟门控以节省电力。

待机模式可以快速进入和退出(通常在两个时钟周期内)。
因此,它对内核的延迟和响应性几乎没有影响。

对于 OSPM 来说,待机状态和保持状态几乎无法区分。
这种区别在外部调试器和硬件实现中显而易见,但在操作系统的空闲管理子系统中则不明显。

\BlockDesc{保持模式}

内核状态(包括调试设置)保存在低功耗结构中,使内核至少部分关闭。
从低功耗保持状态切换到运行状态不需要重置内核。
在从低功耗保持状态切换到运行状态时,会恢复已保存的内核状态。
从操作系统的角度来看,保持状态和待机状态之间除了进入方式、延迟和使用相关的约束外,没有区别。
然而,从外部调试器的角度来看,这些状态有所不同,因为外部调试请求调试事件保持挂起状态,并且不能访问内核电源域中的调试寄存器。

\BlockDesc{关机模式}

在这种状态下,内核被断电。
设备上的软件必须保存所有内核状态,以便在断电期间保留这些状态。
从断电状态切换到运行状态必须包括:

\begin{itemize}
\item
  在恢复电源水平后重置内核。
\item
  恢复已保存的内核状态。
\end{itemize}

断电状态的定义特征是它们会破坏上下文。
这会影响给定状态下关闭的所有组件,包括内核,在更深的状态下还包括系统的其他组件,如 GIC 或特定平台的 IP。
根据调试和跟踪电源域的组织方式,在某些断电状态下,调试和跟踪上下文中的一个或两个可能会丢失。
必须提供机制以使操作系统能够为每个给定状态执行相关的上下文保存和恢复。
执行的恢复从重置向量开始,之后每个操作系统必须恢复其上下文。

\BlockDesc{休眠模式}

休眠模式(Dormant mode)是一种断电状态的实现。
在休眠模式下,内核逻辑被断电,但缓存 RAM 保持通电。
通常,这些 RAM 处于低功耗保持状态,保留其内容但不具备其他功能。
这提供了比完全关机更快的重启,因为缓存中的活动数据和代码得以保留。
同样,在多核系统中,单个内核可以进入休眠模式。

在允许集群内各个内核进入休眠模式的多核系统中,当内核断电时,无法保持一致性。
因此,这些内核必须首先将自己从一致性域中隔离。
在此之前,它们需要清除所有脏数据,并且通常通过另一个内核向外部逻辑发出信号重新通电来唤醒处于休眠模式的内核。

被唤醒的内核必须在重新加入一致性域之前恢复原始内核状态。
由于内核处于休眠模式期间内存状态可能发生变化,内核可能需要无效化缓存。
因此,休眠模式在单核环境中比在集群环境中更有用。
这是因为离开和重新加入一致性域的额外开销。
在集群中,休眠模式通常只有在其他内核已经关闭时,才由最后一个内核使用。

\BlockDesc{热插拔}

CPU 热插拔是一种可以动态切换内核开关的技术。
OSPM 可以使用热插拔根据当前的计算需求改变可用的计算能力。
热插拔有时也用于可靠性方面的原因。
热插拔与使用断电状态进行空闲管理之间有几个区别:

\begin{enumerate}
\item
  当一个内核被热插拔时,监督软件会停止该内核在中断和线程处理中的所有使用。
  调用操作系统不再将该内核视为可用。
\item
  OSPM 必须发出明确的命令才能使一个内核重新上线,即热插拔一个内核。
  在此命令之后,适当的监督软件才会开始在该内核上进行调度或启用中断。
\end{enumerate}

操作系统通常在一个主内核上执行大部分内核引导过程,并在稍后阶段使次要内核上线。
次要内核的引导行为与热插拔一个内核到系统中非常相似。
这两种情况下的操作几乎是相同的。

\subsubsection{动态电压和频率调节}\label{sec:pm-dyn-vol-freq-scale}

许多系统在负载可变的条件下运行,因此具有根据预期负载调整内核性能的能力是很有用的。
通过降低内核的时钟速度可以减少动态功耗。

动态电压和频率调节(DVFS)是一种利用以下关系的节能技术:

\begin{itemize}
\item
  功耗与工作频率之间的线性关系。
\item
  功耗与工作电压之间的平方关系。
\end{itemize}

这种关系可以表示为:

$$ P = C \times V^2 \times f $$

其中:
\begin{itemize}
\item
(P)是动态功率。
\item
(C)是逻辑电路的开关电容。
\item
(V)是工作电压。
\item
(f)是工作频率。
\end{itemize}

通过调整内核时钟的频率来实现节能。
在较低频率下,内核也可以在较低电压下工作。
降低供电电压的优点是既减少动态功耗也减少静态功耗。

对于给定电路,工作电压与该电路可安全运行的频率范围之间的关系是实现特定的。
给定的工作频率及其对应的工作电压被表示为一个元组,称为操作性能点(OPP)。
对于一个系统来说,可实现的 OPP 范围统称为系统的 DVFS 曲线。

操作系统使用 DVFS 来节省能源,并在必要时保持在热限内。
操作系统提供 DVFS 策略来管理消耗的功率和所需的性能。
旨在高性能的策略选择更高的频率并消耗更多的能源。
旨在节能的策略选择较低的频率,从而导致较低的性能。

\subsubsection{汇编电源指令}

ARM 汇编语言包括可以用于将内核置于低功耗状态的指令。
架构将这些指令定义为提示(hints),这意味着内核在执行它们时不需要采取任何特定的动作。
然而,在 Cortex-A 处理器系列中,这些指令被实现为几乎关闭核心的所有部分的时钟。
这意味着核心的功耗大大降低,只有静态泄漏电流被吸引,而没有动态功耗。

WFI 指令的作用是暂停执行,直到内核被以下条件之一唤醒:

\begin{itemize}
\item
  IRQ 中断,即使 PSTATE I 位被设置。
\item
  FIQ 中断,即使 PSTATE F 位被设置。
\item
  异步异常。
\end{itemize}

如果内核在相关的 PSTATE 中断标志被禁用时被中断唤醒,内核会执行 WFI 指令后的下一条指令。

WFI 指令在使用电池供电的系统中广泛使用。
例如,移动电话可以在等待您按下按钮时将内核置于待机模式,而这个过程可以每秒多次重复。

WFE 类似于 WFI。
它会暂停执行,直到发生事件。
这可以是列出的事件条件之一,也可以是由集群中的另一个内核发出的事件。
其他内核可以通过执行 SEV 指令来发出事件。
SEV 向所有内核发出事件信号。
通用定时器也可以编程以触发周期性事件,从而从 WFE 中唤醒一个内核。

\subsubsection{电源状态协调接口}

Power State Coordination Interface (PSCI) 提供了一种与操作系统无关的方法,用于实现可以启动或关闭内核的电源管理用例。
这包括:

\begin{itemize}
\item
  内核空闲管理。
\item
  内核的动态添加和移除(热插拔),以及次要内核引导。
\item
  big.LITTLE 迁移。
\item
  系统关闭和重置。
\end{itemize}

使用此接口发送的消息被所有相关的执行级别接收。
也就是说,如果实现了 EL2 和 EL3,那么由客户中的 Rich OS 发送的消息必须被 hypervisor 接收。
如果 hypervisor 发送了消息,则该消息必须被安全固件接收,然后与 Trusted OS 协调。
这允许每个操作系统确定是否需要上下文保存。

要了解更多信息,请参阅 Power State Coordination Interface (PSCI) 规范。

