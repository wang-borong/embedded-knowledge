\subsection{ARMv8 基础}

在 ARMv8 中,代码执行在四个异常级之一。
异常级别决定特权级别,因此在 $EL_n$ 执行对应于特权 $PL_n$。
更大的 n 值的异常级别处于更高的异常级别。

异常级别提供了软件执行权限的逻辑分离,适用于ARMv8架构的所有操作状态。
它类似并支持计算机科学中常见的分层保护域的概念。

\begin{description}
    \item[EL0] Normal user applications.
    \item[EL1] Operating system kernel typically described as privileged.
    \item[EL2] Hypervisor.
    \item[EL3] Low-level firmware, including the Secure Monitor.
\end{description}

\subsubsection{异常级切换}

异常级别之间可以转换,但是要遵循以下规则:

\begin{itemize}
  \item 移动到更高的异常级别,例如从 EL0 到 EL1,表示软件增加执行特权。
  \item 不能在较低的异常级别下进行异常处理。
  \item EL0 级别没有异常处理,必须在更高的异常级别处理异常。
  \item 异常导致程序流程发生变化。
    异常处理程序的执行从高于EL0的异常级别开始,起始于与所发生的异常相关的定义向量。
    异常包括:IRQ 和 FIQ 等中断、内存系统中止、未定义的指令、系统调用和安全监视器或虚拟机管理程序 trap。
  \item 通过执行 ERET 指令来结束异常处理并返回到上一个异常级别。
  \item 从异常返回可以保持相同的异常级别或进入较低的异常级别,但不能移动到更高的异常级别。
  \item 安全状态确实会随着异常级别的变化而变化,除非从 EL3 重新调整到非安全状态。
\end{itemize}

\begin{probsolu}[title={Problem and Solution \theprob}]{
    如何切换 AArch64 EL?写个切换的实际例子?(课后作业)
  }\label{pb:el_changing}

  当处理异常时,会涉及几个寄存器的操作:
  \begin{enumerate}
    \item 处理器将当前正在执行的指令地址(PC 寄存器)存储在 ELR\_ELn(Exception link register)中。
    \item 将当前处理器的状态(PSTATE)存储在 SPSR\_ELn(Saved Program Status Register)中。
    \item 处理器根据异常向量表跳转到异常处理程序。
    异常处理程序可以修改 ELR 和 SPSR。
    \item 异常处理程序执行 eret 指令推出异常状态。
    这个指令会从 SPSR\_Eln 寄存器恢复处理器的状态,并且恢复 ELE\_Eln 中储存的指令的执行。
  \end{enumerate}
  据上所述,异常处理程序\textcolor{red}{可以修改 ELR\_ELn 和 SPSR\_ELn 寄存器},所以异常处理程序能够间接的修改 EL 等参数,达到切换 EL 的目的。

  比如,想要从 EL3 异常级切换到 EL1,示例代码如下~\ref{lst:change_el}。
  那么需要配置一些系统寄存器,然后调用 eret 指令触发处理器切换异常运行级。

  \begin{enumerate}
    \item 配置 SCTLR\_EL1(System Control Register)。
      sctlr\_eln 寄存器被用来配置处理器的不同参数。
      存在 sctlr\_el1、sctlr\_el2 和 sctlr\_el3 分别对应 EL1、EL2 和 EL3 的寄存器。
      sctlr\_el1 寄存器能够配置 EL0 和 EL1 级别的内存等配置。
      通过修改 sctlr\_el1 某些位的值能达到配置处理器在 EL0 和 EL1 级别运行时的行为。
    \item 配置 HCR\_EL2(Hypervisor Configuration Register)。
      HCR\_EL2 寄存器提供了虚拟化的配置,包括定义是否将各种操作限制在 EL2 中。
      因为只有 EL2 支持 Hypervisor,所以只存在一个 HCR\_EL2 寄存器。
    \item 配置 SCR\_EL3(Secure Configuration Register)。
      SCR\_EL3 寄存器定义当前安全状态的配置:
      \begin{itemize}
        \item EL0,EL1 和 EL2 的安全状态为 Secure 或 Non-Secure
        \item EL2 的 Execution State
      \end{itemize}
    \item 配置 SPSR\_EL3(Saved Program Status Register)。
      EL3 发生异常时,SPSR\_EL3 寄存器用来保存处理器的状态。
    \item 配置 ELR\_EL3(Exception Link Register (EL3))。
      在 EL3 进行异常处理时,ELR\_EL3 寄存器将用来指定即将要返回的地址。
  \end{enumerate}
  通过配置上述系统寄存器,然后调用 eret 触发处理器的执行状态的重恢复,就能将异常级别从 EL3 切换到 EL1。
\end{probsolu}
\begin{lstlisting}[
  language={[ARM]Assembler},
  caption={切换异常级},
  label={lst:change_el}
]
  master:
  ldr    x0, =SCTLR_VALUE_MMU_DISABLED
  msr    sctlr_el1, x0

  ldr    x0, =HCR_VALUE
  msr    hcr_el2, x0

  ldr    x0, =SCR_VALUE
  msr    scr_el3, x0

  ldr    x0, =SPSR_VALUE
  msr    spsr_el3, x0

  adr    x0, el1_entry
  msr    elr_el3, x0

  eret
\end{lstlisting}

\subsubsection{运行状态切换}

ARMv8 架构定义了两种执行状态,AArch64 和 AArch32。
每个状态分别用于描述使用 64 位宽通用寄存器或 32 位宽通用寄存器的执行。
虽然 ARMv8 AArch32 保留了 ARMv7 对特权的定义,但在 AArch64 中,特权级别由异常级别决定。
因此,在 $EL_n$ 的执行对应于特权 $PL_n$。

当处于 AArch64 状态时,处理器执行 A64 指令集。
当处于 AArch32 状态时,处理器可以执行 A32(在早期版本的架构中称为 ARM)或 T32 (Thumb) 指令集。

\begin{probsolu}[title={Problem and Solution \theprob}]{
    如何切换 AArch64 状态到 AArch32 状态?写个切换的实际例子?(课后作业)
  }\label{pb:state_changing}
  
  例如:在 EL3 下进行运行切换,EL3 为 AArch64,将 EL2 切换成 aarch32。
  在 EL3 异常级下,设置 EL2 的架构为 aarch32,设置好返回地址,通过 ERET 指令,即可将 EL2 状态切换成 EL2。
  设置中,主要涉及配置 elr\_el3 寄存器(保存下一异常级的指令地址)和 spsr\_el3 寄存器(保存下一异常级的 pstate 值)。
  对于 spsr\_el3,要设置正确,则要参考 AArch32 的 cpsr 寄存器值进行设置。

  如果需要将 A32 状态切换到 T32 状态,则使用 bx 指令,并且跳转地址的最低位要为 1;
  从 T32 状态切回 A32 状态同样使用 bx 指令,且跳转地址最低位为 0。
  
  总结:EL2 的 A64 和 A32 状态,由 EL3 决定,也就是 SCR\_EL3.RW 寄存器决定。

  EL1 的 A64 和 A32 状态,由 EL2 决定,也就是 HCR\_EL3.RW 寄存器决定。

  EL0 的 A64 和 A32 状态,由 EL1 决定,也就是 CPSR.M[4] 位决定。
\end{probsolu}
