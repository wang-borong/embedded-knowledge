\chapter*{前言}
\addcontentsline{toc}{chapter}{前言}%

\begin{minipage}[b]{0.9\textwidth}
\setlength{\parskip}{0.5\baselineskip}
嵌入式知识体系总结  © 2024 by 王伯榕
is licensed under CC BY-NC-SA 4.0.
To view a copy of this license, visit
\url{http://creativecommons.org/licenses/by-nc-sa/4.0/}
\end{minipage}

\vspace*{2\baselineskip}

\section*{因由}

有许多朋友问我嵌入式该怎么学?
如何入门?
有没有一个可以快速入门,快速学习的路线?

从接触嵌入式开始,这些问题也困扰我良久。
在不断的学习摸索中,我对这个领域也有了一些见解,头脑中的嵌入式知识体系也逐渐清晰。
我想,在这个领域这么多年的积累,也许微不足道,但是不把它们梳理出来,实在浪费。
如果我把个人的认知梳理出来,或许还能帮助到一些朋友,实乃万幸。

于是,我便下定决心在业余时间将嵌入式知识体系整理成册。

\section*{所涉内容}

该手册所涉及的内容是嵌入式系统开发。
嵌入式系统一般由系统硬件、操作系统和特定应用程序组成。
所以本文档主要分为两大部分,软件和硬件。
工具部分作为补充,提供一些开发过程中常用工具的简略介绍与教程。

软件部分又可以分为:

\begin{itemize}
  \item 微处理器体系架构 —— 本章主要涉及常用的为处理器架构协议与编程手册的翻译与讲解。
  \item 编程 —— 本章包含 GNU toolchain 的介绍;ELF 介绍;汇编语言、C 语言以及一些其它语言编程介绍。
  \item 数据结构与算法 —— 对数据结构和算法的梳理。
  \item Firmware 与 Bootloader —— 设备树的协议讲解以及一些 BL 的讲解。
  \item 操作系统原理 —— 操作系统相关知识的梳理。
  \item Linux 基础 —— Linux 基础知识梳理。
  \item Linux 应用开发
  \item Linux 内核开发
\end{itemize}

硬件部分的划分为:

\begin{itemize}
  \item 电路基础
  \item 分立半导体元件
  \item 运放及反馈系统
  \item IC
  \item FPGA 简介
  \item 外设及总线通信协议
  \item 电路设计
  \item 高速 PCB 设计
\end{itemize}

\section*{受众}

广大的嵌入式软件工程师、系统软件工程师、Linux 开发工程师以及电子爱好者。
