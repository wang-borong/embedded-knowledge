\chapter{硬件设计软件使用}

硬件设计软件是电子工程师进行电路设计和 PCB 制作的核心工具。本章将系统介绍主流 PCB 设计软件的使用方法,包括原理图绘制、PCB 布局、封装库管理、设计规则设置、制造文件输出等关键内容,帮助读者掌握专业的 PCB 设计工具使用技巧。

\section{OrCAD 原理图绘制}

OrCAD 是 Cadence 公司推出的专业原理图设计工具,广泛应用于工业界。OrCAD Capture 是其中的原理图编辑器,功能强大且易于使用。

\subsection{OrCAD Capture 基础}

\subsubsection{项目结构}

OrCAD 项目采用层次化结构:

\begin{itemize}
  \item \textbf{项目文件(.opj)}:项目主文件,包含项目配置信息
  \item \textbf{原理图文件(.dsn)}:设计文件,包含所有原理图页面
  \item \textbf{库文件(.olb)}:元件符号库文件
  \item \textbf{输出文件}:网表、BOM 等输出文件
\end{itemize}

\subsubsection{工作界面}

OrCAD Capture 的主要工作区域包括:

\begin{enumerate}
  \item \textbf{项目管理器(Project Manager)}
    \begin{itemize}
      \item 显示项目层次结构
      \item 管理设计文件、库文件
      \item 访问设计规则检查(DRC)工具
    \end{itemize}

  \item \textbf{原理图编辑窗口}
    \begin{itemize}
      \item 绘制和编辑原理图
      \item 放置元件、连接网络
      \item 添加标注和说明
    \end{itemize}

  \item \textbf{元件库浏览器}
    \begin{itemize}
      \item 浏览和搜索元件符号
      \item 预览元件符号
      \item 放置元件到原理图
    \end{itemize}

  \item \textbf{属性编辑器}
    \begin{itemize}
      \item 编辑元件属性
      \item 设置网络属性
      \item 配置设计参数
    \end{itemize}
\end{enumerate}

\subsection{创建新项目}

\BlockDesc{创建新项目}

创建新项目的步骤:

\begin{enumerate}
  \item 启动 OrCAD Capture
  \item 选择 \lstinline{File > New > Project}
  \item 选择项目类型:
    \begin{itemize}
      \item \textbf{Analog or Mixed A/D}:模拟或混合信号设计
      \item \textbf{PC Board Wizard}:PCB 设计向导
      \item \textbf{Programmable Logic Wizard}:可编程逻辑设计
      \item \textbf{Schematic}:纯原理图设计
    \end{itemize}
  \item 指定项目名称和保存路径
  \item 选择原理图页面大小(如 A4、A3 等)
  \item 完成项目创建
\end{enumerate}

\subsection{元件库管理}

\subsubsection{库文件结构}

OrCAD 库文件(.olb)包含元件符号,库文件可以包含多个元件。每个元件包含:

\begin{itemize}
  \item \textbf{符号图形}:元件的图形表示
  \item \textbf{引脚定义}:引脚名称、编号、类型
  \item \textbf{属性}:元件值、封装、制造商等
\end{itemize}

\subsubsection{使用现有库}

OrCAD 自带丰富的元件库,位于安装目录的 \lstinline{Library} 文件夹:

\begin{itemize}
  \item \textbf{discrete.olb}:分立元件(电阻、电容、电感等)
  \item \textbf{connector.olb}:连接器
  \item \textbf{opamp.olb}:运算放大器
  \item \textbf{74xx.olb}:74 系列逻辑芯片
  \item \textbf{memory.olb}:存储器
\end{itemize}

\BlockDesc{添加库到项目}

添加库文件的步骤:

\begin{enumerate}
  \item 在项目管理器中,右键点击 \lstinline{Library} 文件夹
  \item 选择 \lstinline{Add File}
  \item 浏览并选择库文件(.olb)
  \item 库文件将出现在库列表中
\end{enumerate}

\subsubsection{创建自定义元件}

当需要使用的元件不在现有库中时,需要创建自定义元件:

\BlockDesc{创建新元件}

\begin{enumerate}
  \item 创建新库文件或打开现有库文件
  \item 选择 \lstinline{Design > New Part}
  \item 设置元件属性:
    \begin{itemize}
      \item \textbf{Name}:元件名称
      \item \textbf{Part Reference Prefix}:元件编号前缀(如 R、C、U)
      \item \textbf{PCB Footprint}:PCB 封装名称
    \end{itemize}
  \item 绘制元件符号:
    \begin{itemize}
      \item 使用绘图工具绘制元件外形
      \item 添加引脚(\lstinline{Place > Pin})
      \item 设置引脚属性(名称、编号、类型)
    \end{itemize}
  \item 保存元件到库文件
\end{enumerate}

\subsubsection{引脚类型}

OrCAD 中引脚类型包括:

\begin{itemize}
  \item \textbf{Input}:输入引脚
  \item \textbf{Output}:输出引脚
  \item \textbf{Bidirectional}:双向引脚
  \item \textbf{Power}:电源引脚
  \item \textbf{Passive}:无源引脚
  \item \textbf{3-State}:三态引脚
  \item \textbf{Open Collector}:开集电极
  \item \textbf{Open Emitter}:开发射极
\end{itemize}

\subsection{原理图绘制}

\subsubsection{放置元件}

\BlockDesc{放置元件到原理图}

\begin{enumerate}
  \item 在原理图编辑窗口中,选择 \lstinline{Place > Part} 或按快捷键 \lstinline{P}
  \item 在元件库浏览器中选择元件
  \item 点击 \lstinline{Place} 按钮
  \item 在原理图上点击放置位置
  \item 可以连续放置多个相同元件
  \item 按 \lstinline{ESC} 键退出放置模式
\end{enumerate}

\subsubsection{连接网络}

\BlockDesc{连接元件}

\begin{enumerate}
  \item 选择 \lstinline{Place > Wire} 或按快捷键 \lstinline{W}
  \item 点击起始引脚
  \item 移动鼠标到目标引脚
  \item 点击目标引脚完成连接
  \item 可以继续连接其他网络
  \item 按 \lstinline{ESC} 键退出连线模式
\end{enumerate}

\subsubsection{网络命名}

\BlockDesc{命名网络}

\begin{enumerate}
  \item 选择网络(点击网络线)
  \item 右键选择 \lstinline{Edit Properties} 或按快捷键 \lstinline{Alt+Enter}
  \item 在属性对话框中设置网络名称
  \item 也可以使用 \lstinline{Place > Net Alias} 添加网络标签
\end{enumerate}

网络命名规范:

\begin{itemize}
  \item 电源网络:$V_{CC}$、$V_{DD}$、$V_{SS}$、GND、AGND 等
  \item 信号网络:使用有意义的名称,如 \lstinline{CLK}、\lstinline{RESET}、\lstinline{DATA[0:7]} 等
  \item 差分信号:使用 \lstinline{+} 和 \lstinline{-} 后缀,如 \lstinline{USB_D+}、\lstinline{USB_D-}
\end{itemize}

\subsubsection{总线设计}

对于多位总线,可以使用总线符号简化原理图:

\BlockDesc{使用总线}

\begin{enumerate}
  \item 选择 \lstinline{Place > Bus} 或按快捷键 \lstinline{B}
  \item 绘制总线路径
  \item 使用 \lstinline{Place > Bus Entry} 添加总线入口
  \item 连接信号线到总线入口
  \item 使用 \lstinline{Place > Net Alias} 命名总线,如 \lstinline{DATA[0:7]}
  \item 命名连接到总线的信号,如 \lstinline{DATA0}、\lstinline{DATA1} 等
\end{enumerate}

\subsubsection{电源和地符号}

\BlockDesc{放置电源和地符号}

\begin{enumerate}
  \item 选择 \lstinline{Place > Power} 或按快捷键 \lstinline{F}
  \item 在对话框中选择电源或地符号类型
  \item 设置网络名称(如 \lstinline{VCC}、\lstinline{GND})
  \item 放置到原理图上
\end{enumerate}

常用电源符号:

\begin{itemize}
  \item \textbf{VCC\_CIRCLE}:圆形 $V_{CC}$ 符号
  \item \textbf{VDD\_CIRCLE}:圆形 $V_{DD}$ 符号
  \item \textbf{GND\_EARTH}:地符号
  \item \textbf{GND\_POWER}:电源地符号
  \item \textbf{GND\_SIGNAL}:信号地符号
\end{itemize}

\subsubsection{层次化设计}

对于复杂电路,可以使用层次化设计:

\BlockDesc{创建层次化设计}

\begin{enumerate}
  \item 创建顶层原理图
  \item 选择 \lstinline{Place > Hierarchical Block}
  \item 设置模块名称和实例名称
  \item 绘制模块框图
  \item 添加端口(\lstinline{Place > Hierarchical Port})
  \item 创建子图(\lstinline{Design > New Page})
  \item 在子图中实现具体电路
  \item 使用 \lstinline{Place > Off-Page Connector} 连接跨页信号
\end{enumerate}

\subsection{元件属性设置}

\subsubsection{元件属性}

每个元件都有多个属性,重要的属性包括:

\begin{itemize}
  \item \textbf{Reference}:元件编号(如 R1、C1、U1)
  \item \textbf{Value}:元件值(如 10k、100nF)
  \item \textbf{PCB Footprint}:PCB 封装名称
  \item \textbf{Part}:元件型号
  \item \textbf{Manufacturer}:制造商
  \item \textbf{Part Number}:料号
\end{itemize}

\BlockDesc{编辑元件属性}

\begin{enumerate}
  \item 选择元件
  \item 右键选择 \lstinline{Edit Properties} 或按快捷键 \lstinline{Alt+Enter}
  \item 在属性对话框中编辑各项属性
  \item 可以批量编辑多个相同类型的元件
\end{enumerate}

\subsubsection{自动编号}

\BlockDesc{自动编号元件}

\begin{enumerate}
  \item 选择 \lstinline{Tools > Annotate}
  \item 在对话框中设置编号规则:
    \begin{itemize}
      \item \textbf{Scope}:编号范围(整个设计或选中的页面)
      \item \textbf{Action}:编号动作(增量、重置等)
      \item \textbf{Physical Packaging}:物理封装选项
    \end{itemize}
  \item 点击 \lstinline{OK} 执行自动编号
\end{enumerate}

\subsection{设计规则检查(DRC)}

\subsubsection{电气规则检查(ERC)}

OrCAD 可以检查原理图的电气规则:

\BlockDesc{运行 ERC}

\begin{enumerate}
  \item 选择 \lstinline{Tools > Design Rules Check}
  \item 在对话框中设置检查选项:
    \begin{itemize}
      \item \textbf{Check design rules}:检查设计规则
      \item \textbf{Check hierarchical port connection}:检查层次端口连接
      \item \textbf{Check unconnected nets}:检查未连接的网络
      \item \textbf{Check SDT compatibility}:检查 SDT 兼容性
    \end{itemize}
  \item 点击 \lstinline{OK} 运行检查
  \item 查看检查报告,修复错误和警告
\end{enumerate}

常见 ERC 错误:

\begin{itemize}
  \item 未连接的输入引脚
  \item 电源冲突
  \item 网络名称不一致
  \item 层次端口不匹配
\end{itemize}

\subsection{生成输出文件}

\subsubsection{生成网表}

网表(Netlist)是连接原理图和 PCB 布局的桥梁:

\BlockDesc{生成网表}

\begin{enumerate}
  \item 选择 \lstinline{Tools > Create Netlist}
  \item 选择网表格式:
    \begin{itemize}
      \item \textbf{Allegro}:用于 Cadence Allegro PCB Editor
      \item \textbf{Altium}:用于 Altium Designer
      \item \textbf{EDIF}:通用格式
      \item \textbf{SPICE}:用于电路仿真
    \end{itemize}
  \item 设置输出路径和文件名
  \item 点击 \lstinline{OK} 生成网表
\end{enumerate}

\subsubsection{生成物料清单(BOM)}

\BlockDesc{生成 BOM}

\begin{enumerate}
  \item 选择 \lstinline{Tools > Bill of Materials}
  \item 在对话框中设置:
    \begin{itemize}
      \item \textbf{Header}:表头信息
      \item \textbf{Combined property string}:组合属性字符串
      \item \textbf{Place each part entry on a separate line}:每个元件占一行
    \end{itemize}
  \item 选择输出格式(文本、CSV、Excel 等)
  \item 设置输出路径
  \item 点击 \lstinline{OK} 生成 BOM
\end{enumerate}

\subsubsection{生成 PDF}

\BlockDesc{导出原理图为 PDF}

\begin{enumerate}
  \item 选择 \lstinline{File > Print}
  \item 在打印对话框中选择 PDF 打印机
  \item 设置打印选项:
    \begin{itemize}
      \item \textbf{Scale}:缩放比例
      \item \textbf{Color}:颜色选项
      \item \textbf{Page Setup}:页面设置
    \end{itemize}
  \item 点击 \lstinline{Print} 生成 PDF
\end{enumerate}

\section{Altium Designer PCB 设计}

Altium Designer 是业界领先的 PCB 设计工具,集成了原理图设计、PCB 布局、仿真、制造输出等功能。

\subsection{Altium Designer 基础}

\subsubsection{项目结构}

Altium Designer 项目(.PrjPcb)包含:

\begin{itemize}
  \item \textbf{原理图文件(.SchDoc)}:原理图设计文件
  \item \textbf{PCB 文件(.PcbDoc)}:PCB 布局文件
  \item \textbf{库文件}:
    \begin{itemize}
      \item 原理图库(.SchLib)
      \item PCB 封装库(.PcbLib)
      \item 集成库(.IntLib)
    \end{itemize}
  \item \textbf{输出文件}:Gerber、钻孔文件、BOM 等
\end{itemize}

\subsubsection{工作界面}

Altium Designer 的主要工作区域:

\begin{enumerate}
  \item \textbf{项目管理器(Projects Panel)}
    \begin{itemize}
      \item 显示项目文件结构
      \item 管理设计文件
    \end{itemize}

  \item \textbf{原理图编辑器}
    \begin{itemize}
      \item 绘制原理图
      \item 放置元件和连接
    \end{itemize}

  \item \textbf{PCB 编辑器}
    \begin{itemize}
      \item PCB 布局和走线
      \item 3D 视图
      \item 设计规则检查
    \end{itemize}

  \item \textbf{库编辑器}
    \begin{itemize}
      \item 创建和编辑元件符号
      \item 创建和编辑 PCB 封装
    \end{itemize}

  \item \textbf{属性面板(Properties Panel)}
    \begin{itemize}
      \item 编辑选中对象的属性
      \item 实时更新
    \end{itemize}
\end{enumerate}

\subsection{创建新项目}

\BlockDesc{创建 Altium Designer 项目}

\begin{enumerate}
  \item 选择 \lstinline{File > New > Project > PCB Project}
  \item 在项目管理器中右键项目,选择 \lstinline{Add New to Project > Schematic}
  \item 添加 PCB 文件:\lstinline{Add New to Project > PCB}
  \item 保存项目和相关文件
\end{enumerate}

\subsection{原理图设计}

\subsubsection{放置元件}

\BlockDesc{从库中放置元件}

\begin{enumerate}
  \item 在原理图编辑器中,点击 \lstinline{Place > Part} 或按快捷键 \lstinline{P}
  \item 在元件放置对话框中:
    \begin{itemize}
      \item 选择库文件
      \item 搜索元件
      \item 设置元件属性
    \end{itemize}
  \item 点击 \lstinline{OK} 放置元件
  \item 在原理图上点击放置位置
\end{enumerate}

\subsubsection{连接网络}

\BlockDesc{连接元件}

\begin{enumerate}
  \item 选择 \lstinline{Place > Wire} 或按快捷键 \lstinline{W}
  \item 点击起始引脚
  \item 移动鼠标绘制走线
  \item 点击目标引脚完成连接
  \item 可以继续连接其他网络
  \item 按 \lstinline{ESC} 键退出连线模式
\end{enumerate}

\subsubsection{网络标签}

\BlockDesc{添加网络标签}

\begin{enumerate}
  \item 选择 \lstinline{Place > Net Label} 或按快捷键 \lstinline{N}
  \item 在属性面板中设置网络名称
  \item 放置到网络线上
  \item 相同名称的网络标签会自动连接
\end{enumerate}

\subsubsection{更新到 PCB}

\BlockDesc{将原理图更新到 PCB}

\begin{enumerate}
  \item 在原理图编辑器中,选择 \lstinline{Design > Update PCB Document}
  \item 在工程变更订单(ECO)对话框中:
    \begin{itemize}
      \item 查看变更列表
      \item 验证变更
      \item 执行变更
    \end{itemize}
  \item 点击 \lstinline{Execute Changes} 应用变更
  \item 元件和网络将出现在 PCB 编辑器中
\end{enumerate}

\subsection{PCB 布局}

\subsubsection{板框设计}

\BlockDesc{定义板框}

\begin{enumerate}
  \item 在 PCB 编辑器中,切换到 \lstinline{Keep-Out Layer}
  \item 选择 \lstinline{Place > Line} 或按快捷键 \lstinline{P > L}
  \item 绘制板框轮廓
  \item 选择所有板框线
  \item 选择 \lstinline{Design > Board Shape > Define from selected objects}
  \item 板框将根据选中的线条定义
\end{enumerate}

也可以从机械层导入板框:

\begin{enumerate}
  \item 在机械层绘制板框
  \item 选择板框线条
  \item 选择 \lstinline{Design > Board Shape > Define from selected objects}
\end{enumerate}

\subsubsection{层叠管理}

\BlockDesc{设置层叠结构}

\begin{enumerate}
  \item 选择 \lstinline{Design > Layer Stack Manager}
  \item 在层叠管理器中:
    \begin{itemize}
      \item 添加或删除层
      \item 设置层类型(信号层、平面层)
      \item 设置介质厚度
      \item 设置铜箔厚度
      \item 设置介电常数
    \end{itemize}
  \item 可以保存和加载层叠模板
  \item 点击 \lstinline{OK} 应用设置
\end{enumerate}

\subsubsection{元件布局}

\BlockDesc{元件布局原则}

\begin{enumerate}
  \item \textbf{功能分区}
    \begin{itemize}
      \item 模拟和数字电路分开
      \item 高速和低速电路分开
      \item 电源电路单独分区
    \end{itemize}

  \item \textbf{关键元件优先}
    \begin{itemize}
      \item 先放置核心 IC(MCU、FPGA 等)
      \item 再放置外围元件
      \item 考虑信号流向
    \end{itemize}

  \item \textbf{热考虑}
    \begin{itemize}
      \item 发热元件分散放置
      \item 远离温度敏感元件
      \item 考虑散热路径
    \end{itemize}

  \item \textbf{机械约束}
    \begin{itemize}
      \item 接插件放在边缘
      \item 考虑安装和维修
      \item 符合机械设计要求
    \end{itemize}
\end{enumerate}

\BlockDesc{元件对齐和分布}

\begin{enumerate}
  \item 选择多个元件
  \item 使用对齐工具:
    \begin{itemize}
      \item \lstinline{Edit > Align > Align Left/Right/Top/Bottom}
      \item \lstinline{Edit > Align > Distribute Horizontally/Vertically}
    \end{itemize}
  \item 或使用快捷键和右键菜单
\end{enumerate}

\subsubsection{设计规则设置}

设计规则(Design Rules)是 PCB 设计的核心约束:

\BlockDesc{设置设计规则}

\begin{enumerate}
  \item 选择 \lstinline{Design > Rules}
  \item 在规则编辑器中设置各类规则:
    \begin{itemize}
      \item \textbf{Electrical}:电气规则
        \begin{itemize}
          \item Clearance:间距规则
          \item Short-Circuit:短路规则
          \item Un-Routed Net:未布线网络
          \item Un-Connected Pin:未连接引脚
        \end{itemize}
      \item \textbf{Routing}:布线规则
        \begin{itemize}
          \item Width:线宽规则
          \item Routing Topology:拓扑规则
          \item Routing Priority:优先级
          \item Routing Layers:布线层
          \item Routing Corners:拐角规则
          \item Routing Via Style:过孔样式
        \end{itemize}
      \item \textbf{Manufacturing}:制造规则
        \begin{itemize}
          \item Minimum Solder Mask Sliver:阻焊间隙
          \item Silk to Solder Mask Clearance:丝印到阻焊间距
          \item Hole Size:孔径规则
        \end{itemize}
      \item \textbf{Plane}:平面规则
        \begin{itemize}
          \item Power Plane Connect Style:电源平面连接样式
          \item Power Plane Clearance:电源平面间距
        \end{itemize}
      \item \textbf{Testpoint}:测试点规则
      \item \textbf{High Speed}:高速规则
        \begin{itemize}
          \item Parallel Segment:平行段规则
          \item Length:长度规则
          \item Matched Net Lengths:等长规则
          \item Differential Pairs Routing:差分对布线
        \end{itemize}
      \item \textbf{Placement}:布局规则
      \item \textbf{Signal Integrity}:信号完整性规则
    \end{itemize}
  \item 为不同网络类设置不同规则
  \item 点击 \lstinline{OK} 应用规则
\end{enumerate}

\subsubsection{阻抗控制}

\BlockDesc{设置阻抗控制规则}

\begin{enumerate}
  \item 在层叠管理器中设置正确的介质参数
  \item 选择 \lstinline{Design > Rules > Routing > Width}
  \item 创建新的线宽规则
  \item 设置:
    \begin{itemize}
      \item \textbf{Name}:规则名称(如 \lstinline{50Ohm_Single})
      \item \textbf{Where the First object matches}:适用范围
      \item \textbf{Preferred Width}:首选线宽
      \item \textbf{Min Width}:最小线宽
      \item \textbf{Max Width}:最大线宽
    \end{itemize}
  \item 使用阻抗计算工具验证:
    \begin{itemize}
      \item \lstinline{Tools > Impedance Calculation}
      \item 输入层叠参数
      \item 计算特性阻抗
      \item 调整线宽以达到目标阻抗
    \end{itemize}
\end{enumerate}

\subsubsection{走线}

\BlockDesc{手动走线}

\begin{enumerate}
  \item 选择 \lstinline{Route > Interactive Routing} 或按快捷键 \lstinline{P > T}
  \item 点击起始焊盘
  \item 移动鼠标绘制走线
  \item 可以切换层(按数字键或 \lstinline{+/-} 键)
  \item 自动添加过孔
  \item 点击目标焊盘完成走线
  \item 按 \lstinline{ESC} 键退出走线模式
\end{enumerate}

走线技巧:

\begin{itemize}
  \item 使用 \lstinline{Shift+Space} 切换走线模式(45°、90°、圆弧等)
  \item 使用 \lstinline{Space} 切换走线方向
  \item 使用 \lstinline{Backspace} 撤销上一步
  \item 使用 \lstinline{Ctrl+Click} 完成走线并继续
  \item 使用 \lstinline{Tab} 键打开属性面板,调整走线参数
\end{itemize}

\BlockDesc{差分对走线}

\begin{enumerate}
  \item 在原理图中定义差分对网络(使用 \lstinline{_P} 和 \lstinline{_N} 后缀)
  \item 更新到 PCB
  \item 在 PCB 编辑器中,选择 \lstinline{Design > Classes > Differential Pair Classes}
  \item 创建差分对类并添加差分对
  \item 选择 \lstinline{Route > Interactive Differential Pair Routing} 或按快捷键 \lstinline{P > I}
  \item 点击差分对的任意一个焊盘开始走线
  \item 两条线将同时走线,保持间距
  \item 可以调整间距(\lstinline{Shift+,} 减小,\lstinline{Shift+.} 增大)
\end{enumerate}

\subsubsection{过孔}

\BlockDesc{放置过孔}

\begin{enumerate}
  \item 在走线过程中,按 \lstinline{+} 或 \lstinline{-} 键切换层
  \item 自动添加过孔
  \item 也可以手动放置:\lstinline{Place > Via} 或按快捷键 \lstinline{P > V}
  \item 在属性面板中设置过孔参数:
    \begin{itemize}
      \item \textbf{Hole Size}:孔径
      \item \textbf{Diameter}:过孔直径
      \item \textbf{Start Layer}:起始层
      \item \textbf{End Layer}:结束层
    \end{itemize}
\end{enumerate}

\subsubsection{铺铜}

\BlockDesc{铺铜操作}

\begin{enumerate}
  \item 选择 \lstinline{Place > Polygon Pour} 或按快捷键 \lstinline{P > G}
  \item 在属性面板中设置:
    \begin{itemize}
      \item \textbf{Net}:连接到哪个网络(通常是 GND 或电源)
      \item \textbf{Layer}:铺铜层
      \item \textbf{Pour Over All Same Net Objects}:覆盖同网络对象
      \item \textbf{Remove Dead Copper}:移除死铜
      \item \textbf{Grid Size}:网格大小
      \item \textbf{Track Width}:网格线宽
    \end{itemize}
  \item 绘制铺铜区域轮廓
  \item 右键完成绘制
  \item 铺铜将自动填充并避让其他对象
\end{enumerate}

\BlockDesc{重新铺铜}

\begin{enumerate}
  \item 选择铺铜对象
  \item 右键选择 \lstinline{Polygon Actions > Repour Selected}
  \item 或选择 \lstinline{Tools > Polygon Pours > Repour All}
\end{enumerate}

\subsubsection{等长走线}

对于需要等长的信号(如 DDR、并行总线),需要等长走线:

\BlockDesc{等长走线}

\begin{enumerate}
  \item 选择 \lstinline{Route > Interactive Length Tuning} 或按快捷键 \lstinline{P > Y}
  \item 点击需要调整长度的走线
  \item 移动鼠标创建蛇形走线
  \item 在属性面板中设置:
    \begin{itemize}
      \item \textbf{Target Length}:目标长度
      \item \textbf{Amplitude}:振幅
      \item \textbf{Gap}:间距
      \item \textbf{Style}:样式(Mitered、Rounded、Rounded 45°)
    \end{itemize}
  \item 可以查看长度信息(\lstinline{Reports > Measure Distance})
\end{enumerate}

\BlockDesc{使用等长规则}

\begin{enumerate}
  \item 选择 \lstinline{Design > Rules > High Speed > Matched Net Lengths}
  \item 创建新规则
  \item 设置:
    \begin{itemize}
      \item \textbf{Scope}:适用范围
      \item \textbf{Tolerance}:容差
      \item \textbf{Style}:等长样式
    \end{itemize}
  \item 在 PCB 编辑器中,选择 \lstinline{Route > Interactive Length Tuning}
  \item 走线长度信息会实时显示
  \item 达到目标长度后完成
\end{enumerate}

\subsection{设计规则检查(DRC)}

\BlockDesc{运行 DRC}

\begin{enumerate}
  \item 选择 \lstinline{Tools > Design Rule Check}
  \item 在 DRC 对话框中:
    \begin{itemize}
      \item 选择要检查的规则类别
      \item 选择检查范围(整个板或选中的对象)
      \item 设置报告选项
    \end{itemize}
  \item 点击 \lstinline{Run Design Rule Check}
  \item 查看检查报告
  \item 在 PCB 编辑器中,违反规则的地方会高亮显示
  \item 修复所有错误和警告
\end{enumerate}

\subsection{3D 视图}

\BlockDesc{查看 3D 视图}

\begin{enumerate}
  \item 选择 \lstinline{View > 3D Layout Mode} 或按快捷键 \lstinline{3}
  \item 使用鼠标操作:
    \begin{itemize}
      \item 左键拖动:旋转视图
      \item 右键拖动:平移视图
      \item 滚轮:缩放
      \item \lstinline{Shift + 右键拖动}:调整视角
    \end{itemize}
  \item 可以导出 3D 模型(\lstinline{File > Export > STEP 3D})
\end{enumerate}

\subsection{输出制造文件}

\subsubsection{Gerber 文件}

\BlockDesc{生成 Gerber 文件}

\begin{enumerate}
  \item 选择 \lstinline{File > Fabrication Outputs > Gerber Files}
  \item 在 Gerber 设置对话框中:
    \begin{itemize}
      \item \textbf{General}:通用设置
        \begin{itemize}
          \item Units:单位(Inches 或 Millimeters)
          \item Format:格式(2:3、2:4、2:5)
        \end{itemize}
      \item \textbf{Layers}:层设置
        \begin{itemize}
          \item Plot Layers:选择要输出的层
          \item Mirror Layers:镜像层
        \end{itemize}
      \item \textbf{Drill Drawing}:钻孔图
      \item \textbf{Apertures}:光圈表
      \item \textbf{Advanced}:高级设置
    \end{itemize}
  \item 点击 \lstinline{OK} 生成 Gerber 文件
  \item 文件将保存在项目输出文件夹中
\end{enumerate}

\subsubsection{钻孔文件}

\BlockDesc{生成钻孔文件}

\begin{enumerate}
  \item 选择 \lstinline{File > Fabrication Outputs > NC Drill Files}
  \item 在钻孔设置对话框中:
    \begin{itemize}
      \item \textbf{Units}:单位
      \item \textbf{Format}:格式
      \item \textbf{Leading/Trailing Zeroes}:前导/尾随零
    \end{itemize}
  \item 点击 \lstinline{OK} 生成钻孔文件
  \item 通常生成 .drl 和 .txt 文件
\end{enumerate}

\subsubsection{贴片坐标文件}

\BlockDesc{生成贴片坐标文件}

\begin{enumerate}
  \item 选择 \lstinline{File > Assembly Outputs > Generates pick and place files}
  \item 在对话框中设置:
    \begin{itemize}
      \item \textbf{Format}:格式(CSV、TXT)
      \item \textbf{Units}:单位
      \item \textbf{Include}:包含的信息
    \end{itemize}
  \item 点击 \lstinline{OK} 生成文件
\end{enumerate}

\subsubsection{物料清单(BOM)}

\BlockDesc{生成 BOM}

\begin{enumerate}
  \item 选择 \lstinline{Reports > Bill of Materials}
  \item 在 BOM 对话框中:
    \begin{itemize}
      \item 选择要包含的列
      \item 设置分组和排序
      \item 选择输出格式(Excel、CSV、PDF 等)
    \end{itemize}
  \item 点击 \lstinline{Export} 导出 BOM
\end{enumerate}

\subsubsection{装配图}

\BlockDesc{生成装配图}

\begin{enumerate}
  \item 在 PCB 编辑器中,切换到装配层(Top Overlay、Bottom Overlay)
  \item 选择 \lstinline{File > Smart PDF}
  \item 在对话框中:
    \begin{itemize}
      \item 选择要包含的页面
      \item 设置 PDF 选项
      \item 选择输出路径
    \end{itemize}
  \item 点击 \lstinline{OK} 生成 PDF
\end{enumerate}

\section{KiCad PCB 设计}

KiCad 是一款开源、跨平台的 PCB 设计工具,功能强大且完全免费,适合个人和小团队使用。

\subsection{KiCad 基础}

\subsubsection{项目结构}

KiCad 项目包含以下文件:

\begin{itemize}
  \item \textbf{项目文件(.kicad\_pro)}:项目配置文件
  \item \textbf{原理图文件(.kicad\_sch)}:原理图设计文件
  \item \textbf{PCB 文件(.kicad\_pcb)}:PCB 布局文件
  \item \textbf{库文件}:
    \begin{itemize}
      \item 原理图符号库(.kicad\_sym)
      \item PCB 封装库(.kicad\_mod)
      \item 3D 模型库
    \end{itemize}
\end{itemize}

\subsubsection{工作界面}

KiCad 采用模块化设计,主要程序包括:

\begin{enumerate}
  \item \textbf{KiCad}:项目管理器
  \item \textbf{Eeschema}:原理图编辑器
  \item \textbf{PCB Editor}:PCB 布局编辑器
  \item \textbf{Footprint Editor}:封装编辑器
  \item \textbf{Symbol Editor}:符号编辑器
  \item \textbf{3D Viewer}:3D 查看器
\end{enumerate}

\subsection{创建新项目}

\BlockDesc{创建 KiCad 项目}

\begin{enumerate}
  \item 启动 KiCad
  \item 选择 \lstinline{File > New Project}
  \item 选择保存位置和项目名称
  \item 点击 \lstinline{Save} 创建项目
  \item 项目将包含原理图和 PCB 文件
\end{enumerate}

\subsection{原理图设计}

\subsubsection{放置元件}

\BlockDesc{从库中放置元件}

\begin{enumerate}
  \item 在 Eeschema 中,点击工具栏的 \lstinline{Add a symbol} 按钮或按快捷键 \lstinline{A}
  \item 在符号选择器中:
    \begin{itemize}
      \item 浏览或搜索元件
      \item 预览元件符号
    \end{itemize}
  \item 双击元件或点击 \lstinline{OK} 放置
  \item 在原理图上点击放置位置
  \item 可以连续放置多个相同元件
  \item 按 \lstinline{ESC} 键退出放置模式
\end{enumerate}

\subsubsection{连接网络}

\BlockDesc{连接元件}

\begin{enumerate}
  \item 点击工具栏的 \lstinline{Add a wire} 按钮或按快捷键 \lstinline{W}
  \item 点击起始引脚
  \item 移动鼠标绘制走线
  \item 点击目标引脚完成连接
  \item 可以继续连接其他网络
  \item 按 \lstinline{ESC} 键退出连线模式
\end{enumerate}

\subsubsection{网络标签}

\BlockDesc{添加网络标签}

\begin{enumerate}
  \item 点击工具栏的 \lstinline{Add a net label} 按钮或按快捷键 \lstinline{L}
  \item 在属性对话框中输入网络名称
  \item 放置到网络线上
  \item 相同名称的网络标签会自动连接
\end{enumerate}

\subsubsection{更新到 PCB}

\BlockDesc{将原理图更新到 PCB}

\begin{enumerate}
  \item 在 Eeschema 中,选择 \lstinline{Tools > Update PCB from Schematic} 或按快捷键 \lstinline{F8}
  \item 在更新对话框中:
    \begin{itemize}
      \item 查看变更列表
      \item 选择要应用的变更
    \end{itemize}
  \item 点击 \lstinline{Update PCB} 应用变更
  \item 切换到 PCB Editor 查看更新结果
\end{enumerate}

\subsection{PCB 布局}

\subsubsection{板框设计}

\BlockDesc{定义板框}

\begin{enumerate}
  \item 在 PCB Editor 中,切换到 \lstinline{Edge.Cuts} 层
  \item 选择 \lstinline{Add > Graphic Line} 或按快捷键 \lstinline{E}
  \item 绘制板框轮廓
  \item 板框将根据绘制的线条自动定义
\end{enumerate}

也可以从 DXF 文件导入板框:

\begin{enumerate}
  \item 选择 \lstinline{File > Import > Graphics}
  \item 选择 DXF 文件
  \item 设置导入选项
  \item 导入后调整到 \lstinline{Edge.Cuts} 层
\end{enumerate}

\subsubsection{层叠管理}

\BlockDesc{设置层叠结构}

\begin{enumerate}
  \item 选择 \lstinline{Setup > Board Setup > Physical Stackup}
  \item 在层叠管理器中:
    \begin{itemize}
      \item 添加或删除层
      \item 设置层类型(信号层、平面层)
      \item 设置介质厚度
      \item 设置铜箔厚度
      \item 设置介电常数
    \end{itemize}
  \item 可以保存和加载层叠模板
  \item 点击 \lstinline{OK} 应用设置
\end{enumerate}

\subsubsection{设计规则设置}

\BlockDesc{设置设计规则}

\begin{enumerate}
  \item 选择 \lstinline{Setup > Design Rules}
  \item 在规则编辑器中设置:
    \begin{itemize}
      \item \textbf{Net Classes}:网络类
      \item \textbf{Clearance}:间距规则
      \item \textbf{Track Width}:线宽规则
      \item \textbf{Differential Pairs}:差分对规则
      \item \textbf{Via Sizes}:过孔尺寸规则
      \item \textbf{Solder Mask/Paste}:阻焊和钢网规则
    \end{itemize}
  \item 为不同网络类设置不同规则
  \item 点击 \lstinline{OK} 应用规则
\end{enumerate}

\subsubsection{走线}

\BlockDesc{手动走线}

\begin{enumerate}
  \item 选择 \lstinline{Route > Add Filled Zone} 或按快捷键 \lstinline{X}
  \item 点击起始焊盘
  \item 移动鼠标绘制走线
  \item 可以切换层(按 \lstinline{V} 键添加过孔)
  \item 点击目标焊盘完成走线
  \item 按 \lstinline{ESC} 键退出走线模式
\end{enumerate}

\BlockDesc{差分对走线}

\begin{enumerate}
  \item 在原理图中定义差分对(使用 \lstinline{_P} 和 \lstinline{_N} 后缀)
  \item 更新到 PCB
  \item 在 PCB Editor 中,选择 \lstinline{Setup > Design Rules > Differential Pairs}
  \item 创建差分对规则
  \item 选择 \lstinline{Route > Interactive Router}
  \item 选择差分对开始走线
  \item 两条线将同时走线,保持间距
\end{enumerate}

\subsubsection{铺铜}

\BlockDesc{铺铜操作}

\begin{enumerate}
  \item 选择 \lstinline{Add > Zone} 或按快捷键 \lstinline{Z}
  \item 在属性对话框中设置:
    \begin{itemize}
      \item \textbf{Net}:连接到哪个网络
      \item \textbf{Layer}:铺铜层
      \item \textbf{Fill Settings}:填充设置
    \end{itemize}
  \item 绘制铺铜区域轮廓
  \item 双击完成绘制
  \item 铺铜将自动填充并避让其他对象
\end{enumerate}

\subsection{输出制造文件}

\subsubsection{Gerber 文件}

\BlockDesc{生成 Gerber 文件}

\begin{enumerate}
  \item 选择 \lstinline{File > Fabrication Outputs > Gerbers}
  \item 在 Gerber 输出对话框中:
    \begin{itemize}
      \item 选择要输出的层
      \item 设置格式选项
      \item 选择输出目录
    \end{itemize}
  \item 点击 \lstinline{Generate Drill Files} 生成钻孔文件
  \item 点击 \lstinline{Plot} 生成 Gerber 文件
\end{enumerate}

\subsubsection{钻孔文件}

\BlockDesc{生成钻孔文件}

\begin{enumerate}
  \item 在 Gerber 输出对话框中,点击 \lstinline{Generate Drill Files}
  \item 在钻孔文件对话框中:
    \begin{itemize}
      \item 设置格式选项
      \item 选择输出目录
    \end{itemize}
  \item 点击 \lstinline{Generate Drill File} 生成钻孔文件
\end{enumerate}

\subsubsection{物料清单(BOM)}

\BlockDesc{生成 BOM}

\begin{enumerate}
  \item 在 Eeschema 中,选择 \lstinline{Tools > Generate Bill of Materials}
  \item 在 BOM 对话框中:
    \begin{itemize}
      \item 选择 BOM 插件
      \item 设置输出格式
      \item 选择输出路径
    \end{itemize}
  \item 点击 \lstinline{Generate} 生成 BOM
\end{enumerate}

\section{封装库管理}

封装库是 PCB 设计的基础,良好的封装库管理能提高设计效率和质量。

\subsection{封装库结构}

\subsubsection{封装组成}

PCB 封装(Footprint)通常包含:

\begin{itemize}
  \item \textbf{焊盘(Pad)}:元件引脚连接点
  \item \textbf{丝印(Silkscreen)}:元件外形和标识
  \item \textbf{阻焊(Solder Mask)}:焊盘开窗
  \item \textbf{钢网(Paste Mask)}:贴片钢网开窗
  \item \textbf{3D 模型}:3D 显示用
  \item \textbf{装配层(Assembly)}:装配图用
\end{itemize}

\subsubsection{封装命名规范}

封装命名应遵循规范,便于识别和管理:

\begin{itemize}
  \item \textbf{贴片电阻}:\lstinline{R_0805}、\lstinline{R_0603} 等
  \item \textbf{贴片电容}:\lstinline{C_0805}、\lstinline{C_0603} 等
  \item \textbf{IC 封装}:\lstinline{SOIC-8}、\lstinline{QFN-48}、\lstinline{BGA-256} 等
  \item \textbf{连接器}:\lstinline{CONN_2P54_2X5}、\lstinline{USB-C} 等
\end{itemize}

\subsection{创建封装}

\subsubsection{从数据手册创建}

\BlockDesc{根据数据手册创建封装}

\begin{enumerate}
  \item 获取元件数据手册
  \item 查找封装尺寸图(Package Dimensions)
  \item 在封装编辑器中:
    \begin{itemize}
      \item 设置网格和单位
      \item 放置焊盘,设置尺寸和位置
      \item 绘制丝印外形
      \item 添加标识(如引脚 1 标记)
    \end{itemize}
  \item 验证尺寸是否正确
  \item 保存封装到库文件
\end{enumerate}

\subsubsection{使用封装向导}

大多数 PCB 工具提供封装向导,可以快速创建标准封装:

\BlockDesc{使用封装向导}

\begin{enumerate}
  \item 在封装编辑器中,选择封装向导
  \item 选择封装类型(如 SOIC、QFP、BGA 等)
  \item 输入封装参数:
    \begin{itemize}
      \item 引脚数量
      \item 引脚间距(Pitch)
      \item 封装尺寸
      \item 焊盘尺寸
    \end{itemize}
  \item 预览封装
  \item 生成封装
  \item 检查和调整
  \item 保存封装
\end{enumerate}

\subsection{封装库管理}

\subsubsection{库文件组织}

良好的库文件组织:

\begin{itemize}
  \item \textbf{按元件类型分类}:电阻库、电容库、IC 库等
  \item \textbf{按制造商分类}:不同制造商的封装分开
  \item \textbf{按封装类型分类}:贴片库、通孔库等
  \item \textbf{使用版本控制}:使用 Git 等工具管理库文件
\end{itemize}

\subsubsection{封装验证}

\BlockDesc{验证封装正确性}

\begin{enumerate}
  \item \textbf{尺寸检查}
    \begin{itemize}
      \item 对比数据手册尺寸
      \item 检查焊盘尺寸是否合适
      \item 检查引脚间距是否正确
    \end{itemize}

  \item \textbf{3D 模型检查}
    \begin{itemize}
      \item 加载 3D 模型
      \item 检查模型是否匹配
      \item 检查高度是否正确
    \end{itemize}

  \item \textbf{设计规则检查}
    \begin{itemize}
      \item 运行 DRC 检查
      \item 检查焊盘间距
      \item 检查丝印是否覆盖焊盘
    \end{itemize}

  \item \textbf{实际验证}
    \begin{itemize}
      \item 制作测试板
      \item 实际贴装验证
      \item 记录问题并修正
    \end{itemize}
\end{enumerate}

\section{设计规则和约束}

设计规则是确保 PCB 设计质量和可制造性的关键。

\subsection{间距规则}

\subsubsection{走线间距}

走线间距影响:

\begin{itemize}
  \item \textbf{串扰}:间距越小,串扰越大
  \item \textbf{可制造性}:间距必须符合制造能力
  \item \textbf{成本}:更小的间距通常意味着更高的成本
\end{itemize}

典型间距设置:

\begin{itemize}
  \item \textbf{信号线}:3--4 mil(0.075--0.1 mm)
  \item \textbf{电源线}:根据电压等级,通常 8--20 mil
  \item \textbf{差分对内部}:根据阻抗要求
  \item \textbf{差分对之间}:至少 3 倍线宽
\end{itemize}

\subsubsection{焊盘间距}

焊盘间距必须符合元件封装要求:

\begin{itemize}
  \item 检查元件数据手册
  \item 考虑制造公差
  \item 留有余量
\end{itemize}

\subsection{线宽规则}

\subsubsection{载流能力}

线宽必须满足载流要求:

\begin{itemize}
  \item 使用 IPC-2221 标准计算
  \item 考虑温升限制
  \item 考虑内层和外层差异
  \item 考虑铜箔厚度
\end{itemize}

\subsubsection{阻抗控制}

对于高速信号,线宽必须满足阻抗要求:

\begin{itemize}
  \item 使用阻抗计算工具
  \item 考虑层叠结构
  \item 考虑介电常数
  \item 与 PCB 厂商确认
\end{itemize}

\subsection{过孔规则}

\subsubsection{过孔尺寸}

过孔尺寸影响:

\begin{itemize}
  \item \textbf{可制造性}:最小孔径受制造能力限制
  \item \textbf{成本}:更小的过孔通常更贵
  \item \textbf{可靠性}:过孔太小可能影响可靠性
\end{itemize}

典型过孔设置:

\begin{itemize}
  \item \textbf{通孔}:孔径 8--12 mil,直径 16--24 mil
  \item \textbf{盲埋孔}:孔径 4--6 mil,直径 8--12 mil
  \item \textbf{微孔}:孔径 2--4 mil,直径 4--8 mil
\end{itemize}

\subsubsection{过孔间距}

过孔间距规则:

\begin{itemize}
  \item 过孔之间最小间距:通常为过孔直径
  \item 过孔到走线间距:至少 5 mil
  \item 过孔到焊盘间距:至少 8 mil
\end{itemize}

\section{制造文件输出}

制造文件是 PCB 设计和制造之间的桥梁,必须准确无误。

\subsection{Gerber 文件格式}

\subsubsection{Gerber 文件类型}

Gerber 文件包括:

\begin{itemize}
  \item \textbf{顶层铜层(.GTL)}:Top Layer
  \item \textbf{底层铜层(.GBL)}:Bottom Layer
  \item \textbf{内层(.G1、.G2 等)}:Inner Layers
  \item \textbf{顶层阻焊(.GTS)}:Top Solder Mask
  \item \textbf{底层阻焊(.GBS)}:Bottom Solder Mask
  \item \textbf{顶层丝印(.GTO)}:Top Overlay
  \item \textbf{底层丝印(.GBO)}:Bottom Overlay
  \item \textbf{顶层钢网(.GTP)}:Top Paste
  \item \textbf{底层钢网(.GBP)}:Bottom Paste
  \item \textbf{板框(.GKO 或 .GM1)}:Keep-Out 或 Mechanical
\end{itemize}

\subsubsection{Gerber 格式}

常用 Gerber 格式:

\begin{itemize}
  \item \textbf{RS-274X}:扩展 Gerber 格式,包含光圈表
  \item \textbf{RS-274D}:标准 Gerber 格式,需要单独的光圈文件
  \item \textbf{ODB++}:更高级的格式,包含更多信息
\end{itemize}

\subsection{钻孔文件}

\subsubsection{钻孔文件格式}

常用格式:

\begin{itemize}
  \item \textbf{Excellon}:最常用的格式(.drl、.txt)
  \item \textbf{IPC-NC-349}:IPC 标准格式
\end{itemize}

\subsubsection{钻孔文件内容}

钻孔文件包含:

\begin{itemize}
  \item \textbf{工具列表}:孔径和工具编号对应
  \item \textbf{钻孔坐标}:每个孔的坐标和工具编号
  \item \textbf{单位}:英寸或毫米
  \item \textbf{格式}:坐标精度
\end{itemize}

\subsection{制造文件检查}

\BlockDesc{检查制造文件}

\begin{enumerate}
  \item \textbf{使用 Gerber 查看器}
    \begin{itemize}
      \item 使用 GC-Prevue、ViewMate 等工具
      \item 检查所有层是否正确
      \item 检查尺寸是否正确
      \item 检查是否有遗漏
    \end{itemize}

  \item \textbf{对比原始设计}
    \begin{itemize}
      \item 在 PCB 工具中生成 PDF
      \item 与 Gerber 文件对比
      \item 检查差异
    \end{itemize}

  \item \textbf{钻孔文件检查}
    \begin{itemize}
      \item 检查孔径是否正确
      \item 检查坐标是否正确
      \item 检查是否有遗漏的孔
    \end{itemize}

  \item \textbf{与 PCB 厂商沟通}
    \begin{itemize}
      \item 发送文件前先沟通
      \item 确认文件格式和版本
      \item 确认特殊要求
    \end{itemize}
\end{enumerate}

\section{仿真工具使用}

仿真工具可以帮助在设计阶段验证电路性能,减少设计迭代。

\subsection{SPICE 仿真}

\subsubsection{SPICE 基础}

SPICE(Simulation Program with Integrated Circuit Emphasis)是电路仿真的标准工具。

\BlockDesc{运行 SPICE 仿真}

\begin{enumerate}
  \item 在原理图工具中,添加仿真模型
  \item 设置仿真参数:
    \begin{itemize}
      \item \textbf{DC Analysis}:直流分析
      \item \textbf{AC Analysis}:交流分析
      \item \textbf{Transient Analysis}:瞬态分析
    \end{itemize}
  \item 设置激励源
  \item 运行仿真
  \item 查看仿真结果
\end{enumerate}

\subsubsection{常用 SPICE 仿真}

\begin{itemize}
  \item \textbf{直流工作点分析}:计算静态工作点
  \item \textbf{交流小信号分析}:频率响应
  \item \textbf{瞬态分析}:时域响应
  \item \textbf{参数扫描}:参数变化对性能的影响
  \item \textbf{蒙特卡洛分析}:考虑元件容差的影响
\end{itemize}

\subsection{信号完整性仿真}

\subsubsection{SI 仿真工具}

常用 SI 仿真工具:

\begin{itemize}
  \item \textbf{Altium Designer}:内置 SI 仿真
  \item \textbf{Cadence Sigrity}:专业 SI/PI 工具
  \item \textbf{Keysight ADS}:高级 SI 仿真
  \item \textbf{HyperLynx}:Mentor 的 SI 工具
\end{itemize}

\BlockDesc{运行 SI 仿真}

\begin{enumerate}
  \item 在 PCB 工具中,选择要仿真的网络
  \item 设置仿真参数:
    \begin{itemize}
      \item 激励信号(上升时间、幅度等)
      \item 仿真时间
      \item 输出节点
    \end{itemize}
  \item 运行仿真
  \item 分析结果:
    \begin{itemize}
      \item 眼图
      \item 信号完整性
      \item 时序裕量
    \end{itemize}
  \item 根据结果调整设计
\end{enumerate}

\subsection{电源完整性仿真}

\subsubsection{PI 仿真工具}

常用 PI 仿真工具:

\begin{itemize}
  \item \textbf{Cadence Sigrity PowerDC}:PI 分析工具
  \item \textbf{Keysight ADS PIPro}:PI 仿真
  \item \textbf{Altium Designer}:内置 PI 分析
\end{itemize}

\BlockDesc{运行 PI 仿真}

\begin{enumerate}
  \item 设置电源分配网络(PDN)
  \item 设置电流负载
  \item 运行仿真
  \item 分析结果:
    \begin{itemize}
      \item 电压降
      \item 电流密度
      \item 阻抗
    \end{itemize}
  \item 优化去耦电容布局
\end{enumerate}

\section{设计流程和最佳实践}

\subsection{标准设计流程}

\subsubsection{设计阶段}

标准 PCB 设计流程:

\begin{enumerate}
  \item \textbf{需求分析}
    \begin{itemize}
      \item 功能需求
      \item 性能指标
      \item 约束条件(尺寸、成本等)
    \end{itemize}

  \item \textbf{原理图设计}
    \begin{itemize}
      \item 电路设计
      \item 元件选型
      \item 仿真验证
    \end{itemize}

  \item \textbf{PCB 布局}
    \begin{itemize}
      \item 元件布局
      \item 层叠设计
      \item 走线设计
    \end{itemize}

  \item \textbf{设计验证}
    \begin{itemize}
      \item DRC 检查
      \item SI/PI 仿真
      \item EMC 考虑
    \end{itemize}

  \item \textbf{制造文件输出}
    \begin{itemize}
      \item Gerber 文件
      \item 钻孔文件
      \item BOM 和装配图
    \end{itemize}

  \item \textbf{制造和测试}
    \begin{itemize}
      \item PCB 制造
      \item 元件贴装
      \item 功能测试
    \end{itemize}
\end{enumerate}

\subsection{设计最佳实践}

\subsubsection{原理图设计}

\begin{itemize}
  \item 使用有意义的网络名称
  \item 添加必要的注释和说明
  \item 使用层次化设计管理复杂电路
  \item 保持原理图整洁和易读
  \item 进行 ERC 检查
\end{itemize}

\subsubsection{PCB 布局}

\begin{itemize}
  \item 遵循布局原则(功能分区、信号流向等)
  \item 合理使用层叠结构
  \item 设置合适的设计规则
  \item 控制阻抗
  \item 考虑热设计
  \item 考虑 EMC
\end{itemize}

\subsubsection{走线设计}

\begin{itemize}
  \item 关键信号优先走线
  \item 避免长距离平行走线
  \item 使用差分对传输高速信号
  \item 等长走线满足时序要求
  \item 避免锐角
  \item 保持返回路径连续
\end{itemize}

\subsubsection{文档管理}

\begin{itemize}
  \item 保持设计文件版本控制
  \item 记录设计变更
  \item 保存设计评审记录
  \item 维护元件库
  \item 保存制造文件
\end{itemize}

\section{常见问题和解决方案}

\subsection{原理图问题}

\subsubsection{常见错误}

\begin{itemize}
  \item \textbf{未连接的引脚}:检查 ERC 报告,连接所有必要的引脚
  \item \textbf{网络名称不一致}:检查网络标签,确保名称一致
  \item \textbf{电源冲突}:检查电源网络,避免冲突
  \item \textbf{封装缺失}:为所有元件指定正确的封装
\end{itemize}

\subsection{PCB 布局问题}

\subsubsection{常见错误}

\begin{itemize}
  \item \textbf{DRC 错误}:运行 DRC,修复所有错误
  \item \textbf{阻抗不匹配}:检查层叠和线宽设置
  \item \textbf{等长不满足}:使用等长工具调整走线长度
  \item \textbf{去耦不足}:添加足够的去耦电容
\end{itemize}

\subsection{制造问题}

\subsubsection{常见问题}

\begin{itemize}
  \item \textbf{文件格式错误}:确认使用正确的文件格式
  \item \textbf{尺寸错误}:检查单位设置
  \item \textbf{层缺失}:确认所有必要的层都已输出
  \item \textbf{钻孔文件错误}:检查钻孔文件格式和内容
\end{itemize}

\section{总结}

PCB 设计工具是电子工程师进行电路设计的核心工具。掌握主流 PCB 设计软件的使用方法,包括原理图绘制、PCB 布局、封装库管理、设计规则设置、制造文件输出等,是完成高质量 PCB 设计的基础。随着电子系统向高速、高密度、多功能方向发展,PCB 设计工具也在不断更新和完善。持续学习和实践,掌握工具的高级功能和最佳实践,是提高设计效率和质量的关键。
